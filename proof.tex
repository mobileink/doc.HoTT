%%%%%%%%%%%%%%%%%%%%%%%%%%%%%%%%
\section{Proof}
\label{sect:proof}

Traditional (classic) view: a proof is an epistemic device; it
displays, exhibits, makes \textit{visible} (if only to the mind's eye)
a form of \textit{certain knowledge}.\sidenote{The link between
  knowing and seeing runs very deep in Western culture.  Not
  surprisingly it is closely connected with representationalism and
  cartesianism generally.  It has pretty much dominated Western
  thinking since Descartes, but has come under strong attack from
  Pragmatists.  Dewey called it ``the spectator theory of knowledge.''f
  See \citep{rorty_philosophy_2009} etc.}

Alternatives to the spectator theory: pragmatism, know-how over know-that.

\begin{ednote}
  TODO: summary of concepts of proof.  Emphasize contrast between
  representationalism and inferentialism.  Representationalism is
  atomistic: you could have only one concept.  Inferentialism is
  holistic: you have to start out with at least two concepts, since
  every inference involves a premise and a conclusion.  Inferentialism
  is a natural fit for \HoTT.

  Question: can you have only one type?  In other words, is type
  theory essentially holistic or atomistic?
\end{ednote}


For \HoTT{}, as for most varieties of constructivism, it is better to
abandon traditional notions of proof as something you see in favor of a
more pragmatic notion of proof as something you do.

etc.

Critical point: in \HoTT we have two ``kinds'' of types: propositional
types and non-propositional types.\sidenote{This is not in general
  recognized in \HoTTB, but I think it should be emphasized, if only
  because it reflects intuition.}  If we are to also treat ``proof''
(or witness or whatever) as a fundamental principle of \HoTT, one that
complements the concept of type, then we need to treat both ``type''
and ``proof'' as genuses (genii?) of which propositional and non-propositional
are species.

\begin{ednote}
  General point (to be made elsewhere, maybe in
  \cref{sect:foundations}: the concepts of type and proof go together.
  You cannot have one without the other.  That's very different than
  set theory.  You can have sets and elements without proofs.
\end{ednote}



Long story short: we are in dire need of improved terminology.  My
suggestion is as follows:

\begin{description}
\item [Proof of a proposition] In contrast to the classic spectator
  view, we treat proof not as the exhibition (or: making available for
  inspection) of the form of a bit of certain knowledge, but as the
  \textit{demonstrative expression} of the proposition.
  Alternatively, the expressive demonstration of the proposition.  So
  whereas a classic proof is something that must be ``seen'' in order
  to be grasped, a type-theoretic proof is something that must be
  actively \textit{done}, not merely passively observed.  One must be
  able to follow the construction of the proof.

\item [Proof of a non-propositional type] Classically, one only proves
  propositions, not terms.  So the idea of e.g. ``proving'' the
  natural numbers doesn't even make sense; it reflects a category
  mistake.  But in \HoTT, the concept of ``proving'' a type is
  primitive; the problem is that ``proving'' is the wrong word.
\end{description}

So here's one way to look at it: we construct (make) proofs; but the
proofs we construct are expressions of the type (the thing we prove).

%%%%%%%%
\subsection{Of the Ambiguity of Of}
\label{subs:ofofof}

``Of'' supports two distinct readings.  Consider ``the conviction of
the defendant''.  If the court did the convicting, then ``of'' acts as
a kind of intermediary between a verbal noun (``conviction'' as act or
action of convicting) and its direct object (e.g. ``The court
convicted the defendant'').  The conviction affects the defendant from
the outside; it does not ``belong'' to the defendant but to the court.
On the other hand, if we take ``the conviction of the defendant'' to
refer to a belief to which the defendant is firmly committed, then the
conviction is ``internal''; it belongs to and comes from the
defendant.

This ambiguity of ``of'' afflicts phrases like ``proof of a
proposition'' as well.  If we can disambiguate it some of the mystery
of the relation between types and proofs will vanish.

%%%%%%%%
\subsection{Demonstrations and Demonstratives}
\label{subs:}

When we \textit{exhibit} a classic proof of a proposition, the proof
comes out as external to the proposition proved, just as a court's
conviction of a defendant is external to the defendant.  Such a proof
is something added or attached to the proposition.

But when we \textit{demonstrate} a proposition,\sidenote{Note: we
  demonstrate propositions, not proofs; a demonstration of a
  proposition \textit{is} a proof.} the demonstration (that is, proof)
is to be deemed an expression of the proposition in the internal
sense: an expression whose source, so to speak, is the proposition
itself, rather than the writer of the proof.  This may sound odd or
even ridiculously anthropomorphic, but if you think about it a bit it
makes perfect sense.  The mathematical proofs we write down are not
really expressions our our thought, but of mathematical structures,
entities, relations etc.  So they express
mathematics.\sidenote{Actually we should probably think of them as
  having a dual expressivism.  On the one hand they clearly express
  mathematics; but on the other hand, the particular form a proof
  takes is an expression of the writer's style or way of thinking.}

We can think of a demonstration in this sense as expressing a type's
structure, construed as the inferential articulation of the concept of
the type.\sidenote{See \cref{sect:brandom} for more on the inferential
articulation of conceptual content.}

The nice thing about this way of thinking is that it resolves the
tension between propositional and non-propositional types with respect
to proof.  In both cases, what \HoTT{} calls proof or witness is to be
taken as a demonstrative expression, or expressive demonstration, of
the type itself.  In the case of propositional types, favor the term
``demonstration'', with its connotations of progressive unfolding of a
logical structure, or better, rational argument.  In the case of
non-propositional types like \N, favor the term ``demonstrative'',
with its adjectival sense of ``something having a demonstrative
function'', rather than a nominal sense of ``act or action of
demonstrating''.  So an element\sidenote{We really must get rid of
  ``element''; it's too suggestive of set theory.  Maybe
  ``demonstrative'' fits the bill; instead of ``element of a type'' we
  would say ``demonstrative of a type''.  Or maybe ``demonstrator''.}
of a propositional type we would call a demonstration of the type, and
an element of a non-propositional type we would call a demonstrative
of the type.\marginnote{So $2$ is a demonstrative of the natural
  numbers; a proof that ``$2$ is even'' is a demonstration that
  expresses just that ``$2$ is even''.}

\begin{ednote}
  Demonstration qua demonstration of know-how?  Expression as
  expression of a type's structure - that is, its inferential
  articulation?
\end{ednote}

In both cases we have demonstration rather than proof of the type.

\begin{ednote}
  ``Demonstrator'' as the genus of ``demonstration'' and
  ``demonstrative''.  It has the virtue of paralleling
  ``constructor''.
\end{ednote}
