\section{Misc. Niceties}

\begin{ednote}
  tait: ``objects are given or constructed as object of a given
  type''.  The expression ``a : A'' expresses the idea that we are
  given a of type A.  It does this by stipulation rather than
  assertion.  Assertions are challengable and must be justified on
  demand; stipulations are not and need not.
\end{ednote}

ML Type Theory is centered (more or less) on one of the major
logico-philosophical topics of the 20th century, namely the nature of
assertion and its relation to propositions and inferences.

You don't have to understand the arcana of this debate in order to
understand type theory (or HoTT), but some familiarity with the main
outline is very helpful.  Actually, I think it's essential, if you
want to understand the HoTT Book's account of \textit{judgement},
presented in HoTT Chapter 1 (reproduced below).  Fortunately the
presentation is relatively straightforward.

\begin{remark}
  Stress: this is largely a philosophical issue, or perhaps an issue
  in Philosophy of Language.  It's really about how our utterances
  come to have the significances they do.
\end{remark}

Outline:

\begin{itemize}
\item Frege's elevation of \textit{force} as essential
\item Dealing with embedded (and therefore forceless) propositions
\item Wittgenstein
\item Dummett
\item etc.
\item Brandom's recent innovation: decompose ``assertion'' into ``commitment'' and ``entitlement''
\end{itemize}

What the HoTT Book refers to as judgment (following ML) could also be
called assertion.  Brandom's account of the ``fine structure'' of
assertion is very helpful here.  Among other things, it provides a
very simple explanation of how embedded propositions work.  Embedded
propositions are unasserted; the problem is how to reconcile this with
the fact that they are function as assertions if unembedded.  On
Brandom's account, [todo...]

In other words, we can have commitment with or without entitlement,
and vice-versa.

A set membership statement can be explained in terms of commitments
and entitlements.  A free occurance of e.g. \(a\in A\) is ordinarily
taken as an assertion (judgment).  We can follow Frege and make this
\textit{force} explicit: \(\vdash a\in A\).  The problem with this,
however, is that, in contemporary usage, this would make \(a\in A\)
\textit{logically} true, which is not what we want.  Instead we want a
representation of committment to the proposition, as at least
ordinarily true, without regard to its logical truth.\marginnote{TODO:
  logical v. ordinary truth is pretty hairy for non-logicians so the
  distinction should be explicated.}

In sum:  the implicity sense of \(a\in A\) is something like: 
\[\exists \Gamma, a, A | \Gamma\vdash a\in A\]

Informally: there exists a set of propositions \(\Gamma\), a value (or
object) \(a\), and a set \(A\) such that the propostion \(a\in A\) is
deducible from \(\Gamma\).

So the meaning of \(a\in A\) essentially involves existential
quantification.  It is a statement about the world, that it contains
the relavant entities, not about the entities themselves.

\begin{remark}
  Not quite; \(a\in A\) is surely a statement about \(a\), maybe also
  about \(A\), no?  But still there must be an implicity existential
  quantification over the propositions that entail the statement.
\end{remark}

There is a logical subtlety here.  \(a\in A\) seems to be about a
determinate \(a\) and a determinate \(A\), but it isn't, not if we
take it to be an existentially quantified statement.  That's because
\(\exists a, A | a\in A\) does not pick out determinate individuals;
it just says that \textit{some} such individuals exist in the domain
of interpretation.  True, \(a\) and \(A\) are said to be bound by
\(\exists\), but that's not entirely accurate; quantified variables
are not bound in the way that constant symbols like \(\pi\) or \(0\)
are bound.  Whatever we go on to say about \(a\) and \(A\) --
e.g. \(a\in A\) -- remains within the scope of the quantifier, so it
does not count as a statement about determinate individuals.  It's a
statement about the world, that it contains entities that satisfy the
predicate.

On the other hand, the same seems to be true of \(a : A\): though
these symbols be bound, we don't know what they are bound to.  They
are not bound by an implicit existential quantifier; \(a : A\) does
\textit{not} mean \(\exists a, A | a : A\).

\begin{remark}
  Plus, quantifiers have to be used with a predicate; strictly
  speaking, \(\exists a, A\) is not a complete statement.
\end{remark}

By contrast, the Type Theoretic analogue \(a : A\) is a statement
about a specific value and a specific type, without any
quantification.  It is not directly a statement about the world, but
about part of the world.  Or: it expresses both commitment and
entitlement.  That's why it cannot be embedded in e.g. ``if \(a : A\)
then it is not the case that \(b : B\)''.  Embedded propositions
cannot carry force, but \(a : A\) always carries force, intrinsically,
as it were.

%%%%%%%%
\subsection{a : A}
\label{subs:aA}

Forms go from symbols to terms to sentences; from \(a\) to \(a+b\) to
\(a+b=c\).

The ``judgment'' \(a : A\) is clearly a compound term, so it cannot
merely name something.  But is it a sentence?  Does it denote a
proposition?  Or is it analogous to terms like \(a+b\) which are names
of a sort but involve some additional meaning beyond mere reference.

It seems it must involve a proposition, or let's say propositional
content.  We take \(a : A\) as a statement of fact, rather than a mere
reference to some part of the world.  Then how is it distinct from
\(a\in A\)?

The HoTT Book says it is ``analogous'' to the set-theoretic statement
\(a\in A\), but essentially different, since \(a\in A\) is a
proposition but \(a : A\) is a judgment.  It says that, \textit{when
  working internally in type theory}, \(a : A\) cannot be embedded, as
in `` if \(a : A\) then it is not the case that \(b : B\)'', nor can
the judgment \(a : A\) be disproved.

So let's look closely at what this means.  Earlier, HoTT says that
(some) judgments involving A ``exist at a different level from the
\textit{proposition} \(A\) itself, which is an internal statement of
the theory.''  (p. 18) There's a bit of circularity there; what is an
``internal statement''?

{\todo The nature of ``proposition'' has been a topic of
  considerable debate.  Review some of the alternative accounts on
  offer.}


The basic idea seems to be based on the well-known concept that
propositions by themselves are devoid of force, and must be asserted.
HoTT seems to imply that judgments are asserted propositions -- or
more correctly, assertings of propositions.

This seems a little bit off.  Assertion is something only people do.
An inked form on a page cannot really be construed as an assertion.
So we need to work out the mechanics of how a written form like \(a :
A\) can be viewed as a ``judgment'' in this sense.  I think Brandom's
model of assertion works.  It would say, I think, that \(a : A\)
counts as a judgment (assertion) because by convention we agree to
treat it that way, whereas we treat \(a\in A\) slightly differently,
because of the conventions elaborated by 20th century logic.

When HoTTB refers to ``working internally in type theory'', it seems,
the idea is to consider propositions in isolation from their
assertion.  Assertion, on this view, is something that comes from
outside of the world of propositions.  This is perfectly in tune with
the idea that asserting is something people do, but that what gets
assert\textit{ed} -- the \textit{content} of an assertion -- is
distinct from the assert\textit{ing}.

\begin{remark}
  Sellars called this the notorious -ing/-ed distinction.
\end{remark}

This would seem to make \(a : A\) an assert\textit{ing}.

We can think of \atypeA{} as a \textit{given} proposition: one that,
while unasserted, has the same force as a propositional assertion.  Or
another way to put it would be to say that use of \atypeA{} is
inalienably performative.

In fact \atypeA{} corresponds nicely to a common linguistic practice,
namely combining a proper name and a description, as in ``Joan of
Arc'', ``King George'', or ``Slick Willy''.  Or, more colloquially,
``poor Tom'', ``angry Joe'', or ``Gimpel the fool''.  And the
primitive nature of types can be clearly illustrated by analogy with
the military.  In type theory, every object has a type, just as
everybody in the military has a rank.  You cannot be in the military
unless you have a rank.  Within the military, the proper way refer to
someone in the miliary is to combine rank and name: General Custer,
Sergeant York, Private Bilko.  So the difference between \atypeA{} and
\(a\in A\) is like the difference between ``This is General Custer''
and ``This is Custer; he is a General''.

On the other hand, ``This is General Custer'' doesn't look much like a
\textit{judgment}, although it does look like a \textit{claim}.  But
not that it is not a claim about the meaning of ``General Custer'';
rather it is a claim about the relation between ``This'' and ``General
Custer''.  You could be wrong about the name or the rank of whomever
you mean by ``This'', but you cannot be wrong about ``General
Custer''; that's just a qualified name.  Being wrong in this sense
about ``This is General Custer'' is an empirical matter; in type
theory, the question of whether \atypeA (``this is a-of-A'') is
correct or not never even arises.  It doesn't make an empirical
assertion, it states a \textit{given}.  Or we might say it gives a
fact.  By contrast, \(a\in A\), as a proposition, may be either true
or false; when we say ``let \(a\in A\)'', we implicitly stipulate that
\(a\in A\) is to be \textit{assumed} to be true, but it is not
\textit{given} as true.  In other words, we can gloss it as ``\(a\in
A\) has a truth-value like any proposition, so it could be false, but
please assume that it is true.''

Another critical distinction: in standard set theory and logic,
judgments come from the outside, as it were.  But in HoTT, judgments
of the form \atypeA are internal.  They may be derivable inside the
system (by production of a proof or witness.)  In other words,
inference in set theory comes from outside of the world of sets, but
inference in HoTT is built in to the structure of types.  Inference
(construction) is part of the intrinsic meaning of types.

%%%%%%%%
\subsection{Justification}
\label{subs:just}

The HoTT Book's account of judgments in Chapter 1 section 1 seems to
conflate the distinction\sidenote{This is only to be
expected, since Brandom is the first (so far as I know) to see that
assertion (judgment) has an internal structure involving commitment
and entitlement (and some other stuff like a social dimension.)} Brandom makes between commitment and
entitlement.  ``Informally, a deductive system is a collection of
rules for deriving things called judgments.''

But derivation (proof) starts and ends in propositions; commitment is
something else.  The derivation or proof provide warrant for
entitlement to the commitment - justification of the conclusion.  So
how would Brandom parse ``judgment'' as HoTT uses the term?

