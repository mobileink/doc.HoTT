%%%%%%%%%%%%%%%%%%%%%%%%%%%%%%%%
\section{What's the Big Deal about Equality?}
\label{sect:equality}

\begin{ednote}
  Equality is arguably the most important concept of \HoTT{}, as far
  as I can tell, because of the ``Univalence Axiom''.
\end{ednote}

``In the intensional version of the theory, with which we are
concerned here, one thus has two different notions of equality:
propositional equality is the notion represented by the identity
types, in that two terms are propositionally equal just if their
identity type IdA(a,b) is inhabited by a term. By contrast,
definitional equality is a primitive relation on terms and is not
represented by a type; it behaves much like equality between terms in
the simply-typed lambda-calculus, or any conventional equational
theory.

If the terms a and b are definitially equal, then (since they can be
freely substituted for each other) they are also propositionally
equal; but the converse is generally not true in the intensional
version of the theory''\cite{awodey_tth}

``The constructive character, computational tractability, and proof-
theoretic clarity of the type theory are owed in part to this rather
subtle treatment of equality between terms, which itself is
expressible within the theory using the identity types IdA(a, b).''\cite{awodey_tth}

%%%%%%%%
\subsection{Substitution}
\label{subs:substitution}

As the quote from Awodey above suggests, the concept of
substitutability plays a basic role.

\begin{ednote}
  Compare substitution in lambda calculus, and in Brandom's model.
  Maybe something about combinatory logic and the elimination of
  variables?
\end{ednote}

