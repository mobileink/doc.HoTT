\section{Inferential Pluralism}\label{sec:pluralism}

We can classify inferences in at least the following ways:
\begin{itemize}
\item Qualitative kind (direct, inductive, and coinductive inference)
\item Form (type, sequent, and proof inferences)
\item Orientation (apomorphic, promorphic, endomorphic, etc)
\end{itemize}

\subsection{Varieties of Inference}

Direct, inductive, and coinductive inference are species of a single
genus.

Examples:

\begin{description}
\item[Direct inference:]  \(\tj{\seq{x}}{\List{X}}\linfer \tj{\head(n)}{X}\)
\item[Inductive inference:] x
\item[Coinductive inference:] x
\end{description}

\subsection{Formal inference types}

Reasoning with types in a formal calculus involves three distinct
forms of inference:

\begin{enumerate}
\item type inferences (\(\tinfer\)) go from tokens to types; e.g. \(a\tinfer
  A (= \mbox{\tj{a}{A}})\). The proposition ``a has type A'' is underwritten by the
  (material) inference from particular a to general A.
\item sequent inferences (\(\sso\)) go from type inference to type inference.  E.g. \(a:A\sso b:B\)
\item proof inferences (\(\pso\) or \(\Vdash\) or  \(\Vert\)) go
  from sequent inferences to sequent inferences. The horizontal line
  in proofs.
\end{enumerate}

To these we might add a four kind, for dealing with equality:

\begin{itemize}
\item equality inferences ...??? what kind of inference is \(a=b:B\)?
  Note this is token equality. In a sense all tokens of a type are
  equal, for example tokens of the word type ``the'' on this page are
  equally tokens of the type, even if they are distinct individuals.
  In HoTT different paths can be equal. Different proofs of the same
  proposition serve equally \textit{as} proofs. So evidently equals
  need not mean ``same particular''.
\end{itemize}

Whether or not these are all species of the same genus is a
philosophical question. Here it suffices to observe that in the
sequent calculus inferences adhere to a type discipline that
constrains what can serve as premises and conclusion for each type of
inference. This is compatible with the notion that the same kind of
inference is used in every case, but used in three different ways.

A more substantial argument in favor of inferential pluralism would
start by arguing that each type of inference presupposes a different
set of practical skills. Type inference presupposes the ability to
subsume particulars under general concepts. The ability to go from
``Rover'' to ``Rover is a dog'', for example.

\subsubsection{Type Inference}

The standard symbol for type inference is the colon \(\ulcorner :\urcorner\).

Alternatively, we can use a turnstile to emphasize commonality across inference types: \(\linfer_{\tau}\).

To emphasize another kind of equivalence (power?) we can use a horizontal line, possibly annotated to indicate inference type:
%% \AxiomC{$a$}
%% \RightLabel{$\scriptstyle\linfer_{\tau}$}
%% \UnaryInfC{$A$}
%% \DisplayProof

Instead of a turnstile annotation, we could use a box:
%% \fbox{
%% \AxiomC{$a$}
%% \UnaryInfC{$A$}
%% \DisplayProof
%% }

Example: \(\land\text{-intro}\):

%% \begin{displaymath}
%%   \prftree[r]{$\scriptstyle\supset\mathrm{I}$}
%%           {\prftree[r]{$\scriptstyle\supset\mathrm{I}$}
%%             {\prftree[r]{$\scriptstyle\supset\mathrm{E}$}
%%               {\prfboundedassumption{A}}
%%               {\prfboundedassumption{\neg A}}
%%               {\bot}}
%%             {\neg\neg A}}
%%           {A \supset \neg\neg A}
%% \end{displaymath}


%% \begin{prfenv}
  %% \prftree[r]{$\supset\mathrm{I}_{\prfref<assum:A>}$}
  %%         {\prftree[r]{$\supset\mathrm{I}_{\prfref<assum:not_A>}$}
  %%           {\prftree[r]{$\supset$E}
  %%             {\prfboundedassumption<assum:A>{A}}
  %%             {\prfboundedassumption<assum:not_A>{\neg A}}
  %%             {\bot}}
  %%           {\neg\neg A}}
  %%         {A \supset \neg\neg A}
%% \end{prfenv}

\subsubsection{Sequent Inference}
Sequent inference presupposes the ability to combine such
generalizations to form new generalizations, e.g. that ``Rover is a
dog \textit{and} Mittens is a cat''. That is, the ability to master
the practices involving the introduction and elimination of the
``logical'' constants.

\subsubsection{Proof Inference}
Finally, proof inference presupposes the ability to operate at a higher
level of sequent inference management. A detailed argument would show
how each level presupposes the previous level but adds new skills.

Both sequent and proof inference presuppose the ability to draw an
inference to a disjunction, to conclude that premises warrant the
assertion of one or more conclusions. So those two are clearly
substantially different than type inference.


All of Martin-Löf's ``forms of judgment'' are type inferences. There
is no need to add another level of inference types, by saying for
example that \mbox{\(a:A\)} and \mbox{\(a=b:A\)} are distinct
\textit{kinds} of inference. It suffices to say that they are both
type inferences. So we can dispense with ``forms of judgment''.

[In fairness, \(a=b:A\) clearly seems to involve more than mere type
  inference. At first glance the inference from \(a=b\) to ``type A''
  seems a bit off. So maybe we do need to make a distinction between
  primitive type inference (\(a:A\)) and typed equality inferences.]


The forms according to \citetitle{martin1984intuitionistic} \parencite{martin1984intuitionistic}:

\begin{itemize}
\item A is a set
\item A and B are equal sets
\item a is an element of the set A  (\(a:A\))
\item a and b are equal elements of the set A (\(a=b:A\))
\end{itemize}

Elsewhere he gives \(A \text{prop}\) and \(A \text{true}\).

\subsection{Orientation}

We can borrow the taxonomy of morphisms to classify inferences. See
(\ref{morph:classes}) for more information.

