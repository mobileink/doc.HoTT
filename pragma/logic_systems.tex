\section{Logic Systems}

%%%%%%%%%%%%%%%%
\subsection{Natural Deduction}

%% Logical And
\begin{center}
\AxiomC{$A\kern-1.2em$}
\AxiomC{$B$}
\RightLabel{$\scriptstyle{[\lkand \textrm{-intro}]}$}
\BinaryInfC{$A\lkand B$}
\DisplayProof
\hskip 1.4em
\AxiomC{$A\lkand B$}
\RightLabel{$\scriptstyle{[\lkand\ \text{-exit}_{\text{L}}]}$}
\UnaryInfC{$A$}
\DisplayProof
\hskip 1.4em
\AxiomC{$A\lkand B$}
\RightLabel{$\scriptstyle{[\lkand\ \text{-exit}_{\text{R}}]}$}
\UnaryInfC{$B$}
\DisplayProof
\end{center}

%%%%%%%%%%%%%%%%
\subsection{LK}

First some helpful vocabulary:

\begin{description}
\item[Sequent]
  \item[Antecedent] The part of a sequent preceding \(\linfer\).
  \item[Consequent] The part of a sequent following \(\linfer\).
  \item[Principle formula] The (sub-)formula containing the logical
    constant being defined. For example, in rule \(\lkand\linfer\), the
    principle formula is \mbox{\(A\lkand B,\Gamma\linfer \Delta\)}.
  \item[Structure] A sequence of formulae. Uppercase Greek letters
    \(\Gamma, \Delta, \Theta, \Sigma\), etc. are used as
    meta-variables to represent sequences of formulae, which may be
    combined with formula meta-variables (\(A, B, C, etc.\)); for
    example, \(\ulcorner A,\Gamma\urcorner\) and \(\ulcorner \Gamma,
    A\urcorner\) are structures.
    \item[Structure connectives] There is only one structure
      connective: comma. Represents conjunction (\(\lkand\)) in
      antecedents and disjunction (\(\lor\)) in consequents.
    \item[Sequent connective] There are two sequent connectivess,
      \(\seqand\), meaning ``and also'' and \(\seqor\), meaning
      ``either\ldots or''. A rule that uses \(\seqor\) represents the
      merger of two separate rules; most presentations of the calculus
      list them separately and do not use an explicit sequent
      connective. In most presentations of sequent calculi, sequent
      connectives are not made explicit, but making them explicit
      brings out features, especially symmetries, that otherwise may
      not be so clear.
    \item[Logical connectives] The usual logical constants: \(\lkand,
      \lor\), etc.
\end{description}

By convention, rules are categorized as left or right, depending on 1)
the location of the principle formula on the bottom; and 2) the
location of the propositional variables in the sequents. For example,
rule \(\lkand\linfer\) is the left rule for \(\lkand\).

Left rules correspond to elimination rules, right rules, to
introduction rules. We want the introduction rules on the left, since
by convention they are usually presented before elimination rules.
This means that right rules are in the left column here.

Left rules show that when the principle formula is an antecedent of a
bottom sequent (i.e. the conclusion of an inference), inference to the
consequent may be acheived by decomposing it and using one of the
sequents in the premise to make the inference. I.e. it represents a
kind of backward reasoning.

\subsubsection{Axioms}

%% Id, Cut
\begin{center}
\AxiomC{}
\RightLabel{$\text{Id}$}
\UnaryInfC{$A\lkand A$}
\DisplayProof
\hskip 1.4em
\AxiomC{$\Gamma\linfer \Delta_1,C,\Delta_2\kern-1.2em$}
\AxiomC{$\seqand\kern-1.2em$}
\AxiomC{$\Theta_1,C,\Theta_2\linfer\Lambda$}
\RightLabel{$\text{cut}$}
\TrinaryInfC{$\Theta_1,\Gamma,\Theta_2\linfer \Delta_1,\Lambda,\Delta_2$}
\DisplayProof
\end{center}

\subsubsection{Logical Rules}

%% Logical And
\begin{center}  %% and-intro
\AxiomC{$\Gamma\linfer \Delta, A\kern-1.2em$}
\AxiomC{$\seqand\kern-1.2em$}
\AxiomC{$\Gamma\linfer \Delta, B$}
\RightLabel{$\linfer\lkand$}
\TrinaryInfC{$\Gamma\linfer \Delta, A\lkand B$}
\DisplayProof
\hskip 1.4em  %% and-elim
\AxiomC{$A,\Gamma\linfer \Delta\kern-1.2em$}
\AxiomC{$\seqor\kern-1.2em$}
\AxiomC{$B,\Gamma\linfer \Delta$}
\RightLabel{$\lkand\linfer$}
\TrinaryInfC{$A\lkand B,\Gamma\linfer \Delta$}
\DisplayProof
\end{center}

%% Logical Or
\begin{center}
\AxiomC{$\Gamma\linfer \Delta, A\kern-1.2em$}
\AxiomC{$\seqor\kern-1.2em$}
\AxiomC{$\Gamma\linfer \Delta, B$}
\RightLabel{$\linfer\lor$}
\TrinaryInfC{$\Gamma\linfer\Delta, A\lor B$}
\DisplayProof
\hskip 1.4em
\AxiomC{$A,\Gamma\linfer \Delta\kern-1.2em$}
\AxiomC{$\seqand\kern-1.2em$}
\AxiomC{$B,\Gamma\linfer \Delta$}
\RightLabel{$\lor\linfer$}
\TrinaryInfC{$A\lor B,\Gamma\linfer \Delta$}
\DisplayProof
\end{center}

%% Implication
\begin{center}
\AxiomC{$A,\Gamma\linfer \Delta,B\kern-1.2em$}
\RightLabel{$\linfer\supset$}
\UnaryInfC{$\Gamma\linfer \Delta,A\supset B$}
\DisplayProof
\hskip 1.4em
\AxiomC{$\Gamma\linfer \Delta, A\kern-1.2em$}
\AxiomC{$\seqand\kern-1.2em$}
\AxiomC{$B,\Theta\linfer \Lambda$}
\RightLabel{$\supset\linfer$}
\TrinaryInfC{$A\supset B,\Gamma,\Theta\linfer\Delta,\Lambda$}
\DisplayProof
\end{center}

%% Not
\begin{center}
\AxiomC{$A,\Gamma\linfer \Delta\kern-1.2em$}
\RightLabel{$\linfer\neg$}
\UnaryInfC{$\Gamma\linfer \Delta,\neg A$}
\DisplayProof
\hskip 1.4em
\AxiomC{$\Gamma\linfer \Delta, A\kern-1.2em$}
\RightLabel{$\neg\linfer$}
\UnaryInfC{$\neg A,\Gamma\linfer\Delta$}
\DisplayProof
\end{center}

%% Universal quantification
\begin{center}
\AxiomC{$A,\Gamma\linfer \Delta\kern-1.2em$}
\RightLabel{$\linfer\forall$}
\UnaryInfC{$\Gamma\linfer \Delta,\neg A$}
\DisplayProof
\hskip 1.4em
\AxiomC{$\Gamma\linfer \Delta, A\kern-1.2em$}
\RightLabel{$\forall\linfer$}
\UnaryInfC{$\neg A,\Gamma\linfer\Delta$}
\DisplayProof
\end{center}

%% Existential quantification
\begin{center}
\AxiomC{$A,\Gamma\linfer \Delta\kern-1.2em$}
\RightLabel{$\linfer\exists$}
\UnaryInfC{$\Gamma\linfer \Delta,\neg A$}
\DisplayProof
\hskip 1.4em
\AxiomC{$\Gamma\linfer \Delta, A\kern-1.2em$}
\RightLabel{$\exists\linfer$}
\UnaryInfC{$\neg A,\Gamma\linfer\Delta$}
\DisplayProof
\end{center}


\subsubsection{Structural Rules}

%% thinning
\begin{center}
\AxiomC{$\Gamma\linfer \Delta\kern-1.2em$}
\RightLabel{$\linfer\text{K}$}
\UnaryInfC{$\Gamma\linfer \Delta, A$}
\DisplayProof
\hskip 1.4em
\AxiomC{$\Gamma\linfer \Delta\kern-1.2em$}
\RightLabel{$\text{K}\linfer$}
\UnaryInfC{$A,\Gamma\linfer\Delta$}
\DisplayProof
\end{center}

%% contraction
\begin{center}
\AxiomC{$\Gamma\linfer \Delta, A, A$}
\RightLabel{$\linfer\text{W}$}
\UnaryInfC{$\Gamma\linfer\Delta, A$}
\DisplayProof
\hskip 1.4em
\AxiomC{$A,A,\Gamma\linfer \Delta$}
\RightLabel{$\text{W}\linfer$}
\UnaryInfC{$A,\Gamma\linfer \Delta$}
\DisplayProof
\end{center}

%% Exchange
\begin{center}
\AxiomC{$\Theta\linfer\Gamma,A,B,\Delta$}
\RightLabel{$\linfer\text{C}$}
\UnaryInfC{$\Theta\linfer\Gamma,B,A,\Delta$}
\DisplayProof
\hskip 1.4em
\AxiomC{$\Gamma,A,B,\Delta\linfer\Theta$}
\RightLabel{$\text{C}\linfer$}
\UnaryInfC{$\Gamma,B,A,\Delta\linfer\Theta$}
\DisplayProof
\end{center}

\subsection{LL}

Source: \parencite{Girard95linearlogic}, p. 10

\subsubsection{Notation}

We use custom symbols in order to bring the meanings more clearly to
the surface.

\begin{itemize}
\item Additive conjunction: \(A \addand B\). You have both A
  \textit{and} B, but you can only use one: A \textit{or} B.
\item Additive disjunction: \(A \addor B\)
\item Multiplicative conjunction: \(A\fusion B\). You have both, and
  you can only use both together to produce a single output. Alternatively, you can only use it as input to rules that need both.
\item Multiplicative disjunction: \(A\fission B\). You have both, and
  you can use A or B or both to produce either of two possible outputs.
\end{itemize}



\subsubsection{Axioms}

%% Id, Cut
\begin{center}
\AxiomC{}
\RightLabel{$\text{Id}$}
\UnaryInfC{$A\lkand A$}
\DisplayProof
\hskip 1.4em
\AxiomC{$\Gamma\linfer \Delta_1,C,\Delta_2\kern-1.2em$}
\AxiomC{$\seqand\kern-1.2em$}
\AxiomC{$\Theta_1,C,\Theta_2\linfer\Lambda$}
\RightLabel{$\text{cut}$}
\TrinaryInfC{$\Theta_1,\Gamma,\Theta_2\linfer \Delta_1,\Lambda,\Delta_2$}
\DisplayProof
\end{center}

\subsubsection{Logical Rules}

\paragraph{Both but not each (Fusion)}
\begin{center}
\AxiomC{$\Gamma\linfer A\structor\Delta\kern-1.2em$}
\AxiomC{$\seqand\kern-1.2em$}
\AxiomC{$\Theta\linfer B\structor\Lambda$}
\RightLabel{$\fusion\linfer$}
\TrinaryInfC{$\Gamma;\Theta\linfer A\fusion B\structor\Delta\structor\Lambda$}
\DisplayProof
\hskip 1.4em
\AxiomC{$\Gamma;A;B\linfer\Delta$}
\RightLabel{$\linfer\fusion$}
\UnaryInfC{$\Gamma;A\fusion B\linfer\Delta$}
\DisplayProof
\end{center}

%%%%%%%%%%%%%%%%
\paragraph{Both or either (Fission)}
\begin{center}
\AxiomC{$\Gamma\linfer A\structor B\structor\Delta$}
\RightLabel{$\linfer\fission$}
\UnaryInfC{$\Gamma\linfer A\fission B\structor\Delta$}
\DisplayProof
\hskip 1.4em
\AxiomC{$\Gamma,A\linfer \Delta\kern-1.2em$}
\AxiomC{$\seqand\kern-1.2em$}
\AxiomC{$\Theta, B\linfer \Lambda$}
\RightLabel{$\fission\linfer$}
\TrinaryInfC{$\Gamma;\Theta;A\fission B\linfer \Delta\structor\Lambda$}
\DisplayProof
\end{center}

%%%%%%%%%%%%%%%%
\paragraph{Each but not both (additive conjunction)}
\begin{center}
\AxiomC{$\Gamma\linfer \Delta, A\kern-1.2em$}
\AxiomC{$\seqand\kern-1.2em$}
\AxiomC{$\Gamma\linfer \Delta, B$}
\RightLabel{$\linfer\lkand$}
\TrinaryInfC{$\Gamma\linfer \Delta, A\lkand B$}
\DisplayProof
\hskip 1.4em
\AxiomC{$A,\Gamma\linfer \Delta\kern-1.2em$}
\AxiomC{$\seqor\kern-1.2em$}
\AxiomC{$B,\Gamma\linfer \Delta$}
\RightLabel{$\lkand\linfer$}
\TrinaryInfC{$A\lkand B,\Gamma\linfer \Delta$}
\DisplayProof
\end{center}

%%%%%%%%%%%%%%%%
\paragraph{Either but not both (additive disjunction)}
\begin{center}
\AxiomC{$\Gamma\linfer \Delta, A\kern-1.2em$}
\AxiomC{$\seqor\kern-1.2em$}
\AxiomC{$\Gamma\linfer \Delta, B$}
\RightLabel{$\linfer\lor$}
\TrinaryInfC{$\Gamma\linfer\Delta, A\lor B$}
\DisplayProof
\hskip 1.4em
\AxiomC{$A,\Gamma\linfer \Delta\kern-1.2em$}
\AxiomC{$\seqand\kern-1.2em$}
\AxiomC{$B,\Gamma\linfer \Delta$}
\RightLabel{$\lor\linfer$}
\TrinaryInfC{$A\lor B,\Gamma\linfer \Delta$}
\DisplayProof
\end{center}

%%%%%%%%%%%%%%%%
\paragraph{Linear implication}
\begin{center}
\AxiomC{$A,\Gamma\linfer \Delta,B\kern-1.2em$}
\RightLabel{$\linfer\llso$}
\UnaryInfC{$\Gamma\linfer \Delta,A\llso B$}
\DisplayProof
\hskip 1.4em
\AxiomC{$\Gamma\linfer \Delta, A\kern-1.2em$}
\AxiomC{$\seqand\kern-1.2em$}
\AxiomC{$B,\Theta\linfer \Lambda$}
\RightLabel{$\llso\linfer$}
\TrinaryInfC{$A\llso B,\Gamma,\Theta\linfer\Delta,\Lambda$}
\DisplayProof
\end{center}

%%%%%%%%%%%%%%%%
\paragraph{Negation}
\begin{center}
\AxiomC{$A,\Gamma\linfer \Delta\kern-1.2em$}
\RightLabel{$\linfer\neg$}
\UnaryInfC{$\Gamma\linfer \Delta,\neg A$}
\DisplayProof
\hskip 1.4em
\AxiomC{$\Gamma\linfer \Delta, A\kern-1.2em$}
\RightLabel{$\neg\linfer$}
\UnaryInfC{$\neg A,\Gamma\linfer\Delta$}
\DisplayProof
\end{center}

%%%%%%%%%%%%%%%%
\paragraph{Universal quantification}
\begin{center}
\AxiomC{$A,\Gamma\linfer \Delta\kern-1.2em$}
\RightLabel{$\linfer\forall$}
\UnaryInfC{$\Gamma\linfer \Delta,\neg A$}
\DisplayProof
\hskip 1.4em
\AxiomC{$\Gamma\linfer \Delta, A\kern-1.2em$}
\RightLabel{$\forall\linfer$}
\UnaryInfC{$\neg A,\Gamma\linfer\Delta$}
\DisplayProof
\end{center}

%%%%%%%%%%%%%%%%
\paragraph{Existential quantification}
\begin{center}
\AxiomC{$A,\Gamma\linfer \Delta\kern-1.2em$}
\RightLabel{$\linfer\exists$}
\UnaryInfC{$\Gamma\linfer \Delta,\neg A$}
\DisplayProof
\hskip 1.4em
\AxiomC{$\Gamma\linfer \Delta, A\kern-1.2em$}
\RightLabel{$\exists\linfer$}
\UnaryInfC{$\neg A,\Gamma\linfer\Delta$}
\DisplayProof
\end{center}

\subsubsection{Structural Rules}

%% thinning
\begin{center}
\AxiomC{$\Gamma\linfer \Delta\kern-1.2em$}
\RightLabel{$\linfer\text{K}$}
\UnaryInfC{$\Gamma\linfer \Delta, A$}
\DisplayProof
\hskip 1.4em
\AxiomC{$\Gamma\linfer \Delta\kern-1.2em$}
\RightLabel{$\text{K}\linfer$}
\UnaryInfC{$A,\Gamma\linfer\Delta$}
\DisplayProof
\end{center}

%% contraction
\begin{center}
\AxiomC{$\Gamma\linfer \Delta, A, A$}
\RightLabel{$\linfer\text{W}$}
\UnaryInfC{$\Gamma\linfer\Delta, A$}
\DisplayProof
\hskip 1.4em
\AxiomC{$A,A,\Gamma\linfer \Delta$}
\RightLabel{$\text{W}\linfer$}
\UnaryInfC{$A,\Gamma\linfer \Delta$}
\DisplayProof
\end{center}

%% Exchange
\begin{center}
\AxiomC{$\Theta\linfer\Gamma,A,B,\Delta$}
\RightLabel{$\linfer\text{C}$}
\UnaryInfC{$\Theta\linfer\Gamma,B,A,\Delta$}
\DisplayProof
\hskip 1.4em
\AxiomC{$\Gamma,A,B,\Delta\linfer\Theta$}
\RightLabel{$\text{C}\linfer$}
\UnaryInfC{$\Gamma,B,A,\Delta\linfer\Theta$}
\DisplayProof
\end{center}

