\section{Compositionality}\label{sec:composition}

Virtually all treatments of logical and functional calculi take a
function-oriented approach. They treat \modus as a rule of (function)
application.

We can also take a composition-oriented perspective. Then \modus
becomes a rule of composition. This requires that we treat arguments
as arrows. Under this perspective ctors and co-ctors are also treated
as arrows that compose rather than functions that apply.

The composition-oriented approach is basically a category-theoretic
interpretion. So we can use the diagrammatic methods of CT. This gives
a nice picture of lists, for example: for \(x\) we have
\(rightarrow\), for \(\tj{seq{x}}{\List{X}}\) we have \(

\subsection{Semantic Compositionality}
We are not obligated to asssume semantic compositionality.

Notation:  [X] refers to whatever ``X'' denotes.

From A and also B we get A\&B. The latter is syntactically composite;
it does not follow that [A\&B] is semantically composite.

This matters because in the sequent calculus the key idea is inference
from context, not from consequents (conclusions of a sequent). We get
A\&B not from A and B, but from the context that suffices for them:


%% \begin{prooftree}
%% \AxiomC{$\Gamma\linfer a:A$}
%% \AxiomC{$\medtriangleup$}
%% \AxiomC{$\Gamma\linfer b:B$}
%% \TrinaryInfC{$\Gamma\linfer (a,b): A\times B$}
%% \end{prooftree}

If we write this as a natural deduction, without sequents and thus
without context \ContextG , we get
%% \AxiomC{$a:A$} \AxiomC{$\medtriangleup$}
%% \AxiomC{$b:B$} \TrinaryInfC{$(a,b): A\times B$} \DisplayProof ,
which
makes it look like the conclusion is constructed directly from the
pieces of the premises. But with sequents and antecedents it becomes
clear that it is built from the context, so long as the context is
sufficient for A and also B. It does not follow that building
\(\llbracket (a,b):A\times B\rrbracket\) from \ContextG\ involves
building either \semantic{A} or \semantic{B} (e.g. as intermediate
results). The conclusion only says that the context
\ContextG\ \textit{alone} is sufficent for \((a,b):A\times B\). It
does not say or imply that ``getting'' the conclusion from
\ContextG\ involves first getting a:A and also b:A, that is, that
\(\llbracket (a,b):A\times B\rrbracket\) must be a composite.

In other words, nothing says or implies a principle of semantic
compositionality.

On the other hand, the conclusion does use the symbols a,b,A,B, and
they must come from somewhere. And they cannot be merely syntactic,
since we want the conclusion to mean something.

To resolve this apparent paradox we can appeal to the notion of
equality. We can say that the context is sufficient for producing
\textit{some} \(p:A\times B\), such that \(\llbracket p\rrbracket\) =
\(\llbracket(a,b)\rrbracket\). I.e. instead of depending on a
Principle of Semantic Compositionality we use a notion of equality. We
can make this explicit:

%% \begin{prooftree}
%% \AxiomC{$\Gamma\linfer a:A$}
%% \AxiomC{\(\medtriangledown\)}
%% \AxiomC{$\Gamma\linfer b:B$}
%% \TrinaryInfC{$\Gamma\linfer p=(a,b): A\times B$}
%% \end{prooftree}

Intuition pump: colors. Set the context to the primary colors \(\Gamma
= R,G,B\), and make A and B mixed colors. Then A\&B is a mix of mixes,
which can be produced directly from R,G,B without first producing A and
B.

Also, without context Truth becomes the central notion. ``Assume A''
can only mean assume A is true, not ``assume there exists a p that
proves A''.

ML's concept of judgment as involving both truth and proof is not
essential; he needs it though, because he wants to treat Logic as
essentially epistemic. So his judgments and assumptions always involve
some kind of (implicit) proof. This is not necessary if we start with
inference as the central notion and make the context explicit using
the sequent calculus. This does involve inference (obviously) but not
proof.

