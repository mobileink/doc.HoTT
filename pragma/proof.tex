\section{Proof}\label{sec:proof}

\subsection{Detachment}

Modus ponens is sometimes called ``detachment''. The justification for
this is that a modus ponens inference allows us to detach the
conclusion of an implication and assert it as a free-standing
proposition. It is what allows us to go from \(P\rightarrow Q\) to
\(Q\).

A proof proves a proposition; it licenses assertion of its conclusion
as a free-standing proposition. But within the proof itself the
conclusion is glued to the preceding inferences. Therefore to go from
proof to free-standing proposition we need a Rule of Detachment. Modus
ponens for proofs.

Is modus ponens for implication different in kind from modus ponens
for proofs? Formally it certainly is. Implication is a propositional
connector, binding a single premise to a single conclusion, both
propositions. By contrast, the inferential operator in a proof, which
is usually represented by a horizontal line, binds the conclusion to
one or more premises.

For implication, the rule of modus ponents is easy to express; it's
usually called the eliminator rule for implication
(\(\rightarrow\text{-elim}\)).

