\section{Case studies in induction and coinduction}\label{sec:case_studies}

\subsection{Numbers and Conumbers}

Conatural numbers? See
\href{https://www.cl.cam.ac.uk/archive/mjcg/plans/Coinduction.pdf}{Corecursion
  and coinduction: what they are and how they relate to recursion and
  induction}

\subsection{Dark Numbers}

To demonstrate the minimality of types defined by induction, we can
expand \(\Nat\) to \(\NDark\) by adding two ``dark'' constructors,
\(\ZDark\) and \(\SDark\). Define \(\NDark\) by:

\begin{align}
  & \linfer \tj{\ZNat}{\NDark} \\
  & \linfer \tj{\ZDark}{\NDark} \\
  \tj{n}{\ZDark} & \linfer \tj{n\SNat}{\NDark} \\
  \tj{n}{\ZDark} & \linfer \tj{n\SDark}{\NDark} \\
\end{align}

Our reasoning with these ``dark numbers'' will be the same as our
reasoning with natural numbers, except that we have two more cases to
consider when we reason by case analysis.

Ordinary \(\Nat\) and its expansion \(\NDark\) have the same
cardinality; they're both infinite (by construction), and their
elements can be put in one-to-one correspondance. But \(\NDark\) has
``more structure'' than \(\Nat\). We can express this vague notion
mathematically by observing that there is a homomorphism from \(\Nat\)
to \(\NDark\). We can draw a one-to-one map that respects construction
from the former to the latter, but not from the latter to the former.
There are ``too many'' elements in \(\NDark\), even though the sets
have the same cardinality. We can map \(\ZNat\) from \(\Nat\) to its
counterpart in \(\NDark\), but going the other way around we would
need to map both \(\ZNat\) and \(\ZDark\) to \(\ZNat\).

[TODO: make the homomorphsim explicit with a diagram or two]

Homomorphisms induce an ordering. If we have a homomorphism \(A~\hmorph~B\), then we can make this ordering explicit by replacing the
homomorphism arrow \(\hmorph\) by a more suggestive symbol, for
example \(A\preceq B\). Then we can state the minimality principle
more concisely: \(\Nat\) is the least element of the set \(\{A\ |\
  \Nat\preceq A\}\).

The point of this exercise is to provide a vivid demonstration of what
it means for an inductively defined type to be the least of its kind.
But we can have some fun with dark numbers too.

Here are some examples. We use the standard numerals \(0, 1,\cdots\)
to express the number of manifest \(\Nat\) subconstructions are in a
construction that starts from \(\ZNat\), and a parallel set of ``dark
numerals'' \(\dark{0}, \dark{1}, \cdots\) to count dark
subconstructions in constructions starting from \(\ZDark\).

We use \(\bot\) resp. \(\dark{\bot}\) to indicate ``improper''
constructions. An improper construction is a hybrid that contains
subexpressions that do not match the base case. For example
\(\ZNat\SDark\) is \(0\) and \(\dark{\bot}\).

Equation (\ref{dark:card}) demonstrates a function \textsf{abs} that
counts the total number \(\SNat + \SDark\) of successor constructions.

\begin{align}
  & \ZS & = 1 \\
  & \ZDark\SDark & = \dark{1}\\
  & \ZNat\SDark & = 0 = \dark{\bot}\\
  & \ZDark\SNat & = \dark{0} = \bot \\
  & \ZDark(\SDark\SNat)(\SDark\SNat) & = \bot = \dark{2} \\
  & \ZNat\SDark\SDark\SDark\SNat & = 1 = \dark{\bot} \label{dark:hybrid} \\
  & \textsf{\slshape abs}(\ZNat\SDark\SDark\SDark\SNat) & = 4 \label{dark:card} \\
  & \ZDark\SDark\SDark\SDark\SNat & = \bot = \dark{3} \\
  & \textsf{\slshape abs}(\ZDark\SDark\SDark\SDark\SNat) & = 4
\end{align}

We could also design the constructor rules to force parallel successor
operations, so that every \(\SNat\) is accompanied by a \(SDark\) and
vice-versa.  Maybe \(\tj{n}{\ZDark} \linfer \tj{n\SDark\SNat}{\NDark}\)


The real fun begins when we try to define operations like addition. We
can easily define standard addition by ignoring dark numbers, but we
could also define other kinds of addition, such as a total sum that
sums the total number of manifest and dark constructions. Etc.

[TODO:  A
  filter to remove dark constructors, leaving a manifest number
  (within \(\NDark\)). Dark numbers look wierd, but they behave like
  manifest numbers if you squint.]


We can play metaphysical games with dark numbers. You say the
``ordinary'' natural numbers, \(0,1,2\cdots\) exist. The Dark Number
theorist agrees, with a caveat: they're not the \textit{only} natural
numbers that exist. The dark numbers also exist; you've just never
noticed them.

The intuitionist might remain unperturbed by dark numbers. The
Fundamental Principle of Intuitionism says that the only numbers that
exist are the ones we construct. \(\Nat\) exists by construction that
does not involve dark numbers, so whether or not we can also construct
dark numbers is irrelevant. But the Fundamental Principle makes a
presupposition, namely that manifest construction is the only kind
there is. This the Dark Theorist denies, arguing that dark
construction is implicit in all manifest construction. A manifest
construction like \(\ZSS\) conceals some dark construction, like
\((\ZDark\ZNat)(\SDark\SNat)(\SDark\SNat)\) -- one dark construction
for each manifest construction. We might even go so far as to claim
that dark construction is primitive: without it we would not have
manifest construction.

Of course these are just games, intended to demonstrate something
about coinductive reasoning. Mathematics gets along fine without dark
numbers; we do not need them. But then again, if physicists think that
the universe is full of dark matter, why shouldn't mathematicians
think that the mathematical universe is full of dark numbers?
Metaphysically, they make at least as much sense as manifest numbers.

\subsection{Coinduction: stream contraction}

Cotypes are the greatest elements of their class.  etc.

To show what this means we show how to derive a ``smaller'' type from
the type of streams.

It's intuitively obvious that the type ``stream of \(X\)''
(\(\colist{X}\)) is maximal, in the sense that it ``already'' contains
all possible particular \(X\)-streams. But it's also minimal, in that
its co-constructors are underdetermined. They are typed, but not
defined.

``Maximal'' means that it is the greatest element of the class of all
types that are homomorphic to it. We give an example to illustrate
just what that means.

First let's consider a type that is homomorphic to \(\colist{X}\),
without yet worrying about how to define it. The type
\(\eqstream{X}\) is the type of all streams whose heads return a
constant. In other words, streams whose elements are equal, such as
\(\langle 1,1,\ldots\rangle\).

The members of such a type will clearly form a subset\footnote{Types
are not sets, but since we have not yet defined ``subtype'' we'll
treat them as such for now.} of the members of \(\colist{X}\), so if
we can show that there is a homomorphism from \(\eqstream{X}\) to
\(\colist{X}\), we'll have a better idea of what it is for the latter
to be the greatest element of this class. To do so we need to
(co-)define \(\eqstream{X}\) and show how this entails the
homomorphism.

The short version: a cotype like \(\eqstream{X}\) is codefined by
refining the definitions of the co-constructors of \(\colist{X}\).
Their meanings are underdetermined by the definition of
\(\colist{X}\); in fact, their definitions give them no meaning at
all. Their ``definitions'' merely declare their types, so there are no
semantic constraints on them. They can be thought of as specifications
that jointly determine a \textit{protocol} for their cotype. The
definitions of cotypes like \(\eqstream{X}\) work by partially
defining these co-constructors. They provide an implementation of the
protocol, as it were. By giving the co-constructors specific meanings,
they narrow their range. So the cotype they determine comes out as a
kind of contraction of \(\colist{X}\).

This is easily expressible in set theoretic terms. We start with a
domain, an unconstrained set that contains ``everything'' we're
interested in. We produce a subset by adding constraints on
membership; this can be expressed as a characteristic function, or a
list of ``such that'' clauses. In this particular case, we would have
something like the following in informal terms: ``The set of all
members of \(\colist{X}\) \textit{such that} their
\(\ulcorner\head{}\urcorner\) operation always returns the same
value''.

Our task is thus to show how these and similar \textit{such that}
constraints can be expressed in a type calculus. This requires more
than merely working out the syntax. The informal example we gave above
is expressed in the idiom of classical logic; it states conditions
that must be \textit{true of} a stream in order for it to be a member
of the set. We cannot do that, because the tools we have at our
disposal are essentially \textit{inferential}. What we're after is a
way to positively infer membership rather than simply observe it,
which is what \textit{such that} conditions do. They just state a
fact, that members have such-and-such a property \textit{in virtue of
  being members}, not that they earn membership in virtue of having
the property. So our more fundamental task is to show how such
\textit{such that} constraints can be expressed inferentially. Then we
will be able to infer membership from the (inferential) properties of
the candidate stream. This is quite different than merely expressing
\textit{such that} conditions. In fact we'll see below that the
difference is vast.

Co-constructors are \textit{apomorphisms}; the go from the cotype they
define.

\begin{align}
 \cohead(\cotail(\coseq{x})) & = \cohead(\coseq{x}) & \text{equational} \\
  \cohead(\cotail(\coseq{x})) & \linfer \cohead(\coseq{x}) & \text{inferential}
\end{align}

This expresses a genuinely coinductive inference. It cannot be
expressed (or made) in any simply way. It is different than direct
inference.

Such ``definitional'' coinductive inference rules -- those that play a
defining role, similar to an axiom -- express the internal structure
of the streams whose cotype they define. The structure of those
streams is expressed at the most abstract level by the co-ctors of
\(\colist{X}\). The only structure they have is concatenation.
Co-definitions like the one above express further structure - that the
head of the whole is related in a particular way to the head of the
tail\footnote{Note the difference with the \textit{such that} method,
which says nothing about the structure of the things that satisfy the
condition.}. And note that the co-ctors defined in the subtype are not
the ctors of \(\colist{X}\). But they have the same names and type
signatures, and that's what matters: it's what makes for the
homomorphism, and it is what allows us to treat them as
implementations of a protocol.

[But that's not quite right. We're defining a cofunction over
  \(\colist{X}\), so we \textit{must} use its co-ctors. We do, but we
  also (re)define them. So cofunction definitions literally modify
  (augment, extend) the semantics of the ``root'' co-ctors.]

[Do we ever extend/redefine type ctors?]

\subsection{Finite Lists}

\subsection{Streams}

Examples of various types of functions/cofunctions.

To define:

\begin{itemize}
\item projection ops, e.g. second, butlast, nth, etc.
\item pathological (co)functions: constant hd, tl, cons, etc.
\item Take(n)
\item Sum, Sum(n)
\end{itemize}


\subsubsection{Promorphisms (generators)}

Generators are promorphisms.

\paragraph{\(\texttt{upFrom}_{\scriptscriptstyle\top}\)}
generates a stream that is a subsequence of \(\Nat\) starting from a
given number. Usually we omit the coinduction subscript. E.g.
\(\texttt{upFrom}(3) = \stream{3}{4}\).


\begin{prooftree}
  \AxiomC{$\Gamma\linfer\tj{\codefn{upFrom}}{\Nat\func\colist{X}}$}
  \AxiomC{$\Delta\linfer\tj{n}{\Nat}$}
  \BinaryInfC{$\Gamma,\Delta\linfer\tj{\codefn{upFrom}(n)}{\colist{X}}$}
\end{prooftree}

\begin{alignat}{2}
  \codefn{upFrom}(n) &= \cocons(n, \codefn{upFrom}(n+1)) \\
  \cohead(\codefn{upFrom}(n)) &= n \\
  \cotail(\codefn{upFrom}(n)) &= \codefn{upFrom}(n+1)
\end{alignat}


General pattern:

\begin{alignat}{2}
  \cohead\circ\Lift{f}{\omega} &= \phantom{\Lift{f}{\omega}(}n \\
  \cotail\circ\Lift{f}{\omega} &= \Lift{f}{\omega}(n+1)
\end{alignat}


For example, generate \(\langle 1,1,1\rangle\), \(\langle 1,1,\cdots\rangle\).

\subsubsection{Apomorphisms (projectors)}

%%%%%%%%%%%%%%%%
\paragraph{\(\texttt{nth}_{\scriptscriptstyle\top}\)}
projects the \textit{nth} element of a stream.

\begin{prooftree}
  \AxiomC{$\Gamma\linfer\tj{\texttt{nth}}{\Nat\times\colist{X}\func{}X}$}
  \AxiomC{$\Delta\linfer\tj{n}{\Nat}$}
  \AxiomC{$\Lambda\linfer\tj{\coseq{x}}{\colist{X}}$}
  \RightLabel{\raisebox{4pt}{$\scriptstyle[nth\texttt{-intro}]$}}
  \TrinaryInfC{$\Gamma,\Delta,\Lambda\linfer\tj{\texttt{nth}(n,\coseq{x})}{X}$}
\end{prooftree}

\begin{align} %%at}{2}
  \cohead(\texttt{nth}(0,\coseq{x})) &= \cohead(\coseq{x}) \\
  %% & \textit{\small base case, induction on n} \\
  \cohead(\texttt{nth}(\SNat{}n,\coseq{x})) &= \cohead(\texttt{nth}(n,\coseq{x}))
  %% \hspace{-18pt} & \textit{\small both \*ductions}\\
\end{align} %%at}

%%%%%%%%%%%%%%%%
\paragraph{\(\texttt{drop}_{\scriptscriptstyle\top}\)}
removes first \(n\) elements from a stream. Defined by coinduction.

\begin{prooftree}
  \AxiomC{$\Gamma\linfer\tj{\texttt{drop}}{\Nat\times\colist{X}\func{}\colist{X}}$}
  \AxiomC{$\Delta\linfer\tj{n}{\Nat}$}
  \AxiomC{$\Gamma\linfer\tj{\coseq{x}}{\colist{X}}$}
  \RightLabel{\raisebox{4pt}{$\scriptstyle[drop\texttt{-intro}]$}}
  \TrinaryInfC{$\Gamma,\Delta,\Lambda\linfer\tj{\texttt{drop}(n,\coseq{x})}{\colist{X}}$}
\end{prooftree}


%%%%%%%%%%%%%%%%
\paragraph{\(\texttt{take}_{\scriptscriptstyle\top}\)}
returns finite list of the first \(n\) elements of a stream. Defined
by induction on the \(\Nat\) arg (it tests for the base case).

E.g.
\[\texttt{take}(3, \langle 5,6,\ldots\rangle) = [5,6,7]\]

\begin{prooftree}
  \AxiomC{$\Gamma\linfer\tj{\texttt{take}}{\Nat\times\colist{X}\func{}\Lift{X}{\Nat}}$}
  \AxiomC{$\Delta\linfer\tj{n}{\Nat}$}
  \AxiomC{$\Gamma\linfer\tj{\coseq{x}}{\colist{X}}$}
  \RightLabel{\raisebox{4pt}{$\scriptstyle[take\texttt{-intro}]$}}
  \TrinaryInfC{$\Gamma,\Delta,\Lambda\linfer\tj{\texttt{take}(n,\coseq{x})}{\Lift{X}{\Nat}}$}
\end{prooftree}

\texttt{take} must be inductively defined, since both head and tail
are finite, by induction on the indexing arg.

\begin{align} %%at}{2}
  \texttt{take}(0,\coseq{x}) &= \nillist \\
 \texttt{take}(\SNat{}n,\coseq{x})) &=
 \defnup{\Cons}(\cohead(\coseq{x}),\texttt{take}(n,\cotail(\coseq{x}))) \label{take:ind}
  %% \\
  %% \cohead(\texttt{take}(\SNat{}n,\coseq{x})) &= \cohead(\coseq{x}) \\
  %% \cotail(\texttt{take}(\SNat{}n,\coseq{x})) &=
  %% \Cons(\cohead(\coseq{x}),\texttt{take}(n,\cotail(\coseq{x}))) \label{take:tail}
\end{align} %%at}


\subsubsection{Endomorphisms (transformers)}

 evens, odds, merge

\subsubsection{Pathological cases}

\subsection{Transfinite Lists}

The idea is that we start with ordinary infinite lists, and append
transfinite ordinals. E.g. \(\langle 0, 1, \ldots, \omega, \omega+1,
\ldots\rangle\). So our new lists will look like a pair of an infinite
list of \(X\) and an infinite list of transfinite ordinals, but in
theory they are concatenated infinities. This is expressed by the
cotype discipline, which is different that that of pair types. The
tails of the sublists have the same type as the whole list. That is,
the sublists do not get their own types.

Not much use except for demonstrating some things about coinductive
reasoning.

We can express this by expanding the stream type, just as we expanded
\(\Nat\) to get \(\NDark\). We just add new co-constructors so that we
can extract the transfinite ordinals. The \(\cohead\) co-constructor
will not change; the new \(\xcohead\) co-constructor will project the
head of the transfinite sublist. Similarly, the new \(\xcotail\) will
return the tail of the transfinite sublist.

But first we need a type for transfinite ordinals. We won't define it
here (yet), we just need to declare it so we can express tokens of the
type. We use \(\Omega\), and we'll use \(\omega_{i}\) for tokens so we
can write \(\tj{\omega_0}{\Omega}\).\footnote{A subscripted \(\omega\)
symbol as a variable must not be confused with its use as a constant
to name the first transfinite ordinal.}

\vspace{2ex}

We need two list types, \(X^{\Nat}\) and \(X^{\Omega}\) and their
concatenation \(X^{\Nat\Omega}\).

Here are the defining inference rules for cotype \(\xcolist{X}\)
(transfinite list of \(X\)).

\begin{align}
  & \linfer \tj{\Omega}{\Univ} & \Omega\ \text{is a type} \\
  \tj{\seq{x}}{\xcolist{X}} & \linfer \tj{\head(\seq{x})}{X} & \\
  \tj{\seq{x}}{\xcolist{X}} & \linfer \tj{\xcohead(\seq{x})}{\Omega} & \\
  \tj{\seq{x}}{\xcolist{X}} & \linfer \tj{\tail(\seq{x})}{\xcolist{X}} & \\
  \tj{\seq{x}}{\xcolist{X}} & \linfer \tj{\xcotail(\seq{x})}{\xcolist{X}} &
\end{align}

[But then we expect \(\tj{\xcohead(\xcotail(\seq{x}))}{\Omega}\). How can we ensure this?]

\subsubsection{Notes}

The infinite list type is less than the new transfinite list type. So
the latter is not included in the ``terminal coalgebra'' set of the
former. So this example does not demonstrate the maximality of the
infinite list cotype.

