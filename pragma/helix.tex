\section{Interpreting Equality: The HoTT Helix}\label{appendix:helix}

Interpreting equality in a calculus is like interpreting truth. The
method is to provide not a meaning but a mapping from the language of
the logic to some other language that serves as a semantic determiner.
So for example, we might map \textsf{True} to \(1\) and \textsf{False}
to 0. For equality, the symbol \(=\) will normally be mapped to a
similar symbol in the semantic language.

One of the distinctive features of HoTT is its interpretation of
equality. It does not even use the symbol \(=\) (except, optionally,
as an abbreviation). Instead it defines equalities as types, and maps
\textit{proofs} of equality to paths in a space.

Purpose of this section: give an intuitive account of HoTT's
path-based equality. Problems with the HoTT Book's
\textit{explanation} of equality are addressed in a different section.

To start: equality as a path. We do not need to understand homotopy to
get the general idea. The challenge is to make sense of \(a=b\) where
\(a\) and \(b\) are not merely different names for the same thing.

We need to make a distinction between equality and identity. It seems
counter-intuitive to talk of two different things being identical.
Wittgenstein certainly thought so: \enquote{Roughly speaking: to say
  of two things that they are identical is nonsense, and to say of one
  thing that it is identical with itself is to say nothing.}

Equality is a different matter. It's easy to see why we might decide
it makes sense to say that two distinct things are equal. For example,
think of the ways a number can be partitioned:
\(5=1+4\ \text{and}\ 5=2+3\) etc. In each case the RHS
is extensionaly equal to LHS, because it can be reduced to the same
thing. But \textit{intensionally} each RHS is clearly distinct. At a
minimum, if we \textit{compute} these terms, the exact computation
process will be different for each, and each will consume a different
amount of energy. In traditional non-constructive mathematics equality
has meant extensional equality, but in constructive mathematics, where
proof of equality must be constructed, extensional equality is not
always enough. This is the case with HoTT.

In traditional mathematics, extensional meaning was king. The
difference between \(1+4\) and \(2+3\) could be safely ignored, since
they both reduce to the same thing. Once an equation was proven, it
became eligible to be used in further proofs without regard to its its
provenance. The particular proof that proved it could be disregarded,
since it had no bearing on further uses of the (proven) proposition.
In other words, other proofs that depend on a proven proposition \(A\)
do not depend on the proof of \(A\) itself. That may not always be the
case in constructive mathematics. [TODO: simple example demonstrating
  use of an equality-proof after it had proven its proposition.]

Diagrammatic intuition pump: proof of equality as a loop whose start
point is the same as its end point. There's an obvious circularity
here, since we've said the start and end points must be equal. So we
just pick a distinguished point in the loop, and say by definition
that it is both the start point and the end point.

We can even use the loop metaphor to set out a criterion of adequacy
for theories of equality. Start = end iff path from former to latter
is a closed loop.

Or: a = b if there is a path between them. But that's not enough, it
would make Paris equal to Rome. So the paths must all be ``the same''
as a canonical loop, which we call ``refl''. Equality paths must have
the same start points and end points, and the start point must
``equal'' the end point. How can we get rid of the circularity?

By codefinition? We add a clause that says if p is a path between a
and b, then \textit{by definition} refl(p) is the canonical path at a.
That still presupposes that the end point equals the start point. We
have to say p is an equality-path, not just any path. And equality
paths are loops.

To prove that a = a, draw a loop from a to itself.

To prove that a = b, draw a path from a to b and prove that it is closed.

But we have no means to prove that such a path is closed (no
constructors for a=b).

However if it is in fact closed, even if we cannot prove it, then we
can deform it until it becomes ``equal'' to the refl loop.

In other words, if we assume that p is an equality path between a and
b, then we have thereby assumed that it is a closed loop. We can work
with that. But we have no way to construct such a loop.

This amounts to a co-inductive definition.

Or: make a distinction between proof and verification. To prove a=b,
draw a closed loop from a to b. To verify the proof, prove that the
loop is closed. One way to do that is to deform it until it is equal
to the refl loop.

Note the shift from definition of equality to proof of equality.
Classically, this is completely absent. Euclid said nothing about what
constitues a \textit{proof} of equality.

Shift to proof motivates the need for two kinds of equality, one
for provables and one definitional.

We still have not said what equality \textit{is}, metaphysically, or
even epistemically.



Real-world analogue: roadways, literal paths. Starting from Paris
there are many paths leading to Rome. There are also many paths from
Paris back to Paris. So when we say ``Paris = Paris'', we back it up
by pointing to one of those loops.

\subsection{Diagrams}

\begin{figure}[h]
  %% \begin{adjustwidth}{-4cm}{}
\centering
%% \begin{subcaptiongroup}%%{.4\textwidth}
%%   \subcaptionlistentry{Proof of a=b}
\begin{tikzpicture}[scale=2.0,
    %% background rectangle/.style={fill=olive!45},
    %% show background rectangle
  ]
  \begin{axis}[
      view={-20}{-20},
      %% axis line style = ultra thick,
      %% axis lines=middle,
      axis lines=none,
      zmax=80,
      xmax=4,
      ymax=4,
      height=8cm,
      xtick=\empty,
      ytick=\empty,
      ztick=\empty,
      clip=false,
      %% x label style={at={(axis cs:2,0.051)},anchor=north},
      %%   xlabel={$y$},
      %% y label style={at={(axis cs:0.05,2)},anchor=north},
      %%   ylabel={$x$},
      %% z label style={at={(axis cs:0.075,0,80)},anchor=north},
      %%   zlabel={$z$},
    ]
    %% \draw[<-] (-1.2,-5.7) -- node [label={west:foobar}]
    %%      {equality pencil} (0.45,-5.0);

    %% refl a
    \addplot3[domain=0:360,samples=300,samples y=1,variable=t,
      black,no marks,thick]
      %% smooth,variable=t,ultra thick]
    ({cos(t)},{sin(t)},4.5);
    \node at (-0.8,0.05) [name=refl]{};
    \node at (0,1.5)
          [name=refllabel,black,label={[scale=0.5]center:$\tj{\refl_a}{a=_A a}$}]
          {};
    %% \draw [<-] (-1.0,-5.2) -- (-0.5,-5) node
    %%       [label={[scale=0.5]east:equipoint pencil}]
    %%       {};
    %% \node at (1.1,-2.7)
    %%       [black,label={[scale=0.6,distance=-4pt]
    %%           center:{$\textcolor{red}{p}:a=_Ab$}}]
    %%       {};
    %% \node at (2.0,0)
    %%       [black,label={[scale=0.6,distance=-4pt]
    %%           center:{$\mathsf{refl_a}:a=_A a$}}]
    %%       {};
    \node at (-0.1,-0.35) [name=omphalos,black,circle,scale=0.2,draw]{};

    %%helix
    \addplot3+[domain=0.5:3*pi,samples=500,samples y=1,
      black,no marks]
    ({sin(deg(x))}, {cos(deg(x))}, {10*x/(pi)});
    %% p:a=b
    \addplot3+[domain=1.3:2.41*pi,samples=500,samples y=1,
      %% shorten >= 2pt,
      red,
      -{Latex[red,length=1.5mm,width=1.5mm]},no marks,thick]
    ({sin(deg(x))},{cos(deg(x))},{10*x/(pi)})%% NB: no semi-colon
    node [name=A,black,circle,scale=0.2,fill,pos=0,
      label={[black,scale=0.6,distance=-4pt]south east:$a$}]{}
    %% node [red,pos=0.7,label={north:$p$}]{}
    coordinate [name=pab,pos=0.5]{}
    node [name=pablabel,pos=0.4,
      xshift=6pt, yshift=10pt,
      label={[black,scale=0.6] %% ,distance=-18pt]
        center:$\tj{\textcolor{red}{p}}{a=_A b}$}]{}
    node [name=B,black,circle,scale=0.2,fill,pos=1,
      label={[black,scale=0.6,distance=-4pt]south east:$b$}]{};
      %% label={[black,label distance=0pt]east:$b$}]{};
    %% equality chord
    %%
    \draw[black,shorten >= -1cm, shorten <=-0.5cm] (A)--(B);
    \draw[-{Latex[length=1mm,width=.7mm]},black,densely dotted]
    (A)--(omphalos);
    \draw[-{Latex[length=1mm,width=.7mm]},black,densely dotted]
    ([xshift=-3pt,yshift=-3pt]B.south west)--(omphalos);
    \draw[-{Latex[length=1mm,width=.7mm]},black]
    ([xshift=-9pt]pablabel.west) to[out=180,in=90] (pab);
    \draw[-{Latex[length=1mm,width=.7mm]},black]
    ([xshift=-11pt]refllabel.west) to[out=180,in=-130] (refl);

\end{axis}
\end{tikzpicture}
\caption{Equality of two distinct values}
\label{fig:aeqb}
%% \end{subcaptiongroup}
%% \begin{subcaptiongroup}{.4\textwidth}
\end{figure}

Figure \ref{fig:aeqb} illustrates proof of equality of two distinct
values. The geometric forms are to be read as semantic symbols: not
symbols of geometric objects, but of whatever concepts we please. The
vertical line (equipoints pencil) represents identity; all points on
the line are identical. Meaning, we treat the points as coreferring
symbols. The segment joining \(a\) and \(b\) represents the type
\(a=_A b\). The red curve joining \(a\) and \(b\) (``helical loop'')
represents proof that \(a=b\). The black circle represents
\(\textsf{refl}_a:a=a\).

[To represent the coreference, draw arrows from each point in the pencil to the center of the base refl circle.

Per Frege: if a=b, then they must be coreferential, but may have
different sense. So if our helix is to represent that, we have to
treat the points a and b as coreferential, and separated in space only
to emphasize the two different ``modes of presentation'' of a and b.

Warning: the vertical equipoints pencil has no counterpart in HoTT; we
just add it for illustrative purposes, to give a graphic
representation of equality types. It makes it look like an equality
proof (helical curve) is just a deformation of an equality type
(straight line), or vice-versa, but that is not the intended meaning.
The equipoints pencil should be thought of as a kind of meta-line.

%% \subcaptionlistentry{Symmetry}
\begin{figure}[h]
\centering
\begin{tikzpicture}[scale=1.0,show background rectangle]
  \begin{axis}[
      view={-20}{-20},
      %% axis line style = ultra thick,
      %% axis lines=middle,
      axis lines=none,
      zmax=80,
      xmax=4,
      ymax=4,
      height=8cm,
      xtick=\empty,
      ytick=\empty,
      ztick=\empty,
      clip=false,
      %% x label style={at={(axis cs:2,0.051)},anchor=north},
      %%   xlabel={$y$},
      %% y label style={at={(axis cs:0.05,2)},anchor=north},
      %%   ylabel={$x$},
      %% z label style={at={(axis cs:0.075,0,80)},anchor=north},
      %%   zlabel={$z$},
    ]
    %% \draw[<-] (-1.2,-5.7) -- node [label={west:foobar}]
    %%      {equality pencil} (0.45,-5.0);

    %% refl a
    \addplot3[samples=30,smooth,variable=t,domain=0:360,ultra thick]
    ({cos(t)},{sin(t)},4.5);
    %% \node [pos=.5,circle,fill]{};
    \node at (-0.5,1.5) [black]{$\mathsf{\scriptstyle refl_a}$};
    \node at (2.6,0) [black]{$\mathsf{refl_a}:a=_A a$};
    %% \draw [<-] (-0.98,-5.2) -- (0,-5) node [right]
    %%       {\small equipoint pencil};
    %%helix
    \addplot3+[domain=0.5:5*pi,samples=500,samples y=1,
      black,no marks]
    ({sin(deg(x))}, {cos(deg(x))}, {10*x/(pi)})
    node [name=C,black,circle,scale=0.4,fill,pos=0.875,
      label={[black,label distance=0pt]east:$c$}]{};

    %% q:b=c
    \addplot3+[domain=1.5:4.41*pi,samples=500,samples y=1,
      shorten >= 3pt,
      red,->,no marks,thick]
    ({sin(deg(x))},{cos(deg(x))},{10*x/(pi)})
    node at (0.7,-5.7) [black] {$\textcolor{red}{q}:b=_Ac$}
    node [red,pos=0.75,label={north:$q$}]{};
    %% p:a=b
    \addplot3+[domain=1.3:2.41*pi,samples=500,samples y=1,
      shorten >= 3pt,
      red,->,no marks,thick]
    ({sin(deg(x))},{cos(deg(x))},{10*x/(pi)})%% NB: no semi-colon
    node [name=A,black,circle,scale=0.4,fill,pos=0,
      label={[black,label distance=0pt]east:$a$}]{}
    node [red,pos=0.7,label={north:$p$}]{}
    node [name=B,black,circle,scale=0.4,fill,pos=1,
      label={[black,label distance=0pt]east:$b$}]{};
    %% equality pencil
    %%
    \draw[black,shorten >= -1cm, shorten <=-0.5cm] (A)--(C);
    \draw[red,fill,circle,scale=0.6] (0,1);
    \addplot3[domain=0.0:7.31*pi,samples=500,samples y=1,
      %% shorten >= 3pt,
      black,no marks,thin]
    ({1*sin(deg(x))},{0.3-1*cos(deg(x))},{9.8*x/(pi)});

\node at (1.8,-2.7) {$\textcolor{red}{p}:a=_Ab$};

\end{axis}
\end{tikzpicture}
\caption{Transitivity}
\label{fig:transitivity}
%% \end{subcaptiongroup}
%% \captionsetup{subrefformat=parens}
%% \caption{Equality proofs: \subref{fig:peqq} a huge cat,
%% and \subref{fig:aeqb} an elephant}
%% \end{adjustwidth}
\end{figure}

Figure \ref{fig:transitivity} illustrates transitivity.

%% \subcaptionlistentry{Symmetry}
\begin{figure}[h]
\centering
\begin{tikzpicture}[scale=1.0,show background rectangle]
  \begin{axis}[
      view={-20}{-20},
      %% axis line style = ultra thick,
      %% axis lines=middle,
      axis lines=none,
      zmax=80,
      xmax=4,
      ymax=4,
      height=8cm,
      xtick=\empty,
      ytick=\empty,
      ztick=\empty,
      clip=false,
      %% x label style={at={(axis cs:2,0.051)},anchor=north},
      %%   xlabel={$y$},
      %% y label style={at={(axis cs:0.05,2)},anchor=north},
      %%   ylabel={$x$},
      %% z label style={at={(axis cs:0.075,0,80)},anchor=north},
      %%   zlabel={$z$},
    ]
    %% \draw[<-] (-1.2,-5.7) -- node [label={west:foobar}]
    %%      {equality pencil} (0.45,-5.0);

    %% refl a
    \addplot3[samples=30,smooth,variable=t,domain=0:360,ultra thick]
    ({cos(t)},{sin(t)},4.5);
    %% \node [pos=.5,circle,fill]{};
    \node at (-0.5,1.5) [black]{$\mathsf{\scriptstyle refl_a}$};
    \node at (2.6,0) [black]{$\mathsf{refl_a}:a=_A a$};
    \draw [<-] (-0.98,-5.2) -- (0,-5) node [right]
          {\small equipoint pencil};
    %%helix
    \addplot3+[domain=0.5:3*pi,samples=500,samples y=1,
      black,no marks]
    ({sin(deg(x))}, {cos(deg(x))}, {10*x/(pi)});
    %% p:a=b
    \addplot3+[domain=1.3:2.41*pi,samples=500,samples y=1,
      shorten >= 3pt,
      red,->,no marks,thick]
    ({sin(deg(x))},{cos(deg(x))},{10*x/(pi)})%% NB: no semi-colon
    node [name=A,black,circle,scale=0.4,fill,pos=0,
      label={[black,label distance=0pt]east:$a$}]{}
    node [red,pos=0.7,label={north:$p$}]{}
    node [name=B,black,circle,scale=0.4,fill,pos=1,
      label={[black,label distance=0pt]east:$b$}]{};
    %% equality chord
    %%
    \draw[black,shorten >= -1cm, shorten <=-0.5cm] (A)--(B);

\node at (1.8,-2.7) {$\textcolor{red}{p}:a=_Ab$};

\end{axis}
\end{tikzpicture}
\caption{Second-order Equality}
\label{fig:peqq}
%% \end{subcaptiongroup}
%% \captionsetup{subrefformat=parens}
%% \caption{Equality proofs: \subref{fig:peqq} a huge cat,
%% and \subref{fig:aeqb} an elephant}
%% \end{adjustwidth}
\end{figure}

To illustrate second-level equality (figure \ref{fig:peqq}), we would
add a third point \(c\) and a helical loop \(q\) representing proof of
\(b=c\). Then we would shade the region on the surface of the cylinder
between the adjacent helical loops \(p\) and \(q\). This would look
like a ribbon wrapped helically around the cylinder. The surface would
proof of equality of \(p\) and \(q\). This figure would however lack a
graphical representation of the second-level equality type \(p =_{a=_A
  b} q\)

[To really make that work, the ribbon would meta-represent the
  equality type, and the proof would be shown by ``inflating'' the
  ribbon to make a kind of tube.

We can show the three standard properties of the equality relation. We
already show reflexivity. To show symmetry, draw another helical loop
between \(a\) and \(b\), with opposite chirality (polarity). To show
transitivity, draw loops from a to b, and from b to c, and then from a
to c.

