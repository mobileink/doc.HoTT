\section{Identity}\label{sec:identity}

Every thing has an identity, and every thing is identical to itself.

Identicality is a relation. Reflexivity is the identicality relation
of a thing to itself. Two distinct things cannot be identical.

Identity is a property. Every thing as a unique identity property. Or
we can think of it as a function that takes each thing to a unique
value.

Two things are identical if they have the same properties, including
identity. In that case, they are not really two things.

Equality need not entail identicality. That is, distinct things may be
equal, if they have the same set of properties except for identity.

Then the principle of Identity of Indiscernables would not hold. Two
indiscernable things could yet have distinct identities, since
identity cannot be discerned.

Is identity a relation? Seems like a property of individuals. On the
other hand, we only need the concept if we have a plurality of
individuals. It gives us individuation, in a sense. So at the least it
presupposes relations (or relatedness), just as ``face of a coin''
presupposes a relation.

\subsection{Frege}

\textquote[\cite{May2001-MAYFOI-2}, p.1-2]{ But the presence of this
  [identity] symbol raised a problem that still perplexes us today. On
  the one hand, identity statements play a logical role, licensing
  substitutivity; but yet they also express substantive propositions,
  to be proved or established. Is the identity symbol to be a logical
  or non-logical symbol?... To achieve this generality, Frege
  understood that not only must this symbol appear in propositions
  that can be true or false, it must also be a logical symbol; the
  truth of a statement of identity allows for a transition between
  propositions by substitution in the course of proof.}

I.e. \(\fregeq\) must play a role in inference, not just reference.

\textquote[Frege, cited in \cite{May2001-MAYFOI-2} p. 8]{Identity of content differs from conditionality and
negation by relating to names, not to contents.}

p. 56: Frege does not explicitly state that it is
firmly establishing this point, and definitively refuting the formalists
argument against equality understood as identity, that is the goal of the
essay.

\textquote[\citetitle{frege_sense_ref} \cite{frege_sense_ref}, final paragraph]{When we found ‘a=a’
  and ‘a=b’ to have different cognitive values, the explanation is
  that for the purpose of knowledge, the sense of the sentence, viz.,
  the thought expressed by it, is no less relevant than its reference,
  i.e. its truth value. If now a=b, then indeed the reference of ‘b’
  is the same as that of ‘a,’ and hence the truth value of ‘a=b’ is
  the same as ‘a=a.’ In spite of this, the sense of ‘b’ may differ
  from that of ‘a’, and thereby the thought expressed in ‘a=b’ differs
  from that of ‘a=a.’ In that case the two sentences do not have the
  same cognitive value. If we understand by ‘judgment’ the advance
  from the thought to its truth value, as in the above paper, we can
  also say that the judgments are different... There is only one
  possible meaning of equality, according to Frege, that can restore
  the peace: identity.}


\begin{displayquote}[
\cite{Geach1952-GEATFT} p. 12, cited in \cite{May2001-MAYFOI-2} p. 8
]
  Now, let
  \[\vlongdash (A\fregeq B)\]

  mean: the symbol \(A\) and the symbol \(B\) have the same conceptual
  content, so that \(A\) can always be replaced by \(B\) and conversely.

\end{displayquote}

\textquote[\cite{May2001-MAYFOI-2}, p. 8]{What Frege maintains here is that identity statements can
  play the logical role they do just because the conceptual content of
  the symbols to be substituted one for another is the same.}

\begin{displayquote}
In the case of substitution, if it is to be licensed by a
statement of identity, then that statement must carry the pertinent
information about the symbols such that they can be substituted one for
another. If we have a judgement that says of symbols that they have the
same content, e.g. “\(c\fregeq d\),” then we can move in a proof from the judgement:
\[\vlongdash P(c)\]
%% |––— P(c)

to the judgement:
\[\vlongdash P(d)\]
%% |––— P(d).

Frege characterizes such inferences by proposition (52):
\[\vlongdash (c \fregeq d) \supset (f(c) \supset f(d))\]
%% |––— (c / d) e (f(c) e f(d)),
glossed with the remark that it “says that we may replace \(c\) everywhere
by \(d\), if \(c \fregeq d\)” (Begriffsschrift, p. 162).
\end{displayquote}

``Judgments of identity of content'' - makes sense for English words,
we can judge whether or not Hesperus is identical to Phosphorus, and
our judgment could be wrong. But in a logical calculus the same
considerations do not apply. Once we've defined things we do not have
to make judgments of identity.

textquote[p. 11]{Although the symbols that occur in the judgements
  that are inferentially related are, as Frege puts it,
  “representatives of their contents,” what are substituted in the
  conceptual notation are symbols, so we must be able to recognize
  that it is the symbol “b” that is being substituted for the symbol
  “a.” “Thus,” Frege says, “with the introduction of a symbol for
  identity of content, a bifurcation is necessarily introduced into
  the meaning of every symbol, the same symbols standing at times for
  their contents, at times for themselves.” (Begriffsschrift, p.
  124).}

\textquote[p.12]{But notice that if symbols are no longer bifurcated,
  then there is no longer any place for an identity of content symbol;
  identity will have to be otherwise defined.}

\textquote[\cite{May2001-MAYFOI-2} p. 54]{But identity of reference is
  not the same as identity of conceptual content. This is because only
  the latter relation is one that holds between expressions qua how
  their contribution to propositional content is determined by their
  associated modes of determination.}
