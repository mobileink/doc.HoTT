\section{Inferential Modalities}\label{sec:modalities}

The logical constants do not in themselves determine how to produce
and use the propositions containing them. It's the other way around:
the rules governing production and use determine the constants. In a
sense, the constant symbols are just abbreviations, concise
representations of the rules. Using them saves us a lot of words.
Consider how we could say that \(P\) is true because \(P\land Q\)
implies \(P\).. We would have to state the construction rule, saying
something like ``If \(P\) is true and \(Q\) is true, then \(P\) is
true; but \(P\) is in fact true, and so is \(Q\), so \(P\) is true.''
We wouldn't get much logic done that way.

Inference rules essentially involve the modalities of possibiility and
necessity. These modalities are almost never explicitly addressed in
accounts of inference.

Inference preserves modality? \(P\rightarrow Q\) means Q necessarily
follows from P. Which means that it is possible to get Q by
providing P.

The inference rule for products may be glossed: ``if
context \ContextG\ suffices for a:A and also b:B, then it also suffices for \( (a,b):A\times B\)''.

``Suffices for'' expresses \textit{possibility}; ``then'' expresses
\textit{necessity}. So the rule may be expressed more explicitly as
``if it is \textit{possible} to produce A and also B from \Gamma\,
then \textit{necessarily} it follows that it is (also)
\textit{possible} to produce \( (a,b):A\times B \) from \Gamma.''

This is hard to express symbolically, since possiblity and necessity
are already implied by the symbols we use.

Glosses for \(\Gamma\linfer B\):

\begin{itemize}
\item \Gamma\ suffices for B
\item Inference from \Gamma\ to B is licensed/valid
\item \Gamma\ entails B
\item B is a consequence of \Gamma
\end{itemize}

They key idea is that using this (or any sentence) as premise of an
inference rule automatically endows it with the modality of
\textit{possibility}.

Put differently: in \(A\rightarrow B\) (and \(A\linfer B\) the
component propositions A and B are not \textit{asserted}. To
\textit{assert that} A we just write \(A\) as a standalone sentence.
To assert that something follows from A, we embed it in a composite
proposition like \(A\rightarrow B\). Then the composite depends on the
\textit{propositional content} of A, but it does not \textit{assert}
A.

Incidentally, this is where Martin-Lof's account of ``judgment'' goes
off the rails. His judgment is assertion, and he puts judgments in
both the premises and conclusion of inference rules. But inference
rules do not assert their premises or conclusions, so they cannot be
``judgments'' in Martin-Lof's sense. At best they can be possible
judgments.

Symbolically, using \( \Diamond \) for \textit{possibly}, \(\Box\)
for \textit{necessarily}, and \(\models\) for ``actually produces'':

%% \begin{prooftree}
%% \AxiomC{$\Diamond\Gamma\models a:A$}
%% \AxiomC{$;$}
%% \AxiomC{$\Diamond\Gamma\models b:B$}
%% \LeftLabel{$\Box$}
%% \TrinaryInfC{$\Diamond\Gamma\models (a,b): A\times B$}
%% \end{prooftree}

(Here the necessity symbol is just a reminder that consequences are
always necessary.)

Important: the inference symbol \(\linfer\) does \textit{not} mean that
the context \textit{does in fact} produce something; it expresses
validity of inference, so it just means that it is \textit{sufficient}
to do so, that it is \textit{possible} to do so (because the inference
is valid). In other words, the modality is built-in; adding the modal
symbol \(\Diamond\) is intended to make this (redundantly) explicit.
Under a strictly reading, \(\Diamond\Gamma\linfer B\) would mean ``it
is possible that (it is possible that \Gamma\ produces B)'', which is
not the intended reading. To make the modality explicit, we need
another symbol \(\models\) to complement \(\linfer\); we gloss
\(\models\) as ``actually produces'' rather than ``suffices for''
(thus it is not an inference symbol).

Note the difference between rules with and without contexts:

%% \begin{prooftree}
%% \AxiomC{$a:A$}
%% \AxiomC{$;$}
%% \AxiomC{$b:B$}
%% \TrinaryInfC{$(a,b): A\times B$}
%% \end{prooftree}

%% \begin{prooftree}
%% \AxiomC{$\Gamma\linfer a:A$}
%% \AxiomC{$;$}
%% \AxiomC{$\Gamma\linfer b:B$}
%% \TrinaryInfC{$\Gamma\linfer (a,b): A\times B$}
%% \end{prooftree}

The former involves ``if \textit{something} then ..''; it expresses
inference from ``judgment'' to ``judgment''. The latter involves ``if
something \textit{suffices for} something else, then ...''; it
expresses an inference whose premises and conclusion are also
inferences (sequents).

