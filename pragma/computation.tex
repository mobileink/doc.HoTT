\section{Computation}\label{sec:computation}

\subsection{Environments}

\subsection{Continuations}

A full and explicit account of duction must include explicity an
account of continuations (and environments).

Continuations are one thing induction and coinduction have in common.

This is easy to see from a simple example: counting the number of
elements in a finite list.
\begin{align}
  \texttt{count}(\nillist) &= 0 \\ \texttt{count}(x\cons\seq{x})) &= 1
  + \texttt{count}(\seq{x}) \label{continuations:count}
\end{align}

The RHS of equation (\ref{continuations:count}) must wait for its
recursive call to complete before it can continue, by adding \(1\) and
then returning.

To make this explicit we can wrap the continuation in a lambda and
pass it to the recursive call. Etc. Continuation Passing Style (CPS).

Then the question is how well this integrates into inferential logic.

\paragraph{Resources}

For an excellent practical explication of continuations see \citetitle{queinnec2003lisp} (\cite{queinnec2003lisp}).

\begin{itemize}
\item \citetitle{Strachey2000ContinuationsAM} \cite{Strachey2000ContinuationsAM}
  \item \citetitle{10.1007/BF01019459} \cite{10.1007/BF01019459}
\item \citetitle{10.1145/62678.62684} \cite{10.1145/62678.62684}
\end{itemize}
