\section{Fixed Points}\label{sec:fixpoints}

Think of a function as expressing denotation of names. It takes a name
to the thing it names, which is another name. Then a fixed point is
self-reference. Hence the connection to recursion/induction etc.

If you add a constructor to a type, you expand its range. If you add a
co-constructor to a co-type, you contract its domain. Because you're
effectively adding a filter. Except not a filter: filters do not
change anything; coinduction changes the meanings of co-ctors. More
info means narrower space.

Resources:

\begin{itemize}
\item \citetitle{lawvere1997conceptual} \cite{lawvere1997conceptual}
\item \citetitle{davey2002introduction} \cite{davey2002introduction}
\item \citetitle{stoy1977denotational} \cite{stoy1977denotational}
\end{itemize}

\subsection{Least Fixed Points}

LFP of an inductive type is the constructors. Any operations you add
going \textit{into} the underlying set can be reduced to the
contructors. Whatever you can do with the new ops could also be done
without them, just using the constructors. IOW, the LFP is a basis.

\subsection{Greatest Fixed Points}

It's not the constructors that are least, it's the type they
determine. Inductive types are built up from their base cases by
expansion. Coinductive cotypes are greatest, because their
co-constructors allow us to ``shrink'' them.

Functions automatically expand, they add new stuff. Cofunctions
automatically contract, the exclude stuff. So \(\defn{evens}\) creates
a new finite list from an old one, by construction. But
\(\codefn{evens}\) can't create a new colist, because they already
exist, and because we have no colist constructors. Instead it
contracts an old one. Remember that a codefinition does not define
\textit{how} a cofunction manages to do whatever it does, it just
specifies \textit{what} must be done. So we interpret the ``action''
of a cofunction as essentially subtractive. Furthermore subtraction or
contraction is the only thing cofunctions can ``do''.

Outdated:

GFP of a coinductive type is the co-constructors. You can add
operations going out of the underlying set, but such operations cannot
be reduced to the co-constructors. On the contrary, the operation of
the co-constructors must be defined over the results of the new
operations.

In other words, we start with co-constructors that are defined for
arbitrary elements of the type. For any \(\tj{\seq{x}}{\colist X}\) we have
\(\tj{\head(\seq{x})}{X}\) and \(\tj{\tail(\seq{x})}{\colist X}\).

Then to define a new function over \(\colist X\), we have to give
definitions of \(\head\) and \(\tail\) for outputs of the function.
This is the exact opposite of induction, where we define the
constructors once. It means that the co-constructors of a co-inductive
type are never completely defined. They always operate on arguments of
type \(\colist X\), but their meaning (the way they function) depends
on how those arguments were produced.

For example, we can define \codefn{even} and \codefn{odd} over type
\(\colist{X}\):

\begin{align}
  \coseq{\seq{x}} & \defeq [x_0, x_1, \ldots] \\
  \codefn{even}(\coseq{\seq{x}}) & = [x_0, x_2, \ldots] \\
  \codefn{odd}(\coseq{\seq{x}})  & = [x_1, x_3, \ldots]
\end{align}

Note first of all that we cannot define such functions by induction.
There is no base case for infinite colists, so we cannot inductively
construct them. For a finite list, we would define \defn{even} and
\defn{odd} by pattern-matching against the internal structure of the
argument, expressed by constructor patterns starting with the base case:

\begin{align}
  \defn{even}([]) & = [] \\
  \defn{odd}([]) & = [] \\
  \defn{even}(x::seq{x}) & = x::\defn{even}(\tail(\seq{x})) \\
  \defn{odd}(x::seq{x}) & = \head(\seq{x})::\defn{odd}(\tail(\seq{x}))
\end{align}

Here we give the meaning of the function applied to constructors of
the type. We define the \textit{circumstances of application} of the
function: what constructions it may be applied to, with what result.
And there are only two possibilities, corresponding to the two
constructors, nil (\([]\)) and cons (\(::\)).

Compare the corresponding codefinitions. We have no base case, and we
cannot start with \(x::seq{x}\), because \cons\ (\(::\)) is not a
primitive constructor. Moreover we have to start by making
\textit{two} assumptions: first that we're given an opaque token
\(\tj{\coseq{x}}{\colist{X}}\), and second that the function we're
codefining may already be applied to such a token to produce another
value of type \(\colist{X}\). Then we define the consequences of
applying the function: the values taken by the co-constructors when
applied to the function's output.

\begin{align}
  \cohead(\codefn{even}(\coseq{xs})) & = \cohead(\coseq{xs}) \\
  \cotail(\codefn{even}(\coseq{xs})) & = \cotail(\cotail(\coseq{xs})) \\
  \cohead(\codefn{odd}(\coseq{xs})) & = \cohead(\cotail(\coseq{xs})) \\
  \cotail(\codefn{odd}(\coseq{xs})) & = \cotail(\cotail(\cotail(\coseq{xs})))
\end{align}

For co-inductive codefinitions, we give the meaning of the
co-constructors of the type when applied to the results of the
function. In other words, we never actually \textit{define} the
function; instead we define the \textit{consequences} that use of the
function has for the co-constructors - thereby extending the meaning
of the co-constructors.

So there's a perfect symmetry the corresponds exactly to the tenets of
inferential semantics: meanings are determined jointly by both
\textit{circumstances} (upcolist) and \textit{consequences}
(downcolist) of application. Put differently, rules of construction
and rules of use.

Now back to greatest fixed point. The idea is that the co-ctors are
the greatest fixed point, or greatest solution set, or some such. Such
that they would subsume any added functions. In what way? For
induction, the new functions reduce to the base ctors. In coinduction,
reduction is the wrong concept. The relation is some other kind of
subsumption.

The coconstructors are just type specifications. We can treat them as
maximally general. Then we codefine some cofunctions - that narrows
the meanings of the co-ctors, for particular cases. So the raw
co-ctors subsume the function-specific meanings assigned by the
codefinitions.

What precisely is the relation between the raw co-ctors and the
codefinitions given them for particular cofunctions?

Key point is that the codefinitions do not define values for the
co-ctors, they only relate them to the input colist. So they confer
structure, not content.  Although they could add content.
