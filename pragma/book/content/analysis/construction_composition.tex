\chapter{Construction and Composition}

\section{Making Composition Explicit}

General principle: composition is primitive; it is presupposed by
construction.

Case study: \(\Nat\)

The successor operation for \(\Nat\) is commonly expressed as a
function: \(\tj{S}{\morph{\Nat}{\Nat}}\). But this is a mistake. For
one thing, it presupposes the concept ``function'', which we have not
yet defined. More fundamentally it hides the notion of composition,
which is the heart of the matter.

A better approach is to use an inference rule:
\AxiomC{$\Gamma\linfer\tj{n}{\Nat}$}
\RightLabel{$\linfer{}\SNat$}
\UnaryInfC{$\Gamma\linfer\tj{n\SNat}{\Nat}$}
\DisplayProof

The problem with this is that the rule does double duty. In explicitly
introduces \(\SNat\); implicitly it introduces \textit{composition}.
The syntactic concatenation of the symbols \(n\) and \(\SNat\) in the
conclusion has semantic significance.

Compare the (natural deduction) rule for \(\lkand\) introduction:
\AxiomC{$A\kern-1.2em$}
\AxiomC{$B$}
\RightLabel{$\scriptstyle{[\lkand \textrm{-intro}]}$}
\BinaryInfC{$A\lkand B$}
\DisplayProof

This rule explicity introduces \(\lkand\), but it does not
surreptitiously introduce composition. Rather, \(\lkand\) itself
explicitly expresses \textit{logical} composition. The concatenation
of the symbols \(A, \lkand\), and \(B\) is purely syntactic; it has no
semantic significance.

To make the rule for \(\SNat\) introduction fully explicit, we need an
explicit symbol for \textit{numeric} composition. Any symbol would do;
for example we could
use\ \(\ulcorner\,\text{\upshape\textasciicircum}\,\urcorner\) and
write \(n\text{\upshape\textasciicircum}\SNat\). The drawback with
this sort of infix symbol is that it will be cumbersome when we need
to express additional types of numeric composition, which we will do
shortly. Subscripting doesn't look so great: \(n\text{\upshape
  \textasciicircum}_{0}\SNat\) So instead we will use bracketing so we
can write things like \(\langle n\SNat\SNat\SNat\rangle_0\). This
gives us:
\AxiomC{$\Gamma\linfer\tj{n}{\Nat}$}
\RightLabel{$\linfer{}\langle\SNat\rangle$}
\UnaryInfC{$\Gamma\linfer\tj{\langle{}n\SNat\rangle}{\Nat}$}
\DisplayProof

Note that the rule has been renamed to indicate explicitly that it
introduces both \(SNat\) and concatenation \(\langle\,\rangle\).




