\chapter{Continuations}

The concept of a \gls{continuation} emerged from the world of
programming\footnote{\cite{discovery_continuations}}. But we can have
logical continuations too.

A full and explicit account of duction must include explicity an
account of continuations (and environments).

Continuations are one thing induction and coinduction have in common.

This is easy to see from a simple example: counting the number of
elements in a finite list.
\begin{align}
  \texttt{count}(\nillist) &= 0 \\ \texttt{count}(x\cons\seq{x})) &= 1
  + \texttt{count}(\seq{x}) \label{continuations:count}
\end{align}

The RHS of equation (\ref{continuations:count}) must wait for its
recursive call to complete before it can continue, by adding \(1\) and
then returning.

To make this explicit we can wrap the continuation in a lambda and
pass it to the recursive call. Etc. Continuation Passing Style (CPS).

Then the question is how well this integrates into inferential logic.

\section{Logical Continuations}

The concept of `continuation' that emerged from the practical work of
programmers might seem inherently computational. But perhaps
surprisingly it has a direct analog in formal logic when viewed from
an inferentialist perspective.

Every concept is determined jointly by its upstream and and downstream
inferences, those that lead to and from it. Introduction rules express
upstream inference, and elimination rules express downstream
inferences.

Both kinds can be thought of as rules of \textit{use}. This is easily
seen for elimination rules; they tell us what we can do,
\textit{given} their premise. By contrast, introduction rules tell us
how to \textit{produce} their conclusion, not how to use it. Yet they
too count as rules of use, because they tell us how to use
\textit{other} propositions to make the inference.

But each kind of rule expresses only one side of the story. Both are
incomplete; neither alone fully determines the meaning of the concept they
jointly define.

What is the point of producing something, except to use it?

Both kinds of rule leave half of the meaning implicit. Specifically,
each presupposes the other. Introduction rules implicitly ``depend''
on elimination rules, and vice-versa. It's not that we could not have
introduction rules without elimination rules; it's rather that
elimination is \textit{the piont} of introduction rules. We make stuff
so that we can use it, so the intended use is always implicitly
present in the construction.

Put differently, we make inferences when we reach a point in our
reasoning where we need to prove something in order to continue. At
that point, our general line of reasoning is put on hold while we work
out the new inference whose conclusion we need. Once we've made the
needed inference, we \textit{continue to reason} using the conclusion
of the inference -- and the elimination rules are what enable us to
use the conclusion, possibly in conjuction with other propositions and
rules.

The analogy with computational continuations should be clear. When we
come to make an inference, there is an implicit ``rest of the proof''
that awaits the completion of the inference. An inferential
continuation, as it were; we do not have a name for this concept,
because it has never (to my knowledge) been noticed in the history of
logic\footnote{That may well turn out to be untrue, but what we can
say confidently is that if ``logical continuation'' has been noticed,
nobody paid any attention. The possibility of interpreting
continuations under Curry-Howard has been noticed, but that's a
different topic. I'm talking about a purely logical concept that
corresponds to computational continuation but does not itself
presuppose any notion of computation.}

So the Curry-Howard isomorphism preserves continuations.

It follows that to make inference rules fully explicit, we need to make logical continuations explicit.

We need first-class continuations within logic. Maybe first-class
environments as well.

We might as well make first-class ``co-continuations''. The
``current'' inference (function) splits the reasoning into pre- and
post- (or temporal before and after). Continuations represent
``after'' co-continuations, ``before''. Co-continuations are called
``environments''. Actually, they're both environments. The difference
is in orientation or directionality relative to the ``current'' state.

NB: this is not new; it was noticed as early as the 1980s
\parencite{griffin_ch_control} that computational continuations were
compatible with Curry-Howard and thus with logic. But an explanation
based on inferential semantics is new, to my knowledge. Probably
because nobody uses inferential semantics. What we want is in the
sequent calculus of logic, not the lambda calculus of functions.

For the computer scientist, continuations are about control. For the
logician, continuation is about explanation, specifically of inductive
and co-inductive inference.

\section{Resources}

For an excellent practical explication of continuations see \citetitle{queinnec2003lisp} (\cite{queinnec2003lisp}).

\begin{itemize}
\item \citetitle{griffin_ch_control} \cite{griffin_ch_control}
\item \citetitle{Strachey2000ContinuationsAM} \cite{Strachey2000ContinuationsAM}
\item \citetitle{discovery_continuations} \cite{discovery_continuations}
\item \citetitle{10.1145/62678.62684} \cite{10.1145/62678.62684}
\item \citetitle{reasoning_continuations_2} \cite{reasoning_continuations_2}
\end{itemize}

