\chapter{Initial Algebras and Final Coalgebras}

\section{Ethnonumbers}

To demonstrate the concepts of initial algebra and final coalgebra, we
show how to define variant versions of \(\Nat\).
\begin{itemize}
\item \(\Nat^1\)\quad equivalence class of \(\Nat\)-algebras with one successor operation
\item \(\Nat^2\)\quad equivalence class of \(\Nat\)-algebras with two successor operation
\end{itemize}

Standard \(\Nat\) definition:

\begin{center}
  \AxiomC{$\Gamma\linfer$}
  \RightLabel{$\linfer\ZNat$}
  \UnaryInfC{$\Gamma\linfer\tj{\ZNat}{\Nat}$}
  \DisplayProof
  \hspace{2em}
  \AxiomC{$\Gamma\linfer\tj{n}{\Nat}$}
  \RightLabel{$\linfer\langle\SNat\rangle$}
  \UnaryInfC{$\Gamma\linfer\tj{\langle{}n\SNat\rangle}{\Nat}$}
  \DisplayProof
\end{center}

Since numbers and counting originated in Mesopotamia, we'll start by
defining \textit{Babylonian numbers}, \(\Nat_B\):

\begin{center}
  \AxiomC{$\Gamma\linfer$}
  \RightLabel{$\linfer\ZNat_B$}
  \UnaryInfC{$\Gamma\linfer\tj{\ZNat_B}{\Nat_B}$}
  \DisplayProof
  \hspace{2em}
  \AxiomC{$\Gamma\linfer\tj{n}{\Nat_B}$}
  \RightLabel{$\linfer\langle\SNat\rangle_B$}
  \UnaryInfC{$\Gamma\linfer\tj{\langle{}n\SNat\rangle_B}{\Nat_B}$}
  \DisplayProof
\end{center}

Now we're going to define two additional versions of \(\Nat\). Since
``successor'' is a Latin word, we'll design one version for the Romans
and one for the Greeks. These definitions will have the exact same
form as \(\Nat_B\) except for the symbols. That's enough to make them
different systems.

Greek:
\begin{center}
  \AxiomC{$\Gamma\linfer$}
  \RightLabel{$\linfer\ZNat_{g}$}
  \UnaryInfC{$\Gamma\linfer\tj{\ZNat_{g}}{\Nat_{g}}$}
  \DisplayProof
  \hspace{2em}
  \AxiomC{$\Gamma\linfer\tj{n}{\Nat_{g}}$}
  \RightLabel{$\linfer\langle\SNat\rangle_{g}$}
  \UnaryInfC{$\Gamma\linfer\tj{\langle{}n\SNat\rangle_{g}}{\Nat_{g}}$}
  \DisplayProof
\end{center}

Latin:
\begin{center}
  \AxiomC{$\Gamma\linfer$}
  \RightLabel{$\linfer\ZNat_l$}
  \UnaryInfC{$\Gamma\linfer\tj{\ZNat_l}{\Nat_l}$}
  \DisplayProof
  \hspace{2em}
  \AxiomC{$\Gamma\linfer\tj{n}{\Nat_l}$}
  \RightLabel{$\linfer\langle\SNat\rangle_l$}
  \UnaryInfC{$\Gamma\linfer\tj{\langle{}n\SNat\rangle_l}{\Nat_l}$}
  \DisplayProof
\end{center}

These systems are all isomorphic. So they form an equivalence class,
which we'll call \(\Nat^1\). We can nominate any member of that class
to serve as its canonical member; we'll use \(\Nat_B\). Then
\(\Nat_B\) is the \textit{initial algebra} of \(\Nat^1\), because
there is a unique homomorphism from every member of that class to
\(\Nat_B\).

Since the members of \(\Nat^1\) are isomorphic, we can treat them as
``equal'': they have the same extension, which is an abstraction over
their isomorphic elements:

\begin{align}
  0 \eqdef \ZNat & \cong \ZNat_B \cong \ZNat_{\gamma} \cong \ZNat_L \\
  \SNat & \cong \SNat_B \cong \SNat_{\gamma} \cong \SNat_L
\end{align}

But they are all extensionally distinct, since they do have different
constructors, which must be used to define functions and prove
propositions involving them. That would of course be extremely
cumbersome. The nice thing about an initial algebra is that you can
use it as a proxy for any other algebra to which it has a
homomorphism. So for example a proof of something for the initial
algebra counts as a proof for all those other algebras.

That's all fine, but its pretty boring, since all we have so far is
isomorphisms. What really reveals the usefulness of initial algebras
is the existence of other algebras to which the initial algebra is
homomorphic, but which have some additional structure. It's not so
easy to imagine such a beast if we stick to the standard collection of
mathematical objects like products and sums. But it's very easy to
clearly demonstrate the concept by simply adding a constructor to our
\(\Nat\)-algebras.\footnote{What about the rationals and reals?
There's a homomorphism from \(\Nat\) to \(\Real\), but a proof of
something for all \(\Nat\) does not prove it for all \(\Real\), does
it? But it does prove it for the subset of \(\Real\) that is
isomorphic to \(\Nat\). That's the target of the homomorphism. So for
example a proof that \(a+b=b+a\) for \(\Nat\) suffices to prove the
same for the reals that correspond to the nats - and note that they are
not natural numbers, they have a different type, which we express by
adding a decimal point, e.g. \(2.0\) not \(2\). So we could use
\(\Real\) to demonstrate the point of initial algebras, but that has
the disadvantage that constructing the reals from \(\Nat\) is
complicated. Merely adding ctors to \(\Nat\) demonstrates the same
thing in the simplest possible manner.}

So our next step is to define the Greco-Roman numbers. We'll need only
one \(\ZNat\) constructor but two successors:

\begin{center}
  \AxiomC{$\Gamma\linfer$}
  \RightLabel{$\linfer\ZNat_{GL}$}
  \UnaryInfC{$\Gamma\linfer\tj{\ZNat_{GL}}{\Nat_{GL}}$}
  \DisplayProof
\end{center}
\vspace{2ex}
\begin{center}
  \AxiomC{$\Gamma\linfer\tj{n}{\Nat_{GL}}$}
  \RightLabel{$\linfer\langle\SNat\rangle_G$}
  \UnaryInfC{$\Gamma\linfer\tj{\langle{}n\SNat\rangle_{G}}{\Nat_{GL}}$}
  \DisplayProof
  \hspace{2em}
  \AxiomC{$\Gamma\linfer\tj{n}{\Nat_{GL}}$}
  \RightLabel{$\linfer\langle\SNat\rangle_L$}
  \UnaryInfC{$\Gamma\linfer\tj{\langle{}n\SNat\rangle_L}{\Nat_{GL}}$}
  \DisplayProof
\end{center}

Now we can construct hybrid numbers like
\(\langle\ZNat_{GL}\SNat_G\SNat_L\rangle_{GL}\)


Clearly \(Nat_{GL}\) has the same cardinality as \(\Nat_B\), but are
they isomorphic? No. The latter is homomorphic to the former, but not
the other way around.

[TODO: demonstrate the morphisms]

Then we can make more ethnonumbers. For example, we can define Chinese
and Indic \(\Nat\), and then combine them to form Sino-Indic \(\Nat\).
The latter would be isomorphic to Greco-Roman \(\Nat\), so they would
form an equivalence class \(\Nat^2\). We can go on to form \(\Nat^3
\ldots \Nat^n\) just by adding successor operators (the superscript
indicating the number of successor operators). They would all have the
same initial algebra, \(\Nat_B\).
