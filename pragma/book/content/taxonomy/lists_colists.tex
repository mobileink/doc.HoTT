\chapter{Lists and Colists}

Lists and colists are really extensions of products and sums.

Cobasis = cobase set = all elements associated with the cotype treated
as a set.

\section{Lists and Colists as functions}

Any list can be viewed as a function \(f\) defined over \(\Nat\); then
\(n^{th}\) element of the sequence is \(f(n)\).\footnote{The idea of
representing lists as functions in explaining coinduction comes from
\cite{jacobs_intro_coalgebra}, p. 1.} To express the type of infinite
lists in functional terms, we start with the set of all functions from
\(\Nat\) to arbitrary type \(A\). The usual notation for such a type
is \(\Nat\fnarrow{}A\), but we can use the more compact notation common
in category theory: \(\morph{A}{\Nat}\defeq\Nat\rightarrow A\). This
has the advantage of allowing us to express our cotype more concisely:
\(\morphtop{A}{\Nat}\) is the cotype corresponding to
\(\morph{A}{\Nat}\). We'll explain this correspondance more precisely
as we go. For now, the point is that \(\morph{A}{\Nat}\) is a
function-oriented equivalent to \(\colist{A}\). Recall that the latter
is our list-oriented notation for ``infinite (\(\omegaup\)) stream of
terms of type \(A\)''.

A function-oriented perspective allows us to give a somewhat more
constructive account of the \textit{behavior} of streams, but this
must not be mistaken for a constructive account of the streams
themselves.

The set of functions associated with type \(\morphtop{A}{\Nat}\) is
equal to the set associated with type \(\morph{A}{\Nat}\). The
difference between the two is the rules. In effect,
\(\morphtop{A}{\Nat}\) is \(\morph{A}{\Nat}\) extended by a pair of
inference rules governing the behavior of elements of
\(\morphtop{A}{\Nat}\). It is a cottype, not a set; the rules that
determine the cotype \textit{use} the functions of the set
\(\morph{A}{\Nat}\), but those functions do not in and of themselves
suffice to \textit{define} the cotype.

Each cofunction \(\sigmatop\) in \(\morphtop{A}{\Nat}\) is the image
of a function \(\sigmabot\) in \(\morph{A}{\Nat}\) under a ``lifting''
map. But these lifted functions cannot be directly applied to an
argument. The decoration on the notation is there for a purpose: the
elements of \(\morph{A}{\Nat}\) are functions, but the elements of
\(\morphtop{A}{\Nat}\) are not. Or maybe they are; but it does not
matter, because they are hidden. The only way to use them is to use
these two rules:

\begin{itemize}
\item the \(\cohead\) of an arbitrary element
  \(\sigma_{\scriptscriptstyle{\top}}\) of type \morphtop{A}{\Nat} is
  the value of \(\sigma_{\scriptscriptstyle\bot}\) (of type \(A^{\Nat}\)) at \(0\). (Note the \(\bot\) and \(\top\) decorations.)
\item the \(\cotail\) is \(\lambda{}n.\sigma(n+1)\) -- a ``wrapper''
  function that, when passed arg \(n\), applies that same \(\sigma_i\)
  to \(n+1\). Since this is a function of type \(\Nat\fnarrow{}A\), it
  counts as an element of type \morphtop{A}{\Nat}.
\end{itemize}

The result is that iteration over an element of \morphtop{A}{\Nat}
using \(\cohead\) and \(\cotail\) is just like iteratively applying a function from \(A^{\Nat}\) to the elements of \(\Nat\), in order.

In this example we have for tail \(\lambda{}n.\sigma(n+1)\). We can
rewrite this as \(\lambda{}n.\sigma(f(n))\), where
\(f\eqdef\lambda{}x.x+1\). We could instead use
\(f\eqdef\lambda{}x.x+2\); for that matter, we could use any function
of type \(\morph{\Nat}{\Nat}\) so long as it is injective. The end
result would be the same: \(\sigma\) would (eventually) be applied to
all \(\Nat\), so all streams would be represented (but not
\textit{generated}). That's one reason we do not want to take such
rules as definitions of the cotype.

[Terminology: \(\lambda{}n.\sigma(g(n))\) as a ``biased'' version of
  \(\sigma\).]

\vspace{3ex}

Note that we are not compelled to think of streams as functions over
N. We don't even need N. We just need head and tail.

Here a \textit{major, \textbf{MAJOR}} caveat is in order:

\begin{itemize}
\item \morphtop{A}{\Nat} is \textit{\textbf{not constructed}}, so
  there is no sense it which it could be said to be derived from
  \(A^{\Nat}\)
\item There is \textit{\textbf{no}} one-to-one mapping from the elements of \(A^{\Nat}\) to the elements of \morphtop{A}{\Nat}
\item The elements of \morphtop{A}{\Nat} are \textit{\textbf{not}}
  images of the elements of \(A^{\Nat}\)
\end{itemize}

That's the negative part.  Here's the affirmative part:

\begin{itemize}
\item The elements (functions) of \(A^{\Nat}\) are used to
  \textit{specify} the ``behavior'' of the elements (cofunctions) of
  \morphtop{A}{\Nat}.
\item There \textit{is} a mapping from \morphtop{A}{\Nat} to \(A^{\Nat}\).
\end{itemize}

Here is how we informally describe \morphtop{A}{\Nat}: for arbitrary
\(\tj{n}{\Nat}\) and \(\tj{\sigma_{\top}}{A_{\top}^{\Nat}}\) we have:
\begin{align}
  \cohead(\sigma_{\top}(n)) & = \sigma(0) \label{jacobs:head}
\end{align}

Now \(\sigma\) is a function from our original type \(A^{\Nat}\), so
this equation strongly suggests that \(\sigma_{\top}\) (on the LHS)
must be none other than \(\sigma\) (on the LSH) ``lifted'' into
\morphtop{A}{\Nat}. But that's too strong. It \textit{could} be the
case that there is a mapping
\begin{equation}
  A^{\Nat}\rightarrow A_{\top}^{\Nat}\label{eq:slift}
\end{equation}
\noindent
but it is not \textit{necessarily} the case. Remember that cotypes are
completely opaque, so we have no way of knowing how they or their
members were constructed. Even if there were such a mapping, simple
logic tells us that
\begin{equation}
  A_{\top}^{\Nat}\rightarrow A^{\Nat}
\end{equation}
\noindent
does not follow from \ref{eq:slift}, which is what we need for (\ref{jacobs:head}).

Here is how Jacobs diagrams this:

\vspace{2ex}

\begin{tikzpicture}
  \node at (0,0) (a) {$A^{\Nat}$};
  \node at (2.2in,0) (b) {$A\times{}A^{\Nat}$};
  \draw[-Latex] (a) -- (b) node[pos = .5, above]
       {\scriptsize\(\sigma\mapsto (\sigma(0), \lambda{}n.\sigma(n+1))\)};
\end{tikzpicture}%

But this image is misleading. It makes it look like \(A^{\Nat}\) was
responsible for building the cotype, or something like that. But it's
the other way around. We started with the cotype and then rooted
around till we found \(A^{\Nat}\) as a source of suitable behaviors.

Just as bad, it suggests that \(A\times{}A^{\Nat}\) just \textit{is}
the cotype. But that's wrong, it's the type of the \textit{behaviors}
of the cotype.

But maybe it's a typo. The picture could easily be fixed by adding some \(\bot\) annotations:

\begin{tikzpicture}
  \node at (0,0) (a) {$A_{\scriptsize\top}^{\Nat}$};
  \node at (2.2in,0) (b) {$A\times{}A^{\Nat}$};
  \draw[-Latex] (a) -- (b) node[pos = .5, above]
       {\scriptsize\(\sigma_{\top}\mapsto (\sigma(0), \lambda{}n.\sigma(n+1))\)};
\end{tikzpicture}%

But that still can't be quite right. The lambda is wrong; it needs to
be deliver element of the cotype, i.e. a lifted function, so we would need
a lifted lambda.

\section{Promotion (lifting) and Demotion (lowering)}

\section{Lists}

\section{Colists (Streams)}

\subsection{Copermutations}

E.g. swap and other stack operations. swap(n) moves the nth element to
the front of the list.

Copermutations rather than permutations, because all permutations
already exist. Copermutation functions just map between existing
streams.  Permutations on finite lists, by contrast, ...

Copermutations are conservative; they do not contract the cobasis.

For example suppose we start with \(\langle 1,2,\ldots\rangle\) and swap the
top two elements, yielding \(\langle 2,1,\ldots\rangle\). Since all streams
already exist, this does not expand the cobasis. Nor does it contract
it, since the same operation goes the other way around.

\subsection{Contraction}

Functions like \(\codefn{evens}\) and \(\codefn{odds}\) contract the
cobasis.

\subsection{Demotion (co-lifting)}

See also: promoting (lifting) one \(f:\Nat\fnarrow\Nat\) (e.g. dbl) to
operate pointwise on a list (\(\cofn{f}\)). Jacobs' example lifts all
such functions and does more than just apply them pointwise.

Jacobs gives a wonderful example of a colist on the first page of
\parencite{2017intro_coalgebra}. He calls it a coalgebra; we'll call
it a cotype. The concept illustrated is enlightening, but the reader
is warned that the presentation rather misleading, and sometimes
wrong. For example, right at the beginning the text says
\textquote{...a coalgebra is given by a set S and a function c with S
  as domain and with a “structured” codomain...}. But this gets it
exactly backwards: it says or at least suggests that such cotypes are
constructed, which is plainly false.

