%% \section{Notation \& Terminology}\label{appendix:notation}

Remember that it's duality all the way down. We need parallel sets of symbols, one for induction and one for coinduction.

\paragraph{Kind decorations}

\begin{itemize}
\item \(\inductive{A}\)\hfill \verb|\inductive{A}|, \verb|\type{A}|
\item \(\coinductive{A}\)\hfill \verb|\coinductive{A}|, \verb|\cotype{A}|
\item \(\fn{f}\)\hfill \verb|\fn{f}|
\item \(\cofn{f}\)\hfill \verb|\cofn{f}|
\end{itemize}

\paragraph{Equalities}

\begin{itemize}
\item \(=\)\hfill \verb|=|

  Inductive equality. This is 'ordinary' equality.
\item \(\simeq\)\hfill \verb|\simeq|

Bisimilarity (Coinductive equality) . There is no universally
agreed-upon symbol for this; some authors use \(\sim\), others use a
double-headed arrow of some kind, e.g. \(\bisimarrow\) is used by
\parencite{2017intro_coalgebra}.
\end{itemize}

\paragraph{Infinities}

\begin{itemize}
\item \(\Nat\)\quad the natural numbers w/o zero\hfill \verb|\Nat|
\item \(\NatZ\)\quad with zero\hfill \verb|\NatZ|
\end{itemize}

\paragraph{Morphisms}

\begin{itemize}
\item \(\morph{A}{B}\)\hfill \verb|\morph{A}{B}|
\item \(\exp{B}{A}\)\hfill \verb|\exp{B}{A}|
\item \(\type{(\exp{B}{A})}\)\hfill\verb|\type{(\exp{B}{A})}|
\item \(\expbot{B}{A}\eqdef\type{(\exp{B}{A})}\)\hfill \verb|\expbot{B}{A}|
\item \(\exp{{\type{B}}}{A}\neq\expbot{B}{A}\) \hfill \verb|\exp{{\type{B}}}{A}|
%%
\item \(\coexp{B}{A}\)\hfill \verb|\coexp{B}{A}|
\item \(\cotype{(\exp{B}{A})}\)\hfill\verb|\cotype{(\exp{B}{A})}|
\item \(\coexp{B}{A}\eqdef\ttop{(\exp{B}{A})}\)
\item \(\exptop{B}{A}\)\neq\(\exp{{\cotype{B}}}{A}\)
\end{itemize}

The exponential notation \(\exp{Y}{X}\) for \(\morph{X}{Y}\) comes from
category theory. In CT, arrows are reserved to express morphisms in
the most general sense. Since functions are a species of morphism,
clarity demands a different notation, hence exponentiation. (There are
deeper reasons to favor exponential notation, but for our purposes we
just need distinct notation.)

\textbf{Caveat}: exponentiation does double duty. We also use it to
indicate that a function is ``lifted'', e.g.
\(\promote{\expbot{f}{X}}{\coexp{f}{X}}\). Here \(\exp{f}{X}\)
indicates that \(X\) is the domain of \(f\); it does not mean
``functions from \(X\) to \(f\)''.

Exponentiation examples:
\begin{itemize}
\item \(\exp{X}{\Nat}\)\quad lists expressed as functions
\item \(\exp{X}{\omega}\)\quad infinite lists expressed as functions
\item \(f\) lifted over finite lists: \(\Lift{f}{\scriptscriptstyle\Nat}\)
\item \(f\) lifted over streams: \(\Lift{f}{\omega}\)
\end{itemize}

See \ref{notation:definition} below for decorations that add even more
detail.



\paragraph{Promotion and demotion}

\begin{itemize}
\item \(\promote{X}{Y}\)\quad promotion\hfill \verb|\promote{X}{Y}|
\item \(\revpromote{X}{Y}\)\quad reverse promotion\hfill \verb|\revpromote{X}{Y}|
\item \(\demote{Y}{X}\)\quad demotion\hfill\verb|\demote{Y}{X}|
\item \(\revdemote{Y}{X}\)\quad reverse demotion\hfill\verb|\revdemote{Y}{X}|
\item \(\prodemote{Y}{X}\)\quad both\hfill \verb|\prodemote{Y}{X}|
\item \(\depromote{Y}{X}\)\hfill\verb|\depromote{Y}{X}|
\end{itemize}

Examples:

\begin{itemize}
\item \(\promote{\funcbot{A}{B}}{\morphtop{B}{A}}\)
\item \(\revdemote{\funcbot{A}{B}}{\morphtop{B}{A}}\)
\item \(\demote{\exp{B}{A}}{(A\fnarrow{}B)}\)
\end{itemize}


\paragraph{Meta-variables}

Statements: \(\stmt{A}, \stmt{B}, \ldots\)

Propositions: \(\prop{A}, \prop{B}, \ldots\)


\paragraph{Definition}\label{notation:definition}

\begin{itemize}
\item Definition operator: \(\equiv\) or \(\eqdef\)
\item Inductive definition annotation: subscript
  \(\bot\). E.g. \(\defnup{f}\) indicates that
  \(f\) is defined by induction. Same for sets: \(\defnup{X}\) means
  that the set \(X\) is defined by induction. (Latex:
  \texttt{\textbackslash{}defnup\{X\}})
\item Coinductive definition annotation: subscript \(\top\). E.g.
  \(\defndn{f}\) indicates that \(f\) is defined by coinduction. Same
  for sets: \(\defndn{X}\) means that the set \(X\) is defined by
  coinduction. (Latex: \texttt{\textbackslash{}defndn\{X\}})
\end{itemize}

More specific information may be given by superscripts; for example,
to indicate that function \(f\) is defined by coinduction over the
type of infinite streams of \(X\):
\(\tj{\defndn{f}^{\omegaup}}{\defndn{X}^{\omegaup}}\). Needless to
say, this level of detail is usually not needed, but it is available
if absolute clarity is required.

\subsection{Determinateness}

\begin{itemize}
\item Indeterminacy: \(\xcancel{a}\) or \(\overset{\times}{a}\) - two
  strokes meaning untyped and undefined; an empty, unclassified symbol
\item Underdetermined: ??
\item Partially determined (i.e. typed): \(\cancel{a}_{\scriptscriptstyle A}\) (if \(\tj{a}{A}\), but \(a\) is not defined.)
\item Fully determinate: \(\overbar{\underbar{a}}\) (or: \fbox{\(a\)})
  -- typed and defined. In display environments like diagrams we may
  use the ground symbol; see (\ref{depgraph:coinduction}) for an
  example.
\end{itemize}

Possibly: three-sided box for underdetermined? Box suggests category,
so sym in open box suggests typing.

\subsection{Symbols}

TODO: use \(\supset\) for implication, \(\rightarrow\) for function.
(make Curry-Howard explicit)

\begin{itemize}
\item Equality: \(=\)
\item approx: \(\approx\)
\item sim: \(\sim\)
\item simeq: \(\simeq\)
\item cong: \(\cong\)
\item Eqdef: \(\eqdef\)
\item Coloneqq: \(\coloneqq\)
\item Questeq: \(\questeq\)
\item bumpeq: \(\bumpeq\)
\item Congruence: \(\equiv\) (\texttt{\textbackslash{}equiv})
\item Triangle eq: \(\triangleq\)
\item triangle: \(\triangle\)
\item triangledown: \(\triangledown\)
\item smalltriangledown: \(\smalltriangledown\)
\item medtriangledown: \(\medtriangledown\)
\item largetriangledown: \(\largetriangledown\)
\item Wedge eq: \(\wedgeq\)
\item owedge: \(\owedge\)
\item varowedge: \(\varowedge\)
\item ovee: \(\ovee\)
\item varovee: \(\varovee\)
%% \item Corresponds: \(\corresponds\)
\item Dot eq: \(\doteq\)
\item Circle eq: \(\circeq\)
\item Semantic brackets: \(\llbracket a\rrbracket\) refers to semantic
  object denoted by symbol ``a''. Useful for disambiguation. For
  example ``the structure of \(A\land B\)'' is ambiguous, since it
  could refer to either the syntactic or semantic structure. But ``the
  structure of \(\llbracket A\land B\rrbracket\) refers unambiguously
  to semantic structure.
\item Corner quotes: \(\ulcorner a\urcorner\) to disambiguate
  reference to syntax.
\item Syntactic entailment: \(\linfer\)
\item Particular entailment: \(\linfer_\alpha\) or
  \(\sststile{\alpha}{}\). E.g. \(A;B\sststile{\land}{}{A\land B}\).
  Another example: \(\Gamma\sststile{(a,b)}{=}{p:A\times B}\) for
  \(\Gamma\linfer p=(a,b):A\times B\).
\item Semantic entailment: \(\models\), \(\sdtstile{}{}\) (Ebbinghaus also uses this for the satisfaction relation)
\end{itemize}

