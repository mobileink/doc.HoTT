%% \section{Notation \& Terminology}\label{appendix:notation}

\subsection{Infinities}

\begin{itemize}
\item Nat w/o zero: \(\Nat\)
\item Nat w/zero: \(\NatZ\)
\end{itemize}

\subsection{Lifting}

\begin{itemize}
\item Finite lists: \(\Lift{X}{\Nat}\)
\item Streams: \(\Lift{X}{\omega}\)
\item \(f\) lifted over finite lists: \(\Lift{f}{\scriptscriptstyle\Nat}\)
\item \(f\) lifted over streams: \(\Lift{f}{\omega}\)
\end{itemize}

See \ref{notation:definition} below for decorations that add even more
detail.


\subsection{Meta-variables}

Statements: \(\stmt{A}, \stmt{B}, \ldots\)

Propositions: \(\prop{A}, \prop{B}, \ldots\)


\subsection{Definition}\label{notation:definition}

\begin{itemize}
\item Definition operator: \(\equiv\) or \(\eqdef\)
\item Inductive definition annotation: subscript
  \(\bot\). E.g. \(\defnup{f}\) indicates that
  \(f\) is defined by induction. Same for sets: \(\defnup{X}\) means
  that the set \(X\) is defined by induction. (Latex:
  \texttt{\textbackslash{}defnup\{X\}})
\item Coinductive definition annotation: subscript \(\top\). E.g.
  \(\defndn{f}\) indicates that \(f\) is defined by coinduction. Same
  for sets: \(\defndn{X}\) means that the set \(X\) is defined by
  coinduction. (Latex: \texttt{\textbackslash{}defndn\{X\}})
\end{itemize}

More specific information may be given by superscripts; for example,
to indicate that function \(f\) is defined by coinduction over the
type of infinite streams of \(X\):
\(\tj{\defndn{f}^{\omegaup}}{\defndn{X}^{\omegaup}}\). Needless to
say, this level of detail is usually not needed, but it is available
if absolute clarity is required.

\subsection{Determinateness}

\begin{itemize}
\item Indeterminacy: \(\xcancel{a}\) or \(\overset{\times}{a}\) - two
  strokes meaning untyped and undefined; an empty, unclassified symbol
\item Underdetermined: ??
\item Partially determined (i.e. typed): \(\cancel{a}_{\scriptscriptstyle A}\) (if \(\tj{a}{A}\), but \(a\) is not defined.)
\item Fully determinate: \(\overbar{\underbar{a}}\) (or: \fbox{\(a\)})
  -- typed and defined. In display environments like diagrams we may
  use the ground symbol; see (\ref{depgraph:coinduction}) for an
  example.
\end{itemize}

Possibly: three-sided box for underdetermined? Box suggests category,
so sym in open box suggests typing.

\subsection{Symbols}

TODO: use \(\supset\) for implication, \(\rightarrow\) for function.
(make Curry-Howard explicit)

\begin{itemize}
\item Equality: \(=\)
\item approx: \(\approx\)
\item sim: \(\sim\)
\item simeq: \(\simeq\)
\item cong: \(\cong\)
\item Eqdef: \(\eqdef\)
\item Coloneqq: \(\coloneqq\)
\item Questeq: \(\questeq\)
\item bumpeq: \(\bumpeq\)
\item Congruence: \(\equiv\) (\texttt{\textbackslash{}equiv})
\item Triangle eq: \(\triangleq\)
\item triangle: \(\triangle\)
\item triangledown: \(\triangledown\)
\item smalltriangledown: \(\smalltriangledown\)
\item medtriangledown: \(\medtriangledown\)
\item largetriangledown: \(\largetriangledown\)
\item Wedge eq: \(\wedgeq\)
\item owedge: \(\owedge\)
\item varowedge: \(\varowedge\)
\item ovee: \(\ovee\)
\item varovee: \(\varovee\)
%% \item Corresponds: \(\corresponds\)
\item Dot eq: \(\doteq\)
\item Circle eq: \(\circeq\)
\item Semantic brackets: \(\llbracket a\rrbracket\) refers to semantic
  object denoted by symbol ``a''. Useful for disambiguation. For
  example ``the structure of \(A\land B\)'' is ambiguous, since it
  could refer to either the syntactic or semantic structure. But ``the
  structure of \(\llbracket A\land B\rrbracket\) refers unambiguously
  to semantic structure.
\item Corner quotes: \(\ulcorner a\urcorner\) to disambiguate
  reference to syntax.
\item Syntactic entailment: \(\linfer\)
\item Particular entailment: \(\linfer_\alpha\) or
  \(\sststile{\alpha}{}\). E.g. \(A;B\sststile{\land}{}{A\land B}\).
  Another example: \(\Gamma\sststile{(a,b)}{=}{p:A\times B}\) for
  \(\Gamma\linfer p=(a,b):A\times B\).
\item Semantic entailment: \(\models\), \(\sdtstile{}{}\) (Ebbinghaus also uses this for the satisfaction relation)
\end{itemize}

