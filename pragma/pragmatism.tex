\section{Pragmatism}

MLTT is based on an epistemic notion of logic. The central concept is
knowledge.

\begin{displayquote}
And, when the relation between judgement, or assertion, if you prefer,
and knowledge is understood in this way, \textbf{logic itself is
  naturally understood as the theory of knowledge}, that is, of
demonstrative knowledge, Aristotle’s {\greekfont{ἐπιστὴμη ἀποδειχτιχή}}.
Thus logic studies, from an objective point of view, our pieces of
knowledge as they are organized in demonstrative science, or, if you
think about it from the act point of view, it studies our acts of
judging, or knowing, and how they are interrelated.
(\parencite{martin1996meanings} p. 20, emphasis added)
\end{displayquote}


Under a pragamatic perspective, propositional content is fixed by the
inferential role of propsitions. Rules of inference are fixed by
normative inferential practice. (They are implicit in our normative
practices). So the central notion is inference, not knowlege. The job
of logical vocabulary is to enable explicit expression of implicit
inferential practices. With this in hand, we have the tools we need to
do the traditional job of Logic: express and explain explain
\textit{logical} consequence.

