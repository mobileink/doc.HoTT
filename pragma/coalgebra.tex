\section{Algebra \& Coalgebra}\label{sec:coalgebra}

It's much easier to find algebraic/coalgebraic treatments of induction
and coinduction than it is to find logical treatments. Much
\textit{much} easier.

What such treatments lack (at least, all that I have found) is
\textit{explanation}. They do show how to recognize inductive and
coinductive definitions (by their algebraic properties), and they may
offer guidance as to how to write one's own inductive or coinductive
definitions, but they to not explain what they are. They do not
address the question of how or why reasoning by induction and
coinduction seems to work. That's understandable, given that such
explanation is generally not the goal.

A logical treatment -- specifically, an inferential treatment -- can
do more. By treating induction and coinduction as species of
reasoning, we can hope to uncover something more fundamental than the
algebraic and coalgebraic structures that characterize uses of such
reasoning.

For example, many texts start off by characterizing inductive and
coinductive definitions in terms of set inclusion:

\begin{displayquote}[\cite{Gordon94atutorial}, p. 1]
  To say a set is inductively defined just means it is the least
  solution of a certain form of inequation. For instance, the set of
  natural numbers \(\Nat\) is the least solution (ordered by set
  inclusion, \(\subseteq\)) of the inequation
  \begin{align}
    \{0\} \cup \{S(x)\ |\  x\in X\} \subseteq X.
  \end{align}

  ...

  Dually, a set is co-inductively defined if it is the greatest
  solution of a certain form of inequation. For instance, suppose that
  \(\rightsquigarrow\) is the reduction relation in a functional
  language. The set of divergent programs, \(\uparrow\), is the
  greatest solution of the inequation
  \begin{align}
    X \subseteq \{a\ |\ \exists b(a\rightsquigarrow b \& b\in X)\}
  \end{align}
\end{displayquote}

\vspace{1ex}

They go on to offer definitions based on fixed points:

\begin{displayquote}[\textit{ibid.}, p. 3,4]
  Let \(U\) be some universal set and \(F :
  \wp(U)\rightarrow\wp(U)\) be a monotone function (that is,
  \(F(X)\subseteq F(Y)\) whenever \(X\subseteq Y\)). Induction and
  co-induction are dual proof principles that derive from the
  definition of a set to be the least or greatest solution,
  respectively, of equations of the form \(X = F (X)\).
  \vspace{1ex}

  [...definition of \(\mu X.F(X)\) and \(\nu X.F(X)\) omitted for brevity...]
  \vspace{1ex}

  We say that \(\mu X.F(X)\), the least solution of \(X = F(X)\), is
  the set \textbf{inductively defined} by \(F\) , and dually, that
  \(\nu X.F(X)\), the greatest solution of \(X = F(X)\), is the set
  \textbf{co-inductively defined} by \(F\).
\end{displayquote}

\vspace{2ex}

\blockquote[\cite{sangiorgi2011introduction}, p. 28]{The central ingredient for our explanation of induction and coinduction is the Fixed Point Theorem, which says that monotone functions in complete lattices have a least and a greatest fixed point.}

[Note the problem with the fixed point perspective: it depends not
  only on the non-trivial concept of fixed points, but also on the
  concepts ``complete lattice'' and ``monotone function''.]

Chapter 2 of \citetitle{sangiorgi2011introduction} gives no fewer than
six different characterizations of inductively and coinductively
defined sets.

They're not all technical, though. Another good tutorial offers this:

\blockquote[\cite{Jacobs1997ATO}, p. 7]{In an \textit{inductive definition} of a function \(f\), one defines the value of \(f\) on all constructors.

In a \textit{coinductive definition} of a function \(f\), one defines
  the values of all destructors on each outcome of \(f\).}

Many authors include a treatment based on the concept ``polynomial
functor'' from Category Theory. Another tutorial

\paragraph{Resources}

\begin{itemize}
\item \citetitle{Jacobs1997ATO} \cite{Jacobs1997ATO}
\item \citetitle{2017intro_coalgebra} \cite{2017intro_coalgebra}
\item \citetitle{sangiorgi2011introduction} \cite{sangiorgi2011introduction}
\item \citetitle{sangiorgi2011advanced} \cite{sangiorgi2011advanced}
\item \citetitle{Gordon94atutorial} \cite{Gordon94atutorial}
\item \citetitle{rutten_uni_coalgebra} \cite{rutten_uni_coalgebra}
\end{itemize}

