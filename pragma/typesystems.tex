\section{Type Systems}\label{sec:typesystems}

\subsection{Typing ``judgments''}

The standard way to express ``a has type A'' is \(\tj{a}{A}\). But that is
not the only way. Some authors write the type symbol as a superscript:
\(a^A\). Another way would be to harmonize with the sequent calculus
and write \(a\linfer A\).

We could just define \(\tj{a}{A}\eqdef a\linfer A\). Then what would our
rules look like?

%% \AxiomC{$A\seqso a$}\AxiomC{$B\seqso b$}\BinaryInfC{$A,B\seqso (a,b)$} \DisplayProof

One problem is we would need one structure connector on the LHS for
each logical connector. In this case \(A,B\) would have to be read
\(A\times B\); for disjunction it would have to be \(A\plus B\). In
other words our structures would have to be replaced with types. I'm
not so sure that would work, but it might. From context of a
conjunction of propositions to a product of types?

%% \AxiomC{$A\seqso a$}\AxiomC{$B\seqso b$}\BinaryInfC{$A\times B\seqso (a,b)$} \DisplayProof

If this works out, we would get a notation that is dual to the
standard one. If so it should be easy.

The problem with \(a\linfer A\) is that it severs the link between type
and token. If we have multiple types and tokens, it would be
impossible to see which tokens have which types.

But as an explanatory device \(a\lso A\eqdef \tj{a}{A}\) works rather
well. Then reasoning from premises of that form to conclusion that
form looks like sequent reasoning. So if our conclusion is
\((a,b):A\times B\), we get \( (a,b)\lso A\times B\).

Even better: \(a\lso A\pso\tj{a}{A}\) or \(a\lso A\Vdash\tj{a}{A}\) or
\(a\lso A\,\Vert\,\tj{a}{A}\). Here the second inference op corresponds
the the horizontal rule in a sequent proof.

\subsection{Propositions: free-standing and embedded}

Is the concept ``proposition'' primitive?  It is for Brandom.

Do calculi presuppose a semantic domain of propositions? Not
necessarily. ML's talk about A prop first then A true tries to address
this. If we do not want a representationalist logic, where
propositions are supplied from outside, then we need to account for
them in some other way. For Brandom they would be presupposed by the
very ability to reason and vice-versa, since they are instituted by
normative inferential practices. The way out of the apparent
circularity is to appeal to practice.

ML does not have such a refined theory, he makes A prop a judgment we
have to make before we can get to A true, but his account of that kind
of judgment is not very convincing. He does seem broadly within the
pragmatist current, though.

In other words, A prop is ML's way of bootstrapping an uninterpreted
calculus into a meaningful language of logic.

Brandom: proposition is fundamental unit of accountability. Same for
logic. You can use a calculus to derive formulae, but you are not
reasoning if no propositions are involved.

The logical constants come to have meaning in virtual of their rules
of use. The task is to account for the meanings of the non-logical
symbols. If they are to be propositions, must they be supplied by some
external source? Well yes, but that source can be the same set of
normative practices that provide the fuel for the logical constants.

The status of ``proposition'' is a fundamental issue.

Bifurcation Principle: propositions have two ``roles'', free-standing
and ingredient (as premise or conclusion of a proof). Propositional
content is the same.

Martin-Lof's ``judgment'' stuff tries to reconcile these?


\subsection{Proofs}


 \(a:A\) is true (horizontally) iff \(a:A\) is
proven (vertially).

 \(a:A\) is true as a free-standing proposition iff \(a:A\) is the
 conclusion of a valid proof (ingredient sense).

Or \(a:A\) is true iff it is the end sequent of a proof (object).

Or \(a:A\) is true iff it is correctly constructed.


This bifurcation of proof/truth is what motivated ML to develop the
distinction between propositions and judgments.

\subsection{Notes}
A theory of types should explain types. But type calculi are logical
calculi. Since logic is the science of proof and consequence, ...

We use type systems to prove things. What kind of things? The
conclusion of a proof always has the form \(p:P\). What kind of thing
is that? We can interpret it as a proposition, glossed as ``p is a
proof of P''. Or we can treat it as an inference and gloss it as ``p
entails P''. ML calls it a ``judgment'', or ``form of judgment''.

If we treat it as a proposition, then all proofs in a type system are
meta-proofs. They prove a proof claim. This is starkly different from
untyped calculi, which place no substanstial constraints on the
propositions they prove.

Maybe that's why ML felt the need to call them judgments instead of
propositions.

What is a proof? We can prove that a proposition is true; can we also
prove that an inference is valid? We can certainly \textit{claim} that
an inference is valid; that's what inference rules do. A proof of a
proposition proves its conclusion, but it does not prove its own
validity. If an inference step is licensed by an inference rule, we
take that as proof of the validity of that step. But that does not
give us a proof \textit{object}; if we think of a proof as a tree or
chain of inferences, then the justification of an inference step by
reference to an inference rule cannot count as a proof. To do that we
would have to write a meta-proof that displays the inference step
itself as the conclusion of a proof that starts from the inference
rule.  That seems like a tall order.

On the other hand, if we take \(a:A\) as an inference, then a proof in
a type calculus \textit{does} prove the validity of an inference. The
inference steps in the proof are themselves meta-inferences; they go
from inference to inference. So again, a proof in a type system is
essentially a meta-proof.

But that would also mean that \(a:A\) is not a judgment.

Of course, this is based on the propositions-as-types interpretation.
Even if we treat propositions as types, we are not thereby compelled
to treat tokens of such types as proofs of a proposition. We can just
say that they are tokenings and leave it at that. After all, when we
say that 3 is a token of type Nat, we don't ordinarily think of it as
proof, and we certainly do not thing of Nat as a proposition.

Then a proof in a type system would prove a ``tokening'', and we could
gloss \(a:A\) as ``a is a token(ing) of A''.

If a proposition is a type, then what is a token of such a type, if
not a proof? For example, if the proposition is \(2<3\), we might
express its type as something like \(T_{2<3}\), or \(\overline{2<3}\).
In other words, we could come up with the equivalent of the
\(\lambda\) operator. The latter turns an open formula into the name
of a function; our new operator would turn it into the name of a type.
E.g. \(p: \overline{2+2=4}\).  But we already have equality types!

What about something like ``every n is odd or even''? Or just a
complex logical expression?

If propositions are types, then we can think of the proposition's
formula as the name of the type?

Consider the intro rule for products. The conclusion is
\((a,b):A\times B\). If we read this as \((a,b)\) is proof of
\(A\times B\), then something is off, because what we have proven
directly is the type inference \((a,b):A\times B\). The proof is the
entire proof tree, and that gets forgotten if we treat \((a,b)\) as
the (principle) proof. So there's an issue of ``proof objects''
involved. Why should we treat \((a,b)\) as a proof object, when proofs
are trees?

IOW, Curry-Howard induces an inconsistency that goes beyond mere
terminology. On the one hand, our proof objects are trees; on the
other hand, what our proof-trees prove is that a token, which is not a
tree, is a proof.

Curry-Howard is based on calculi? It says something about formal
representations, from which we infer it says something about the real
stuff. Propositions and types end up looking the same in the calculi,
so we infer they are the same. And that's probably based on the
isomorphism between implications and functions. It's harder to see if
you start with equations.

Remember: propositions-as-types means \textit{logical} propositions.
Mathematical or other propositions must be first converted to logical
form.

BTW, proof-trees also construct (and thus ``prove'') the type part of
the concluding inference.

NB: \(A\land B\) is a logical proposition; typed, it is a product.
Suggesting we can read \(A\times B\) as a proposition. But proof of
\(A\land B\) is a tree, whereas a ``proof'' of \(A\times B\) is a
token, \((a,b)\) which is not a tree. To see it as a proof, we have to
view it as representing a proof-tree.

We're forced to adopt two notions of proof, one for proof trees, and
one for tokens. There's an epistemic/cognitive difference. A token of
a type is ipso facto a kind of non-demonstrative proof of the type.

But proof-as-program must refer to proof-trees?

The ``proofs'' in proofs as programs are meta-proofs; their
conclusions are the type ``judgments'' saying the token instantiaes
(``proves'') the type. So it should be ``typing metaproofs as
programs''.

\subsection{Martin-Lof}

\textquote[\cite{psh_judgments_martin_lof} 494-5]{According to
  Martin-Löf, [propositional logic] does not deal with
  ``propositions'' which are given as a domain of discourse from
  outside. Whether a closed expression is a proposition is something
  that is to be established within the theory... Therefore
  Martin-Löf's theory distinguishes two forms of categorical
  judgments, A is a proposition (A prop) and A is true (A true) which
  are explained in such a way that the latter presupposes the former.}

But doesn't ``closed expression'' already presuppose denotation?

And doesn't truth always presuppose proposition?

\textquote[\cite{psh_judgments_martin_lof} 495]{To know A prop
  means to know what one must do in orfer to verify A, i.e. what
  counts as a verification of A. So if I have grasped what a
  verification of A looks like, I have proved A prop.}

This seems preposterous. If it were so, we would be unable to reason
about conjectures, for example, by assuming them true and following
out the consequences.

HoTT p. 18: \enquote{Note that the judgment “A has a proof” exists at
  a different level from the proposition A itself, which is an
  internal statement of the theory.} This notion of ``internal
statement of the theory'' seems to reflect the notion that we cannot
be ``given'' propositions from the outside to serve as denotatums; we
have to construct them somehow within the theory. I don't think that
works very well. Anyway the difference between \(a:A\) and \(A\) is
pretty obvious, both syntactically and semantically. But why are we
compelled to think that ``the proposition A itself'' is ``an internal
satement of the theory''? This seems to be trying to make fine
metaphysical distinctions. Propositions are types, so what does it
mean to say that a type is ``an internal statement of the theory''?

What

Brandom: propositions are primitive and articulated and indeed
instituted inferentially. There's no sense in which propisitions could
form an ``external'' domain for reasoning, and no need for a judgment
A prop in order to support A true. ``A true'' is just another way of
saying ``A'' (prosentential account of ``is true'').

\subsubsection{Notes}

Kant's concept of judgement: ML makes it out to be epistemic (or
doxastic). Brandom makes it out to be deontic.

\paragraph{\textit{On the meanings of the logical constants and the justifications of the logical laws} (\parencite{martin1996meanings})
  \newline}

This paper goes off the rails almost immediately, when it claims that
the introduction rule for logical conjuction,
%% \AxiomC{$A$}\AxiomC{$B$}\BinaryInfC{$A\& B$} \DisplayProof
``...takes us from the affirmation of \(A\) and the affirmation of \(B\) to
the affirmation of \(A\&B\)...''. He later hedges a bit, using
``assertion'' or ``judgment'' instead of ``affirmation''. But the
claim is patently false, for all three terms.

First, introduction rules, like all rules, are conditionals, and
conditionals assert neither their premises nor their conclusion. ``If
P then Q'' does not assert either P or Q. So the premises and
conclusions of rules cannot be affirmations (or assertions or
judgments).

Less obviously, introduction rules use propositional variables (like
\(A, B, P, Q\), etc.) that range over propositions. But it is a
category mistake to affirm or assert a propositional variable; we can
only assert propositions, and a variable is not a proposition. If
we replace the propositional variables in a rule with particular
propositions, we get a particular proposition, not a rule. Going from
\(A; B \linfer A\land B\) to ``Snow is white and also roses are red, so
both snow is white and roses are red'' is a proposition; to assert it
is \textit{ipso facto} to assert its component subpropositions. But it
is not a rule.  Nor is it a conditional.

\paragraph{\textit{Truth of a proposition, evidence of a judgement, validity of a proof}}

\begin{displayquote}
  My answer to the questions, What is a judgement? and, What is a proof of a judgement? is simply that a proof of a judgement is an act of knowing and that the judgement which it proves is the object of that act of knowing, that is, an object of knowledge.
  \parencite{martin1987truth} 417
\end{displayquote}

Of course, this does not tell us what a judgment \textit{is}, it just
tells us that it is something we can know.

But he does tell us, on p. 409, that ``A is true'' is a judgment, in
which A is a proposition. That would make not A but ``A is true'' an
object of knowledge. Evidently the idea is that the truth of A is the
object of knowledge? Can we ``know'' just A? It doesn't make sense to
say that we know a proposition; we can only know \textit{that} it is
true or false.

The entire thing falls apart if the premises and conclusion are not
judgments. What a muddle!

\begin{displayquote}
  [T]he proof of a judgement is nothing but the act of knowing, or,
  perhaps better, the act of understanding or grasping, and that what
  you grasp, namely, the object of knowledge, is the same as what you
  prove, namely, the assertion or judgement. \parencite{martin1987truth}
  417
\end{displayquote}

Needless to say, this is at odds with the proof-theoretic approach
that treats a proof as an object. He's effectively just playing with
words here; calling proof an ``act of knowing'' just avoids the
question of what is a proof. It's kind of a meta-claim, that by
recognising that a purported proof does in fact prove its conclusion
puts you into a state of knowing. But that's not saying much about
what a proof is, beyond ``it proves something''.

\medskip

Other problems:

Illocutionary force. He uses it (incorrectly) to explain ``I know'',
but does not seem to realize that illocutionary force is precisely
what distinguishes assertion from interrogation, command and the other
various kinds of ``speech act''. You cannot write down an assertion;
you can only write down a sentence, and count on your reader to
observe the (universal?) norm that declares a written sentence should
be granted (by the reader) assertional force.

Judgment and ``evident judgment''. Very muddled.

The source of the troubles would seem to be the perceived need to cast
logic as an essentially epistemic enterprise. Hence the presentation
of judgment etc. in terms of knowledge. But knowledge in the sense of
being in possession of some kind of abstract knowledge thing, or
having some kind of special ``knowing'' property, really has very
little to do with it. Logic is a matter of mastery of normative
practices. You can call that ``knowing'' \textit{how to do} things in
the logic game, but that's just a way of speaking, as when we say that
somebody who has mastered chess ``knows'' how to play chess. It's
practical mastery that matters.. It's not psychology, and it will not
do to \textit{explain} logic in terms of knowledge. You just end up
going in circles. To know something is ... to know that it is true.

Assertion: plays no role in logic, although it does play a role in
metalogic. That is, we do assert that our inference rules (as
propositions) are true (valid), but we need not assert that any
formulas in our proofs are true. We just need to make sure our
inferences/proofs are valid, and that does not require assertion of
propositional forms.

A proof is a kind of conditional assertion license - if it is valid,
then it licenses assertion of its conclusion \textit{provided that}
its premises are true (we're entitled to assert them).

So there are no judgments in logic.

No wait. It depends on how we think of a proof. If we think of it as
an instance of rules, then it is a proposition that is not a
conditional. Of the form ``A and B ... therefore C''. Then the
horizontal line in the steps means not ``implies'' or ``entails'' but
``therefore''. This is evident in modus ponens. Schematically, stated
as a rule, we have ``If \(A\rightarrow B\) and A then \(B\)''.
Asserting this does not assert A or B (nor the implication). But if we
instantiate it we get ````\(A\rightarrow B\) and A, therefore \(B\)'',
which asserts both A and B (and \(A\rightarrow B\)).

But we do not reason with asserted propositions; we reason with
assumptions and rules. To use a rule in a proof is not to instantiate
it. Is it? Rules legitimize inferences, they do not assert the
components of the inferences. We can think of the uses of a rule as
construction of another rule. We always work schematically, so when we
build a proof using the inference rules what we do is create another
schema. All the metavariables remain metavariables, and the inferences
therefore remain general (the do not become particular
``therefores''). We only instantiate it when we apply it to concrete
premises.

After all, when we use a rule in a proof, it looks just like the rule.
We just copy the rule into the proof, so to speak. It remains schematic.

Were that not the case, then the inference line would be overloaded.
It would mean licensed inference in the inference rules, but actual
inference in proofs.

Plus, to reason we would have to continually go back and forth between
the particular propositions and inferences in our proof and the
generalities of the inference rules to see what applies. That would be
a very unnatural kind of deduction. Who does it? It's much more
efficient to reason entirely in terms of generalities - rules and
metavariables.

All inference rules are implicitly universally quantified. We can
express this in two ways. On is to say ``for all propositions A, B,
etc...''. But that would be cheating, since we have not yet defined
``for all''. The other way is to say ``for arbitrary propositions A,
B, ...etc.'' That's a little bit better, since we do not have a formal
way to say ``for arbitrary''. By using it we implicitly acknowledge
that the quantification is of an informal, undefined sort.

\paragraph{HoTT Book (\parencite{hottbook})\newline}

p. 20:
\begin{displayquote}
``Judgments may depend on assumptions of the form x :
A, where x is a variable and A is a type... Note that under this
meaning of the word assumption, we can assume a propositional equality
(by assuming a variable p : x = y), but we cannot assume a judgmental
equality x ≡ y, since it is not a type that can have an element.''
\end{displayquote}

But this is plainly false, or at least inconsistent with the preceding
discussion, which lists ``x ≡ y : A'' (``judgemental equality'') as a
primitive form of judgment. And ``x ≡ y'' is an abbreviation for ``x ≡
y : A''. If we can assume one primitive form of judgment, a:A, why can
we not assume the other, a=b:A? It is true that we cannot assume x ≡
y, but that is because it is syntactically ill-formed (it is not a
judgmental equality).

This claim that ``we cannot assume a judgmental equality x ≡ y'' is
also contradicted on page 19, where we have `The best way to represent
this is with a rule saying that given the judgments a : A and A ≡ B,
we may derive the judgment a : B.''. But ``given'' is just another way
of saying ``assuming'', so this says that we can assume A ≡ B. We're
not told what A ≡ B means; presumably it abbreviates A ≡ B : U, but in
any case if it is not a judgmental equality, then we have no
indication of what it is.

Same page:

\begin{displayquote}
``By the same token, we cannot prove a judgmental equality
either, since it is not a type in which we can exhibit a witness.''
\end{displayquote}

But this too must be false. If we can prove \(a:A\) then why not
\(a≡b:A\)? They're both judgments.

In both cases the text seems to be making a fundamental error, namely
it forgets that x ≡ y is an abbreviation for x ≡ y :A. By itself, x ≡
y is the \textit{premise} (or \textit{antecedent}, if you prefer to
think of judgments as sequents) of the judgment x ≡ y :A, just as a is
the premise of a:A. And since the ``unit of reasoning'' so to speak,
is the judgment, we can neither assume nor prove premises alone. But
if that's the intended meaning, then it is a category error to call x
≡ y a judgment.

HoTT book says judgmental equality is definitional. But is it really a
good idea to treat judgments and definitions as the same thing?

Martin-Loff makes a distinction. His ``definitional equality'' is
purely syntactic and not the same as a=b:A.

We can treat \(=\) the same way we treat logical operators like
\(\&\), on the the principle that logical operators recapitulate the
(prelogical) structure of the premises. Example: \(\&-intro\).
Premises are the structure \(A ; B\), read ``A and also B''; this
structural operator \(;\) expresses the prelogical concept of
\textit{and}. From this we infer \(A \& B\). Glossing: the logical
combination \(A\&B\) expresses the prelogical combination \(A;B\).
Reversing direction, from \(A\&B\) we can infer \(A\) and also we can
infer \(B\); that is, the prelogical combination \(A;B\) follows from
the logical combination \(A\&B\). So the entry/exit rules for \(\&\)
serve to bridge the prelogical and the logical. (Caveat: \(A;B\) is
formal, structured syntax, but it expresses prelogical rather than
logical concepts.)

This justifies entry/exit nomenclature, instead of intro/elim. The
latter idiom describes syntactic operations; but what the rules
express is entry and exit transitions between prelogic and logic. The
entry rule expresses the transition from a prelogical notion of
aggregation to a logical notion of conjunction; the exit rule expresses
the reverse transition.

Compare Sellars' concepts of language entries and exits.

We can the same for the equality operator. First we need to unpack \(x
≡ y : A\), which gives us two simple judgments \(x:A\) and \(y:A\), a
structural combinator, and something that expresses the concept that x
and y are (prelogically) equal. The aggregation operator \(;\) used
above for \(\&\) expresses simple aggregation of one thing \textit{and
  also} another thing, or one thing \textit{together with} another thing.
But for \(=\) mere aggregation of premises is not sufficient; we need
an additional notion of equality, which moreover must be a prelogical
concept. That is, our ``ordinary'', intuitive notion of equality.
We'll use the traditional ``such that'' notation, which captures the
idea that we have an aggregation under a constraint. So instead of \(x
≡ y : A\) we will write \(x:A ; y:A | x≡y\), which we gloss ``x:A and
also y:A such that x equals y''. (or: and also the assumption that
x=y). The point of this is just to express the conceptual structure
more conspicuously. The problem is that this seems to have the wrong
form.

Alternatively, we could express the equality as an assumption rather
than a constraint. Might be better, since then the inference to
logical equality would discharge the assumption. I.e. the explicit
equality \( =_A(a,b) \) recapitulates (and discharges) the prelogical
assumption that a equals b.

[Note that \(;\), \(|\), and \(≡\) are metasymbols.]

From this we can infer logical (or at least quasi-logical) equality;
the conclusion is \( =_A(x,y) : Id_A(x,y) \).

Glossing: from the prelogical aggregation of x:A and also y:A, under
the constraint that they are (prelogically) equal, we can infer the
(logical) eq-junction of x and y as a token of the (logical) identity
combination type \(Id_A(x,y)\).

This gives us a primitive binary constructor \( =_A(\ ,\ ) \)

Just as with \(\&\), we can reverse this, and infer the ``equality
judgment'': from \( p: Id_A(x,y) \) we can infer \(x ≡ y : A\). This
is exactly analogous to \(\&\) (i.e. product types): [todo...]

Untyped:  from assumption \(x, y | x≡y \) infer \(x = y\).

We're reading '=' as a kind of constrained combinator: x together with
y under constraint = (prelogically) entails equality combination of x
and y.

HoTT p. 20: ``By the same token, we cannot prove a judgmental equality
either, since it is not a type in which we can exhibit a witness.''
But this too must be false. If we can prove \(a:A\) then why not
\(a=b:A\)? They're both judgments. The trick is to discard the idea
that judgmental equality is definitional. The ``such that equality''
is a constraint, not a definition. We cannot assume definitions, but
we can assume judgments, including equality judgments. Actually


\subsection{HoTT}

The claim is that equality types are inductively defined. Where's the
induction?

\textquote[\cite{Hintikka1992-HINTCO}]{[I]nduction means, in the first
  place, inference from particular cases to general truths and,
  secondarily, inference from particular cases to other particular
  cases... If inferences from particulars to particulars satisfy
  certain conditions, the principles according to which they are made
  are logically equivalent to principles governing inferences from
  particulars to generalizations.}

He disregards the second since it is equivalent to the first.

Mathematical induction reasons from two particular cases: the base
case and the inductive case. The inductive case is itself an inference
from one particular case (i.e. arbitrary n) to another particular case
(n+1). Coinduction reasons only from particular to particulars (e.g.
hd and tl of an arbitrary infinite list). In both cases, the inference
goes to the general (universal quantification). This is compatible
with Hintikka's characterization of induction.

How can we account for inductive inference pragmatically, under an
inferentialist expressivist order of explanation? Our (first order)
logical calculi do not have formal rules for inductive inference. But
we do make such inferences, not only in mathematics but also in
defining the syntax of our logical calculi. So our formalized
reasonings presuppose inductive reasoning. Maybe our inference rules
do too, since they express ``general truths''.

So our question translates to: once we have instituted \(\rightarrow\)
or \(\linfer\) for particular cases (like Pittsburgh-Princeton), how do
we get to a general rule, e.g. \(P\rightarrow Q\) where P and Q are
metavariables? \(P_{\alpha}\linfer Q_{\beta}\) is the
\textit{vernacular} expression of the correctness of a
\textit{particular} inference; \(P\rightarrow Q\) is the
\textit{logical} expression of the correctness of a \textit{general}
rule. Getting from the former to the latter involves two transitions,
one from the vernacular to the logical language, and one from a
particular case to a general rule. How can we explain this
pragmatically?

It looks like we're compelled to think that some kind of inductive
reasoning is implicitly at work. Intuitionistic logic makes it
explicit: to justify the rule, one must demonstrate a particular case,
that is, one must pick an arbitrary (but particular) case P and show
that it entails Q. We can express this as the
\(\rightarrow\scriptstyle{\text{-intro}}\) rule. But that rule is
again a general rule, so we have not yet explained how we get from
particular to universal. After all ``arbitrary but particular'' is
itself a generalization, which we express using a metavariable and
brackets.

Does the generality of inference rules presuppose inductive reasoning?
Only if there is no other way to generalize. But e.g. lambda
abstraction does not seem to involve induction. A lambda abstraction
is a kind of general rule that is not justified by induction.

But \(\rightarrow\scriptstyle{\text{-intro}}\) does involve induction.
Maybe the general principle is that any rule that starts with an
assumption uses induction. Since assumption means ``arbitrary
particular case''. But then, all rules, being conditionals, start with
an assumption, in a sense: ``if the premises are true'' means ``if
they are true for an arbitrary particular case'', then the conclusion
follows for that case. But that's the inverse of induction, it goes
from universal to particular. Deduction, dual to induction.

But the rules must be instituted by induction.

This makes ``assumption'' the key to induction. Assume a particular
case, show another particular case follows, conclude a general rule.
So our ability to generalize by induction presupposes the ability to
assume a particular case.

Nat and for all n:Nat are generalizations; ``arbitrary n:Nat'' is a
particular. But it is an indeterminate particular. No, it's
determinately a Nat. Arbitrary means arbitrary, not indeterminate. But
its also an assumption, so in fact it is not a particular in hand.

We can express this as a counterfactual: ``if n were an arbitrary Nat,
then ...''. This is a counterfactual, because n is not a Nat, its an
unbound variable, or at least a variable not bound to a particular
value (variables bound by a quantifier are not bound to particular
values, they're bound to the quantified parameter; iow ``bound'' means
``not free'', as opposed to ``bound to a particular value'', that is
what allows its meaning to be overloaded.).

The concept ``arbitrary particular'' would seem to implicate an axiom
or principle of choice. It must be at least possible to settle on an
arbitrary particular. What is the pragmatic account of the Axiom of
Choice? Pragmatic does not mean intuitionistic.

Maybe we can just say that the rule
expresses the inductive inference of which it is the conclusion. Just
like \(apple\land orange\) expresses the conclusion of an
\(\land\scriptstyle{\text{-intro}}\) rule.

%% \bookmark[named=CH,level=0]{Curry-Howard}
\section{Curry-Howard}\label{sec:curry-howard}

First a caveat: the Curry-Howard isomorphism is often referred to as
``propositions-as-types''. This is incorrect; it is not particular
propositions that are types, but propositional formulae.

It's very easy to see the relation between propositions (that is,
formulae) and types even in a minimal \textit{untyped} calculus for
first-order logic. Take for example the introduction rule for
conjunction, which would look something like \(A, B\seqso A\lkand B\).
The standard intended interpretation of such formulae is
propositional: we are to treat \(A\) and \(B\) as meta-variables
ranging over propositions, and we implicitly add ``...is true''.

The standard intended interpretation in terms of propositions and
truth is so common and so easily understood that it is easy to forget
that it is optional, and to think that such calculi are
\textit{essentially} about reasoning with propositions and truth. But
the standard \textit{intended} interpretation is not the only one
possible. We can draw on other semantic domains to provide
interpretations; in particular, we can use the world of types and
tokens.

Under a types-and-tokens interpretation, we have two options. We can
use types only, and gloss \(A, B\seqso A\lkand B\) as ``If A is a type
and B is a type, then A\lkand B is a type''. This is straightforward.

But we can also use tokens as our semantic domain; then we would gloss
the same formula as ``If A is a token and B is a token then A\lkand B
is a token''. This too is straighforward but not very useful. The
obvious problem is it omits all information about types.

In both cases, we also need to provide an alternate interpretation for
\(lkand\). Instead of ``conjunction of propositions'', we use
``product of types'' and ``pair of tokens''.

We normally need both types and tokens, so the usual practice is to
merge these to interpretations and add some new syntax, such as the
standard \(a:A\) notation.

What we need is an interpretation (and corresponding notation) that
marries types and tokens. The usual strategy is to use special syntax
such as \(a:A\) to express ``a is a token of type A''. We're not
compelled to us such syntax. We could also us superscripting and write
\(a^A\), for example. But we could also avoid such type annotations
and settle on a particular convention of interpretation. For example,
we can stipulate that \(A\) is to be read as ''arbitrary (and unnamed)
token of type A'' - leaving implicit what is made explicit by \(a:A\).

The difference between \(a:A\) and ``arbitrary token of type A'' is
just the name symbol \(a\). We can make this explicit be defining a
type abstraction operator analogous to the lambda abstraction
operator.

Under propositional semantics, \(\choice A\) means ``arbitrary
proposition A'', i.e. ``let the symbol A designate an arbitrary
proposition.''  The symbol \(A\) names the chosen proposition.

\subsection{Type Abstraction}

We need a type abstraction operator analogous to the \(\lambda\)
operator.

Why? Its what we need to provide an integrated about of types as sets
and types as propositions. The problem is that propositions are
particulars, so they cannot be types. What ``propositions as types''
really means is that each proposition is associated with a type whose
members are proofs of the proposition. We cannot express this concept
with standard notation. All we have is \(p:P\), interpreted as ``\(p\)
is a proof of proposition (type) \(P\)''. But if \(P\) is a
proposition, it cannot be a type. So we need a notation that allows us
to explicitly say ``the type associated with the proposition \(P\)''.

Not ``propositions as types'', but ``propositions as proof-types''.

There are two kinds of abstraction involved. We can abstract directly
from tokens to types, and we can abstract indirectly from token
construction rules to a type.

Example: \(\Nat\). The definition of the type \(\Nat\) is given by two
construction rules. Informally, \(\Nat = ...\). But we need to make a
distinction between the symbol \(\Nat\) and the type, just as be make
a distinction between a function name and the function it names.

For functions, we can \textit{determine} functions without naming
them, and this is what allows us to name them. To make \(f\) the name
of a function, we write \(f\defeq\lambda x.M\). This means: ``\(f\)
names the function determined by the lambda expression \(\lambda
x.M\)''. Without lambda, we would have no way of naming functions.

Similar considerations apply to types. We cannot name a (determinate)
type unless we have a type abstraction operator, call it \(\tau\).
Then we would have \(T\defeq \tau.\textsf{expr}\), meaning ``\(T\)
names the type determined by the tau expression
\(\tau.\textsf{expr}\)''.  Let's try this with \(\Nat\):
\[\Nat\defeq\tau.\tj{\ZNat}{\Nat}, \tj{n}{\Nat}\linfer\tj{Sn}{\Nat}\]

However, this is a recursive definition. To eliminate recursion, we need
to be able to replace \(\Nat\) on the RHS with something that means
``the type under definition''.  Howsabout:
\[\tau x.\tj{\ZNat}{x}, \tj{n}{x}\linfer\tj{Sn}{x}\]

Here we use the bound variable \(x\) to mean not ``for all'', as is
the case with lambda, but something like ``the type.'' A kind of
pronominal, or Russell's iota.

A \(\lambda\)-bound variable ranges over the domain of the function. A
lambda expression like \(\lambda x.M\) may be glossed ``the function
that ...''

But wait, we don't need a bound variable. We're defining a type, all
we need is a way to say ``the type we're defining'', and we can just
use \(\tau\) for that. On the RHS it means ``it'', referring to ``the
type''.
\[\tau.\tj{\ZNat}{\tau}, \tj{n}{\tau}\linfer\tj{Sn}{\tau}\]

That gives us the desired ``anonymous'' type. It determines the type
whose members are constructed by \ZNat and \SNat. Now we can give it a
name:
\[\Nat\defeq\tau.\tj{\ZNat}{\tau}, \tj{n}{\tau}\linfer\tj{Sn}{\tau}\]

which we gloss ``\(\Nat\) names the type whose members are constructed
by \ZNat and \SNat''.

\(\tau\) is a kind of \textit{indirect abstraction} operator for
types. It abstracts over the constructors and thus indirectly over the
tokens. This is a different kind of abstraction than the ordinary
token-to-type abstraction, which we can call \textit{token
  abstraction}, because it generalizes over tokens. We see a token
instance like ``the'' and abstract to the associated word-type (which
we name ``the \(\ulcorner the\urcorner\) type'').

So much for indirect abstraction from token construction rules to
types. What about direct token abstraction? We should be able to
abstract from a particular natural number to its type \(\Nat\).

For starters, we can define a kind of naive token abstraction operator
that expresses the type of any token. For example, using
superscription, we might say that \(2^{\tau}\) determines the type of
which \(2\) is a member. But that would be a pretty indeterminate
type, because we would have no way of determining any other tokens of
the type. Furthermore, each particular determines what we will call a
``homoiconic'' type, for lack of a better term, whose tokens are
``instances'' similar to the particular. In the case of \(2^{\tau}\),
that would mean the type whose members include all instances of the
figure \(\ulcorner 2\urcorner\) in the document you are reading; there
are three such instances in this very paragraph.

\subsection{Propositional Type Abstraction}

Now since propositions are particulars, we need a we to abstract from
a proposition to the type of its proofs.

We need something like \(\tau.(a+b=b+a)\) to mean ``the type whose tokens are proof of the proposition \(a+b=b+a\)''.  Maybe:
\[\tau.(a+b=b+a)^{\Pi}\]

But to integrate with set-ish types like \(\Nat\), we need a more
specific notion of constructors. \(\Nat\) is determined by the
specific constructors express in its \(\tau\)-expression. For
propositions, the constructors are all the constructors, primitive and
derived, in the type logic. So if we call the logic \textsf{\slshape LJ}:
\[\tau.(a+b=b+a)^{\textsf{\slshape LJ}}\]

This gives us a kind of exponentiation that dovetails with that used
in category theory, where \(A^B\) may be taken to mean ``functions
from \(B\) to \(A\)''. So \(\tau\) would give us ``the type whose
tokens are constructed \textit{from} the rules of \textsf{\slshape
  LJ}''. But since the logic and rules will always be understood, we
can use \(\Pi\) instead of naming the logic:
\[\tau.(a+b=b+a)^{\Pi}\]

Now we can directly express the relations between propositions,
proofs, and types:
\[P\defeq\tau.(a+b=b+a)^{\Pi}\]
so \(\tj{p}{P}\) means that \(p\) is a token of the type of proofs of
the proposition \(a+b=b+a\).

Note that a proposition is not an element of its associated proof
type.

We can use this with set-ish tokens too; we just need to restrict the
exponent, so we get ``the type whose tokens constructed from the rules
for some type.''  E.g.
\[\tau.2^{\Nat}\text{, or } 2^{\tau\Nat}\]

would be glossed ``the type associated with \(2\) whose elements are
constructed from the rules of type \(\Nat\)''.

Of course, since we already know the type of \(2\) is \(\Nat\), we
could write \(2^{\tau}\), meaning ``the type whose rules were used to
determine \(2\)'', i.e. \(\Nat\).

Note use of ``determine'' instead of ``construct''. That's because the
tokens of cotypes (codefined, by coinduction) do not have
constructors.

Every type has a set of rules and corules that jointly determine the
tokens of the type; and every token has a type. So superscript
\(\_^{\Pi}\) applies to particular tokens always refers to the rules
and corules of its type. For propositions this means the entire set of
types.

\subsection{Token Promotion and Type Demotion}

The token promotion operator is \(\tau\).  It is analogous to the
lambda abstraction operator. The idea is that it allows us to refer
to the type of a token when we do not have a proper name for the type.
So for any token \(a\) the expression \(\tau a\) means ``\textit{the} type of
token \(a\)''; in a typed calculus: \(a: \tau a\).

The type demotion operator is \(\choice\); it is analogous to
``application'' in the lambda calculus. For any type \(A\) the
expression \(\choice A\) means ``arbitrary token of type A''. In a
typed calculus: \(\choice A: A\).

Now we can give a types-and-tokens interpretation of untyped calculi.
For example, \(\choice A, \choice B\seqso \riota A\lkand \riota B\) would
mean ``if \(\choice\)A is a token type A and \(\choice\)B is a token
type B, then \(\iota A \lkand\iota B\) is the token formed by
combining them.''

This is just an alternative notation to \(a:A\). It allows us to avoid
naming token and/or type particulars. It's also too complicated, since
\(\iota A\) must be bound to \(\choice A\).

It looks like there is just no way to give a typed semantics to an
untyped calculus. But that's ok, the goal here is just to show how
propositions (formulae) and types go together.


\subsection{Notes}

Thinking out loud...

``Snow is white'' is a proposition. It's counter-intuitive to call it
a type. What would count as a token of such a type? It just does not
make sense.

The problem is that we routinely overload the term ``proposition''. We
call expressions like \(A\land B\) and \(a+b=b+a\) propositions, but
they are not propositions. What they are is \textit{propositional
  schemas}. Propositions are particulars; propositional schemas are
universals. So ``(Snow is white)\(\land\)(Roses are red)'' is a
particular proposition that is an instance of the schema \(A\land B\).

In fact Curry-Howard is often stated more accurately as ``formulae are
types''; that is indeed the term Howard used in his landmark paper.

It makes perfect sense to say that a propositional schema or formula
is a type; both are generalizations.

But wait, its more than a schema. More like a schematic
meta-meta-variable. \(A\land B\) ranges over not just A and B, but
over all tokens of its type, which need not necessarily match the form
of the schema. So \(A\land B\) is just like A as a meta-variable, but
its more than that.  Or its just a compound metavariable.

But A and B are type metavariables, not token metavariables. They
can't be instatiated as tokens of their type, but only as particular
type symbols. So crap, all this schema reasoning is bogus.

So instead of ``propositions as types'' we should say ``propositional
schemas are types''. The the proofs of such types are instances of the
schema, not proofs of a proposition. So \((2,3)\), being an instance
of the schema \(\), counts as a proof of that type; but proving that
it is an instance of the schema is involves a different kind of proof.

Maybe this confusion is due to the fact that logicians usually prove
propositional formulas, but they say they are proving propositions.
What they are doing is providing a general proof for a class of
particular propositions. Mathematicians too are mostly interesting in
proving generalities, which they express by formulae.

Well, to be fastidious, what logicians usually prove is a universal
closure over a propositional formula. In effect, ``for all particular
formulae P, Q, the following holds''; or ``for arbitrary
formulae...''. So what gets proven is a universally quantified
implication whose conclusion is the formula of interest. But ``proof
of a schematic formula'' makes no sense; what the proof (of the
closure) shows is that all \textit{instances} of the formula are true.

But does this reading really work? What about the idea that the
meaning of a proposition is given by all its proofs, esp. when there
may be many ways to prove it? In that case we would be talking about a
particular proposition? No, we could be talking about propositional
formulae too. For example \(A\land B\rightarrow B\land A\). That's a
schema. We say we can prove it. What that means is that we can prove
it for arbitrary \(A, B\) - universal closure. And we do that by
sticking with formula schemas and metavariables. And such a proof
would not be an instance of the schema. Because it would not be a
proof of the formula, it would be a proof of the universal closure
over the proof structure. The concluding for of an instance of that
proof would be an instance of the formula of interest. That token
particular would represent the proof the produced it. So even
different proofs would end up at the same token. And that's what
programs do, compute a result. For propositional schemas, the result
is a token instance of the schema; for functions, it is a token of a
numeric type.

It all depends on what you mean by proof. A logical proof builds its
conclusion just as a function builds its result. So the proof is like
the function implementation. Its the output that counts as
proof-of-type (instance of schema). So ``program as proof'' or
vice-versa really means ``output builder as conclusion builder''.
Proof as machine to build conclusion, which is a particular
proposition (instance of schema). Proof object.

So proof shows how to build output, it does not build the output. To
prove the proposition schema, show how you would build an instance in
a particular case

Some writers interpret \(a:A\) as ``a is a proof \textit{term} for
proposition A''.

Also the kind of proofs we're talking about only prove type
correctness. Of a function, for example. We're not talking about proof
of the correctness of an implementation. Only proof that it is correctly
typed.  Or, for a dependent pair, that b is twice a.

Also, compare Church's development of lambda. \(x+2\) is an open
formula; \(\lambda x.x+2\) is a closed formula that ranges over all x.

``Proofs are programs'' also overloads ``proof''. On the one hand, the
idea is that tokens of a type are proofs. On the other, a proof of a
proposition is a tree or chain of inferences expressed in a language.
So we have two senses of proof: one meaning ``instance'', and the
other meaning ``demonstration''.

Tokens of a type are never \textit{demonstrative} proofs of a
proposition. But we can offer demonstrative proof that a token is an
instance of a type.

Given a particular proposition, we can infer the type of which it is
an instance. That's a primitive notion in a type system: every token
particular has a type.

\subsection{Equality Types}

Things get a little more complicated when we consider equations. An
equation like \(2+2=4\) is a particular proposition, which we can
prove. From it we can infer the type it betokens. That is, we can
infer its propositional schema, or formula. What should that look
like? We need a symbol to generalize each side of the equation, and we
also need to indicate their type, \(\Nat\). The HoTT convention is to
write \(\Id{A}{a}{b}\) or \(\Eq{A}{a}{b}\). The important thing to
remember whatever we write, it represents a schema or type.

That would give us something like \(\ulcorner 2+2=4 :
\Eq{A}{a}{b}\urcorner\), indicating that \(\ulcorner 2+2=4\urcorner\)
is a token particular of the general schema
\(\ulcorner\Eq{A}{a}{b}\urcorner\). But then \(2+2=4\), being a
particular proposition, is the kind of thing we can proof by
demonstration. We do that by reducing both sides, ending up with
\(4=4\). But that in turn is another token particular, whose inferred
type is \(\ulcorner\Eq{A}{a}{a}\urcorner\). And the only token of that
type, by definition, is \(\refl{a}\).

Types of the form \(\ulcorner\Eq{A}{a}{b}\urcorner\) have no defined
constructors, which means that while we can say that some \(p\) is an
instance of the type by writing \(p:\ulcorner\Eq{A}{a}{b}\urcorner\),
we cannot express \(p\) as an inductively defined token, as we can do
for natural numbers, for example. Every nat can be expressed as S...SZ
for some number of S. We cannot do that with equality types. If we
start with a token that is an equation, we can reduce it to get
\(a=a\), from which we can infer the type
\(\ulcorner\Eq{A}{a}{a}\urcorner\), which we can prove with
\(\refl{a}\). But if we start with a variable like \(p\), then we have
no equation to reduce. [FIXME: finish this. coinduction -
  \textit{assuming} \(p:\ulcorner\Eq{A}{a}{b}\urcorner\), we can infer
  \(\refl{a}:\ulcorner\Eq{A}{a}{a}\urcorner\). But how can we arrive
  at the former as the conclusion of an inference in a proof?]

A formula like \(a+b=b+a\) is a propositional formula. We can
prove such propositional formulae by ...

Then \(2+3=3+2\) would be a token (proof?) of the type. But that's the
kind of particular that we can prove.


%% \bookmarksetup{startatroot}

%%%%%%%%%%%%%%%%%%%%%%%%%%%%%%%%
\section{Proof}
\label{sect:proof}

Traditional (classic) view: a proof is an epistemic device; it
displays, exhibits, makes \textit{visible} (if only to the mind's eye)
a form of \textit{certain knowledge}.\sidenote{The link between
  knowing and seeing runs very deep in Western culture.  Not
  surprisingly it is closely connected with representationalism and
  cartesianism generally.  It has pretty much dominated Western
  thinking since Descartes, but has come under strong attack from
  Pragmatists.  Dewey called it ``the spectator theory of knowledge.''f
  See \citep{rorty_philosophy_2009} etc.}

Alternatives to the spectator theory: pragmatism, know-how over know-that.

\begin{ednote}
  TODO: summary of concepts of proof.  Emphasize contrast between
  representationalism and inferentialism.  Representationalism is
  atomistic: you could have only one concept.  Inferentialism is
  holistic: you have to start out with at least two concepts, since
  every inference involves a premise and a conclusion.  Inferentialism
  is a natural fit for \HoTT.

  Question: can you have only one type?  In other words, is type
  theory essentially holistic or atomistic?
\end{ednote}


For \HoTT{}, as for most varieties of constructivism, it is better to
abandon traditional notions of proof as something you see in favor of a
more pragmatic notion of proof as something you do.

etc.

Critical point: in \HoTT we have two ``kinds'' of types: propositional
types and non-propositional types.\sidenote{This is not in general
  recognized in \HoTTB, but I think it should be emphasized, if only
  because it reflects intuition.}  If we are to also treat ``proof''
(or witness or whatever) as a fundamental principle of \HoTT, one that
complements the concept of type, then we need to treat both ``type''
and ``proof'' as genuses (genii?) of which propositional and non-propositional
are species.

\begin{ednote}
  General point (to be made elsewhere, maybe in
  \cref{sect:foundations}: the concepts of type and proof go together.
  You cannot have one without the other.  That's very different than
  set theory.  You can have sets and elements without proofs.
\end{ednote}



Long story short: we are in dire need of improved terminology.  My
suggestion is as follows:

\begin{description}
\item [Proof of a proposition] In contrast to the classic spectator
  view, we treat proof not as the exhibition (or: making available for
  inspection) of the form of a bit of certain knowledge, but as the
  \textit{demonstrative expression} of the proposition.
  Alternatively, the expressive demonstration of the proposition.  So
  whereas a classic proof is something that must be ``seen'' in order
  to be grasped, a type-theoretic proof is something that must be
  actively \textit{done}, not merely passively observed.  One must be
  able to follow the construction of the proof.

\item [Proof of a non-propositional type] Classically, one only proves
  propositions, not terms.  So the idea of e.g. ``proving'' the
  natural numbers doesn't even make sense; it reflects a category
  mistake.  But in \HoTT, the concept of ``proving'' a type is
  primitive; the problem is that ``proving'' is the wrong word.
\end{description}

So here's one way to look at it: we construct (make) proofs; but the
proofs we construct are expressions of the type (the thing we prove).

%%%%%%%%
\subsection{Of the Ambiguity of Of}
\label{subs:ofofof}

``Of'' supports two distinct readings.  Consider ``the conviction of
the defendant''.  If the court did the convicting, then ``of'' acts as
a kind of intermediary between a verbal noun (``conviction'' as act or
action of convicting) and its direct object (e.g. ``The court
convicted the defendant'').  The conviction affects the defendant from
the outside; it does not ``belong'' to the defendant but to the court.
On the other hand, if we take ``the conviction of the defendant'' to
refer to a belief to which the defendant is firmly committed, then the
conviction is ``internal''; it belongs to and comes from the
defendant.

This ambiguity of ``of'' afflicts phrases like ``proof of a
proposition'' as well.  If we can disambiguate it some of the mystery
of the relation between types and proofs will vanish.

%%%%%%%%
\subsection{Demonstrations and Demonstratives}
\label{subs:}

When we \textit{exhibit} a classic proof of a proposition, the proof
comes out as external to the proposition proved, just as a court's
conviction of a defendant is external to the defendant.  Such a proof
is something added or attached to the proposition.

But when we \textit{demonstrate} a proposition,\sidenote{Note: we
  demonstrate propositions, not proofs; a demonstration of a
  proposition \textit{is} a proof.} the demonstration (that is, proof)
is to be deemed an expression of the proposition in the internal
sense: an expression whose source, so to speak, is the proposition
itself, rather than the writer of the proof.  This may sound odd or
even ridiculously anthropomorphic, but if you think about it a bit it
makes perfect sense.  The mathematical proofs we write down are not
really expressions our our thought, but of mathematical structures,
entities, relations etc.  So they express
mathematics.\sidenote{Actually we should probably think of them as
  having a dual expressivism.  On the one hand they clearly express
  mathematics; but on the other hand, the particular form a proof
  takes is an expression of the writer's style or way of thinking.}

We can think of a demonstration in this sense as expressing a type's
structure, construed as the inferential articulation of the concept of
the type.\sidenote{See \cref{sect:brandom} for more on the inferential
articulation of conceptual content.}

The nice thing about this way of thinking is that it resolves the
tension between propositional and non-propositional types with respect
to proof.  In both cases, what \HoTT{} calls proof or witness is to be
taken as a demonstrative expression, or expressive demonstration, of
the type itself.  In the case of propositional types, favor the term
``demonstration'', with its connotations of progressive unfolding of a
logical structure, or better, rational argument.  In the case of
non-propositional types like \N, favor the term ``demonstrative'',
with its adjectival sense of ``something having a demonstrative
function'', rather than a nominal sense of ``act or action of
demonstrating''.  So an element\sidenote{We really must get rid of
  ``element''; it's too suggestive of set theory.  Maybe
  ``demonstrative'' fits the bill; instead of ``element of a type'' we
  would say ``demonstrative of a type''.  Or maybe ``demonstrator''.}
of a propositional type we would call a demonstration of the type, and
an element of a non-propositional type we would call a demonstrative
of the type.\marginnote{So $2$ is a demonstrative of the natural
  numbers; a proof that ``$2$ is even'' is a demonstration that
  expresses just that ``$2$ is even''.}

\begin{ednote}
  Demonstration qua demonstration of know-how?  Expression as
  expression of a type's structure - that is, its inferential
  articulation?
\end{ednote}

In both cases we have demonstration rather than proof of the type.

\begin{ednote}
  ``Demonstrator'' as the genus of ``demonstration'' and
  ``demonstrative''.  It has the virtue of paralleling
  ``constructor''.
\end{ednote}


\section{Logical Constants}\label{sec:logconsts}

\subsection{Dyads}

\subsection{Implication}

Comes first because the rules themselves use consequence (implication).

A \textit{dyad} is a conjunction: a composite of two things.

Before we can have a dyad (a whole composed of two parts), we must
have the parts individually. We do not have a single word to express
this concept in English. So the best we can do is pick out a
circumlocution and bless it as our designated way of expressing the
idea. We will use ``and also'', so that \(\ulcorner A\) and also
\(B\urcorner\) expresses the idea of A and B \textit{uncomposed};
we'll also use the admittedly paradoxical term \textit{predyad} to
refer to two things before (logically) they have been composed to form
a whole.

Dyads come in various flavors:

\begin{itemize}
\item conjunction: \(A\land B\)
\item additive conjunction: \(A \addand B\). You have both A
  \textit{and} B, but you can only use one: A \textit{or} B.
\item multiplicative conjunction: \(A\fusion B\). You have both, and
  you can only use both together to produce a single output.
\item disjunction: \(A\lor B\)
\item additive disjunction: \(A \addor B\)
\item multiplicative disjunction: \(A\fusion B\). You have both, and
  you can use A or B or both to produce either of two possible outputs.
\end{itemize}

With the dyad concept in hand we can easily see what is wrong with
\ML's ``meaning explanations'' of the logical constants.

According to [cite?], to understand a logical constant one must
understand what counts as its proof; and according to
[\parencite{martin1984intuitionistic} p. 7], ``a proof of the proposition
\(A\&B\) consists of a proof of \(A\) and a proof of \(B\)''. But this
also suffices to prove fusion (\(\fusion\)), so it is not sufficient
to fully explain either conjunction or fusion. It explains how dyads
may be proven/produced, but does not distinguish between the two functional
roles a dyad can play.

The problem is that \ML\ only explains one side of the inferential
articulation of the logical constants, namely the constructors. An
adequate explanation must do more that show how propositions involving
the constants may be proved (or constructed, produced, computed,
etc.). It must also explain the \textit{consequences} of such
propositions - how they may be \textit{used}. The difference between
different kinds of dyads is made clear by explaining how they may be
used. Loosely, with a pair \(A\land B\) we can use \(A\) or \(B\) or
both; with an additive conjunction \(A\&B\), we can use either \(A\)
or \(B\), but not both; and with a multiplicative conjuction \(A\fusion
B\) (fusion), we can only use both.

\paragraph{Ontology}

It is tempting to thing that the logical constants determine different
\textit{kinds of things}. For example, that each conjunction
determines a different kind of dyad. But this is problematic. Dyads
are all formed in the same way, so it's hard to see how they could
belong to different ontological categories.

A better way to think about this is to view each logical constant as
determining not an ontological category but a kind of (logical)
\textit{game}, in which the constructions are the pieces. So a
proposition like \(A\land B\) is a dyad that serves as a game piece in
the conjunction game. Under this perspective, ontology is irrelevant.
It does not matter what kind of a thing a game piece is, except that
it must ``work'' for the game; only the rules of the game matter, and
the same kind of piece may work for different games. Of course, one
rule is that you have to have a game piece in order to play the game;
but the construction rules allow us to produce those pieces, without
regard to ontological status. For the various kinds of conjunction,
the pieces are dyads, which are capable of playing different roles in
the different games.

This puts Linear Logic in a different light. Usually LL is presented
as a ``logic of resources'', and intuitive explanations of the rules
are based on the fiction that propositions represent resources that
can be produced and consumed. So we get glosses like ``\(A\&B\) means
you can consume A or B and you get to choose, and \(A\oplus B\) means
you get only one, but you don't get to choose.''

Or we get half-stories like the manifestly absurd interpretation of
\(A\linfer B\&C\) as ``for A (qty of money) I can buy whichever I
choose of B and C''; while that may contain some technical truth, it
offers no clue as to why a conjunction (\(A\linfer B\&C\)) should
behave like a disjunction (``whichever I choose'').

But LL is is a \textit{logic}, and it traffics in the same stuff as
any other logic, namely propositions, inferences, etc. It can be used
to model the behaviour of resources, but it is a major mistake to
think that the logic is in any way essentially connected to the
concept of a resource. For example, the usage-rule for \(\&\),
additive conjunction, says that the \textit{conjunct} \(A\&B\)
suffices to prove anything that can be proven by A alone \textit{or} B
alone.  That's very different than ``you can choose either A or B''.

IOW, the rules show how propositions formed from the logical constants
can serve a function that can also be accomplished in some other way;
or put another way, that the conclusion of an elimination rule is a
kind of abbreviation for some other rules that do the same thing.

\subsubsection{Conjunction}

If our language were to lack (prelogical) \textit{and}, then we would
not be able to say things like ``He has a dog and a cat''. We would
have to settle for two sentences ``He has a dog'' followed by ``He
has a cat''.

\paragraph{Distributivity
\newline}

Conjunction is distributive. ``He has a dog and a cat if and only if
he has a dog and he has a cat.'' Symbolically: \(P(a\&b)\)\iff
\(P(a)\,\&\,P(b)\). If the predicate P is ``...is true'', then since the
propositional content of ``P is true'' is the same as the content of
``P'', we can drop it. This yields \[(a\&b)\iff (a)\&(b)\] meaning
``a\&b is true if and only if a is true \&\, b is true''.

The grouping expressed by the parentheses is essential. If we omit the
parentheses, we get \(a\& b\iff a\&b\), which fails to express the
distribution of ``...is true'' over conjunction. We can express the
grouping more concisely by writing \(a,b\) for \((a)\&(b)\) (a
semicolon is also commonly used). This yields the equivalent
propositions:

\begin{align}
  & \text{True}(a)\,\&\,\text{True}(b)\iff \text{True}(a\&b) \\
  & (a)\,\&\,(b)\iff (a\,\&\,b) \\
  & a,b\iff a\&b
\end{align}

\textbf{Important}: this is prelogic; it is not yet logic proper. In
particular, conjunction is ambiguous. Intuitively, there is more than
one way to combine two things to form a whole. We can \textit{pair}
the inputs, such they remain distinct and can be retrieved by
decomposing the pair. But we can also \textit{fuse} the inputs,
resulting in a whole from which the individual inputs cannot be
extracted. Real-world examples include color blends formed by mixing
primary colors, a ``smoothie'' formed by mixing ingredients in a
blender, and an alloy formed by melting and mixing two metals. Fusion
can be expressed by formal logic; it is used in Linear Logic,
Relevance Logic, and others. In any case, to get from our prelogical
conjunction to a proper logic we will need to clarify exactly how
conjunction works. So far we've only show that it is distributive for
predicates.

Summary: The rule for \(\&\) introduction expresses the material
inference from ``A and also B, independently'' to ``Both A and B,
together''. The former is expressed formally by a premise structure
\(A ; B\), and the latter by logical conjuction \(A \& B\).

The material inference is simple. If I say ``I have a dog and also I
have a cat, independently'', you can infer that I have ``both a dog and a
cat, together''. If you say it out loud, nobody will object. But ``A
and also B, independently'' is conceptually distinct from ``Both A and B,
together''. The distinction is the difference between a collection of
parts and a whole composed of those parts.

Assuming ``A and also B, independently'', we can infer ``Both A and B,
together''. The \(\&\)-introduction rule makes this inference
explicit. Formally: \(A ; B \linfer A \& B\). Logical \(\&\)
expresses the notion ``both and, together''; it follows from
structural \(;\), which in turn formally expresses the prelogical
notion ``and also, independently''. The \(\linfer\) expresses the inference
from the latter to the former.

\paragraph{Notes\\}

Critical concept is \textit{independence}. ``A suffices for C and also
B suffices for C'' means they suffice independently; ``A and B suffice
for C'' means they suffice together, as a whole, not independently.

Explanatory vignettes for dialogisms: Bob \& Alice. The don't really
explain, they illustrate.

\subsubsection{Disjunction}

It is fairly easy to see how inferential practices involving
conjuction can explain prelogical \textit{and}. Its a bit more
difficult to come up with a good story for \textit{or}.

Our task is to explain what kind of prelogical inferential move is
expressed by saying things like ``David is carved from wood or
marble''. But what can we \textit{do} that involves \textit{or}? Its
easy to see that saying one thing, and then saying another thing,
suggests \textit{and}; after all, such sayings are already
``conjoined'' in time, even if they are semantically independent.
``Roses are red'' followed by ``violets are blue'' leads to ``Roses
are red and violets are blue.'' What leads to ``Roses are red
\textit{or} violets are blue''?

A good start is \textit{travel}. To get from Rome to Paris one can
travel by train, and one can travel by car. But one cannot travel buy
car \textit{and} by train. A choice must be made; that is the nature
of travel.

In other words, some words in our vocabulary force choices on us,
others do not. Or more likely, some parts of the world force choices
on us. Given tea and honey, I can consume them them both at once by
first combining them. Given two paths to a destination, I can only
take one. Given a block of wood and a block of marble, I can sculpt a
statue of one or the other, but not both. I can only watch one movie
at a time. And so on.

\begin{align}
 & \text{Og can get from his cave to the watering hole by following path A} \\
 & \text{Og can get from his cave to the watering hole by following path B}
\end{align}

Inference: Og can get there by path A or path B.

This inference is underwritten by the material content of ``following
a path''. If we know what it means, we know we cannot follow two paths
simultanously. Specifically, we would criticize the inference to ``Og
can get there by following both path A and path B.''

But I don't think we need language to language transitions to explain
inference to \textit{or}. We can make that inference from our
\enquote{knowledge} of how the world works. A coin cannot land with both
faces up. We ``know'' this not because we possess some kind of
theoretical knowledge, but because that is our experience in the
world. A very young child in the world of cave-men might not yet grasp
this simple fact, but will learn it soon enough from experience. So we
can say that the inference to ``heads or tails'' is underwritten by
experience rather than by the content of any propositional premise.

IOW, what is the propositional stimulus (excluding questions) to which
the correct response is ``A or B''? For conjunction, ``A and B'' is the
correct response to a sequence of sayings, ``A'' then ``B''.

Disjunction seems to be tied to both choice and ignorance. You can
choose train or car to get to Paris. Flip a coin and you don't know
what you'll get, but you know it will be heads or tails. Calling a
coin flip involves both choice and ignorance. So disjunction also
seems to be tied to possibility (probability?).

If I'm told a coin has two faces, heads and tails, I can respond by
choosing heads or tails. I can also respond by saying ``So each face
is either heads or tails'', or ``So if you flip it the upface will be
heads or tails'', etc.

Disjunction seems to be parasitic on conjunction. (duals?) For ``A or
B'' to make sense, both A and B must be in play.

Can the same stimulus work for both conjunction and disjunction? The
response to ``he has a dog'' then ``he has a cat'' can be:

he has a dog and a cat
each is a dog or a cat(?)
a question: which is bigger, dog or cat?
a command: choose dog or cat!

\subsubsection{Duality of conjunction and disjunction}
Conjunction and disjunction are dual.

The left-intro rule for \(\lor\) demonstrates (half of) the duality:

%% \begin{prooftree}
%% \AxiomC{$A,\Gamma\linfer C\kern-1.2em$}
%% \AxiomC{$\medtriangleup\kern-1.2em$}
%% \AxiomC{$B,\Gamma\linfer C$}
%% \RightLabel{$\lor\linfer$}
%% \TrinaryInfC{$A\lor B,\Gamma\linfer C$}
%% \end{prooftree}

Gloss: if A suffices for C, and independently B suffices for C,
then \(A\lor B\) suffices for C. Informally: both A \textit{and} B
independently suffice for C; and since \(A\lor B\) will always
``contain'' at least one of A or B, it too suffices for C.

Or: \(A\lor B\) suffices wherever A \textit{and} B (independently)
suffice.

For conjunction \(\land\) we have two left-intro rules:

%% \begin{center}
%% \AxiomC{$A,\Gamma\linfer C\kern-1.2em$}
%% \RightLabel{$\land\linfer_1$}
%% \UnaryInfC{$A\land B,\Gamma,\linfer C$}
%% \DisplayProof
%% \hskip 1.5em
%% \AxiomC{$B,\Gamma\linfer C$}
%% \RightLabel{$\land\linfer_2$}
%% \UnaryInfC{$A\land B,\Gamma,\linfer C$}
%% \DisplayProof
%% \end{center}

These rules say that \(A\land B\) suffices for C wherever either A
(alone) suffices \textit{or} B (alone) suffices. In other words, each
is independently sufficient for C, so we can chose either rule to get
\(A\land B\). If we choose \(\land\linfer_1\), then we know that the
inference to C ``uses'' the A part of \(A\land B\) (so to speak) to
get to C. This is how it corresponds to the elimination rule of
natural deduction: implicitly, it dismantles \(A\land B\) to obtain A,
then uses it to get C.

Note also that it does not follow from rule \(\land\linfer_1\) that B
suffices for C. It may or may not, but we're not using it so we don't
care. Similarly for \(\land\linfer_2\); if we use it, then A may or may
not suffice for C, and we don't care either way, because we only need
B to get C.

Notice that the premises for \(\land\linfer_1\) and \(\land\linfer_2\)
together are the same as those for \(\lor\linfer\), but they're split
into two rules. We can bring out the symmetry more conspicuously by
combining them. To do this we need another symbol to express the
\textit{or}ing of the top sequents. We'll use \(\medtriangledown\):

%% \begin{prooftree}
%% \AxiomC{$A,\Gamma\linfer C\kern-1.2em$}
%% \AxiomC{$\medtriangledown\kern-1.2em$}
%% \AxiomC{$B,\Gamma\linfer C$}
%% \RightLabel{$\land\linfer$}
%% \TrinaryInfC{$A\land B,\Gamma\linfer C$}
%% \end{prooftree}

The intention here is that this one rule expresses exactly what the
two rules \(\land\linfer_1\) and \(\land\linfer_2\) express together in
disjunction. It shows more clearly the duality of sequent disjunction
(\(\medtriangledown\) on the top) and propositional conjunction
(\(\land\) on the bottom). Except for the connectives, it has the same
structure as \(\lor\linfer\), which shows the duality in the other
direction - sequent conjunction (\(\medtriangleup\)) on top, and
propositional disjunction (\(\lor\)) on the bottom.



\include{linearlogic}

