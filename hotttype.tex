%%%%  CAVEAT:  xelatex chokes on pkg soul (loaded by tufte); use lualatex

%% \documentclass[12pt,toc]{tufte-handout}
\documentclass[reqno,12pt]{tufte-book}
%% \usepackage{trace}
%% \documentclass[reqno,12pt]{article}

\usepackage{draftwatermark}

% BLACK & WHITE
\input{opt-black-white}

% FORMATTING DEPENDENT ON PAPER SIZE
\input{opt-letter}

\usepackage{etex}

%%%%%%%%%%%%%%%%%%%%%%%%%%%%%%%%%%%%%%%%%%%%%%%%%%%%%%%%%%%%%%%%
%% packages included by original hott main.tex

%%% For table {tab:theorems}
\usepackage{pifont}

%%% Multi-Columns for long lists of names
\usepackage{multicol}

\usepackage{graphicx}
\usepackage{comment}

\usepackage{fancyhdr} % To set headers and footers

%% \usepackage{nextpage} % So we can jump to odd-numbered pages

\usepackage{amssymb,amsmath,stmaryrd,mathrsfs,wasysym}
\usepackage{enumitem,mathtools,xspace}
%% \numberwithin{equation}{subsection}

\usepackage{xstring} % For generating singluars and plurals in \backref

%% \usepackage{xcolor,mdframed}
\usepackage{xcolor} % For colored cells in tables we need \cellcolor
\usepackage{wallpaper} % For the background image on the cover page

\usepackage{booktabs} % For nice tables
\usepackage{array} % For nice tables

\definecolor{linkcolor}{rgb}{\OPTlinkcolor}
\usepackage{aliascnt}
\usepackage[capitalize]{cleveref}
\usepackage[all,2cell,cmtip]{xy}
\UseAllTwocells
%\usepackage{natbib}
\usepackage{braket} % used for \setof{ ... } macro

\usepackage{tikz}
\usetikzlibrary{decorations.pathmorphing,arrows}

\usepackage{etoolbox}           % hacking commands for TOC

%% \usepackage{mathpartir}         % for formal.tex appendix, section 3

\usepackage[numbered]{bookmark} % add chapter/section numbers to the toc in the pdf metadata
%%%%%%%%%%%%%%%%%%%%%%%%%%%%%%%%%%%%%%%%%%%%%%%%%%%%%%%%%%%%%%%%

%% these two go together!
\usepackage{framed}
\usepackage[standard,framed]{ntheorem}
%% \newtheorem{theorem}{Theorem}
%\newtheorem{cor}{Corollary}
%\newtheorem{lem}{Lemma}
%% \newtheorem*{defn}{Definition}
%% \theoremstyle{remark}
%% \newtheorem{remark}{Remark}
%% \newtheorem*{commentary}{Commentary}

%% \theoremclass{Remark}
%% \theoremstyle{break}
%% \newtheorem{note}{Note}[section]

\theoremstyle{plain}
\theorembodyfont{\upshape}
\theoremsymbol{\ensuremath{\ast}}
\theoremseparator{}
%% \newtheorem{ednote}{Ed. note}[section]
\newframedtheorem{ednote}{Ed. note}[section]

\newtheorem*{todo}{TODO}
%% \newtheorem{eg}{Example}

\input{macros}

\usepackage{appendix}

%% \usepackage{csquotes}

\usepackage{setspace}

%% broken (doesn't work with tufte-handout):
\usepackage{zed-csp}
%% broken:
%% \usepackage{ltcadiz-fam}

\usepackage{fontspec}
%% \usepackage{xltxtra,xunicode}
\defaultfontfeatures{Scale=MatchLowercase}

%% \defaultfontfeatures{Scale=MatchLowercase}
%% \setmainfont[Mapping=tex-text]{Times New Roman}
%% \setsansfont[Mapping=tex-text]{Arial}
%% \setmonofont{Courier}

\setmainfont[Ligatures=TeX]{TeX Gyre Bonum}
\setromanfont[Ligatures=TeX]{TeX Gyre Bonum}
\setsansfont[Ligatures=TeX]{TeX Gyre Adventor}
\setmonofont[Ligatures=TeX]{TeX Gyre Cursor}


%% \setmainfont[Mapping=tex-text]{Minion Pro}
%% \setromanfont[Mapping=tex-text]{Minion Pro}
%% \setsansfont[Mapping=tex-text]{TeX Gyre Heros}

%% Bugfix: see https://code.google.com/p/tufte-latex/issues/detail?id=64
% Set up the spacing using fontspec features
\renewcommand\allcapsspacing[1]{{\addfontfeature{LetterSpace=15}#1}}
\renewcommand\smallcapsspacing[1]{{\addfontfeature{LetterSpace=0.0}#1}}

\usepackage{epigraph}
\setlength{\epigraphwidth}{.8\textwidth}

%% general symbols - degree, etc.
%% \usepackage{gensymb}

\usepackage [english]{babel}
\usepackage [autostyle, english = american]{csquotes}
%% \usepackage{quoting}

%% nice double-stroke fonts
\usepackage{dsfont}

% Small sections of multiple columns
\usepackage{multicol}

% Provides paragraphs of dummy text
\usepackage{lipsum}

% The units package provides nice, non-stacked fractions and better spacing
% for units.
\usepackage{units}

%\usepackage{geometry}                % See geometry.pdf to learn the layout options. There are lots.
%\geometry{letterpaper}                   % ... or a4paper or a5paper or ...

\usepackage{xfrac}

\usepackage{hyperref}
\hypersetup{
  bookmarks=true,         % show bookmarks bar?
  bookmarksdepth=3,
  unicode=true,          % non-Latin characters in Acrobat’s bookmarks
  pdftoolbar=true,        % show Acrobat’s toolbar?
  pdfmenubar=true,        % show Acrobat’s menu?
  pdffitwindow=false,     % window fit to page when opened
  pdfstartview={FitH},    % fits the width of the page to the window
  pdftitle={Intuition and Exponentiation},    % title
  pdfauthor={G. A. Reynolds},     % author
  pdfsubject={Mathematics},   % subject of the document
  pdfcreator={G. A. Reynolds},   % creator of the document
  pdfproducer={Producer}, % producer of the document
  pdfkeywords={Exponentiation} {Mathematics}
  pdfnewwindow=true,      % links in new window
  colorlinks=true,       % false: boxed links; true: colored links
  linkcolor=blue,          % color of internal links
  citecolor=blue,        % color of links to bibliography
  filecolor=magenta,      % color of file links
  urlcolor=cyan           % color of external links
}

%% \usepackage[
%% bibstyle=numeric,
%% citestyle=authoryear,
%% hyperref,
%% bibencoding=utf8,
%% backref=true,
%% backend=biber]{biblatex}

%% http://tex.stackexchange.com/questions/66778/citation-alias-with-multibib-and-natbib
%% \makeatletter
%% \def\@mb@citenamelist{cite,citep,citet,citealp,citealt,citepalias,citetalias}
%% \makeatother

%% http://stackoverflow.com/questions/2496599/how-do-i-cite-the-title-of-an-article-in-latex
\defcitealias{z-iso-13568}{ISO 13568:2002 Information technology -- Z formal specification notation --
  Syntax, type system and semantics}

\usepackage{tikz}
\usepackage[markings,customcolors]{hf-tikz}
\usetikzlibrary{%
  arrows%
  ,calc%
  ,decorations.text%
  ,decorations.pathreplacing%
  ,fadings%
  ,positioning
  ,shapes.geometric%
}

\usepackage{tikz-3dplot}

\usepackage{pgfplots}
\pgfplotsset{height=7cm,compat=1.9}

\usepackage{tkz-euclide}
\usetkzobj{all}

%% prettier integral syms, but broken on miktex
%% \usepackage{esint}


%% \usepackage{MnSymbol}
%% \usepackage[misc]{ifsym}

%% \usepackage{morefloats}

%%%%%%%%%%%%%%%%%%%%%%%%%%%%%%%%%%%%%%%%%%%%%%%%%%%%%%%%%%%%%%%%
\title{HoTT Types}
%% \\
%% \Large Derived from the HoTT Book}
\author{}
%\date{}                                           % Activate to display a given date or no date

%%%%%%%%%%%%%%%%
%% tufte-latex customizations

\makeatletter
\let\runauthor\@author
\let\runtitle\@title
\makeatother

%% running headers
\newcommand{\changefont}{%
  \fontsize{7}{9.5}\selectfont
}
\fancypagestyle{plain}{
  \fancyhead[LO,LE]{\leftmark }
  \fancyhead[RO,RE]{\rightmark}
  \fancyfoot[CO,CE]{\thepage}
  \fancyfoot[LE]{\textsc{\runtitle}}
  \fancyfoot[RO]{\textsc{\runtitle}}
  \renewcommand{\headrulewidth}{0pt}
  \renewcommand{\footrulewidth}{0pt}
}
\pagestyle{plain}

\def\chpcolor{blue!45}
\def\chpcolortxt{blue!60}
\def\sectionfont{\LARGE}

\setcounter{secnumdepth}{5}
\setcounter{tocdepth}{5}        % sections and subsections for the toc

\makeatletter
%% Section:
\def\@sectionstrut{\vrule\@width\z@\@height12.5\p@}
\def\@makesectionhead#1{%
  {%\par\vspace{20pt}%
    \parindent -10pt\raggedleft\sectionfont
    %% \colorbox{\chpcolor}{%
    %%   \parbox[t]{90pt}{\color{white}\@sectionstrut\@depth4.5\p@\hfill
    %%     \ifnum\c@secnumdepth>\z@\thesection\fi}%
    %% }%
    \vspace{10pt}%
    \begin{minipage}[t]{\textwidth}%{\dimexpr\textwidth-90pt-2\fboxsep\relax}
      \@sectionstrut\hspace{-15pt}\textit{\textbf\Huge #1}
    \end{minipage}\par
    \vspace{5pt}%
  }
}
%% \def\@makesectionhead#1{%
%%   {\par\vspace{20pt}%
%%    \parindent 0pt\raggedleft\sectionfont
%%    \colorbox{\chpcolor}{%
%%      \parbox[t]{90pt}{\color{white}\@sectionstrut\@depth4.5\p@\hfill
%%        \ifnum\c@secnumdepth>\z@\thesection\fi}%
%%    }%
%%    \begin{minipage}[t]{\dimexpr\textwidth-90pt-2\fboxsep\relax}
%%    \color{\chpcolortxt}\@sectionstrut\hspace{5pt}\textbf{#1}
%%    \end{minipage}\par
%%    \vspace{10pt}%
%%   }
%% }
\def\section{\@afterindentfalse\secdef\@section\@ssection}
\def\@section[#1]#2{%
  \ifnum\c@secnumdepth>\m@ne
  \refstepcounter{section}%
  \addcontentsline{toc}{section}{\protect\numberline{\thesection}#1}%
  \else
  \phantomsection
  \addcontentsline{toc}{section}{#1}%
  \fi
  \sectionmark{#1}%
  \if@twocolumn
  \@topnewpage[\@makesectionhead{#2}]%
  \else
  \@makesectionhead{#2}\@afterheading
  \fi
}
\def\@ssection#1{%
  \if@twocolumn
  \@topnewpage[\@makesectionhead{#1}]%
  \else
  \@makesectionhead{#1}\@afterheading
  \fi
}
\makeatother

%%%%%%%%%%%%%%%%
%% macros

\newenvironment{important}[1][]{%
  \begin{mdframed}[%
      backgroundcolor={red!15}, hidealllines=true,
      skipabove=0.7\baselineskip, skipbelow=0.7\baselineskip,
      splitbottomskip=2pt, splittopskip=4pt, #1]%
    \makebox[0pt]{% ignore the withd of !
      \smash{% ignor the height of !
        \fontsize{32pt}{32pt}\selectfont% make the ! bigger
        \hspace*{-19pt}% move ! to the left
        \raisebox{-2pt}{% move ! up a little
          {\color{red!70!black}\sffamily\bfseries !}% type the bold red !
        }%
      }%
    }%
}{\end{mdframed}}

%% reversed integral sign
\makeatletter
\providecommand*{\curv}{%
  \mathrel{%
    \mathpalette\@curv\int
  }%
}
\newcommand*{\@curv}[2]{%
  \reflectbox{$\m@th#1#2$}%
}
\makeatother

%% \def\LaTeX{%
%%   L\kern-.36em
%%   {\setbox0=\hbox{T}%
%%     \vbox to \ht0{\hbox{\the\scriptfont0 A}\vss}}%
%%   \kern-.15em
%%   \TeX
%% }

%%%%%%%%%%%%%%%%

\newcommand\cspace{coordinate space}
\newcommand\Cspace{Coordinate space}
\newcommand\CSpace{Coordinate Space}

\newcommand\dspace{design space}
\newcommand\Dspace{Design space}
\newcommand\DSpace{Design Space}

\newcommand\Omg{\(\Omega\)}
\newcommand\sccs{standard cartesian coordinate space}
\newcommand\origin{\((0,0)\)}
\newcommand\ab{\((a,b)\)}

\newcommand\atypeA{\ensuremath{(a : A)}}

%% \newcommand\N{\(\mathds{N}\)}
%% \newcommand\R{\(\mathds{R}\)}
%% \newcommand\RR{\(\mathds{R}\times\mathds{R}\)}
%% \newcommand\Rtwo{\(\mathds{R}^2\)}
%% \newcommand\Z{\(\mathds{Z}\)}


\def\HoTT{%
  H\kern-.7pt
  {\tiny\raisebox{1pt}{o}}%
  %% {\setbox0=\hbox{T}%
  %%  \vbox to \ht0{\vss\hbox{\the\scriptfont0 o}\vss}}%
  \kern-1.5pt
  TT}

\def\HoTTB{%
  the H\kern-.7pt
  {\tiny\raisebox{1pt}{o}}%
  %% {\setbox0=\hbox{T}%
  %%  \vbox to \ht0{\vss\hbox{\the\scriptfont0 o}\vss}}%
  \kern-1.5pt
  TT Book
}

\newcommand\ML{Martin-L\"{o}f}

\newcommand\ITT{Intuitionisti Type Theory}

\newcommand\TTh{Type Theory}
\newcommand\tth{type theory}

\includeonly{%
%% introduction%
%% ,pragmatism
%% ,proof
%% ,semantics
%% ,math
%% ,foundations
%% ,types
,curry-howard
%% ,assertion
%% ,equality
%% ,brandom
%% ,hotttypes
%% ,misc
%% ,introduction
%% ,preliminaries
,lexicon
,proofassistants
}

%%%%%%%%%%%%%%%%%%%%%%%%%%%%%%%%%%%%%%%%%%%%%%%%%%%%%%%%%%%%%%%%
\begin{document}
%% \ifx\traceon\undefined \tracingall \else \traceon \fi

\maketitle

\begin{ednote}
  Currently this doc contains a (mildly organized) set of notes
  followed by the intro and chapter 1 from the
  \href{http://homotopytypetheory.org/book/}{HoTT Book}.  Eventually
  (maybe) the intro and chapter 1 will contain annotations, comments,
  additional examples, etc., but I have not started that yet, so if
  you are already familiar with the text you need not read them -- I
  haven't (so far as I recall) changed anything.

  The idea is to winnow out some of the strictly mathematical stuff
  leaving the core ``philosophical'' stuff, and annotate the text with
  some comments and quotes from Martin-L\"{o}f, Brandom, etc.  Or
  maybe leave the math stuff in, but annotate it with more detailed
  explanation and examples in programming languages.  In any case the
  purpose is to more fully articulate the link between HoTT's ideas of
  type and judgment (etc.) to the philosophical debates about
  language, assertion, proposition from which they emerged.  Why?
  Because I find those bits of the HoTT a little murky, and
  philosophers like Brandom have a lot to say about the issues.  Also,
  to show more clearly how type theory differs from set theory and
  classic logic.  Another goal is to provide more practical guidance
  to programmers interested in exploring dependent types.
\end{ednote}

\tableofcontents
%% \setcounter{tocdepth}{2}        % chapters, sections, and subsections for the
%%                                 % metadata of the pdf
%% \cleartooddpage[\thispagestyle{empty}]

%% \mainmatter % Turn on roman page numbers and numbered chapters

%%%%%%%%%%%%%%%%%%%%%%%%%%%%%%%%
%%%%%%%%%%%%%%%%%%%%%%%%%%%%%%%%
\chapter{The Pragmatist Enlightenment}
\label{sect:enlightenment}

%%%%%%%%
\section{Liberation}
\label{subs:liberation}



%%%%%%%%
\section{Pluralism}
\label{subs:pluralism}

\begin{ednote}
  Not just propositions-as-types, but types-as-propositions.  Example:
  the type \N can be viewed as a proposition ``there exists a natural
  number''.  This means that there is no authoritative definition of
  what a type is, which means that pluralism is an essential aspect of
  type theory.  Is this a sharp contrast with traditional mathematics?
  For pre-modern mathematics, number was unequivocally quantity or
  magnitude - no pluralism there.  Modern mathematics discarded
  quantitative interpretations of number in favor of structural
  notions.  The issue of pluralism is not so clearly decided there.
  Once you have isomorphisms, you can't really say that one structure
  is \emph{the} structure for a given class.  Groups, for example.  So
  isn't modern math essentially pluralistic?  Well let's look at
  foundations - set theory doesn't seem to be very pluralistic; a set
  is a set is a set, and not something else.  You can come up with
  distinct set \emph{theories}, but they all depend on the primitive
  notion of set, or maybe set membership.  Type theory, by contrast,
  seems to be different.  It doesn't have this kind of unity.  In fact
  there are many distinct type theories, so we should probably always
  use the plural.  The primitive seems to be ``type''; but the concept
  of type is not primitive in all type theories---\HoTT{} being a case
  in point.  ``In fact, no type former is 'primitive' to the game of
  type theory in this sense: you can very well have a type theory with
  no type formers! But it won't be very interesting...'' (M. Shulman,
  \href{https://groups.google.com/d/msg/hott-amateurs/U1X0m4r6G-A/K5eeMSPXE5YJ})
\end{ednote}

``Type theory, formal or informal, is a collection of rules for
manipulating types and their elements.  But when writing mathematics
informally in natural language, we generally use familiar words,
particularly logical connectives such as “and” and “or”, and logical
quantifiers such as “for all” and “there exists”. In contrast to set
theory, type theory offers us more than one way to regard these
English phrases as operations on types. This potential ambiguity needs
to be resolved, by setting out local or global conventions, by
introducing new annotations to informal mathematics, or both.''\HoTTB, p. 101

%%%%%%%%
\section{Normative Pragmatics}
\label{subs:normprag}

  Chapter 1 of \cite{brandom_mie}

%%%%%%%%
\section{Inferential Semantics}
\label{subs:inferentialism}

  Chapter 2 of \cite{brandom_mie}


%%%%%%%%
\section{Expressivism}
\label{subs:expressivism}

See \cite{price_expressivism_2013}


%%%%%%%%%%%%%%%%%%%%%%%%%%%%%%%%
\chapter{Logics}
\label{sect:logics}

\begin{description}
\item [Traditional] terms are primitive; propositions are combinations of terms; judgments apply to proposotions
\item [Modern: classic] LEM, AC, etc.
\item [Modern: intuitionistic]
\item [Expressivism]  Brandom's version: propositions are primitive; relation to inferential semantics; Price's global expressivism
\end{description}

\begin{ednote}
  From schema to type.  E.g. \(A,B\vdash A\land B\) --- traditionally
  viewed as a schema (involving either substitution or denotation), no
  construction involved.  Move from this to viewing it as a rule of
  construction or recipe for making something.
\end{ednote}

\chapter{Proposition}

\begin{ednote}
  BHK interpretation.  How \ML{} got it wrong wrt classic interpretation.
\end{ednote}


%%%%%%%%%%%%%%%%%%%%%%%%%%%%%%%%
\section{Proof}
\label{sect:proof}

Traditional (classic) view: a proof is an epistemic device; it
displays, exhibits, makes \textit{visible} (if only to the mind's eye)
a form of \textit{certain knowledge}.\sidenote{The link between
  knowing and seeing runs very deep in Western culture.  Not
  surprisingly it is closely connected with representationalism and
  cartesianism generally.  It has pretty much dominated Western
  thinking since Descartes, but has come under strong attack from
  Pragmatists.  Dewey called it ``the spectator theory of knowledge.''f
  See \citep{rorty_philosophy_2009} etc.}

Alternatives to the spectator theory: pragmatism, know-how over know-that.

\begin{ednote}
  TODO: summary of concepts of proof.  Emphasize contrast between
  representationalism and inferentialism.  Representationalism is
  atomistic: you could have only one concept.  Inferentialism is
  holistic: you have to start out with at least two concepts, since
  every inference involves a premise and a conclusion.  Inferentialism
  is a natural fit for \HoTT.

  Question: can you have only one type?  In other words, is type
  theory essentially holistic or atomistic?
\end{ednote}


For \HoTT{}, as for most varieties of constructivism, it is better to
abandon traditional notions of proof as something you see in favor of a
more pragmatic notion of proof as something you do.

etc.

Critical point: in \HoTT we have two ``kinds'' of types: propositional
types and non-propositional types.\sidenote{This is not in general
  recognized in \HoTTB, but I think it should be emphasized, if only
  because it reflects intuition.}  If we are to also treat ``proof''
(or witness or whatever) as a fundamental principle of \HoTT, one that
complements the concept of type, then we need to treat both ``type''
and ``proof'' as genuses (genii?) of which propositional and non-propositional
are species.

\begin{ednote}
  General point (to be made elsewhere, maybe in
  \cref{sect:foundations}: the concepts of type and proof go together.
  You cannot have one without the other.  That's very different than
  set theory.  You can have sets and elements without proofs.
\end{ednote}



Long story short: we are in dire need of improved terminology.  My
suggestion is as follows:

\begin{description}
\item [Proof of a proposition] In contrast to the classic spectator
  view, we treat proof not as the exhibition (or: making available for
  inspection) of the form of a bit of certain knowledge, but as the
  \textit{demonstrative expression} of the proposition.
  Alternatively, the expressive demonstration of the proposition.  So
  whereas a classic proof is something that must be ``seen'' in order
  to be grasped, a type-theoretic proof is something that must be
  actively \textit{done}, not merely passively observed.  One must be
  able to follow the construction of the proof.

\item [Proof of a non-propositional type] Classically, one only proves
  propositions, not terms.  So the idea of e.g. ``proving'' the
  natural numbers doesn't even make sense; it reflects a category
  mistake.  But in \HoTT, the concept of ``proving'' a type is
  primitive; the problem is that ``proving'' is the wrong word.
\end{description}

So here's one way to look at it: we construct (make) proofs; but the
proofs we construct are expressions of the type (the thing we prove).

%%%%%%%%
\subsection{Of the Ambiguity of Of}
\label{subs:ofofof}

``Of'' supports two distinct readings.  Consider ``the conviction of
the defendant''.  If the court did the convicting, then ``of'' acts as
a kind of intermediary between a verbal noun (``conviction'' as act or
action of convicting) and its direct object (e.g. ``The court
convicted the defendant'').  The conviction affects the defendant from
the outside; it does not ``belong'' to the defendant but to the court.
On the other hand, if we take ``the conviction of the defendant'' to
refer to a belief to which the defendant is firmly committed, then the
conviction is ``internal''; it belongs to and comes from the
defendant.

This ambiguity of ``of'' afflicts phrases like ``proof of a
proposition'' as well.  If we can disambiguate it some of the mystery
of the relation between types and proofs will vanish.

%%%%%%%%
\subsection{Demonstrations and Demonstratives}
\label{subs:}

When we \textit{exhibit} a classic proof of a proposition, the proof
comes out as external to the proposition proved, just as a court's
conviction of a defendant is external to the defendant.  Such a proof
is something added or attached to the proposition.

But when we \textit{demonstrate} a proposition,\sidenote{Note: we
  demonstrate propositions, not proofs; a demonstration of a
  proposition \textit{is} a proof.} the demonstration (that is, proof)
is to be deemed an expression of the proposition in the internal
sense: an expression whose source, so to speak, is the proposition
itself, rather than the writer of the proof.  This may sound odd or
even ridiculously anthropomorphic, but if you think about it a bit it
makes perfect sense.  The mathematical proofs we write down are not
really expressions our our thought, but of mathematical structures,
entities, relations etc.  So they express
mathematics.\sidenote{Actually we should probably think of them as
  having a dual expressivism.  On the one hand they clearly express
  mathematics; but on the other hand, the particular form a proof
  takes is an expression of the writer's style or way of thinking.}

We can think of a demonstration in this sense as expressing a type's
structure, construed as the inferential articulation of the concept of
the type.\sidenote{See \cref{sect:brandom} for more on the inferential
articulation of conceptual content.}

The nice thing about this way of thinking is that it resolves the
tension between propositional and non-propositional types with respect
to proof.  In both cases, what \HoTT{} calls proof or witness is to be
taken as a demonstrative expression, or expressive demonstration, of
the type itself.  In the case of propositional types, favor the term
``demonstration'', with its connotations of progressive unfolding of a
logical structure, or better, rational argument.  In the case of
non-propositional types like \N, favor the term ``demonstrative'',
with its adjectival sense of ``something having a demonstrative
function'', rather than a nominal sense of ``act or action of
demonstrating''.  So an element\sidenote{We really must get rid of
  ``element''; it's too suggestive of set theory.  Maybe
  ``demonstrative'' fits the bill; instead of ``element of a type'' we
  would say ``demonstrative of a type''.  Or maybe ``demonstrator''.}
of a propositional type we would call a demonstration of the type, and
an element of a non-propositional type we would call a demonstrative
of the type.\marginnote{So $2$ is a demonstrative of the natural
  numbers; a proof that ``$2$ is even'' is a demonstration that
  expresses just that ``$2$ is even''.}

\begin{ednote}
  Demonstration qua demonstration of know-how?  Expression as
  expression of a type's structure - that is, its inferential
  articulation?
\end{ednote}

In both cases we have demonstration rather than proof of the type.

\begin{ednote}
  ``Demonstrator'' as the genus of ``demonstration'' and
  ``demonstrative''.  It has the virtue of paralleling
  ``constructor''.
\end{ednote}

%%%%%%%%%%%%%%%%%%%%%%%%%%%%%%%%
\section{Semantics}
\label{sect:semantics}

%%%%%%%%
\subsection{Meaning}
\label{subs:meaning}

%%%%%%%%
\subsection{Model-theoretic Semantics}
\label{subs:modeltheorysem}

%%%%%%%%
\subsection{Proof-theoretic Semantics}
\label{subs:proofsem}

``Proof-theoretic semantics is an alternative to truth-condition semantics. It is based on the fundamental assumption that the central notion in terms of which meanings are assigned to certain expressions of our language, in particular to logical constants, is that of proof rather than truth. In this sense proof-theoretic semantics is semantics in terms of proof . Proof-theoretic semantics also means the semantics of proofs, i.e., the semantics of entities which describe how we arrive at certain assertions given certain assumptions. Both aspects of proof-theoretic semantics can be intertwined, i.e. the semantics of proofs is itself often given in terms of proofs.''\cite{schroeder-heister_proof-theoretic_sep}


%%%%%%%%
\subsection{Inferential Semantics}
\label{subs:inferensem}


%%%%%%%%%%%%%%%%%%%%%%%%%%%%%%%%
\chapter{Mathematics}
\label{sect:math}

%%%%%%%%
\section{Traditional}
\label{subs:mathtrad}

%%%%%%%%
\section{Modern: classic}
\label{subs:mathmodclassic}

%%%%%%%%%%%%%%%%%%%%%%%%%%%%%%%%
\section{Modern: Intuitionism}
\label{sect:mathmodintuit}

\begin{ednote}
  Why Brouwer should be deemed a pragmatist.
\end{ednote}


%%%%%%%%%%%%%%%%%%%%%%%%%%%%%%%%
\section{Mathematical Pragmatism}
\label{sect:mathprag}

\begin{ednote}
  \HoTT is (largely) founded on \ML{}'s account of ``judgment''
  (assertion).  I don't know if that's entirely accurate, but it's my
  story and I'm sticking with it for now.  (\ML{} was quite specific
  that his project was motivated by ``purely philosophical''
  considerations.  See his 1972 paper.)  Brandom's account of
  assertion is part of a larger, very ambitious project that aims to
  explain the structure of rationality.  It's a thoroughly pragmatic
  account; everything comes down in the end to ``proprieties of
  practice'': conceptual activity (thinking and talking) is explained
  in terms not of what we know but of what we do (or what we know
  \textit{how} to do).

  Brandom's account of assertion is much more refined and
  sophisticated than \ML{}'s.  If we replace \ML{}'s account with
  Brandom's, then \HoTT comes out as a piece of ``mathematical
  pragmatism'' (or pragmatic mathematics): an account mathematics
  grounded in practice.
\end{ednote}

\begin{ednote}
  TODO: Brandom's philosophy, like most of contemporary pragmatism,
  subverts the dominant representationalist mode of thinking.  It
  turns things upside-down, or inside-out.  So it is with type theory.
  (In one of his papers \ML{} suggests something similar, pointing out
  that his take on judgment etc. reverts (in some sense) back to
  practices that preceded the ways of thinking that have dominated
  modern ``classic'' mathematics and logic.)  The to-do item here is
  to show how the relation of type theoretic to classic thinking in
  mathematics and logic parallels the relation between pragmatist
  (anti-representational, expressivist) thinking and representational
  (cartesian, platonistic) thinking in philosophy, about rationalism,
  conceptual content, etc.  Show how type-theoretic thinking turns
  traditional classic thinking inside-out.
\end{ednote}



%%%%%%%%%%%%%%%%%%%%%%%%%%%%%%%%
%%%%%%%%%%%%%%%%%%%%%%%%%%%%%%%%
\section{Foundations}
\label{sect:foundations}

\HoTT purports to offer a new foundational concept for mathematics.  If
we take assertion to be the foundational concept of type theory (I'm
not sure this works, but it seems plausible), then Brandom's account
of assertion can link type theory to a foundational account of
discursive practice (rationality).

Today set theory is the reigning foundational theory of mathematics.
It's fairly easy to present it as such: first you list the axioms,
then you show how to ``construct'' the natural numbers from sets,
using a successor function.  Or you might follow the lead of the Z
specification notation\sidenote{\cite{zed_spec}}, and proceed from
sets to relations and then to functions.  However you do it, it's all
pretty intuitive and relatively easy to explain, even to mathophobes.

What would such a foundational presentation look like for \HoTT?  If
\HoTT turns out to be a genuinely foundational theory, then it must be
grounded in intuition; specifically, we should expect that its basic
notions correspond in some way to some collection of pre-theoretic
mathematical intuitions, just as the axioms of set theory do, or as
the axioms of geometry match our ordinary intuitions about the
organization of space as we experience it pre-theoretically.

Presentations of set theory usually begin by discussing the axioms;
but even though axioms serve as ``unexplained explainers'', such a
presentation inevitably depends on a yet more primitive layer of
concepts.  Specifically, not only the (pre-theoretical) concepts of
set, subset, and membership, but also axiom and perhaps proof.  All of
these ``preliminary'' concepts---let's call them
``principles''---correspond more or less directly to intuitions
available to any concept-user.

In general, an explicit account of the fundamental \textit{principles}
of set theory is either omitted or informally glossed, before the
presentation moves on to the axioms.  But type theory, in the end, is
radically different from set theory at a very fundamental level, as
far as I can see.  ``Set'' and ``type'' are so easily grasped that it
is easy think of them as more-or-less the same sort of thing; but if
you look hard at them, they are very different, even fundamentally
different.  So I think a presentation of \HoTT would be well served by
beginning with an explicit account of principles, even before moving
on to consider primitives of the theory.

What are the pre-theoretical principles and primitives of \HoTT?  The
obvious place to start is ``type''.  The concept of ``type'' obviously
emerges from ordinary experience; indeed, it is arguably more
primitive than the concept of ``set''.  Just look at the vast
literature on the emergence of categorization in developmental and
cognitive psychology; the ability to categorize is undoubtedly one of
the most primitive human intellectual skills, if not the most
primitive.  It may even be a primitive animal capability--bees
categorize flowers, and every member of sexually reproducing species
categorizes possible mates.

What about ``axiom''?  At first glance it would seem that any
foundational account of mathematics (or anything else for that matter)
must rest on an axiomatic foundation.  Which is just another way of
saying that any explanation of anything must eventually bottom out on
a bedrock of unexplained explainers.  You can't explain everything
without entering an infinite regression.

On the other hand, we can view axiomatic explanation as just one
``style'' of explanation, one of many.  When you begin with axioms,
you present them as unequivocally (and unquestionnably) true.  But
this is really a bit of salesmanship; sometimes axioms turn out not to
be quite as axiomatic as they seem.  Reconceptualizations happen,
which may lead us to view axioms in a new light in which they do not
look quite as certain.  Then axiomatic explanations are still
intelligible, but are no longer unquestionnable.  The classic example
of this sort of evolution is to be found in the history of geometry.
Before the development of non-Euclidean geometries in the 19th
century, the axioms of Euclidean geometry were not only unquestioned
but unquestionable:\sidenote{I suspect I'm overstating the case here.
  Mathematicians: is this true?} the idea that parallel lines could
meet was not just wrong, but crazy.  Today, using axioms to define a
geometry is just a way of making clear the assumptions necessary to
make the theory work.  They no longer represent essential
connections to externally available bits of certain knowledge.

In other words, axioms are not a necessary condition of adequate
explanation.  So the question is whether or not the axiomatic style is
most appropriate for a presentation of \HoTT?  On the one hand, it
seems to me that it is not necessary; an adequate explanation of \HoTT
without axioms should be possible.  For example, we can treat the
concept of ``type'' as primitive, even if we cannot find a good way to
express it axiomatically.\sidenote{This is a little fuzzy; maybe it
  doesn't even make sense.  But as long as we're rethinking the
  foundations of mathematics, we might as well rethink everything.}

\begin{remark}
  Leaving full presentation of principles for later.  I think it
  includes at least proposition and judgment, maybe inference and
  proof.
\end{remark}

In any case, we'll have to begin somewhere, by stating some
fundamental principles; then we'll need an account of the primitives,
whether they take the form of axioms or not.  What are the principles
upon which \HoTT depends?  And once we have some principles (which are
external to the theory proper), what are the primitives (which are
``inside'' the theory)?\sidenote{Ok, ``primitive'' sounds a lot like
  ``axiom''.  But I think there's a difference, even if I can't quite
  articulate it.  Let's provisionally say that a primitive is an axiom
  without the concommitant commitment to unquestionned certainty.}

Here are some possibilities, based on my understanding of the material
in \HoTTB.  Please keep in mind this is coming from somebody who
thinks he has a fairly good grasp of what type theory is all about,
but is still grappling with \HoTT.

%%%%%%%%
\subsection{\HoTT Principles}
\label{subs:hottprinciples}

\begin{description}
\item [Type] Obviously a fundamental concept.  What to say about it, though, is
  not so obvious.
\item [Proposition]
\item [Judgment]
\item [Proof]\sidenote{from Latin \textit{probare} "to make good;
  esteem, represent as good; make credible, show, demonstrate; test,
  inspect; judge by trial" (source also of Spanish \textit{probar},
  Italian \textit{probare}), from \textit{probus} "worthy, good,
  upright, virtuous,"} Two kinds, corresponding to the two kinds of
  provables:\sidenote{Remember, we're talking about pre-theoretical
    principles (concepts) here, not about \HoTT per se.}
\begin{description}
\item [Demonstration] - \textit{rational argument} that compels assent
  to a proposition\sidenote{``Demonstration'' is intuitively
    satisfying, but conceptually misleading, insofar as it suggests a
    visual metaphor.  That would be classical; but for type theory we
    want metaphors of construction, not inspection.}
\item [Witness] - evidence that bears witness to the existence of a kind or category
\end{description}
\item [Inference]
\item etc.
\end{description}

\newthought{The concept of proof in type theory} deserves special
attention.  \Cref{sect:proof} examines it in detail; here, suffice it
to say that it extends beyond the traditional and intuitive notion of
proof as something one does to or with propositions.  In type theory,
propositional types represent propositions, so a type-theoretic proof
of a propositional type---call it a ``tt-proof''--- corresponds to an
ordinary proof of a proposition; it essentially involves inference,
for example.  But type theory also has lots of non-propositional
types, like \N.  These do \textit{not} represent propositions:
propositions have truth-values, natural numbers do not.  In set
theory, there is no connection between sets, elements, and proofs.  An
element either is, or is not, a member of a given set.  Period, full
stop.  The notion of proof never enters the set-membership
picture.\sidenote{That need not mean that proving membership is never
  an issue.  But you don't prove membership; rather, you prove that
  the element satisfies some predicate, which is a different concept.}
In particular, the existence of a set is not dependent on particular
members, and the fact that some element is a member of some set has no
significance with respect to the existence of the set.  By contrast,
in type theory construction of an element of a type counts as proof of
the type.  Etc.\sidenote{FIXME: fix this language.}  But this kind of
``proof'' is not like proof of a proposition; it does not involve a
proposition that may be true or false, and it does not involve
inference.  Instead it serves as a kind of evidence that shows the
type.

\begin{remark}
  Is there a significant distinction to be made between proof and
  witness?  I suspect there is, based on the difference between
  propositions and names.  Both count as evidence, but there is a
  difference between an inferential proof of a proposition and a
  ``testimonial'' witness to a kind.  Propositions-as-types unifies
  the two ideas, but does not erase the distinction.
\end{remark}

%%%%%%%%
\subsection{\HoTT Primitives}
\label{subs:hottprimitives}

\HoTT primitives are ....\sidenote[][-48pt]{A proper exposition would list 1) the name of the
  primitive, e.g. ``\(\Pi\)-type''; 2) the ``constructor'' symbol,
  e.g. \(\cross\) for product types; 3) the analogous concept from set
  theory, and then the ``rules'' for defining a type (formation,
  construction, elimination, computation, uniqueness).}

\begin{description}
\item [Function] ``Unlike in set theory, functions are not defined as
  functional relations; rather they are a primitive concept in type
  theory.''\sidenote{Or: set theory
    \textit{defines} a function as a set of ordered pairs whose domain
    has no duplicates; in other words, it treats a function and its
    ``graph'' as the same thing.  Question: what happens to the graph
    of a function in type theory?} \citep[p. 21]{hottbook}

\item [Product] Product types correspond to cartesian products in set
  theory.  The constructor symbol is the same as in set theory:
  \(\cross\).\sidenote{Why isn't this called the ``\(\Huge\cross\)-type''?}
  ``[U]nlike in set theory, where we define ordered pairs to be
  particular sets and then collect them all together into the
  cartesian product, in type theory, ordered pairs are a primitive
  concept, as are functions.''\citep[p. 26]{hottbook}

\item [Coproduct type] Coproduct types correspond to disjoint unions
  in set theory.  ``In type theory, as was the case with functions and
  products, the coproduct must be a fundamental construction, since
  there is no previously defined notion of ``union of
  types''.\citep[p. 33]{hottbook}

\item [Proposition type] Conceptually, at least, this seems primitive.
  Especially if the concept of ``proposition'' counts as a
  pre-theoretic principle.  Which implies that proof is also a
  pre-theoretic principle.  Propositions are fundamentally different
  than the other kinds of type, since they have truth-conditions. Etc.
  It follows that proofs are fundamentally different from other kinds
  of witness.

\end{description}

\begin{remark}
  For consistency, we might want to use symbols to designate all of
  the primitives, just as we do for \(\Pi\) and \(\Sigma\).  This
  would give us: \(\Huge\fun\)-types, \(\Huge\cross\)-types, and
  \(\Huge +\)-types.
\end{remark}


%%%%%%%%
\subsection{\HoTT Quasi-primitives}
\label{subs:quasiprim}

\noindent ``Fundamental''\sidenote{Obviously we need a better bit of
  terminology.  ``Quasi-primitives''?  ``Neo-primitives''?  These
  types are not primitive, strictly speaking, but on the other hand
  they are basic.  I think there is another fundamental principle at
  work here.  In set theory, for example, the concept of function is
  not only not primitive, it isn't necessary.  You could discard it
  and still have set theory.  But my intuition tells me that e.g. the
  concept \(\Pi\)-type is in a sense necessary or essential in type
  theory, even if it is not primitive.  Once you have the primitives,
  you necessarily have these non-primitive basic types.  Dunno if
  that's correct, but it would sure be nice if it were.} (but
non-primitive) types.  These types seem to be on a par with the
primitive types as far as importance goes, but they presuppose the
primitives, so cannot themselves be considered primitive.

\begin{description}

\item [Universe]  Is this a primitive?  Probably not, since it builds on the type concept.

\item [\(\Pi\)-type] Informally, ``dependent function''
  types.\sidenote{As a practical matter, I think it would be useful to
    have an informal term for these types that falls between
    ``dependent function type'' and \(\Pi\)-type.  Something like
    ``p-function type''.  \(\Pi\)-type is admirably concise, but I
    think it should mention ``function'', since it names a kind of
    function.}  The concept of \(\Pi\)-type is a generalization of the
  concept of function type, so it isn't primitive.

\item [\(\Sigma\)-type] Informally, ``dependent pair''
  type.\sidenote{Shouldn't this be called ``dependent
    \textit{product}'' type?  The type is product, not pair; pairs are
    ``elements'' of the type.  Informally, maybe ``sig-prod type?}
  The concept of \(\Sigma\)-types is a generalization of the concept
  of product type.

\end{description}

%%%%%%%%
\subsection{\HoTT Standard Type Library}
\label{subs:hottstdlib}

\begin{remark}
  By analogy to the usual ``standard library'' of programming
  languages.  The idea is to list commonly used types that are neither
  primitive nor quasi-primitive; ``application'' types, in a sense.
\end{remark}


\begin{description}
\item [Boolean] \citep[p. 34]{hottbook}
\item [$\nat$] \citep[p. 36]{hottbook}
\item [Propsition]
\item [Identity] 
\end{description}



%%%%%%%%%%%%%%%%%%%%%%%%%%%%%%%%
\chapter{Types}
\label{sect:type}

\HoTTB page 27 describes a ``general pattern for introduction of a new
kind of type''.  Martin-L\"{o}f does this too, somewhere.  In \HoTTB,
the list is

\begin{description}
\item [Formation Rules]
\item [Introduction Rules]  or constructors
\item [Elimination Rules] or eliminators
\item [Computation Rules]  ``which express how an eliminator acts on a constructor''
\item [Uniqueness Principle] which ``expresses uniqueness of maps into
  or out of that type.  Optional.
\end{description}


The question is where to place this stuff in the description of \HoTT.
Are these things primitives?  Do they form essential aspects of a
type?  Or in other words, can we have (think of) types without these rules?

\HoTTB introduces them almost as an afterthought, as a Remark in the
third major construction defined in Chapter 1.  But I suspect this is
a mistake or oversight; it looks to me like these rules are indeed
fundamental, essential to the concept of type.  In that case, they
should be presented along with the introduction of the type concept,
rather than in the middle of a description of a particular type.

%%%%%%%%%%%%%%%%%%%%%%%%%%%%%%%%
\section{Terms}
\label{sect:terms}

\begin{ednote}
  ``Terms'' is Awodey's terminology.  More common terminology include:
  witness; inhabitant.  Also proof.
\end{ednote}

``Under the Curry-Howard cor- respondence, one identifies types
with propositions, and terms with proofs...''\cite{awodey_tth}

%%%%%%%%
\section{Witness}
\label{subs:witness}

\begin{ednote}
  In what sense is a proof a witness to a type, or an ``inhabitant''
  of a type?  Intuitively this language does not work very well; we
  don't intuitively think of a proposition as a type ``inhabited'' by
  proofs.  The notion of proof as ``witness'' to a type is a
  substantive epistemological notion; it not only says that the proof
  is related to the type, but also it says something about the nature
  of that relationship.

  The trick is to see it from the perspective of the machine.  A
  proposition like \(1+1=2\) is just a form to the machine.  We can
  see that it is true just by looking, due to some mysterious
  epistemic capability.  But machines do not have epistemic abilities;
  a form is a form is a form to a machine.  Hammer, nail.  So in order
  for the machine to treat \(1+1=2\) as a \emph{true} proposition, we
  have to give it something more: a proof.  But ``proof'', again, is
  an substantive epistemic notion; the machine analog must be purely
  formal.  From the machine perspective, a proof is just another form,
  or rather, collection of forms (including inference rules as complex
  forms): to give the machine a proof of P we must provide it with a
  form or forms that ``lead to'' (produce, result in) P.  To prove a
  proposition to a machine, we give it forms and reduction rules such
  that the formal use of those forms and rules results in the form of
  the proposition to be proved.  (FIXME: a more accurately way of
  putting this would involve reduction of formulae to normal form,
  confluence, etc.)

  So we can think of a proof as a kind of device---just another
  machine (or machine description), but one whose sole output is the
  proposition to be proved.  Since for any given proposition there may
  be many ways of building such a proving device, we can treat these
  devices as forming a kind of equivalence class, which we can
  identify by taking (the form of) the proposition as a symbol
  referring to the class.  Now the connection to types and witness
  becomes clear: the equivalence class of such proving devices forms a
  type, the type of the devices (proofs), and each device (proof)
  ``inhabits'' (or as we would prefer, expresses) the type.
\end{ednote}


%%%%%%%%%%%%%%%%%%%%%%%%%%%%%%%%
\chapter{Curry-Howard}
\label{sect:curry-howard}

\section{Two kinds of proof}

Proof in logic and math: \textit{discursive} proof.

Proof in ordinary language: evidence, demonstration, etc.

Empirical v. logical proof.

Example: to prove to you that there is a coin in this purse I can open
the purse and display the coin.

The ``proof'' of the proposition is thus a performance of a certain
kind: a proof-performance.  But \textit{kinds} are abstract; the
specific performance should thus be construed as a \textit{token} of
the kind.  In this particular case (displaying a coin in a purse), the
performance can be repeated.  Of course, each performance will differ
from all the others in its fine detail, but insofar as each repeat
performance counts as proof \textit{of the same kind}, each counts as
a \textit{proof token} of the same type.

But it is not merely a proof-token; a proof-token of the type that
proves a particular proposition.  Every specific proof-token is a
proof of a particular proposition P.  So a given performance of this
sort - a repeatable proof-token - is classifiable as a proof-of-P
token (=performance).  The notion of \enquote{proof token} is a
generalization over proof-of-P tokens for all P.

In the case of ordinary provings like the example given above -
consider the sort of ``proving'' that goes on in schoolyards, where
proving means showing - proof is non-discursive: it does not involve
explicit reasoning.  Caveat: this may count as a kind of empirical
proof, but is not to be confused with inductive reasoning.  What makes
a given performance count as a proof is a deep question we won't go
into here, but we all know that e.g. displaying a coin in a purse
counts as proof of the proposition that the purse contains a coin.

NB: the original proposition ends up as the conclusion of a piece of
practical reasoning: I see a coin in the purse; therefore there is a
coin in the purse.

In the case of mathematical and logical reasoning, proof involves
discursive performance.

Written proofs as traces of discursive performances.

The proposition to be proved ends up as the conclusion of the proof.

The type of a proposition (statement, etc.) is \textit{not} the type
of its proofs.  The type of a proof of P is exactly
\textit{proof-of-P}, not P.  More exactly,
proof-whose-\textit{conclusion}-is-P.  That's the kind of thing such a
proof is: it's the sort of thing that counts as a proof of (proves) P.

So what is the type of a proposition?  The question is malformed; what
we really want to know is, what is the type of a proposition
\textit{token}.  The answer is obvious but hard to articulate clearly
in English, due to the inherent circularity of the type/token
distinction.  The type of a token is just its type; the tokens of a
type are, well, its tokens.  This page has many tokens of type
\enquote{the}.

\subsection{Tokens, Terms, Types}
\label{subsec:tokens-terms-types}

To communicate clearly about these issues, we need special notation.
Quote marks are insufficient; they turn an expression into a name of
the expression.  The famous example (Tarski's convention T) is:

\enquote{Snow is white} if and only if snow is white.

This sentence contains two tokens of type `snow is white'; the first
is mentioned, the second used.  Technically the quoted version
functions as a name (thus mention) of the sentence, while the unquoted
version is just the sentence (used).  The quoted version does
\textit{not} denote or indicate the type of the sentence.  For that we
can use a designated notation such as \(\ulcorner \urcorner\), so that
\(\ulcorner\)3\(\urcorner\) refers to the type of tokens of the form
`3'.  Call these token-type quote marks.  When we need to explicitly
refer to some symbol \textit{qua} token, we use \(\llcorner \lrcorner\)
and write \(\llcorner 3\lrcorner\).  So \(\llcorner 3\lrcorner\) is a
token of type \(\ulcorner 3\urcorner\).

\subsection{Structure of Proofs}
\label{subsec:structproofs}

So: we have a proposition P, and we have a discursive proof of P.
What \textit{kinds} of things are involved here?  What is the
structure of the kinds?  Kinds rather than types, because we want to
reserve the notion of type for syntactic duty: a type system is a kind
formal notation that combines the notions of syntactic calculus and
kindedness.

A proof of P is a proof token whose conclusion is a token of type P
(and type P is in turn a token of type Proposition.)

The \textit{written} form of the proof token as a trace of a proof
process or computation.  So conclusion of a (static) written proof ~
end result of a (dynamic) computation or construction.

Compare proof of an int and proof of a proposition.  An integer symbol
like \(23\) is a formula, just like a propositional formula.  It
denotes a device that computes a result.  This is true even of
``simple'' symbols like \(3\): in contrast to the denotational
perspective, under which \(3\) simply denotes the integer, under the
constructive perspective the symbol \(3\) denotes a device that
computes the integer.  The type/token distinction applies here just
like it applies to propositions and proofs: a computation (proof) of
\(3\) is a process/proof/computation/whatever whose conclusion is a
token of type 3.  That type in turn is a token in the type Z
(integer).  Similarly, the type of a proposition token is a token of
type Proposition (or we might call it type Provable, or Proven or even
True or the like).

Remark: token and term.  Same thing?  Not really.  Term contrasts with
type, just like token, but at a different level of abstraction.  By
example: 3 (on the page) is a token of type \(\ulcorner\)3\(\urcorner\),
which in turn is a term of type Z (here ``type Z'' means type of
values, rather than \(\ulcorner Z\urcorner\), the type of `Z' tokens).

Now how does this related to Curry-Howard?  In particular proof checking etc.?

We can interpret the usual formulation \enquote{a proposition is the
  type of its proofs} to be an abbreviated way of saying that the type
of a proposition serves to categorize proofs whose conclusions are
tokens whose type is the proposition.

Key concept: token-repeatability.  In the example of pulling a coin
from my pocket in order to prove the proposition that there is a coin
in my pocket, once I have performed the proof, I cannot repeat it,
since the coin is no longer in my pocket. (See: linear logic.)  But if
the proposition is that there is a coin in purse, I can prove it by
opening the purse and displaying the coin.  Since the coin stays in
the purse, I can repeat this proof as often as I like: produce as many
proof-tokens of this kind as I wish.

In the case of formal logic and computation, proofs are repeatable.

\section{Proofs and Propositions}

The usual formulation is something along the lines of \enquote{a
  proposition is the type of its proofs}.  But this obviously cannot
be quite right: propositions and proofs are distinct \textit{kinds} of
things, so how can a proof be a kind of proposition?.  We would never
say that a building is the type of its blueprint; why say that a
proposition is the type of its proofs?

The problem is that the standard terminology ``forgets'' about
computation.  They type of a compound expression is by definition the
type of its \enquote{result}.  In the case of mathematical
expressions, the result (of a computation) is a value of a certain
type; in the case of propositions, the result (of a proof) is either a
proposition or a truth value, depending on your preferred perspctive.
In both cases, it would be more accurate to talk of both the type of a
computation and the type of the result of a computation.



\section{Misc. notes}
\begin{ednote}
  Usually presented as ``propositions-as-types'', but this suggests an
  asymmetrical relationship; in fact the principle is that
  propositions \emph{are} types, and vice-versa.  This is a major move
  in type theory, introduced by \ML(?) based on work by Curry and
  Howard.  TODO: what exactly are the implications of this principle?
\end{ednote}

\begin{ednote}
  The critical point is that we go minimalist: start with the minimal
  logical language, which means combinatory logic.  It is the
  isomorphism between the logical constants and the combinators
  (Curry) that motivates Curry-Howard.  Once you see the connection at
  this minimalist level, it is easy to see it at any level, since the
  logical constants are the basic building blocks from which all
  propositions are constructed.
\end{ednote}


\begin{ednote}
  Start with Schoenfinkel and Curry, and the goal of finding the
  absolute minimum, which means eliminating variables.  Then the basic
  combinatorys, then the isomorphism to the logical constants.

  Equivalence of combinatory logics (no vars) and lambda calculus (vars)
\end{ednote}


Analogies.  Proof/proposition, term/type: ``There is also a one-to-one
correspondence between proofs of a certain proposition in constructive
predicate logic and terms of the corresponding type.'' (Dependent Types at Work)

%%%%%%%%%%%%%%%%%%%%%%%%%%%%%%%%
\section{bhk}
\label{sect:bhk}


\begin{ednote}
  Importance of metaphors.  See ML on BHK: proof as task to be
  accomplished, problem to be solved.  Add another metaphor:
  destination to be reached.
\end{ednote}


%%%%%%%%%%%%%%%%%%%%%%%%%%%%%%%%
\section{Assertion and Judgment}
\label{sect:assertionjudgment}

%%%%%%%%
\subsection{notes}
\label{subs:notes}

This section needs some serious revision.  Here's the straight dope,
in a nutshell.  In his paper ``Truth of a proposition, evidence of a
judgement, validity of a proof''\citep{martin-lof_truth_1987}, which
is specifically about the philosophical basis of \ITT{}, \ML{}
attempts to explain the concepts proposition, truth, evidence,
proof, and validity.  The first part of the paper, which gives some
historical and conceptual background, is just right for the most
part.  He points out, for example, that for the intuitionist proof
comes before truth.  But he makes a major blunder when he claims
that the classic truth-conditional account of the logical
connectives, an account that is based on truth table semantics, and
the BHK accounts, which treat a proposition as an expectation or
task etc., are just different ways of saying the same thing.
``Façons de parler'', as the saying goes.  But I think this is flat
out wrong.  Classic and intutionistic logic may use the same
formulas, but they could not be more different conceptually.
Classic truth-conditional logic presupposes something like what
Wittgenstein called (in his Tractatus days, at least) a picture
theory of meaning. (I may not be getting the exact wording right
here, but the idea should be clear enough.)  Proof in that kind of
logic has nothing to do with construction; it's all about
correspondence, a representational relation between language and
objective reality.  Obviously there's much more to be said about
this, but suffice it to say that \ML's claim that classic logic and
intuitionistic BHK logic are in the same line of business strikes me
as not only wrong but a little bit shocking.  So wrong that I have
to wonder why he made that sort of claim.  Maybe he was unfamiliar
with the pragmatist literature.  And we'll just proceed on the
assumption that I am not wrong, if you don't mind.  I'll provide a
more detailed justification of this claim later.

Another thing that looks wrong to me is his account of the BHK
interpretation of propositions; but in this case, he has an excuse:
R. Brandom's more refined account of assertion and proposition was
not yet available.  Brandom's account makes the problems with \ML's
account quite clear.  The latter follows BHK in treating a
proposition as a problem to be solved, a task to be accomplished, or
an expectation of a proof, etc.  The problem is that propositions
are clearly exactly \emph{not} these things.  Obviously we may
\emph{treat} a proposition as e.g. a task to be accomplished; but
that does not determine what a proposition \emph{is}.  Or to put it
differently \ML{} seems to ignore the significance of \emph{force},
which is distinct from conceptual content.

For now I don't have time to explicate the point in detail, so
here's the short version: Brandom divides assertion into commitment
and entitlement.  And what makes the proposition primitive is that
it is the minimal unit of \emph{responsibiity} - Brandom traces this
notion to Kant.  To assert a proposition is to undertake a
commitment to it, and also to license others to challenge ones
entitlement to that commitment.  Thus it inescapably involves a kind
of responsibiity: the responsibility to justify (``vindicate'', as
Brandom says) one's commitment.  One way to do this is to
demonstrate the entitlement by giving \emph{reasons} for it.

If you're familiar with the \ML{} paper mentioned above, the
connection should be fairly obvious.  Commitment and entitlement are
deontic attitudes, which institute deontic statuses (e.g. being
correct or incorrect).  They are emphatically \emph{not} properties
of propositions.  So it is just a mistake to think that propositions
are or express tasks, problems, or expectations.  On the other hand,
\emph{assertion} of a proposition does give rise to a responsibility
to vindicate commitment.  Talk of ``expectation of a proof'' is
entirely intelligible as a way of saying that assertion licenses
others to challenge one's commitment---to expect that one can or
will prove it.  Talk of task or problem is really a way of talking
of the justification or vindication that one is responsible
for.\sidenote{And note that this is not a mere matter of voluntary
  acceptance of responsibility; it arises because assertion licenses
  others to \emph{hold} one respondible, treat one \emph{as}
  responsible, and therefore sanction speakers who fail to vindicate
  their commitments.}

So in the end, \ML{} is speaking more or less the right vocabulary,
but his explanation is off, and his characterization of propositions
as involving something in addition to propositional content
(e.g. expectation, task, etc.) is not defensible, at least from a
Brandomian perspective.  \HoTT{}, unfortunately, duplicates his
error in its account of judgment.

The remedy is close at hand, though.  All we need do is recognize that
what \HoTTB calls ``judgments'', like ``a : A'' and ``a := b'', are
really \emph{stipulations}, and what it calls propositions,
\emph{assertions}.  A stipulation, unlike an assertion, does not
require justification.  A stipulator does not license listeners to
demand reasons for the stipulation.  Of course they can make such
demands, but almost by definition we can stipulate \emph{ad libitum}.
Listeners who don't like our stipulations can go elsewhere; but no
\emph{rational} challenge can be mounted against a stipulation.  The
reason it makes no sense to ask for a proof of ``a : A'' is not
because it is a ``judgment'' but because it is a stipulation and
therefore needs no justification.  By contrast, ``propositional
equality'' is just a fancy (and rather unfortunate) term for
``asserted equality''.

Another way to look at it: \HoTT, following \ML{}, attributes
special properties to propositions.  But that's hard to defend,
philosophically, and doesn't amount to genuine explanation; a better
way is to explain assertion in terms of deontic attitudes and
responsibilities.

%%%%%%%%
\subsection{Judgment}
\label{subs:judgment}


The account of judgment offered in the HoTT Book doesn't really work.
Ditto for Martin-L\"{o}f's account.  For example, it makes sense to
say ``P is a proposition'', but it doesn't make sense to say ``P is a
judgment''.  That's because judgment is a act, something one does.

On the other hand, ``judgment'', like ``proposition'', can be treated
as a verbal noun or as a ``plain'' noun.  Saying ``P is a
proposition'' is usually taken to mean that P refers to what has been
proposed.  There is no obvious reason not to treat ``P is a judgment''
in a similar manner: P refers to what has been judged.

However, there is a difference.  Judging a proposition (what was
proposed) amounts to \textit{evaluating} what was proposed, as good or
bad, true or false, or whatever.  By contrast, proposing a proposition
amounts to merely exhibiting it for consideration.  This arguably
involves an implicit evaluation - to propose a proposition is to
implicitly claim that it is good, or true, etc.  But proposing does
not involve offering an evaluation that is distinct from what is
proposed, whereas judgment does.  The two are distinct kinds of speech
act, and refering to the content of a speech act is not the same as
referring to the speech act itself.

Furthermore, it is not correct to treat the nominal sense of
``judgment'' as being the content, what has been judged.  The nominal
sense of ``judgment'' refers to the act of judgment itself, and not
the proposition judged.

Actually, by the same reasoning it is not correct to say that the
nominal sense of ``proposition'' is what-is-proposed; rather, it is
the act proposing, nominalized.  This makes perfect sense when you
consider that ``proposing'' can also be nominalized; ``the proposing''
is another way of saying ``the proposition''.

The same goes for all -tion words: suggestion, opposition, etc.  In
each case, the word can refer to the doing, or to what is done, and
what is done is always the act of doing itself -- not the subject or
object of the doing.

This suggests we should make a distinction between, for example, the
content of a proposition and ``proposition''.  But this term seems to
be a special case; it has the usual plain noun sense of
what-was-proposed, the usual verbal sense of ``proposing'', but also
the nominalized verbal sense of ``act of proposing''.

(But then the same considerations apply to ``judgment''.  The
difference must go back to semantics.)

\begin{remark}
  The Arabic grammatical tradition captures this distinction
  beautifully, mainly because the structure of the language makes it
  simple to do so.
\end{remark}

Or put it this way: when we judge a proposition like ``2+2=4'' to be
true, the what-was-judged is not ``2+2=4'' but the truth of ``2+2=4''.

\begin{remark}
  But how is this different from ordinary predication, like ``The
  triangle is red'' as a proposition?  Should we say that what is
  proposed is not that the triangle is red, but the redness of the
  triangle?  No, since we're treating it as a propostion, and the
  whole thing is proposed (exhibited).  If we judge it to be true,
  then again the judgment 
\end{remark}

So saying ``P is a judgment'' is incoherent if P is taken to refer to
nothing more than what is proposed.  If P refers to a claim of the
form ``X is true'' (or good, etc.), then ``P is a judgment'' seems to
make more sense; but it doesn't, really.  P still refers to an
unasserted content; to make sense, we would have to say something like
``P is a judgment when asserted''.  More explicitly, ``'X is true' is
a judgment'' (or better, ``'X is true' expresses a judgment'') only
\textit{exhibits} ``X is true'', which is a proposition, not a
judgment.  As a proposition it expresses a judgment; but when embedded
(equivalently, quoted) it does not express anything.

\begin{remark}
  Compare: ``Snow is white'' iff snow is white.  The quoted bit is a
  name of the sentence; it counts as a \textit{mention} of the
  sentence, which has no force.  The unquoted version of same is the
  sentence itself; it counts as a \textit{use} of the sentence, which
  has assertional force.  Obviously, the occurances of ``P'' in ``P is
  a proposition'' and ``P is a judgment are names of a proposition and
  thus mentions.  So they have no force.
\end{remark}

The key point is Frege's point: the content of a proposition is
distinct from the force of the utterance.  That means that P in ``P is
a proposition'' is unasserted, just as it is when embedded, as in ``If
P then Q''.  The truth of ``P is a proposition'' is independent of the
truth of P.

So even if we take the act of declaring ``P'' to be an act of
judgment, it does not follow that a reference to P is a reference to
the act of judging that P.  Hence there is no way to make ``P is a
judgment'' work.  If we take P to refer to what was judged, that again
is a proposition (or propositional content), so ``P is a judgment'' is
incoherent.

\begin{remark}
  We can assert that P, and we can assert P.  We can judge that P, but
  we cannot judge P.  I don't think this is a mere grammatical
  distintion; I think it reflects a genuine semantic difference.
\end{remark}



\documentclass{article}

\usepackage{hottmacros}

\usepackage{fontspec,xltxtra,xunicode}
\usepackage{fontspec}
\defaultfontfeatures{Scale=MatchLowercase}

%% \setmainfont[Mapping=tex-text]{Times New Roman}
%% \setromanfont[Mapping=tex-text]{Times New Roman}
%% \setsansfont[Mapping=tex-text]{Arial}

\setmainfont[Mapping=tex-text]{TeX Gyre Pagella}
%% \setromanfont[Mapping=tex-text]{TeX Gyre Pagella}
%% \setsansfont[Mapping=tex-text]{TeX Gyre Heros}

\usepackage{tikz}
%% \usetikzlibrary{arrows,shapes,patterns,backgrounds,spy}
%% \usepackage{pgffor}

%% \usepackage{animate}

%%%%%%%%%%%%%%%%%%%%%%%%%%%%%%%%%%%%%%%%%%%%%%%%%%%%%%%%%%%%%%%%
\begin{document}
\title{HoTT without Equality}
\maketitle
\tableofcontents
\vfill
\large

Often said that what's distinctive about HoTT is its treatment of
equality.  But equality is an optional concept.  We offer two other
ways to get the same thing, without mentioning equality: unit
families, and successor families.

Actually we could do three: units, successors, and relations.  All
have the same structure but distinct symbols and intuitive
interpretations.  None depend on a concept of equality.

Equality emerges as a special case in the ordinary economy of types
and tokens.  Or rather one case; there's nothing special about it.

The real innovation is the introduction of a novel concept of
``induction'', which is motivated by problems that arise from the
possibility of non-canonical tokens.

Whether ``induction'' is the appropriate term is another matter.

\section{Unit Types}

This perspective is suggested by the observation that the equality
type in HoTT looks a lot like a Unit type, in that it has a single
constructor.

\subsection{Atomic Units}

\begin{align}
  & \UnitA{:}\,\SuccT & \textit{family of unit types from}\ A \\
  & a{:}A,\ \UnitA(a):\Univ & \textit{unit type at a from A} \\
  & p{:}\,\UnitA(a) & \textit{proof of unit type at a} \\
  & S_{\UnitA(a)}\prod\limits_{x{:}S_{A(a)}}\Univ & \textit{family of unit types at elts of unit type at a} \\
  & q{:}\,\UnitA(a) & \textit{proof of unit at a} \\
  & S_{\UnitA(a)}(q):\Univ & \textit{unit at q from unit at a from A}
\end{align}

\subsection{Dependent Unit Types}

\begin{align}
  & \prod\limits_{X:\Univ}\prod\limits_{(x,y:X)}\Univ & \textit{type of \(3^{\circ}\) family of unit-types} \\
  %% && \prod\limits_{X:\Univ}\prod\limits_{(x,y:X)}\Univ:\equiv\ \prod\limits_{X:\Univ}\prod\limits_{(x:X)}\prod\limits_{(y:X)}\Univ \notag \\
  & \textsf{Unit}{:}\prod\limits_{X:\Univ}\prod\limits_{(x,y:X)}\Univ & \textit{arbitrary \(3^{\circ}\) family of unit-types} \\
  & \UnitA{:}\DsuccT & \textit{arbitrary \(2^{\circ}\) family of unit-types from A}\\
  & \UnitA(a){:}\prod\limits_{(y:A)}\Univ & \textit{\(1^{\circ}\) family of unit-types at a from A} \\
  & \UnitA(a)(b){:}\,\Univ & \textit{unit type at b for a from A} \\
\end{align}

We can think of \(\UnitA(a,b)\) as expressing the unification of \(a\)
and \(b\), rather than their equality: together they form a binary
unit, or something like that.

As usual, we will drop the type symbol when the context makes it
clear, writing \(\Unit(a,b)\) instead of \(\UnitA(a,b)\).

\subsection{Unit Proofs}

What's the intuition for unit proofs?  Reflexivity is out-of-bounds.


\subsubsection{Canonical Constructors}

Since the concept of reflexivity depends essentially on a concept of
equality, we need a different concept here.  Actually we do not need a
concept, we just need a constructor.  Users can conceptualize it in
whichever way they please:

\begin{align}
  &\star{:}\prod\limits_{X{:}\Univ}\prod\limits_{(a:X)}\Unit^X(a,a) \\
  &\star^A{:}\,\prod\limits_{(a:A)}\UnitA(a,a) \\
  &\star^A(a){:}\,\UnitA(a,a) & \star a{:}\,\Unit(a,a)
\end{align}

So \(\star a\) has type \(\Unit(a,a)\); that is all.  No further
meaning is to be read into it; in particular, it has nothing to do
with reflexivity or equality.

\subsection{Unit Induction}

Following the definition of path induction: the principle of unit
induction says that to prove something for all \(p{:}\,\Unit(x,y)\),
it is sufficient to prove \(\star{:}\,\Unit(x,x)\).

The intuitive justification for this principle is that ...?


\section{Successor Types}

For any type \(A\) we may form \(\succt\); this is the type of
\emph{families of successor types from \(A\) at \(a\)}.  Given
\(a{:}A\), we can specify a successor family \(S^A\) for \(A\) by
writing, \(S^A{:}\succt\); then \(S^A(a){:}\Univ\) is a
\emph{successor type at \(a\) from \(A\)} or just \emph{successor type
  at \(a\)} if the base type is evident.  There are as many successor
families for \(A\) as there are elements of \(\succt\), and for each
one, as many successor types at \(a\) as there are elements of \(A\).

The motivation for calling these successor families and types is that
\(\succt\) takes an element from one level to the next in the type
hierarchy.  Given \aA, if \(A\) is level \(n\), then \(a\) is at level
\(n-1\), so \(S^A(a)\) is at \(n\), the same level as \(A\).

We can make this explicit and general by treating each type \(A\) as a
universe, writing \(\Univ_{n+}\) instead of \(A\).  Then to designate a
family of successor types for \(A\) we will write \(S^A\), which we
can express this formally using a \Pi\ type:

\begin{remark}
  We can write \(a{:}A\succ S^A(a)\) or the like to express
  successorship quasi-formally.
\end{remark}

\begin{align}
  & S{:}\prod\limits_{A{:}\Univ_{n+1}}\prod\limits_{(a:A)}\Univ_{n+1}
\end{align}

Then we can expose the inductive structure:

\begin{align}
  & S^A{:}\prod\limits_{(a:U_{n})}\Univ_{n+1} \\
  & a{:}U_{n} \\
  & S^A(a){:}U_{n+1} & (a\succ S^A(a))
\end{align}

Thus for any element \(a\) of any universe \(\Univ_n\) we can find a
family of successor types \(S^{\Univ_n}\) such that
\(S^{\Univ_n}(a){:}\Univ_{n+1}\).

Since every type may be treated as a proposition, this means that
every element of every type is associated with a proposition.

\subsection{Successor Iteration}

We can iterate successors for atomic types.

\begin{align}
  & S_A{:}\,\SuccT & \textit{family of successor types from}\ A \\
  & a{:}A,\ S_A(a):\Univ & \textit{successor type at a from A} \\
  & p{:}\,S_A(a) & \textit{proof of successor type at a} \\
  & S_{S_A(a)}\prod\limits_{x{:}S_{A(a)}}\Univ & \textit{family of succ types at elts of succ type at a} \\
  & q{:}\,S_A(a) & \textit{proof of proposition at a} \\
  & S_{S_A(a)}(q):\Univ & \textit{successor at q from successor at a from A}
\end{align}

and so on, \emph{ad infinitum}.


\subsection{Dependent Successor Types}

Same as relation types?  Conceptually, not so much?

We can also form ``complex'' successor types.  For example, we can
express relation types as dependent successor types.  Note that in the
following each \(\Pi\) operator corresponds to a family; e.g. a
\(3^{\circ}\) (third order) family is a family of families of families.

NB: in what follows we will try to follow the following terminological
regime: successor type \emph{at} element a \emph{for} element b
(meaning dependent on element b) \emph{from} type.

\begin{remark}
  FIXME: for successor types use \(S^A\) instead of \(\circ^A\)?
  Reserve the latter for relation types?  Do we then need distinct
  \(S\) symbols for unary, binary etc. successor families, or can we
  overload \(S\)?
\end{remark}

\begin{align}
  & \prod\limits_{X:\Univ}\prod\limits_{(x,y:X)}\Univ & \textit{type of \(3^{\circ}\) family of successor-types} \\
  && \prod\limits_{X:\Univ}\prod\limits_{(x,y:X)}\Univ:\equiv\ \prod\limits_{X:\Univ}\prod\limits_{(x:X)}\prod\limits_{(y:X)}\Univ \notag \\
  & S{:}\prod\limits_{X:\Univ}\prod\limits_{(x,y:X)}\Univ & \textit{arbitrary \(3^{\circ}\) family of successor-types} \\
  & S^A{:}\DsuccT & \textit{arbitrary \(2^{\circ}\) family of successor-types from A}\\
  & S^A(a){:}\prod\limits_{(y:A)}\Univ & \textit{\(1^{\circ}\) family of successor-types at a from A} \\
  & S^A(a, b){:}\,\Univ & \textit{successor type at b for a from A} \\
  %% & &  (a\circ_A b) :\equiv \circ_A(a)(b) :\equiv ((a\circ_A)b) \notag
\end{align}

Note the regimented vocabulary: \(S^A(a,b)\) is \emph{a} successor
(type) \emph{at} b \emph{for} a \emph{from} A, where \emph{for a}
means dependent on \(a\), and \emph{from A} means that \(A\) is the
base type from which the elements are drawn.

\begin{remark}
  Two kinds of ``forward motion'': from element \(a\) to ``next''
  element \(b\), and from element \(b\) to ``next'' type \(S^A(a,b)\).
\end{remark}

\begin{remark}
  Since \(S^A(a,b)\) goes from element to type we can think of it as a
  kind of ``next'' that crosses type levels.  But since \(S^A(a,b)\)
  is at the same level as \(A\), we can also think of it as a kind of
  next after \(A\) at that level, i.e. inductively generated by \(A\).
\end{remark}

Structurally this presentation is exactly the same as the Unit
presentation; the sole difference is in how we choose to interpret
things intuitively.

\subsection{Successor Proofs}

What's the intuition for successor proofs?  Reflexivity is out-of-bounds.

\subsubsection{Canonical Constructors}
Canonical constructor for dependent successor types?  Why not Latex'
``succ'' symbol: \(\succ\)?

For that matter, why not write \((b\succ a)\) instead of \(S(a,b)\)?
Because that symbol is for succession within a domain, whereas our
\(S\) is for crossing domain boundary.  Besides, it would be
intuitively incompatible with equality.

But we could write \(b\succ S(a,b)\) or the like.

\subsection{Dependent Successor Iteration}

From \(p,q{:}\,S(a,b)\) to \(S^{S(a,b)}(p,q)\), to \(S^{S^{S(a,b)}(p,q)}(r,s)\), etc.
etc.

\subsection{Successor Induction}

To prove something for all \(p{:}\,S(a,b)\) it suffices to prove it for \(\succ{:}\,S(a,a)\).

%%%%%%%%%%%%%%%%%%%%%%%%%%%%%%%%%%%%%%%%%%%%%%%%%%%%%%%%%%%%%%%%
\section{Relation Types}

\begin{remark}
  Here again we have the same structure with different symbols and a different interpretation.
\end{remark}

\begin{align}
  & \circ{:}\prod\limits_{X:\Univ}\prod\limits_{(x,y:X)}\Univ & \textit{arbitrary \(3^{\circ}\) family of relation-types} \\
  & \circ^A{:}\DsuccT & \textit{arbitrary \(2^{\circ}\) family of relation-types from A}\\
  %% & \circ^A(a){:}\prod\limits_{(y:A)}\Univ & \textit{\(1^{\circ}\) family of relation-types at a from A} \\
  & (a\circ^A b){:}\,\Univ & \textit{type of relation (pair) (a, b)} \\
  %% & &  (a\circ^A b) :\equiv \circ^A(a)(b) :\equiv ((a\circ^A)b) \notag
\end{align}

\begin{remark}
  Think of \(\circ\) as a relation on pairs (i.e. \(A\times B\))?
\end{remark}

Classically, a binary relation is a relation between two elements of a
set; in type-theoretic terms we can think of this as a relation
between two elements of the same type.  But if we treat binary
relation types as dependent successor types the nature of the relation
changes.  Instead of a ``horizontal'' relation between two elements at
the same level (so to speak), a binary relation is a ``vertical''
relation between elements of different types, one constructed from the
other.  The critical point is that \(a\circ b\) is a \emph{type}; not
as a ``horizontal'' relation between \(a\) and \(b\) within \(A\), but
as a successor type that is ``vertically' related to \(b\) (for
``previous'' element \(a\)).  At the same time, \((a\circ b)\) is
``horizontally'' related to \(A\) itself, since \((a\circ b)\) is a type at
the same level as \(A\).

We can iterate successor types for relations just as we did for atoms.
Given \((a\circ_A b)\), then for any element \(p\) of that type --
i.e. proof of the relation -- we can form a successor family \(\circ_{(a\circ b)}\):

%%  is a successor type from \(\circ_A\) at \(b\) for
%% (dependent on) \(a\),

\begin{align}
  & \relAdef &  2^{\circ}\textit{family of relation-types from A} \\
  & \relrelAdef & \\
\intertext{\hfill\(\relrelA\) \textit{= family of reln types for successor type at y for x of}\ \(A\)} \\
  %% & a{:}A, b{:}A \\
  %% & \circ_{(x\circ_A y)}(a,b){:}\prod\limits_{p,q{:}(x\circ_Ay)}\Univ \\
  %% & p{:}(a\circ_A b), q{:}\,(b\circ_A a) \\
  %% & \relrelA(a,b,p,q):\Univ \\
  %% & (p\relrelA q) :\equiv \relrelA(a,b,p,q).
  & (p\relrelA q){:}\,\Univ & \textit{successor type at q for p, from type}\ (x\circ_A y)
  %% \intertext{\textit{\hfill ---successor type for dependent successor type \((a\circ_A b)\) at \(p{:}(a\circ_Ab)\)}}
\end{align}

Normally we will write \((p\relrelA q)\) instead of
\(\relrelA(a,b,p,q)\), or just \((p\circ q)\) if the parameters are
evident from the context.  Then we will say that \((p\relrelA q)\) is
``a successor type of \(\relA\) at \(q\) for \(p\)''.

NB: if we really wanted to, we could also iterate \(\circ_A\) itself,
since it is an element of a function type.  Giving
\(\circ_{\circ_A}\); I'm not sure what it means.

%% and we
%% can go from \(p{:}(a\circ b)\) to a successor type \(S_{(a\circ b)}(p):\Univ\).



\subsection{Reflexivity Types}

The types of relation type families can be partitioned into two classes:

\begin{itemize}
\item \textit{reflexivity relation types} of the form \((a\circ_A a)\)
\item non-reflexiivty types of the form \((a\circ_A b)\)
\end{itemize}

\begin{remark}
Every relation type family has reflexivity types; we'll call these the
reflexivities of the family.  Not all reflexivities are inhabited
(e.g. \(a < a\)), but for a given family, if one is they all are(?).
We'll call any family whose reflexivities are inhabited a reflexivity
family.

All equality families are reflexivity families, but not all
reflexivity families are equality families (e.g. \(a \leq a\)).  To
get equality types, we need to add \(\reflA\) constructors to
relation types, to make sure their reflexivity types are inhabited.
In addition, ...
\end{remark}

Reflexivity successor types are self-dependent: successor at x for x.

\subsubsection{Constructors for Reflexivities}

Instead of \(\reflA\) we use the more general \(\rA\):

\begin{align}
  &\bullet{:}\,\rfamU \\
  &\rA{:}\, \rfama \\
  &\rA(x){:}\, \reflTAx & \bullet x{:}\,\reflTx
\end{align}

\(\rA\) works for both canonical and non-canonical tokens of the base type.

We the type is clear from the context we will drop the annotation and
write just \(\bullet x{:}\,\reflTx\).

\subsection{Symmetry Types}

The HoTT book ``proves'' symmetry by working with a function type
\((x=_Ay)\to (y=_Ax)\).

Here, we stick to successor and relation types and show how to
construct symmetry.  In any relation family, some will be symmetric
and some won't.  The task is to explain what it is to be symmetric,
i.e. what are the \emph{defining constructions} for symmetry.

The types of every successor type family \(\relA\) may be partitioned
into classes:

\begin{itemize}
\item Reflexivity types: types of the form \(a\circ_A a\)
\item Symmetry types: pairs of types of the form \((a\circ_A b), (b\circ_A a)\)
\end{itemize}

These types may or may not be inhabited for any given dependent
successor type family.  If the reflexivity types are inhabited
\emph{for all a}, we say the (relation) family is reflexive.

The rule for symmetry is a little bit different.  For any given
family, some symmetric pairs may be inhabited and some may not be
inhabited.  Or one member of the pair may be inhabited and the
other not (so we have four possibilities.)  To count as symmetric,
either both members of a symmetric pair must be inhabited or neither
member can be inhabited.  In other words, if we can find a symmetric
pair only one of whose members is inhabited then it is not a symmetric
relation family.

For example, the relation \(\neq\) is symmetric for \(\Nat\), but
\(\leq\) is not.

\begin{remark}
  TODO: Give example from HoTT of relation where some symmetric pairs
  are inhabited and some are not.  Equality works since we have
  \(a\neq b\) and \(b\neq a\) for lots of pairs, but we also have \(a=
  b\) and \(b=a\) for lots of pairs, just because there are multiple
  ways to have equality etc. But it would be nice to have an example
  other than equality.
\end{remark}

For every \(a,b{:}A\) and dependent successor type family \(\circ_A\)
we have the dependent successor type \(a\circ_A b\); and for every
proof of that type \(p{:}(a\circ_A b)\) we can form a family of
(\(2^{nd}\)-order) successor types, which we can call \(\circ_{(a\circ_A b)}\):

\begin{align}
  %% &\prod\limits_{(p,p^{-1}:x\circ_Ay)}\Univ & \prod\limits_{(p:x\circ_Ay)}\prod\limits_{(p^{-1}:y\circ_Ax)}\Univ \\
  &\circ_{(a\circ_A b)}{:}\prod\limits_{(p,q:a\circ_Ab)}\Univ & \textit{for all a,b}
%% \intertext{\textit{\hfill ---family of successor types for type \((a\circ_A b)\)}}
  %% &\circ_{(a\circ_A b)}{:}\prod\limits_{(p,q:x\circ_Ay)}\Univ
\end{align}


Then given \(p{:}(x\circ_A y)\) and \(q{:}(y\circ_A x)\), we have
\((p\circ_{x\circ_A y} q){:}\,\Univ\) --- the dependent successor type
at \(q\) for (i.e. that depends on) \(p\) (from type \(\relrelA\)).

Note:

\begin{itemize}
  \item \xrely\ is the successor type at \(y\)\ dependent on \(x\)
  \item \yrelx\ is the successor type at \(x\)\ dependent on \(y\)
\end{itemize}

So not only are they different types, they are successor types of
different elements.  Proof of one cannot count as proof of the other,
so we need a proof for each.  But if we do have both proofs, we can
iterate, constructing successors of these successors.  So proving that
either \((p\circ_{x\circ_A y} q)\) or \((q\circ_{x\circ_A y} p)\) is a
proposition shows that the components have proofs, which in turn is
defined as what it means for the base family of relation types
\(\relA\) to be symmetric.

 \begin{align}
   (a\ &\circ_A b){:}\,\Univ & \textit{---given \(a,b:A\)} \\
   (p\ &\circ_{(x\circ_Ay)} p^{-1}){:}\,\Univ & \textit{---given \(p{:}(a\circ_A b),\  p^{-1}{:}(b\circ_A a)\)}
\end{align}

The problem here is that \(p\) and \(p^{-1}\) prove different types, and
\(\circ_{(x\circ_Ay)}\) is only defined for types \(x\circ_A y\). ???

We should view \(\relrelA\) as the successor to \(\relA\).

\(\relA\) applies only to elements of \(A\).  Its successor
\(\relrelA\) applies to elements of different types.  But each such
type is the successor to an element of \(A\), so in a sense
\(\relrelA\) does indirectly apply to elements of \(A\).

\begin{remark}
All of which suggests we should view a type family as a genuine type,
whose elements are the types in the family.  Then \(\relrelA\) will
apply to elements of the same (family) type.  This suggests in turn
that we should make this explicit, so that we can refer to the
elements of a family (rather than the function that generates them).
E.g. \(\Lambda(\relA)\) names the family of types generated by
\(\relA\), giving e.g. \((a\relA b){:}\Lambda(\relA)\).  At the very
least this would make exposition easier.
\end{remark}


\begin{remark}
The goal is to explain that symmetric relations are those whose
symmetries are inhabited.  And they are ``natural'', just as
reflexivities are natural.  To wit, we picked out types \(a\circ a\)
to talk about reflexivities; to talk about symmetries, we pick out
symmetric pairs of types \((a\circ b)\) and \((b\circ a)\).  Every
non-trivial relation type will have such pairs.  If both types in the
pair are inhabited, then we have a symmetry type, full stop.  What it
means for both to be inhabited is that the higher order relation
\((p\circ p^{-1})\) between inhabitants of the types can be
constructed.  Not inhabited; you can only construct it if you have
both proofs, so that's enough.  So to show symmetry we only need to
show that \(p\circ p^{-1}\) is a proposition, not that it is true.  To
construct an element of such symmetry pair, we need a constructor,
just as we needed refl to construct an element of\(a\circ a\).
Contrast this with the functional approach of the HoTT book.
\end{remark}

\subsubsection{Summary}

For symmetry we need to do what we did with reflexivity, that is for
particular families provide a constructive means.  For reflexivity we
provided a constructor refl that can be associated with any relation
family.  For symmetry we cannot do this so directly since it may be
the case that some symmetry pairs are uninhabited; we cannot have a
symm constructor like refl.

Or can we?  Why not something like \(sym{:}(a=_A b)\)?  Because it
would be useless for e.g. \(\Nat\) since it would prove e.g. \(2=3\).

The only thing we have to go on is induction: the notion that proof
for refl:a=a is as good as proof for all p:a=b.



We form successors, then we use path induction.

\vfill

\subsection{Transitivity Types}

\subsection{Equality Types}

\section{Inductions}

The Hott book calls this path induction, in acknowledgement of the
role of homotopy theory in HoTT.  But what is at issue is purely a
matter of type theory, for which homotopy theory is irrelevant, so we
use ``Equality Induction'' instead.

The HoTT book also gives the impression that equality is a kind of
primitive in HoTT, and is one of the major ideas that sets it apart
from other approaches.  I think this is not quite right.  In HoTT, the
Id/= type (actually family) is not primitive, nor is it the only
possible such family; it is one of many.  The critical notion is not
equality but equality (``path'') induction.

What's the point of induction?  Or rather, what problem does a
principle of induction solve?

The basic problem is infinity.  To prove something over an infinity we
cannot inspect each element; instead we need a (finite) \emph{method}
of proof that makes it \emph{possible} for us to prove an arbitrary
case.  Contrast this with the classical approach, where proof is
always \emph{alethic} and does not involve (formal) proof methods.
Classical proof for infinities depends on quantification over their
elements, and offers an alethic conclusion: either the proposition is
true for all elements, or not.  There is no notion of possibility, or
of method of proof.  (Of course there are informal \emph{strategies}
for discovering a proof, but they are external to mathematics proper.)

  Where the infinity is inductively constructed, as in the case
of \(\Nat\), we can rely on the inductive structure of the type to
support proof by (mathematical, structural, etc.) induction.  That's
intuitively satisfying, since we can see that the structure of proof
corresponds to the structure of the type; we know (or are convinced)
that we can ``reach'' any element of the infinity in a finite number
of steps, so we can apply our proof method in a finite number of
steps.

The secondary problem is that a type may ``contain'' elements that are
not canonically (that is, inductively) constructed.  This is obviously
the case for equality types, but it may also be the case for types
like \(\Nat\).  The problem here is that such ``exotic'' elements are
not reachable by construction in the way that canonical (inductively
defined) elements are reachable from the base case.

There are two basic strategies to solve this problem.  One is to show
that a canonical element can be inferred from every non-canonical
element; the other is to show that, given propositions dependent on
elements of the type, if we can prove those dependent canonical
elements, then we can (construct a method to) prove those dependent on
non-canonical elements.  Note that this is very different from
standard induction.

Both forms of induction involve an unjustifiable but intuitively
obvious assumption.  In the case of traditional mathematical
induction, the assumption is that proving the base case and the
inductive hypothesis justifies inference to the general case that P is
true for all elements.  But this assumption cannot be proven.  This is
almost always expressed in the form of a deductive inference, but in
fact it is not a deduction but an induction.  We are \emph{confident}
that P is true for all n if we are \emph{certain} that it is true for
the base case and inductive hypothesis.  But we do not have
\emph{certain} knowledge that P is true for all n; that would require
inspection of every case, which we cannot do.

Compare Church-Turing: we are all very confident that the various
models of computation are equivalent, but we cannot (so far) prove it,
so we cannot be certain.  Proof by induction is similar: we cannot
prove that such proofs are valid, but we are confident that they are,
so we call them proofs.

In the case of finite types...

In the case of constructors, we do have certainty, just because they
are constructors.  There is no question of proving that a constructor
constructs.  Compare e.g. the introduction rule for product types: in
this case, we can be absolutely certain that \((a,b){:}\,A\times B\)
if we are certain that \(a{:}A\) and that \(b{:}B\), because that is
just what the rule for pairs means.  And we are certain that this is
the case for \emph{every} element of \(A\) and element of \(B\),
because it is in effect a definition.

So the first kind of induction involves inference to all
canonically-formed elements of an infinity.  The second kind involves
inference to all elements, including non-canonical elements.  The
legitimacy of this inference is much more tenuous.  For the first kind
of induction, the inference to infinity is justified by the inductive
structure of the elements; for the second kind, there is no such
inductive structure for the non-canonical elements, so it is harder to
see why we should be confident that proving the base case
etc. justifies inference to the general case.

Now HoTT says that the equality types \((a=_Ab)\) are inductively
defined by their canonical constructors \(\reflA\).  This makes some
sense, since it is always the case that types are (in a sense) defined
by their constructors.  (They are also defined by eliminators, etc.)
On the other hand, the constructors only serve to define reflexivity
types \((a=_Aa)\); non-reflexive types \((a=_Ab)\) have \emph{no}
constructors, and so no canonical elements.  But then why should we
believe that \emph{these} types are inductively defined by the
constructors of \emph{other} types?  There seems to be some magic
going on here.

Notice also that this says nothing about the structure of the elements
of equality types.  In particular, it does not say that the elements
of an equality type have inductive structure.  The canonical elements
of \((a=_Aa)\) are obviously inductively defined by \(\reflA\); but we
should we be convinced that the non-canonical elements of \((a=_Aa)\)
are thereby defined, let alone the elements of \((a=_Ab)\)?  There is
no intuitive support for this concept, as far as I can see.  But
that's okay, since the HoTT book does no claim this.

So the problem is twofold: first, how do we get to \((a=_Ab)\) when
all we have to go on is \(\reflA(a){:}(a=_Aa)\)?  We might express
this by asking how we can get from canonical equality types
(i.e. reflexivities) to non-canonical equality types
(i.e. non-reflexivities).  Second, how do we get from canonical to
non-canonical elements of equality types?

On the other hand, we do have the obvious intuition that it is always
the case that (\(a=_Aa)\), so that if it is also the case that
\((a=_Ab)\), those two propositions are in some sense ``the same'', so
proof of \((a=_Aa)\) should count as proof of \((a=_Ab)\), and
vice-versa.  We also have the intuition that if \(a\) and \(b\) are in
fact equal, then any one proof of that fact is as good as any other.
These intuitions are based directly in the very idea of equality.

Etymonline: ``As a term in logic (early 15c.) it [induction] is from
Cicero's use of inductio to translate Greek epagoge "leading to" in
Aristotle. Induction starts with known instances and arrives at
generalizations; deduction starts from the general principle and
arrives at some individual fact.''

This sense of ``leading to'' is what is missing in path induction.
This also shows why mathematical induction cannot be deemed deductive,
because it moves from specific to general.

A better term might be \emph{transduction}.  Etymonline: from Latin
transducere/traducere "lead across, transfer, carry over," from trans-
"across" + ducere "to lead"

\begin{remark}
  Note that there is a big difference between definition by induction
  and proof by induction.
\end{remark}

\subsection{Varieties of Induction}

There are three well-known and widely accepted principles of induction:

\begin{itemize}
\item Mathematical induction (a/k/a \(\Nat\)-induction)
\item Structural induction
\item Transfinite induction
\item Well-founded induction (http://planetmath.org/wellfoundedinduction)
  ``As an example of application of this principle, we mention the proof of the fundamental theorem of arithmetic : every natural number has a unique factorization into prime numbers. The proof goes by well-founded induction in the set ℕ ordered by division.''
  a/k/a Noetherian induction?
  WFI is basic; the others are specializations.
\item Rule induction (computer science?)
\item Epsilon induction https://en.wikipedia.org/wiki/Epsilon-induction
\end{itemize}

What all these have in common is some notion of ``forward progress'':
a ``successor'' operation that moves from one place or element to the
``next'' place or element.

Equality induction does not involve this kind of movement.  For any
equality type we may have multiple distinct inhabitants, but they are
all ``at the same level'', so to speak; they cannot be arranged in a
sequence involving next or preceding elements.

\subsection{Well-Founded Induction}

 Most accounts of WFI are classical.  You start with a well-founded
 set \(X\) and relation \(\preceq\).  Here's one way to state WFI (from
 Andreas Klappenecker's notes on Noetherian Induction):

 \begin{theorem}
   (The Principle of Noetherian Induction). Let \((A,\preceq)\) be a well-
   founded set. To prove that a property \(P(x)\) is true for all
   elements \(x\) in \(A\) it is sufficient to prove the following two
   properties:
   \begin{enumerate}
   \item Induction basis: \(P(x)\) is true for all minimal elements of \(A\).
   \item Induction step: For each non-minimal \(x\) in \(A\), if
     \(P(y)\) is true for all \(y\prec x\), then \(P(x)\) is true.
   \end{enumerate}
\end{theorem}

The conclusion is then that \(\forall x P(x)\).

This is fine, but it's non-constructive.  In order for it to work
constructively, the relation must be constructive; that is, \(y\prec
x\) must mean that \(x\) can be constructed from \(y\).

We don't have that in path induction.  There is no relation of
construction between the base case \(\reflAx{:}\,\reflTa\) and any
other \(p\) where \(\preflTa\).  So WFI (classical) cannot underwrite
path induction.

\bigskip

The HoTT book says this:

``The basic way to construct an element of \(a = b\) is to know that
\(a\) and \(b\) are the same. Thus, the introduction rule is a
dependent function

\begin{align}
  \textsf{refl}{:}\,\reflfam
\end{align}

called \textbf{reflexivity}, which says that every element of A is
equal to itself (in a specified way).'' (p. 47-8)

The formal definition given in appendix A.2.10 says:

\medskip

\begin{tikzpicture}[
    edge from parent path={
        (\tikzparentnode\tikzparentanchor)
        +(0pt,.5\tikzleveldistance)
        -- (0pt,-.5\tikzleveldistance -| \tikzchildnode\tikzchildanchor)
        -- +(0.75cm,0pt)
        -- +(-0.75cm,0pt)
    },
    grow'=up,
    level distance=4ex,
    level/.style={sibling distance=5em/#1}]
  \draw (2.75cm,.35cm) node {=-\textsc{intro}};
  \node (Concl) {\(\Gamma\vdash \textsf{refl}_a{:}\,a=_Aa\)}
    child { node (Major) {\(\Gamma\vdash A{:}\Univ\)} }
    child { node (Minor) {\(\Gamma\vdash a{:}A\)} } ;
\end{tikzpicture}

(Note that this is called the =-\textsc{intro} rule rather than
the \(\reflA\)-\textsc{intro} rule.)

\medskip

This means that we can always infer \(\reflAa{:}\,\reflAaa\) from just
\(a{:}A\); but it says nothing about any other proof of \(\reflAaa\),
nor about proof of \(\reflAxy\) for any other \(x,y{:}A\).  So even if
we have \(\reflAa\prec p\) for all \(\preflAxy\), we cannot invoke WFI,
since there is no constructive path from the former to the latter.

There is another problem with the informal explanation offered by the
HoTT book: it relies on a notion of induction as explanatory.  This
violates the general principle that types are explained by their
inference rules.  ``Path induction'' is explained by the ind-intro
(aka =-elim) rule, which has nothing to do with any antecedent notion
of induction.  It just lays down what it means to be an equality
proof.  We can call this induction, but then we're just labelling an
elimination rule -- misleadingly, since the elimination rule does not
involve and is not justified by any concept of induction.

On the other hand, one could argue that the formal rule captures some
intuitive notion of induction, just as the \&-intro rule captures our
informal notion of conjunction.  But the problem with that is, as
explained above, we do not seem to have an intuitive notion of
induction that corresponds to the =-elim rule.  On the other hand, we
do have intuitions about equality, but they do not depend on
intuitions about induction; they seem to depend mostly on practices of
exchange.

What we do with equalities seems to be fundamentally different from
what we do with inductive stuff.  Induction takes us from the
particular to the general; equality seems to be already general.  We
don't need to prove that one pebble is as good as any other for
tallying; it's built-in to the notion of tallying.  No induction is
needed; it doesn't make sense to try to ``test'' new pebbles to see if
they really work, or to prove that they do.  Using induction to
justify our ways with equality makes no sense.

\bigskip

\noindent
A possible way out is to reinterpret reflexivity as something that
follows from equality generally.  Then we can define a different
introduction rule for \(\reflA\).

\medskip

\begin{tikzpicture}[
    edge from parent path={
        (\tikzparentnode\tikzparentanchor)
        +(0pt,.5\tikzleveldistance)
        -- (0pt,-.5\tikzleveldistance -| \tikzchildnode\tikzchildanchor)
        -- +(1.5cm,0pt)
        -- +(-1cm,0pt)
    },
    grow'=up,
    level distance=4ex,
    level/.style={sibling distance=6em/#1}]
  \draw (5.25cm,.5cm) node {\(\textsf{refl}^1-\)\textsc{intro}};
  \node (Concl) {\(\Gamma\vdash \textsf{refl}^Aa\,{:}\,\reflAaa\)}
    child { node (Major) {\(\Gamma\vdash A{:}\Univ\)} }
    child { node (Minor) {\(\Gamma\vdash a,b{:}A\)} }
    child { node (Minor) {\(\Gamma,p\vdash \reflAab\)} } ;
\end{tikzpicture}

\medskip

\begin{tikzpicture}[
    edge from parent path={
        (\tikzparentnode\tikzparentanchor)
        +(0pt,.5\tikzleveldistance)
        -- (0pt,-.5\tikzleveldistance -| \tikzchildnode\tikzchildanchor)
        -- +(1.5cm,0pt)
        -- +(-1cm,0pt)
    },
    grow'=up,
    level distance=4ex,
    level/.style={sibling distance=6em/#1}]
  \draw (5.25cm,.5cm) node {\(\textsf{refl}^2-\)\textsc{intro}};
  \node (Concl) {\(\Gamma\vdash \textsf{refl}^Ab\,{:}\,\reflAbb\)}
    child { node (Major) {\(\Gamma\vdash A{:}\Univ\)} }
    child { node (Minor) {\(\Gamma\vdash a,b{:}A\)} }
    child { node (Minor) {\(\Gamma,p\vdash \reflAab\)} } ;
\end{tikzpicture}

\medskip

The idea is that, instead of inferring \textsf{refl} directly from
\(a{:}A\), we can infer it from \emph{any} proof of \(\reflAab\).
This corresponds more closely to treating reflexivity as a binary
relation, rather than a unary predicate.  Reflexivity itself does not
\emph{introduce} equality, as the HoTT definition suggests; after all,
we can specify equality types without also introducing their proofs.

Rather, reflexivity is just the canonical form of proof of reflexivity
types, not of equality types generally.

This suggests another revision to the HoTT definition, this one a refinement:

\medskip

\begin{tikzpicture}[
    edge from parent path={
        (\tikzparentnode\tikzparentanchor)
        +(0pt,.5\tikzleveldistance)
        -- (0pt,-.5\tikzleveldistance -| \tikzchildnode\tikzchildanchor)
        -- +(0.75cm,0pt)
        -- +(-0.75cm,0pt)
    },
    grow'=up,
    level distance=4ex,
    level/.style={sibling distance=5em/#1}]
  \draw (2.75cm,.35cm) node {=-\textsc{intro}};
  \node (Concl) {\(\Gamma\vdash (a=_Ab){:}\,\Univ\)}
    child { node (Major) {\(\Gamma\vdash A{:}\Univ\)} }
    child { node (Minor) {\(\Gamma\vdash a,b{:}A\)} } ;
\end{tikzpicture}

\noindent which just says that given \(a,b{:}A\) we can form the
\emph{type} \(\reflAab\), independently of whether we can find a proof
of that type.  The introduction rules \(\refl{1}\) and \(\refl{2}\)
then just say that, given any proof \(p\) of \(\reln{A}{a}{b}\), we
can introduce the canonical proofs \(\refl{A}a{:}\reln{A}{a}{a}\) and
\(\refl{A}b{:}\reln{A}{b}{b}\).

\medskip

What would this mean for constructive WFI?  The core idea of WFI for
proofs on equalities is that to prove something for any equality proof
for any pair (of type \(A\)), it suffices to prove it for the
\(\refl{A}\) proof for reflexivity pairs; the justification is based
on well-founded structure.

With our revisions, any equality proof can be reduced to \(\refl{A}\),
which is more in line with the notion of canonical form as normal
form.  The intro rules do not mean that other proofs are constructed
inductively from the canonical proofs; but as they are indeed
introduction rules they set the meaning of \(\refl{A}\).  Instead of
proof by induction, we prove by normalization -- which in turn is a
kind of induction.  Normalization has well-founded structure; it
always bottoms out at canonical forms.

So it still suffices to prove the base case \(\refl{A}\).  Not
directly because of inductive construction, nor because of (classical)
well-founded structure, but because of an (axiomatic) principle of
normalization.  (?)

\begin{remark}
  In other words, if our intro rules allow us to infer canonical form
  from non-canonical form -- by fiat or axiom, as it were -- this is
  just as good as having an inductive constructor that builds other
  canonical forms from the base case.  For example, for \(\Nat\) we
  use \(S\) to build from \(Z\); but for =, we reverse this, saying in
  effect that non-canonical forms can be ``deconstructed'' and the
  canonical form pulled out.  We might be able to think of our
  \(\refl{A}\) intro rules as implicit elimination rules or
  co-constructors: they say that for any proof of equality we can
  extract the base case that must have been used to construct the
  proof, even if we do not have an explicit \(S\)-like constructor
  showing exactly how the proof was built up from the base case.
\end{remark}

\begin{remark}
  Or, there are two ways for WFI to work constructively, one based on
  inductive construction, and one based on inductive co-construction,
  so to speak.  The former method establishes a well-founded relation
  by construction; the latter by co-construction.  In the former case,
  you can always construct the ``next'' element; in the latter, you
  can always co-construct the base element.  Is this co-induction?  So
  equality proofs are codata?
\end{remark}

\medskip

Consider what the idea would look like for \(\Nat\):

\medskip

\begin{tikzpicture}[
    edge from parent path={
      (\tikzparentnode\tikzparentanchor)
      +(0pt,.5\tikzleveldistance)
      -- (0pt,-.5\tikzleveldistance -| \tikzchildnode\tikzchildanchor)
      -- +(1.5cm,0pt)
      -- +(-1cm,0pt)
    },
    grow'=up,
    level distance=4ex,
    level/.style={sibling distance=6em/#1}]
  %% \draw (5.25cm,.5cm) node {\(0{:}\,\Nat} ;
  \node (Concl) {\(\Gamma\vdash S(n){:}\,\Nat\)}
  child { node (Major) {\(\Gamma\vdash n{:}\Nat\)} } ;
\end{tikzpicture}
\begin{tikzpicture}[
    edge from parent path={
      (\tikzparentnode\tikzparentanchor)
      +(0pt,.5\tikzleveldistance)
      -- (0pt,-.5\tikzleveldistance -| \tikzchildnode\tikzchildanchor)
      -- +(1.5cm,0pt)
      -- +(-1cm,0pt)
    },
    grow'=up,
    level distance=4ex,
    level/.style={sibling distance=6em/#1}]
  %% \draw (5.25cm,.5cm) node {\(0{:}\,\Nat} ;
  \node (Concl) {\(\Gamma\vdash n{:}\,\Nat\)}
  child { node (Major) {\(\Gamma\vdash S(n){:}\Nat\)} } ;
\end{tikzpicture}
\begin{tikzpicture}[
    edge from parent path={
      (\tikzparentnode\tikzparentanchor)
      +(0pt,.5\tikzleveldistance)
      -- (0pt,-.5\tikzleveldistance -| \tikzchildnode\tikzchildanchor)
      -- +(1.5cm,0pt)
      -- +(-1cm,0pt)
    },
    grow'=up,
    level distance=4ex,
    level/.style={sibling distance=6em/#1}]
  %% \draw (5.25cm,.5cm) node {\(0{:}\,\Nat} ;
  \node (Concl) {\(\Gamma\vdash 0{:}\,\Nat\)}
  child { node (Major) {\(\Gamma\vdash n{:}\Nat\)} } ;
\end{tikzpicture}

\medskip

\noindent Here the rule on the left just expresses the definition of
\(S\) for \(\Nat\).  The rule on the right expresses a kind of
elimination rule - we know any \(n{:}\Nat\) must have the form
\(S(S(...S(0)))\), so it expresses the idea that all of the \(S\)s
could be eliminated.  We could prove this, since any \(n{:}\,\Nat\) is
inductively constructed.  But were that not the case, we could simply
stipulate it as part of the definition of what it is to be of type
\(\Nat\).  And that's what we do for equality types and \(\refl{A}\).

\subsection{Coinduction}



\subsection{Reflexivity Induction}

For any reflexivity type \(\reflAa\) we may have infinitely many proofs
\(\preflAa\).  The HoTT book singles out just one such proof per type
and designates it as the canonical constructor for the type:

\begin{align}
  &\textsf{refl}{:}\,\reflfam &  \textsf{HoTT p. 48} \\
  \intertext{(which should be:)}
  &\textsf{refl}{:}\,\reflfamU \\
  \intertext{giving}
  &\reflA(x):\,(x\circ_A x)
\end{align}

Which makes \(\reflA\) a family of ``provers'' of reflexivity types:
give it an \(x\) and you get a (canonical) proof of \(\reflTAx\).

Since we may have non-canonical proofs \(x{:}\reflAa\), we are
confronted with the task of showing that if a predicate on proofs of
\(\reflAa\) can be satisfied by one such proof it can be satisfied by
any such proof.  This is what path induction is for (although it
extends beyond reflexivity types to all \(x\circ_Ay\) types).

A predicate on proofs of \(\reflAa\) looks like this in the HoTT book:

\begin{align}
  & \Cxyp & \textsf{HoTT p. 49} \\
  \intertext{giving}
  & C(x,x,\reflx){:}\,\Univ
\end{align}

\noindent We rewrite this as:

\begin{align}
  & \SXYPT \\
  \intertext{giving:}
  & \SXX \\
  & \SXXP
  \intertext{and we set}
  & S^{x\circ_Ax}(p) :\equiv \reflTAx
\end{align}

We use \(\circ\) instead of \(=\) for generality, and we use
\(S^{\circ_A}\) intead of \(C\) in order to highlight that fact that
we treat it as a (not ``the''!) successor type family drawn from
\(\circ_A\) types, so \(\SXXP\) is \emph{a} successor type at \(p\)
from type \(\reflTAx\) -- which in turn we treat as \emph{a} successor
type at \(x\) for \(x\) from \(A\).

Note that for any \(\preflTx\) we have \(p{:}\,S^{x\circ_Ax}(p)\):
every proof of \(\reflTAx\) is proof of its own successor type.  Or: if
p proves a successor type at x for x, the it also proves its own
successor drawn from that successor type at x for x.  Reflexivity all the way
up!


\medskip

Now the HoTT book bases path induction on \(\reflAx\), which is an
element of \(\reflTAx\); informally, the path induction principle for
reflexive types can be expressed by saying that a proof of a predicate
on \(\reflAx{:}\,\reflTAx\) suffices as proof on all \(\preflTx\).

\bigskip

The approach adopted here is a little different.  First, we don't
single out a canonical constructor.  Intuitively, the idea is that
since all proofs of \(\reflTAx\) are ``equivalent'', it should be
enough to prove the predication for any such proof; there is no call
to single out a canonical one.

\begin{remark}
I suspect that introducing a canonical element for reflexivity
types was motivated by the example of the inductive definition of
\(\Nat\), which has two canonical constructors.  And it is the
inductive structure of those constructors that makes proof by
mathematical induction work.  But the elements/proofs of equality
types are not constructed in this way; there is no sequential
structure to them.  But this suggests that we do not really need a
canonical constructor.
\end{remark}

\begin{remark}
  Note that mathematical induction is a little ambiguous.  Once you've
  proven for base case and inductive hypothesis, you drawn the
  conclusion to ``all n''.  But the justification for this inference
  is the intuition that ``all n'' are in fact constructed using the
  canonical constructors, so ``all n'' really means ``all canonical
  n''.  There is no good intuition (for me, at least) leading from
  ``all canonical n'' to ``all n including non-canonical/exotic n''.
  But that reading is essential to HoTT induction, and I suspect it is
  the real innovation.
\end{remark}

\bigskip

The basic idea is that we replace the notion of a canonical element
\(\reflAx\) with the notion of a canonical successor function.  This
gives us a genuine kind of induction: starting with any \(\preflTx\)
we get a ``next'' object, namely the successor type at \(p\).  There
are infinitely many successor families, and thus successor types at
\(p\); the intuition behind ``reflexivity induction'' is that proving
the canonical successor type for some \(\preflTx\) suffices as proof
for all such \(p\), for all successor types.  In other words, if you
can prove the canonical successor proposition \emph{at p} for some
\(\preflTx\), you can prove any successor proposition \emph{at p} for
any such \(p\).

So what our approach boils down to is moving from elements/proofs of
reflexivities to their successor types/propositions, which are what we
want to prove.  This gives us an inductive structure, analogous to the
structure of \(\Nat\), that the HoTT approach, based on canonical
elements, lacks.  Well, it's not so much that it goes missing as that
it goes unremarked.  What I sketch here is implicit in the HoTT book.

In other words, it is not the case that all elements of \(\reflTAx\)
are inductively constructed from canonical constructors, but it is the
case that the successor type of every such element is inductively
constructed by a successor function.  In contrast to e.g. \(\Nat\),
whose structure can be thought of as ``vertical'', the inductive
structure of reflexivity proofs combines ``horizontal'' and
``vertical'' aspects.  The successor types are all at the same level,
one level ``up''from the elements to which they are successors;
neither can be ordered.  But nonetheless each pair of element and
successor type has inductive structure, from n to n+1.

A critical point is that we do not need a canonical element to do
this; we need not define \emph{any} constructors for reflexivity
types.

Once this structure is clear it also becomes clear that the inductive
principle required to make this work is clearly not like the inductive
principles involved in the three basic kinds of induction.  The
proposal here is that it is this induction principle that is
distinctive, rather than the Id/= type family specified in the HoTT
book.

The implicit reasoning: we prove that if something is true for
arbitrary \(n\) (i.e. for any \(\preflTx\)) then it is true for
\(n+1\) (i.e. \(p{:}S^{x\circ_Ax}(p)\)).  We then infer that it is
true, not for all \(n\) (there are only two levels, \(n\) and \(n+1\),
and we've already shown the equivalent of \(P(n)\to P(n+1)\) for
arbitrary n) but for all successor types (propositions) in the family
\(\prod\limits_{(x,y:A)}\prod\limits_{p{:}(x\circ_Ay)}{:}\,\Univ\).

In other words, mathematical induction goes from \(P(n)\to P(n+1)\)
for arbitary n to \(P(n)\) for all n; we go from \(P(n)\to P(n+1)\)
for arbitary n to \(Q(n)\) for all \(Q, n\).

Hmm, that can't be right.  Some successor props may not be
satisfiable.  We have to do the inductive proof for each successor
type.

Since we don't use a canonical element, there is no move from ``true
for canonical'' to ``true for everything''.  We don't even need to
move to ``true for everything'', since we prove the inductive step for
arbitrary n?  No, proof for arbitrary n is not same as proof for all;
you have to make the inductive inference to get from the former to the
latter.  So we need it too.

What's our base case?  Refl?  Hmm, that would suggest that path
induction is wrong, its missing the inductive step that we have
articulated.

Writers on HoTT (Ladyman?) say that the inductive step goes missing in
path induction; we go straight from base case to generalization.  That
makes no sense to me.

The formal definition of HoTT weasels out of this be defining ind= as
a primitive elimination rule.  But that sticks in my craw too; it just
seems wrong to encode induction as an inference rule the way HoTT does
it.  It should be structured just like the traditional inductions:
base case, then inductive hypothesis, then generalization:

\begin{itemize}
\item Base case:  \(\reflAx{:}\,\reflTAx\to \reflAx{:}\,S^{x\circ_Ax}(\reflAx)\)
\item Inductive hypothesis:  \(\preflTx\to p{:}\,R^{x\circ_Ax}(p)\) for arbitrary \(p\)
\item Generalization:  \(\preflTx\to p{:}\,S^{x\circ_Ax}(p)\) for arbitrary \(p\)
  %% \(S^{x\circ_Ax}(p)\) for all \(\preflTx\) and for all \(S\)
\end{itemize}

Critical move: defining \(S^{x\circ_Ax}(p) :\equiv (x\circ_Ax)\).  Is
that kosher for all successors?  No, the successor type at p can be
anything.  We need canonical reflexive successor:  \(\RXXP\).

The problem is we have two variables: proofs of x=x, and successor
types at those proofs.  To generalize over the latter we must use
conditionals.

Goal is to show that for arbitrary successor S, if S(p) for some
p:x=x, then S(p) for all p:x=x.

\begin{itemize}
\item Base case:  \(\preflTx\to p{:}\,R^{x\circ_Ax}(p)\) for arbitrary \(p\)
\item Inductive hypothesis: function from S(p) to ?? for some p?
\item Generalization: function from S(p) to ?? for all p?
  %% \(S^{x\circ_Ax}(p)\) for all \(\preflTx\) and for all \(S\)
\end{itemize}


\subsection{Recursion}

%% https://en.wikipedia.org/wiki/Course-of-values_recursion

\section{Construction and Co-construction}

The standard presentation of intuisionistic/constructive mathematics
says, in so many words, that you must be able to construct stuff.
This means that if we think we have something but no means of
constructing it, we're in trouble.  Case in point: equality proofs in
HoTT.

But the concepts of coinduction and bisimilarity suggest a more
refined notion.  Construction builds up; co-construction builds down.
If you can start at the base case and build your object using
canonical constructors, you have a construction.  If you start at your
object and ``build'' - that is, co-construct - a base case, then you
have a co-construction.

In other words, it is sufficient for constructivity to be able to get
(co-) constructively from your object to a base case.  If you can do
that, then you are working (co-)constructively, even if you have no
method of constructing your object from the base case.

This corresponds to the idea of ``hidden'' stuff addressed by
bisimilarity.  In this case, the constructive form of the object is
hidden; but since we can co-construct downwards we can infer that the
object is indeed a construction even though we cannot ``see'' the
construction.

\begin{remark}
  What I'm calling co-construction seems to be called ``observation''
  by some authors.  Suggesting that we might think of elimination
  rules as observational rules.  This makes sense; if we have
  arbitrary \(x{:}\,A\), the elim rules tell us how it behaves, not
  how it is constructed.
\end{remark}

\subsection{State Space v. Construction Space}

Can we think of vars like \(x{:}\,A\) in terms of a (hidden) state
space, by analogy with bisimulation?  We don't know how \(x\) was
constructed, but we can ``observe'' it by using eliminators, etc.
Instead of hidden state, hidden construction (structure).

\section{Grounding Equality: Pragmatics of Fair Exchange}

The task is to capture our ordinary intuitions about equality, just
like Turing captured our intuitions about effective procedures.
Similar: logical consequence.  So far nobody has managed to do for
logical consequence what Turing did for effective procedure:provide
something formal that captures our ordinary intuitions.  Ditto for
equality.  (TODO: quotes about problems with equality)

Paradigm: fiat currency.  Each dollar bill is "equal" to each other.
But each is also different, as a physical particular.  What makes them
equal is exchange value, which is socially instituted.  So equality is
a social institution.

Equality as exchange value depends on token recognition, so the
ability to recognize that a particular is a token of a type - this
thing is a \$1 bill - is more primitive than equality, but also enables
equality.  This dependency can be made explicit by writing a =\$ b
instead of a = b, meaning that a and b are equal under the
\$-denominated currency regime - that is, under a particular
perspective.  So equality is always perspectival, which translates
into equality-as-type in HoTT.

Proof of equality: type theories like HoTT need a "proof" for each
equality type; usually this is defined as refl.  But the formal
structure of such definitions in a type theory does not capture our
ordinary intuitions about equality.  We can replace the (arbitrary)
symbols "=" and "refl" with any others without changing the meaning;
in particular we could use "Unit" and "*", respectively, giving "*a :
Unit a a instead of "refl(a): a = a" (leaving type A implicit).  So
the formal definitions have no conceptual content except what is
instituted by the introduction and elimination rules.  We cannot rely
on antecedent notions of equality and reflexivity; the latter in any
case is a specifically mathematical concept, relatively remote from
our ordinary intuitions, and so cannot be counted as primitive.

What counts as "proof" for our ordinary notion of equality, the one
involving exchange value?  We cannot count on merely "knowing that a
and b are the same" (HoTT p. 47 "The basic way to construct an element
of a = b is to know that a and b are the same.")  That approach runs
straight into Hume's problem.  Recall that Hume pointed out that we
cannot observe causality.  We can observe event a, and event b, and we
can observe that b invariably follows a; but we cannot observe a
causal connection between the two, and so we have no iron-clad
guaranteed that b will always follow a in the future; we cannot "know"
that a causes b.  Similarly, we can observe that a is a \$1 bill, and
that b is a \$1 bill, but we cannot observe a relation of equality
between them; we cannot "know" by observation that they are equal.  Of
course we can know that they are both \$1 bills - tokens of the same
type - but that does not count as knowing that they are equal.  (Or
does it?  We might also say that we can know that they are equally
tokens of the same type but that gives us no means of demonstrating
that knowledge - there is no "type of knowing that two things are
equal".  Then again, knowing that a and b are tokens of a type is
knowledge of a "vertical" relation between token and type; the
"horizontal" relation of equality between such tokens is a different
thing.) As mentioned above, equality is a social institution, not an
observable, "objective" property.  The only way to demonstrate
("prove") that a and b are equal is to actually exchange them and have
both parties to the exchange come away thinking the exchange was fair.
So we could simply define "refl" to mean "act of exchange"; but that
doesn't seem very mathematical, since actions are dynamic and
mathematical objects are not.  What we need is an actual physical
token that counts as proof of the equality in the same way that a \$1
bill (or coin) counts as "proof" of \$1.

Demonstration as proof: note that the notion of proof in HoTT and
similar extends our ordinary notion of proof.  We don't ordinarily go
around saying that, e.g. 2 proves Nat, but in type theory that is
exactly what "2:Nat" means (or at least it is one accepted gloss of
the notation.)  But if we change "proves" to "demonstrates" we have
something a little less odd: SS0 (=2) demonstrates Nat just because on
inspection we can see that it is composed solely of the constructors
that define Nat, assembled according to the rules of the type theory;
it "shows" Nat in action, so to speak.  Note the close connection
between compositionality and demonstration (proof); it's not
accidental.  Similarly it makes sense to say that a \$1 bill or coins
demonstrates the concept of \$1, and that fair exchange of dollar bills
demonstrates their equality.

So what physical token would count as demonstrating the equality of
two \$1 bills?  Here again we rely on a social institution: a contract.
Instead of actually exchanging dollar bills, the parties can agree to
such an exchange, and certify their agreement by writing and signing a
contract that obliges them to make the actual exchange some time in
the future.  The contract then serves as a physical demonstration of
the equality of the goods to be exchanged.

Notice that this does not really count as "proving" anything in the
traditional sense of exhibiting justification.  Indeed, in light of
the fact that equality is a social institution, it doesn't even make
sense to try to prove that one \$1 bill is equal to another.  That
would make it sound like there is some kind of equality property or
relation that we can discover and exhibit, or that there is some fact
of the matter that we can expose.  But there is no fact of the matter,
nor is there any "objective" equality relation to be "proved":
equality is social and perpectival.  Things are equal only because we
agree to treat them as equal, and this applies as much to mathematics
as to medium sized dry goods.  To social instutions of mathematics may
be complex, but they remain social institutions.

To put it another way: proof of equality /institutes/ equality; fair
exchange of a pair of 1\$ bills is just what makes them equal.

(But: isn't that always the case?  Proof of P institutes P?  Not
classically; there, propositions and proofs are ontologically
distinct.  Classical proofs justify (assertion of) propositions;
intuitionistic/inferential proofs institute propositions.  A
proposition demonstrated cannot be ontologically severed from its
demonstration.)

So we can view "refl" as a kind of contractual certification of
equality.  On this view, it has nothing to do conceptually with the
notion of reflexivity - it does not mean that a thing stands in a
binary relation of reflexivity to itself; it means that two distinct
things have equal exchange value.

The critical observation here is that the "reflexive" formula "a = a"
contains not one but two distinct 'a' tokens.  They are not "the
same", any more than two distinct \$1 bills are the same.  But they do
have the same exchange value, which is the same exchange value that
all such 'a' tokens have.  So we can take "a = a" as an inference
rule, one that licenses exchange of 'a' tokens.  Then "refl(a) : a = a"
reifies that license, so to speak, in the form of token "refl" whose
role is to certify that such exchanges are fair.

Key point being that this avoids the notion of reflexivity as a binary
relation of a thing to itself.  We could make this more perspicuous by
replacing "refl" with something like "exch", or even some variation on
the "=" symbol, e.g.  "a=a! : a = a", meaning "a=a!" certifies that
all 'a' tokens have equal exchange value.

Another critical point: we derived "exchange value" from ordinary
practices, but we can also derive it solely from type theory
primitives by means of a notion of token repeatables, or maybe even
token reflexivity.  Token repeatables are always interchangeable.
Each token, qua symbol, "gives rise" to the type of its token
repeatables.  E.g. we have 3:Nat; but we also have the type of such
3-tokens (token repeatables), so every "occurence" of the symbol '3'
is a token of that type.  To make this explicit we probably need a new
type former that turns tokens into token-repeatable types.  Then we
can say e.g. 3:T(3), where T(3) names the type of all 3-tokens.  By
definition, every token of a T-type is interchangeable with every
other token of that type.

Normally we do not actually exchange such tokens, instead we just
write down fresh copies and count that as using the same token.  But
we could physically exchange them; we could cut the tokens out of our
piece of paper and shuffle them around.  So for a = a, we could
actually cut out those 'a' tokens and paste either of them in place
wherever we need an a-token in our proof.

No constants and no vars - the distinction presupposes denotational
semantics.  But in type theory we only have tokens and types.  When we
write \(x:A\) what we mean is that the inference from the particular
token \(x\) (which is token-repeatable) to type \(A\) is
\emph{assumed}, not that \(x\) is a variable of type \(A\) waiting for
assignment.

E.g. x:Nat, x = 1+1.  The latter does not mean ``assign the value of
1+1 to symbol x''; nor does it mean something like ``(it is true that)
the values of x and 1+1 are equal''.  It just means that we are
licensed to exchange x tokens and 2 tokens.



**************** old:


Goal is to explain type theory

    * in terms of the ordinary intuitions of lived experience

    * without relying on representational vocabulary like "refers", "denotes", etc.

    * without metaphysics or psychologism

The approach draws on Brandom:

    * normative pragmatics

    * inferential semantics

    * logical expressivism

Surprising results:

    *  the Unit type is incoherent

    * identity types (aka token-repeatable types) are primitive, all
       others derivative (in the order of explanation); this is
       because tokens are always repeatable, which gives rise to a
       type (token-repeatable type)

    * constructors are not functions; the former are primitive, the
       latter derivative

The starting point is the type-token distinction.  We show how this
relation can be explained in terms of practical norms instituting the
treatment of particulars as tokens of a type.

What is a type?  A type is an institution.  What is a token?  A token
is a particular in a functional role.


Token-particulars and perspectivism.  The only way a particular can
function as (play the role of) a token of a type is for us to treat it
as such.  We move from particular to token-particular - that is, from
particular to particular-as-token.  This is an inferential move, from
to particular to category.  It's inferential because categories
(concepts) are inferentialy articulated.  What Sellars called a
language-entry move.  But this move does not require language.  Or
rather categorization does not, though conceptualization does.  Even
primitive life forms categorize; in fact inanimate things categorize
(Brandom's example of a chunk of iron, rusting.)  But since we're
talking about human practices, it is proper to view the move from
particular to token-particular as an inferential move; let's call it a
token inferencing.

Token inferencing is a two-fold move: from particular to token of
type, i.e. inference to both type and functional role. (?)


A good "intuition pump" for illustrating the pragmatic basis of type
theory is the practice of tallying.  Before we can even begin to tally
- e.g. by dropping pebbles into a pouch, notching a stick, etc. - we
must have mastered the inferential practices involved in recognizing
particulars as tokens of a type.  We must have the practical ability
to treat distinct pebbles as indistinguishable tokens of a type - call
it Counter.  And this is a matter of practical norms - we treat
countings using pebbles as correct or incorrect, and that's about as
far as explanation can go.  It would be fruitless to try to explain
why or how we manage to recognize that a particular object is a
pebble; the best we can say is that treating it as a pebble just like
all the others is the correct thing to do.

So token inferencing is a primitive practice that precedes more
sophisticated token/type practices like tallying; the latter depends
on the former.  Neither depends on language; both are essentially
normative, practical skills.

The critical point here is that (the practice of) this sort of
inferential skill (token inferencing) is what institutes concepts.
Creatures can treat particular pebbles as tokens of a (nameless) type
even if they have no language; in fact, they do not even need an
antecendently available concept of type (or of "concept").  But once
they have instituted such normative practices they can say what they
otherwise can only do by expanding their vocabulary.  They can invent
words like ``that'' and ``pebble'', and go around point at things and
declaring "that pebble"!  Doing so makes explicit the normative
practice of treating things as pebbles; it makes the language-entrance
move explicit as a correct move.  The village idiot can go around
pointing at cows and exlaiming ``that pebble!'', but the other
villagers will treat him as linguistically incompetent.  ``That
pebble!'' as a kind of rule, which can be applied correctly or
incorrectly.

This is just what the standard type-theoretic notation "a : A" does.
Most expositions of type theory gloss this as "a has type A", or "a is
proof/witness/inhabitant of A", etc.  The proposal here is that we
should read this expression as exactly analogous to a propositional
implication like A -> B.  But only analogous; A -> B is a schema
licensing inference from proposition to proposition, whereas we treat
"a : A" as a schema licensing inference from particular to
token-particular.  We gloss it "from a infer A" or "from particular a
infer type A", or "infer that particular a is a token-particular of
type A".  It's also stipulative, not empirical; remember 'a' is a
token, not a symbol.

Note that token inferencing is perspectival; particulars only come to
play the role of tokens (i.e. are correctly treated as tokens) within
the context of particular purposes or "language games".  So it might
be better to gloss "a : A" as "treat particular a as a token of type
A, for the purpose at hand."  (Remembering that inferencing is
something we do that involves /treating/ things in correct ways, so to
treat a as a token of type A is to make a practical inference from
particular to token-particular.)

[Constants: 3:Nat is a rule licensing use of 3-tokens according to the
  rules of Nat.]


[FIXME: Now there is a (very) subtle but critical point here that is
  obscured by our very use of language.  We naturally tend to
  abstraction; most of us will treat the "a" in "a : A" not as a
  particular but as either a constant symbol that denotes a particular
  or a variable symbol that ranges over a collection of particulars,
  or maybe even an "unknown".  In any case, before it is any of that
  it is indeed a particular - a particular bit of ink on paper, or
  illumination on a screen.  And remember we have ruled out any appeal
  to denotation, so we cannot take the tokens in our expressions as
  refering to anything.]

But if "a : A" is an inference license (or: an inference that
institutes a license, i.e. a rule), don't we need to already know what
"a" is to proceed with the inference?  Only under denotational
semantics.  [Cmp. Platonic anamnesis] It can't mean "treat any old
thing as an A-token".  But if it just means "an A-token is an A-token"
then it does not involve inference.  The point is that a:A does not
express some relation or between or ``true'' fact about a and A that
we have discovered.  It makes no sense to say ``a:A'' is true, because
it is not a proposition, it is an inference (rule) that does not
depend no denotation .

The issue comes out more clearly if we go back to tallying practices
using pebbles.  It should be obvious that the pebbles involved in the
practice of tallying do not denote anything; in fact they are not even
symbols.  They are particulars; distinct entities located in space and
time.  Particulars are distinct by definition; to say that one two
particulars are "identical" is nonsensical, and to say that one
particular is identical to itself is vacous (Wittgenstein said
something like this, I believe.)  To move from particular to
token-particular means to ignore the particularities (the identity) of
the particular.  And it is by doing this that the notion of type
emerges (is instituted).  The Platonist might say that a particular is
a token of type A because it somehow (mysteriously) "participates" in
ideal A; the Aristotelean, that we somehow (mysteriously) derive the
abstract A from particulars.  The pragmatic perpective eliminates the
mystery: it is only by virtue of our treating particulars as A-tokens
that the type A and the role of A-token are instituted.

This has two subtle consequences that are not generally noticed in
type theories.  One is that the notion of a Unit type, with only a
single "element", is conceptually incoherent.  A type with only one
element would not be a generalization; the notion of type essentially
involves a plurality of tokens.  But can we not just stipulate that
type Unit has only one token?  We can try, but it won't work, since
that token is itself "repeatable".  In HoTT terms, if we have * :
Unit, we can create as many * tokens as we please.  The critical point
is just that these iterated tokenings are precisely *not* the same, or
identical, or equal: they are particulars.  We /treat/ them as the
same; but just think about it: that means we treat them as "identical
instances" of the one Tau type T(*).  And this illustrates the second
subtlety: it means that we are treating our one token itself as a
(proxy for a) type.  In fact every token of every type gives rise to
this sort of type, which we can call something like a token-repeatable
type.  People familiar with HoTT may recognize this as none other than
the Identity or Equality type of HoTT.

In other words, token-particulars have an inherently dual character.
On the one hand they are particulars; on the other hand, they are
/treated/ as indistinguishable tokens.

For example, we have 3 : Nat, glossed as "3 has type Nat".  But 3 also
is a type, namely the type of all 3-tokens.

Here's a more vivid example.  Using the standard inductive definition
of Nat, with Z and S as base case and successor, respectively:

    S S Z = S S Z

which says that 2 = 2.  The traditional way to explain this is in
denotational terms: the right and left hand sides of this equation
denote the same thing, which obviously must be the case since they
have the same form.

But the right and left sides of the equation are \emph{not} equal;
they are distinct particulars; they are not the same thing.  And since
we cannot appeal to denotation, we cannot say that they are the same
by virtue of denoting the same ``value''.  One way to explain the
meaning of this equation is to treat the two subexpressions as
distinct tokens of type T(2), which yields the following reading:

  "S S Z = S S Z" means that each "S S Z" token has type T(S S Z).

[BUT: what justifies this?  Only our exchange practices.  Go back to
pebbles, where each token is clearly a distinct physical particular,
and emphasize that every S and Z we write down is just as particular.
It is only our practice of treating such particulars (and their
accumulations) as interchangeable that institutes the idea of
equality.]

This is why it makes sense to say that a and b can be equal in more
than one way.  It just means that they are token repeatables.  And it
is always possible to ``create'' a new token, e.g. strike a new tally
mark or pick up a new pebble.

[FIXME: so we have two ideas we need to disentangle.  Once is
token-inferencing, the other exchangeability.  Plus the idea of a constructor.]


[We can treat token repeatables as ``the same'' effortlessly, without
  explicit rules; but this is precisely what machines cannot do, and
  that is why we need equality types in formal languages.]

In other words, the pragmatic perspective, emphasizing normative
inferential practices and particulars, leads directly the HoTT notion
of equality types.  But it explains that concept, not in terms of
equality of identification of tokens, but in terms of inference from
particular to token-particular - that is, to token-repeatable type.

And ultimately this is grounded in norms governing exchange or trade.

Concrete illustration: counting to two using two particular pebbles
can be done in two ways.

Also, treat e.g. S S Z as exactly like dropping pebbles - the S tokens
are particulars; they are every bit as particular as particular real
pebbles.

So every Nat "element" is a token-repeatable type.  E.g. any
particular 3 token has both type 3 and type Nat.

Constructors: in tallying our pebbles are constructors.  They are not
functions.  Ditto for S and Z: SSZ does not denote a number as result
of a function, it just /is/ a token from which we can infer Nat:

    SSZ : Nat

That is, SSZ is a particular (containing three particulars) that we
treat as a 3-token, which we treat as a Nat-token.  For this to be
intelligible we have no need of a function concept.  This is obvious
in the case of Z, which is analogous to our empty pouch.

More precisely: SSZ functions as (has the role of) a Nat in our game,
and S and Z function as tokens in that game; we can use tokens to
construct other tokens (e.g. adding an S token to SZ); but tokens are
not functions.  They don't do anything, although they /have/ a
function: they can be used (by rule) to construct new tokens.

Calling a ctor a function that constructs values of a type is like
calling a brick a function that builds walls.

Actions as types and tokens: concrete act of dropping a pebble as
token of type Tally (or TallyAction).  Then equality becomes
equivalence of action.

Reflexivity: a=a as a type is just the token-repeatability type of a,
and every a has this type; refl just means that each token of this type
is a token of this type.  Its a relation of token to type rather than
token to itself.

Tau types: every token is (naturally?) associated with a type, the
type of its token-repeatables.  The tau operator is analogous to
Church's lambda operator: just as lambda turns an open formula into a
name of its associated function, the tau operator turns a
token-particular into the name of its associated token-repeatable
type.  For example, T(2) names the type of all 2 tokens.

Now every token of a tau type is substitutable for any other token of
the same type; that's just what token-repeatable means.  (Particulars
are not repeatable, but token-particulars, that is tokens, are.)

Recall that "a : A" means that the inference from a to A is good.  So
e.g. a : Tau(2) means that the inference from a to Tau(2) is good.
What a : Tau(2) does not mean is that a is itself a 2-token.  So an
expression like 1+1 : Tau(2) is intelligible; it just means that from
the token 1+1 we can infer Tau(2).  And it makes sense in practice,
because we can (by the rules of the system) convert "1+1" to "2", and
we have 2 : Tau(2) by definition.  And notice that it works just as
well the other way around: 2 : Tau(1+1).  To make the inference, we
just rewrite 2 to obtain a "1+1" token.

Equality is closely related to tau types.  "a = b" means that a and b
are substitutable.  But we interpret substitution as an inference: "a
= b" means that the inference from any expression involving a to the
same expression with b replacing a is a good one.  So the equality
operator "=" is best thought of as an inference op.  To make this more
conspicuous we'll write "==>" instead of "=".

Caveat: a ==> b does not mean "from a infer b"; it means rather
something like "from any expression containing a, e.g. SEXP[a], infer
SEXP[b/a]".

How to relate tau types and this inferential interpretation of
equality?  How should we interpret e.g. "1+1 = 2"?  And what would
count as a "proof" of it?

First, a substitution inference rule is a schema; that means that the
rule a ==> b is not about the particulars embedded in the rule.
Rather, it relies on the implicit tau types T(a) and T(b); the
explicit formulation should be a:T(a) ==> b:T(a).

[NB: problem with the HoTT definition that given a:A and b:A, form
type \(a =_A b\).  This allows us to form e.g. 2 = 3.]

But (in HoTT at least) "a = b" just another way of expressing the Id
type: Id(a,b) (We're omitting the type sym A.), whose canonical
element is refl(a).  And this looks suspiciously like Tau(a).

The problem with the HoTT book is that it does not explain this, it
just glosses it with the very improbably "[t]he basic way to construct
an element of a = b is to know that a and be are the same".  There are
two problems with this gloss.  One is that "the way to construct... is
to know" makes no sense; constructing and knowing are not the same
thing.  The second and more basic problem is that the text offers no
explanation of what it is to know that a and b are "the same".  The
whole discussion is circular.  Equality, identity, sameness, and
related terms come out as unexplained explainers.

By focussing on normative practices of treating particulars as tokens,
treating token exchanges as correct, etc., we can offer a genuine and
non-mysterious explanation of the vocabulary of equality.  Terms like
equals, identification, etc. come out as explicitation devices that
allow us to say explicitly what we can otherwise only do in practice:
treat particulars as tokens, and treat one token as substitutable for
another for a particular purpose.  By foregrounding inference,
equality naturally emerges as a rule of (substitution) inference, and
since rules institute types, exchange practices institute equality
types, which the terminology of equality makes explicit.

Under this perspective the equality or identity types come out as
primitives - that is, primitive inference rules.

What about tau types?  There doesn't seem to be anything like a tau
type in the HoTT book.  But a tau type really just is the HoTT Id type
under a different perspective.  But there are differences.  In HoTT,
equality types are expressed in terms of two tokens of a single type:
given a:A and b:A, form the type \(Id_A(a,b)\) (equivalently, (\(a =_A b\)))
with canonical witness refl(a).  As noted above this does not rule out
things like "2 = 3" with witness refl(2).  Also, HoTT does not really
address the distinctions we've made between particulars,
token-particulars, token-repeatables, etc.  (I suspect that is because
the authors have a tendency to continue to think in
representationalist terms, which is not surprising since its a very
natural tendency).

So a question is: can we recover HoTT's Id types with their refl
witness just from our tau types and substitution-inference rules?


Token indiscernability: to treat a particular as a token-particular
(that is, as a particular that "has" some type) is just to treat it as
indiscernable from other token-particulars of the same type (for the
purpose at hand).  But "indiscernability" is an arcane philosophical
concept.  The practical basis for it is the normative practice of
substitutability or exchangeability.  It is only because we treat
particulars as tokens, and further because we treat tokens of the same
type as interchangeable, that we can speak of indiscernability.  We
don't need the philosophical apparatus of properties, satisfaction,
etc. to make sense of this.

(Compare chess pieces: distinct as particulars, but treated as
indiscernable - it doesn't matter which rook you put on the right or
left, though which rook you move does matter.  But even then you move
right or left rook, not this or that particular.  It's the position on
the board that makes them discernable not their properties as
individuals.)

So substitutability and token inferencing (or "tokenization"?) seem to
be equally primitive.  You cannot go treat a particular as a
token-particular unless you can also treat particulars as
substitutable "under the type".  Alternatively: it is not possible to
play the game of tokens and types if you cannot also play the
substitution game.  And vice-versa.

In more formal terms: tau types are primitive.  They are an
explicitation of the norms of both tokenization and substition
practices.

Natural types v. fiat types: the counter or pebble types in our
examples count as natural types.  They are instituted by our practices
but do not involve explicit rules.  The types of type theories like
HoTT are fiat types: they are instituted by explicitly articulated
rules.  Nevertheless the meaning of those rules is to be
explained in terms of normative practices.

Types as institutions, literally.

refl as token of type Tau(a).  What would count as a physical token of
equality in our tallying game?  In HoTT terms: to prove something for
all T(a) tokens, it suffices to prove it for ... T(a) itself?  Or any
T(a) token?  With the concept of T(a) we loose the notion of canonical
constructor.  Unless we want to treat Tau itself as the canonical
ctor.

But: tokens are not physical, particulars are.  Tokens (and types) are
(functional) roles, rules in the game.  So even our pebbles are not,
strictly speaking, physical tokens.  So a better question to ask is:
what plays the role of an equality token (i.e. a "proof" of equality)?
Answer: tau types?  A tau type T(a) is just the role that particulars
may play, as a-tokens.

Primary v. secondary identity.  Particulars have identities; this is
primary identity.  Token-particulars play roles: this is secondary
identity.  Each red checker piece has an identity as a particular; no
two pieces are identical in this sense.  But qua game pieces, they
play "the same" role; they are all "identical" in this derived sense.
And what is identical is not the pieces themselves, but the role they
play, since there is only one such role: red game piece.  Not the
pieces but their roles are "the same".  Every token-particular has two
identities in this sense, one primary (or primitive) and the other
secondary (or derived).  Derived from both primary identity and the
rules of the game.  The rules of the game institute the derived
identity, but the play that role, a game piece must (antecedently)
have primary identity, i.e. be a particular.

Inferential Type Semantics.  Or, Type-Inferential Semantics.

\section{Refl and Path Induction}

We don't need refl; it's an artifact of denotational semantics.  We
write ``a=a'' so we want to interpret this as a binary relation
between a thing and itself.  Which is (according to some,
e.g. Wittgenstein) nonsense.

Using Unit instead of Id, and ** (or ``exch'' or whatever) instead of
refl eliminates this conceptual wart from our system.  Then instead of
treating refl as the canonical constructor, and interpreting it to
mean that all a=b are ``freely generated'' by refl (nonsense), we
interpret our constructor \(**a: Unit_A a a\) as certification that
every token of type T[a] is exchangeable with every other such token.
We do not need a concept of reflexivity.

But: \(Unit_A a a\) and \(Unit_A a b\) are different types.  How can
one constructor work for both?  But are they really different types,
if a = b?  No; they only look like different types, since tokens 'a'
and 'b' look different.  But they are both tokens of the same Tau
type.  There is only one type \(Unit_A a \_\) for any give a.  Or,
there is only one type \(Unit_A \_ \_\) for every a:A - it doesn't
matter what the arguments to \(Unit_A\) are - fix either one and the
other must be equal to it for the type to be inhabited.

Why can't we say e.g. \(refl_2 : (2 = 3)\)?  In HoTT, refl is only
defined for (a=a), so we cannot write \(refl_a : (a = b)\).  But then
how can we prove a=b?  HoTT says that refl suffices for this, and the
reasoning is that paths a-b are ``freely generated'' by the constant
path at a.  Which implies that \(refl_a\) denotes the constant path at
a, although the HoTT book does not explicitly invoke denotation; it
just says ``We regard \(refl_a\) as being the constant path at the
point \(a\).'' (p. 48) The problem is that this principal of path
induction depends on a denotational interpretation; it can't be a mere
informal gloss.  Freely generated paths at \(a\) cannot count as
proofs of (a=b) unless we transfer this notion of free generation from
the semantic domain back to the type language; that is, we just
stipulate that any \(x:(a-b\) denotes one of the paths freely
generated by the constant path at a.  I.e. the proof tokens
\(x:(a=b)\) are ``freely generated'' by the canonical token
\(refl_a\), in the same way that paths a-b are freely generated by the
constant path at a.  Same concept, different domain.  The problem is
that ``free generation of tokens'' is not a concept of type theory,
since it is non-constructive.  Without (non-constructive) denotational
assignment we have no justification for treating \(refl_a\) as
sufficient for proving \((a=b)\), or for thinking that free generation
of paths justifies free generation of tokens (proofs).

(On the other hand: since we can freely create new tokens, it might
make sense to introduce an idea of ``free construction of tokens'' to
correspond to free generation of structures.)

So the HoTT book's account of path induction is understandable and
useful, but it is not genuinely constructive; it does not explain
equality types.

One possible reponse to this objection is that path induction
is actually defined by the Pi expressions in the book at page 49, not
by the ``informal'' explanation of equality in terms of freely
generated paths.  But what do those expressions really mean?  As far
as I can see, all they do is give two different ways of expressing
reflexivity, two ways of constructing a witness to a type built from
\(x, x, refl_x\).  Specifically, they build a new type from equality
types and refl, and provide two constructors (c and f) for that type;
that's all.  They specifically do not explain how refl can be counted
as sufficient for a=b.

In other words, the formal definition of path induction begs the
question; it does not explain how refl suffices, it just takes it for
granted that it does.

What's the heart of the problem?  Just that we can have multiple
``proofs'' of equality, i.e. a and b can be equal in a plurality of
ways.  The HoTT strategy for dealing with this is to privileged one
way of being equal (i.e. the constant path, the refl constructor), and
treat all the other ways as derivative (freely generated by the
privileged one).  We propose an alternative that does not privilege
any particular ``way of being equal'', in which reflexivity plays no
role.  Note that ``constant path at a'' presupposes that a path is a
function, which is not consistent with the notion that HoTT leaves out
the topology.  It would be more consistent to call it a null path, one
that goes nowhere, is not really a path.  On our approach, which
eschews the path interpretation, this corresponds to trivial notion
that every token has identity, which is a predicate rather than a
(reflexive) relation.

Better: free tokenization.  tau-types are ``freely tokenized''.

\section{Notes}

The HoTT book seems to treat its Id/= type as special.  But there's
nothing special about it; its one of infinitely many equality types
whose properties derive entirely from the structure of the type
theory, in particular the inductive structure of elements and
successor types.

``The key new idea of the homotopy interpretation is that the logical notion of identity a = b of two objects a, b : A of the same type A can be understood as the existence of a path... `` p. 3

`` In type theory, for every type A there is a (formerly somewhat
mysterious) type \(Id_A\) of identifications of two objects of A; ...''
p. 4 I think this is not right; there is no special type Id.  There
are families of relational types etc. and criteria for deciding when
to call something reflexive, symmetric, etc.  So there are lots of
equality types.

Just as every relation has reflexivity types \(a\circ_A a\) which may
or may not be inhabited, so they all have symmetry types \(p
\circ_{\circ_A} q\) which may or may not be inhabited.  No, because we
may not p and q.  We always have the types for which they are proofs,
however; maybe we should try relating those.

%% (p\ &\circ_{(x\circ_Ay)} p^{-1}){:}\,\Univ & \textit{---given \(p{:}(a\circ_A b),\  p^{-1}{:}(b\circ_A 

\subsection{Inductions}

Three kinds: mathematical, structural, transfinite.  What all have in
common is forward progress.  Which is missing from path induction.
Can we use successor induction to recover a notion of forward/upward
progress for path induction?  We need a way to construe \(p{:}x\circ_A
y\) as successor to \(\textsf{refl}_A(x){:}\,x\circ_A x\).

Both refl and p have successor types.  Need to prove conditional: if
tsucc(refl) then tsucc(p).  No way to do that, hence appeal to induction
principle.

Put it another way: if reflexivity types are inhabited, then so are
symmetry pairs.  So symmetry pairs together ``look like'' reflexivity
types.  Maybe their proofs are reflexive?

But how to avoid circularity?  If reflexivity types are inhabited,
then so are symmetry pairs -- but only for those symmetry pairs that
are inhabited, not all of them.  Not 2=3 and 3=2, for example, but 2=x
for all x equal to 2.

Note refl works for both canonical and non-canonical x:A.

\section{ST/FOL=}

``ST/FOL='' = Set Theory/First Order Logic with Equality

\subsection{Relations}

In ST/FOL=, a binary relation is a subset of \(A\times B\), where
\(A\) and \(B\) are sets.

Given set \(X\) and binary relation \(\circ\) on \(X\times X\):

\begin{itemize}
\item \textit{Reflexivity:}\quad \(\forall x\in X\ x\circ x\)
\item \textit{Symmetry:}\quad \(\forall x,y\in X\  (x\circ y)\leftrightarrow (y\circ x)\)

  E.g. both \(\neq\) and \(=\) are symmetric, but only the latter is also reflexive and transitive.
\item \textit{Transitivity:}\quad \(\forall x,y,z\in X\  (x\circ y)\land(y\circ z)\to (x\circ z)\)
\item \textit{Equality:}\quad a binary relation is an equality ``='' iff it satisfies reflexivity, symmetry, and transitivity
\item \textit{Anti-symmetry:}\quad \(\forall x,y\in X\  (x\circ y)\land(y\circ x)\to x=y\)

  E.g. \(\leq\) is anti-symmetric.
\end{itemize}

\subsection{Axioms of Set Theory}

\end{document}


%%%%%%%%%%%%%%%%%%%%%%%%%%%%%%%%%%%%%%%%%%%%%%%%%%%%%%%%%%%%%%%%
\chapter{The Language of \HoTT}
\label{sect:hottlang}

\begin{ednote}
  TODO: disentangle commonly used terminology: parameterized types,
  generic types, algebraic types, generalized algebraic types, types
  indexed/parameterized, polymorphism, parametric polymorphism,
  polytypism etc. etc.
\end{ednote}

\begin{ednote}
  Type formers as entries in a \HoTT{} lexicon, serving not as
  definitions but as (normative) rules of usage.  They don't say what
  the terms mean, they set out how to use them.  That means,
  specifically, the rules governing the notation, not rules governing
  denoted entities.

  NB: rules of use \emph{of vocabulary} \(\neq\) rules of construction
  \emph{of objects}.  But the idea is for one set of rules to work
  both ways.  That's pretty much how model-theoretic semantics
  connects vocab to semantic domain (completeness and consistency).
  The difference is that the TT semantic domain here (i.e. objects and
  their rules of construction) is not a passive, platonistic realm of
  ``real'' objects, but a pragmatic ``field'' of action.  So rules of
  vocab use and rules of construction converge while remaining
  conceptually distinct.  IOW the difference is not metaphysical.

  On this view we treat \HoTT{} as truly a vocabulary rather than a
  theory about something.  Or more precisely as a regimented idiom or
  dialect.  The user is free to treat e.g. \N as ``defining'' a true
  model of the natural numbers, but \HoTT{} makes no such claim.
\end{ednote}

One way to think about mathematics and logic is in terms of objects,
structures, relations, and the like.  etc.

But one can also think of it in terms of vocabularies (or idioms,
etc.).  Then mastering a discipline is not just a matter grasping some
content, but also of acquiring practical mastery over a vocabulary.

The vocabulary of set theory has dominated mathematical discourse for
most of the last 100 years or so.  Starting in the late 1940s, a
competing vocabulary based on category theory began to emerge.  Today
it is not uncommon to see both vocabularies deployed in the same
discourse (lecture, paper).

{\todo Type theory as a vocabulary - mostly confined to logic, then
  computer science.  Etc.  \HoTT{} as the latest distinctive vocab. -
  covering both math and compsci, also regions of logic.
  Significantly different that both set theory and classic logic.}

``it is possible to directly formalize the world of homotopy types
using the class of languages called dependent type systems and in
particular Martin-Lof type systems.'' V. Voevodsky
\url{http://www.math.ias.edu/~vladimir/Site3/Univalent\_Foundations\_files/univalent\_foundations\_project.pdf}

Note: ``class of languages called dependent type systems'' -
languages, not theories

``Type systems are syntactic objects which are specified in several
steps. First one chooses a formal language L which allows the use of
variables and substitution. Then one chooses a collection of relations
on the sets of L-expressions with a given set of free variables which
is stable under the substitutions. These relations are called the
reduction rules and the equivalence relation generated by the
reduction rules is called the conversion relation....  A type system
based on L is defined as a pair of subsets BB and BBg in the sets of
pre-contexts and pre-sequents respectively which satisfy a number of
conditions with respect to reduction and substitution. Elements of BB
are called the (valid) contexts of a type system and elements of BBg
the (valid) sequents of the type system.'' same, p. 3



%%%%%%%%
\chapter{\HoTT{} Types}
\label{subs:hott}

\HoTT{} primitives are ....\sidenote{A proper exposition would
  list 1) the name of the primitive, e.g. ``Dependent Product-type'';
  2) the ``constructor'' symbol, e.g. \(\Pi\); 3) the analogous
  concept from set theory; 4) the ``rules'' for defining a type
  (formation, construction, elimination, computation, uniqueness).}

\begin{ednote}
  The concept of primitives probably isn't going to work for type
  theories proper.  Type theories seem to be inherently pluralistic,
  so there is no way to pick out some things as intrinsically
  primitive in all type theories.  Each theory might \emph{define} a
  set of primitives, but that would define conventions of the theory,
  not the notion of primitive that we're after here.  So if we want to
  talk about primitives it will have to be as above, involving
  principles antecedent to any type-theoretic talk. (?)
\end{ednote}

%%%%%%%%
\section{Simple Types}
\label{subs:simpletypes}

\begin{ednote}
  E.g. \N
\end{ednote}

``Proposition type'' Conceptually, at least, this seems primitive.
Especially if the concept of ``proposition'' counts as a pre-theoretic
principle.  Which implies that proof is also a pre-theoretic
principle.  Proposition types are fundamentally different than the
other kinds of type, since they have
truth-conditions.\sidenote{Actually this isn't quite right.
  Propositions have truth-conditions in classic logic, but not not in
  type theory.  In type theory they have proofs; a propositional type
  is not true or false, but proven or disproven.  But the larger point
  stands: the \textit{concept} of proposition is different than the
  concept of, say, natural number.} Etc.  It follows that proofs are
fundamentally different from other kinds of witness.

\begin{ednote}
  But Curry-Howard means there is no distinction between types and
  propositions, so it makes no sense to try to demarcate a
  ``proposition type''.  E.g. the type \N can be viewed as the type of
  ``there exists a natural number''.  This feature demarcates type
  theory; in classic logic and math, and esp. traditional logic, there
  is a fundamental difference between propositions and the terms from
  which they are constructed.  Not so in \tth{}.
\end{ednote}


%%%%%%%%
\section{Compound Types}
\label{subs:compountypes}

\begin{description}
\item [Function] ``Unlike in set theory, functions are not defined as
  functional relations; rather they are a primitive concept in type
  theory.''\sidenote{Or: set theory
    \textit{defines} a function as a set of ordered pairs whose domain
    has no duplicates; in other words, it treats a function and its
    ``graph'' as the same thing.  Question: what happens to the graph
    of a function in type theory?} \citep[p. 21]{hottbook}

\item [Product] Product types correspond to cartesian products in set
  theory.  The constructor symbol is the same as in set theory:
  \(\cross\).\sidenote{Why isn't this called the ``\(\Huge\cross\)-type''?}
  ``[U]nlike in set theory, where we define ordered pairs to be
  particular sets and then collect them all together into the
  cartesian product, in type theory, ordered pairs are a primitive
  concept, as are functions.''\citep[p. 26]{hottbook}

\item [Coproduct type] Coproduct types correspond to disjoint unions
  in set theory.  ``In type theory, as was the case with functions and
  products, the coproduct must be a fundamental construction, since
  there is no previously defined notion of ``union of
  types''.\citep[p. 33]{hottbook}

\end{description}

\begin{ednote}
  [Updated] [Update: M. Shulman pulled the scales from my eyes: ``From
    the point of view taken in the book, the difference is only one of
    perspective, and any type can represent a proposition by simply
    shifting our perspective on it. For instance, the type Nat
    represents the proposition "there exists a natural number".'']
  What I've called ``proposition type'' is not \textit{formally}
  distinct; the distinction I'm after is conceptual.  But this
  suggests that we need to add at least one more item to our list of
  primitives: ``ordinary'' or simple types.  Maybe it would be best to
  start with the natural numbers as an example of a simple type,
  rather than function types.  Then an example of a proposition type,
  before proceeding to function type.  The items listed (following the
  \HoTTB{}) are really constructions, or let's say complex types,
  built out of two other types.  So maybe the distinction we want is
  between simple and complex or compound types.  Then the simple types
  would come out as primitive, and the complex types as derived (just
  like dependent types.)  Compare the idea of constructions in
  category theory.  There categories are primitive and e.g. the
  product category is an example of a category constructed from other
  categories.
\end{ednote}

\begin{ednote}
  For consistency, we might want to use symbols to designate all of
  the primitives, just as we do for \(\Pi\) and \(\Sigma\).  This
  would give us: \(\Huge\fun\)-types, \(\Huge\cross\)-types, and
  \(\Huge +\)-types.
\end{ednote}


%%%%%%%%
\section{Dependent Types}
\label{subs:quasiprim}

\begin{ednote}
  Add sth re:  types indexed over n v. parameterized over \(\alpha\)
\end{ednote}

\begin{ednote}
  Major TODO: ``basic'' types (as in Z) plus constraints (predicates)
  v. complex dependent types.

  Example: the paradigmatic example for dependent types is a list.
  This combines a type parameter \(\alpha\) and an index \(n\):
  List~\(\alpha, n:\nat\) means a list of length \(n\) of values of
  type \(\alpha\).

  But that's only one possibility.  Let's add a real number:
  List~\(\alpha, n:\nat, x:\real\).  In this case \(x\) could mean
  anything: sum of the elements of the list, product, mean, standard
  deviation, etc.  In short \(x\) can be any \emph{statistic} computed
  over the list.

  This works intuitively; does it work in type theory?  Specifically,
  \ML{}-style theories like \HoTT?  Would these be genuine types or
  types plus additional constraints?  In other words, is the
  interpretation of \(x\) built-in to the type, or is it an ``extra''
  constraint applied after the fact, as it were.
\end{ednote}

\noindent ``Fundamental''\sidenote[][-28pt]{\begin{ednote}
  Obviously we need a better bit of
  terminology.  ``Quasi-primitives''?  ``Neo-primitives''?  These
  types are not primitive, strictly speaking, but on the other hand
  they are basic.  I think there is another fundamental principle at
  work here.  In set theory, for example, the concept of function is
  not only not primitive, it isn't necessary.  You could discard it
  and still have set theory.  But my intuition tells me that e.g. the
  concept \(\Pi\)-type is in a sense necessary or essential in type
  theory, even if it is not primitive.  Once you have the primitives,
  you necessarily have these non-primitive basic types.  Dunno if
  that's correct, but it would sure be nice if it were.
  \end{ednote}}%
(but non-primitive) concepts and types.  These types seem to be on a par
with the primitive types as far as importance goes, but they
presuppose the primitives, so cannot themselves be considered
primitive.

\begin{ednote}
  We can make a distinction between the concept of dependent type, and
  the two specific dependent types introduced here.  Neither is
  primitive; you can have a type theory without the concept of
  dependent types.  Most programming languages fit this description,
  whether they have an explicit type discipline or not.
\end{ednote}

\begin{ednote}
  Regarding the notion ``quasi-primitive'': not a very satisfactory
  term, but I can't come up with a better one at the moment.  What I'd
  like to show is that the concept of dependent type (maybe also type
  universe) follows ``naturally'' or necessarily or essentially from
  the more primitive concepts.  Maybe the right concept here is
  ``induction'': the primitive concepts (types) ``induce'' the concept
  of dependent type.  That would be nice esp. if induction is a
  primary principle.
\end{ednote}

\begin{description}

\item [Universe]  Is this a primitive?  Probably not, since it builds on the type concept.

\item [\(\Pi\)-type] Informally, ``dependent function''
  types.\sidenote{As a practical matter, I think it would be useful to
    have an informal term for these types that falls between
    ``dependent function type'' and \(\Pi\)-type.  Something like
    ``p-function type''.  \(\Pi\)-type is admirably concise, but I
    think it should mention ``function'', since it names a kind of
    function.}  The concept of \(\Pi\)-type is a generalization of the
  concept of function type, so it isn't primitive.

\item [\(\Sigma\)-type] Informally, ``dependent pair''
  type.\sidenote{Shouldn't this be called ``dependent
    \textit{product}'' type?  The type is product, not pair; pairs are
    ``elements'' of the type.  Informally, maybe ``sig-prod type?}
  The concept of \(\Sigma\)-types is a generalization of the concept
  of product type.

\end{description}

%%%%%%%%
\section{Standard Type Library}
\label{subs:hottstdlib}

\begin{ednote}
  By analogy to the usual ``standard library'' of programming
  languages.  The idea is to list commonly used types that are neither
  primitive nor quasi-primitive; ``application'' types, in a sense.
\end{ednote}


\begin{description}
\item [Boolean] \citep[p. 34]{hottbook}
\item [$\nat$] \citep[p. 36]{hottbook} But there's a problem here;
  actually several.  First of all, the section on the naturals in
  chapter one does not actually show how to construct them; its really
  a section about induction, not the natural numbers.  Second, what it
  does say about the naturals is that they start from zero.  That
  obviously won't do; zero is not a natural number, and there is no
  intuitive notion of zero as a number.  You can't even think of it as
  a number until you've severed the link between the concept of number
  and the concepts of quantity and/or magnitude.  So the natural
  numbers really must start with 1, not 0.
\item [Proposition]  Moved to Primitives section.
\item [Identity]
\end{description}


\section{Misc. Niceties}

\begin{ednote}
  tait: ``objects are given or constructed as object of a given
  type''.  The expression ``a : A'' expresses the idea that we are
  given a of type A.  It does this by stipulation rather than
  assertion.  Assertions are challengable and must be justified on
  demand; stipulations are not and need not.
\end{ednote}

ML Type Theory is centered (more or less) on one of the major
logico-philosophical topics of the 20th century, namely the nature of
assertion and its relation to propositions and inferences.

You don't have to understand the arcana of this debate in order to
understand type theory (or HoTT), but some familiarity with the main
outline is very helpful.  Actually, I think it's essential, if you
want to understand the HoTT Book's account of \textit{judgement},
presented in HoTT Chapter 1 (reproduced below).  Fortunately the
presentation is relatively straightforward.

\begin{remark}
  Stress: this is largely a philosophical issue, or perhaps an issue
  in Philosophy of Language.  It's really about how our utterances
  come to have the significances they do.
\end{remark}

Outline:

\begin{itemize}
\item Frege's elevation of \textit{force} as essential
\item Dealing with embedded (and therefore forceless) propositions
\item Wittgenstein
\item Dummett
\item etc.
\item Brandom's recent innovation: decompose ``assertion'' into ``commitment'' and ``entitlement''
\end{itemize}

What the HoTT Book refers to as judgment (following ML) could also be
called assertion.  Brandom's account of the ``fine structure'' of
assertion is very helpful here.  Among other things, it provides a
very simple explanation of how embedded propositions work.  Embedded
propositions are unasserted; the problem is how to reconcile this with
the fact that they are function as assertions if unembedded.  On
Brandom's account, [todo...]

In other words, we can have commitment with or without entitlement,
and vice-versa.

A set membership statement can be explained in terms of commitments
and entitlements.  A free occurance of e.g. \(a\in A\) is ordinarily
taken as an assertion (judgment).  We can follow Frege and make this
\textit{force} explicit: \(\vdash a\in A\).  The problem with this,
however, is that, in contemporary usage, this would make \(a\in A\)
\textit{logically} true, which is not what we want.  Instead we want a
representation of committment to the proposition, as at least
ordinarily true, without regard to its logical truth.\marginnote{TODO:
  logical v. ordinary truth is pretty hairy for non-logicians so the
  distinction should be explicated.}

In sum:  the implicity sense of \(a\in A\) is something like: 
\[\exists \Gamma, a, A | \Gamma\vdash a\in A\]

Informally: there exists a set of propositions \(\Gamma\), a value (or
object) \(a\), and a set \(A\) such that the propostion \(a\in A\) is
deducible from \(\Gamma\).

So the meaning of \(a\in A\) essentially involves existential
quantification.  It is a statement about the world, that it contains
the relavant entities, not about the entities themselves.

\begin{remark}
  Not quite; \(a\in A\) is surely a statement about \(a\), maybe also
  about \(A\), no?  But still there must be an implicity existential
  quantification over the propositions that entail the statement.
\end{remark}

There is a logical subtlety here.  \(a\in A\) seems to be about a
determinate \(a\) and a determinate \(A\), but it isn't, not if we
take it to be an existentially quantified statement.  That's because
\(\exists a, A | a\in A\) does not pick out determinate individuals;
it just says that \textit{some} such individuals exist in the domain
of interpretation.  True, \(a\) and \(A\) are said to be bound by
\(\exists\), but that's not entirely accurate; quantified variables
are not bound in the way that constant symbols like \(\pi\) or \(0\)
are bound.  Whatever we go on to say about \(a\) and \(A\) --
e.g. \(a\in A\) -- remains within the scope of the quantifier, so it
does not count as a statement about determinate individuals.  It's a
statement about the world, that it contains entities that satisfy the
predicate.

On the other hand, the same seems to be true of \(a : A\): though
these symbols be bound, we don't know what they are bound to.  They
are not bound by an implicit existential quantifier; \(a : A\) does
\textit{not} mean \(\exists a, A | a : A\).

\begin{remark}
  Plus, quantifiers have to be used with a predicate; strictly
  speaking, \(\exists a, A\) is not a complete statement.
\end{remark}

By contrast, the Type Theoretic analogue \(a : A\) is a statement
about a specific value and a specific type, without any
quantification.  It is not directly a statement about the world, but
about part of the world.  Or: it expresses both commitment and
entitlement.  That's why it cannot be embedded in e.g. ``if \(a : A\)
then it is not the case that \(b : B\)''.  Embedded propositions
cannot carry force, but \(a : A\) always carries force, intrinsically,
as it were.

%%%%%%%%
\subsection{a : A}
\label{subs:aA}

Forms go from symbols to terms to sentences; from \(a\) to \(a+b\) to
\(a+b=c\).

The ``judgment'' \(a : A\) is clearly a compound term, so it cannot
merely name something.  But is it a sentence?  Does it denote a
proposition?  Or is it analogous to terms like \(a+b\) which are names
of a sort but involve some additional meaning beyond mere reference.

It seems it must involve a proposition, or let's say propositional
content.  We take \(a : A\) as a statement of fact, rather than a mere
reference to some part of the world.  Then how is it distinct from
\(a\in A\)?

The HoTT Book says it is ``analogous'' to the set-theoretic statement
\(a\in A\), but essentially different, since \(a\in A\) is a
proposition but \(a : A\) is a judgment.  It says that, \textit{when
  working internally in type theory}, \(a : A\) cannot be embedded, as
in `` if \(a : A\) then it is not the case that \(b : B\)'', nor can
the judgment \(a : A\) be disproved.

So let's look closely at what this means.  Earlier, HoTT says that
(some) judgments involving A ``exist at a different level from the
\textit{proposition} \(A\) itself, which is an internal statement of
the theory.''  (p. 18) There's a bit of circularity there; what is an
``internal statement''?

{\todo The nature of ``proposition'' has been a topic of
  considerable debate.  Review some of the alternative accounts on
  offer.}


The basic idea seems to be based on the well-known concept that
propositions by themselves are devoid of force, and must be asserted.
HoTT seems to imply that judgments are asserted propositions -- or
more correctly, assertings of propositions.

This seems a little bit off.  Assertion is something only people do.
An inked form on a page cannot really be construed as an assertion.
So we need to work out the mechanics of how a written form like \(a :
A\) can be viewed as a ``judgment'' in this sense.  I think Brandom's
model of assertion works.  It would say, I think, that \(a : A\)
counts as a judgment (assertion) because by convention we agree to
treat it that way, whereas we treat \(a\in A\) slightly differently,
because of the conventions elaborated by 20th century logic.

When HoTTB refers to ``working internally in type theory'', it seems,
the idea is to consider propositions in isolation from their
assertion.  Assertion, on this view, is something that comes from
outside of the world of propositions.  This is perfectly in tune with
the idea that asserting is something people do, but that what gets
assert\textit{ed} -- the \textit{content} of an assertion -- is
distinct from the assert\textit{ing}.

\begin{remark}
  Sellars called this the notorious -ing/-ed distinction.
\end{remark}

This would seem to make \(a : A\) an assert\textit{ing}.

We can think of \atypeA{} as a \textit{given} proposition: one that,
while unasserted, has the same force as a propositional assertion.  Or
another way to put it would be to say that use of \atypeA{} is
inalienably performative.

In fact \atypeA{} corresponds nicely to a common linguistic practice,
namely combining a proper name and a description, as in ``Joan of
Arc'', ``King George'', or ``Slick Willy''.  Or, more colloquially,
``poor Tom'', ``angry Joe'', or ``Gimpel the fool''.  And the
primitive nature of types can be clearly illustrated by analogy with
the military.  In type theory, every object has a type, just as
everybody in the military has a rank.  You cannot be in the military
unless you have a rank.  Within the military, the proper way refer to
someone in the miliary is to combine rank and name: General Custer,
Sergeant York, Private Bilko.  So the difference between \atypeA{} and
\(a\in A\) is like the difference between ``This is General Custer''
and ``This is Custer; he is a General''.

On the other hand, ``This is General Custer'' doesn't look much like a
\textit{judgment}, although it does look like a \textit{claim}.  But
not that it is not a claim about the meaning of ``General Custer'';
rather it is a claim about the relation between ``This'' and ``General
Custer''.  You could be wrong about the name or the rank of whomever
you mean by ``This'', but you cannot be wrong about ``General
Custer''; that's just a qualified name.  Being wrong in this sense
about ``This is General Custer'' is an empirical matter; in type
theory, the question of whether \atypeA (``this is a-of-A'') is
correct or not never even arises.  It doesn't make an empirical
assertion, it states a \textit{given}.  Or we might say it gives a
fact.  By contrast, \(a\in A\), as a proposition, may be either true
or false; when we say ``let \(a\in A\)'', we implicitly stipulate that
\(a\in A\) is to be \textit{assumed} to be true, but it is not
\textit{given} as true.  In other words, we can gloss it as ``\(a\in
A\) has a truth-value like any proposition, so it could be false, but
please assume that it is true.''

Another critical distinction: in standard set theory and logic,
judgments come from the outside, as it were.  But in HoTT, judgments
of the form \atypeA are internal.  They may be derivable inside the
system (by production of a proof or witness.)  In other words,
inference in set theory comes from outside of the world of sets, but
inference in HoTT is built in to the structure of types.  Inference
(construction) is part of the intrinsic meaning of types.

%%%%%%%%
\subsection{Justification}
\label{subs:just}

The HoTT Book's account of judgments in Chapter 1 section 1 seems to
conflate the distinction\sidenote{This is only to be
expected, since Brandom is the first (so far as I know) to see that
assertion (judgment) has an internal structure involving commitment
and entitlement (and some other stuff like a social dimension.)} Brandom makes between commitment and
entitlement.  ``Informally, a deductive system is a collection of
rules for deriving things called judgments.''

But derivation (proof) starts and ends in propositions; commitment is
something else.  The derivation or proof provide warrant for
entitlement to the commitment - justification of the conclusion.  So
how would Brandom parse ``judgment'' as HoTT uses the term?



\clearpage
\appendix
\begin{appendices}
  %%%%%%%%%%%%%%%%%%%%%%%%%%%%%%%%%%%%%%%%%%%%%%%%%%%%%%%%%%%%%%%%
  \chapter{HoTT Book Excerpts}
  \include{introduction}

  \include{preliminaries}

  %%%%%%%%%%%%%%%%%%%%%%%%%%%%%%%%%%%%%%%%%%%%%%%%%%%%%%%%%%%%%%%%
  \chapter{Lexicon}

\begin{description}
\item [BHK]
\item [Witness]
\item [Curry-Howard]
\item [Combinator] 
\item [Proof obligation]
\item [Induction] 
\item [Recursion] 
\end{description}



  %%%% Bibliography
  %% \bibliographystyle{halpha}
  %% \phantomsection % black magic to get TOC to point to correct page
  %% \addcontentsline{toc}{part}{\bibname}
  %% \markboth{}{\textsc{Bibliography}}
  %% {\renewcommand{\markboth}[2]{} % Prevent bibliography from resetting the header to something silly
  %% \OPTbibliographyfont

  \chapter{Proof Assistants}

The \textit{specification} language of Coq is Gallina.

The ``vernacular'' is Coq's command language - it allows you to talk
to the Coq system itself.

\section{Coq}

\subsection{Coq'Art}

\enquote{The relation between a program and its type is the same as
  the relation between a proof and that statement it proves.  Thus
  verifying a proof is done by a type verification algorithm.} p. 4

\enquote{An important characteristic of the Caculus of Constructions is that
every type is also a term and also has a type.  The type of a
proposition is called \texttt{Prop}.  For instance, they proposition
\(3\leq 7\) is at the same time the type of all proofs that 3 is
smaller than 7 and a term of type \texttt{Prop}.} p. 4

\enquote{In the same spirit, a \textit{predicate} makes it possible to
  build a parametric proposition...[E]xamples of predicates are the
  predicate `to be a sorted list' with type \texttt{(list Z)->Prop}
  and the binary relation \(\leq\), with type \texttt{Z->Z->Prop}.} p. 4


  %% \bibliography{references}
  %% \bibliographystyle{plainnat}

\end{appendices}

\end{document}
