\documentclass[12pt,toc]{tufte-handout}
%% \documentclass[reqno,12pt]{tufte-book}
%% \usepackage{trace}
%% \documentclass[reqno,12pt]{article}

\usepackage{draftwatermark}

% BLACK & WHITE
\input{opt-black-white}

% FORMATTING DEPENDENT ON PAPER SIZE
\input{opt-letter}

\usepackage{etex}

%%%%%%%%%%%%%%%%%%%%%%%%%%%%%%%%%%%%%%%%%%%%%%%%%%%%%%%%%%%%%%%%
%% packages included by original hott main.tex

%%% For table {tab:theorems}
\usepackage{pifont}

%%% Multi-Columns for long lists of names
\usepackage{multicol}

\usepackage{graphicx}
\usepackage{comment}

\usepackage{fancyhdr} % To set headers and footers

%% \usepackage{nextpage} % So we can jump to odd-numbered pages

\usepackage{amssymb,amsmath,stmaryrd,mathrsfs,wasysym}
\usepackage{enumitem,mathtools,xspace}
%% \numberwithin{equation}{subsection}

\usepackage{xstring} % For generating singluars and plurals in \backref

%% \usepackage{xcolor,mdframed}
\usepackage{xcolor} % For colored cells in tables we need \cellcolor
\usepackage{wallpaper} % For the background image on the cover page

\usepackage{booktabs} % For nice tables
\usepackage{array} % For nice tables

\definecolor{linkcolor}{rgb}{\OPTlinkcolor}
\usepackage{aliascnt}
\usepackage[capitalize]{cleveref}
\usepackage[all,2cell,cmtip]{xy}
\UseAllTwocells
%\usepackage{natbib}
\usepackage{braket} % used for \setof{ ... } macro

\usepackage{tikz}
\usetikzlibrary{decorations.pathmorphing,arrows}

\usepackage{etoolbox}           % hacking commands for TOC

%% \usepackage{mathpartir}         % for formal.tex appendix, section 3

\usepackage[numbered]{bookmark} % add chapter/section numbers to the toc in the pdf metadata
%%%%%%%%%%%%%%%%%%%%%%%%%%%%%%%%%%%%%%%%%%%%%%%%%%%%%%%%%%%%%%%%

\usepackage{framed}
\usepackage[standard,framed]{ntheorem}
%% \newtheorem{theorem}{Theorem}
%\newtheorem{cor}{Corollary}
%\newtheorem{lem}{Lemma}
%% \newtheorem*{defn}{Definition}
%% \theoremstyle{remark}
%% \newtheorem{remark}{Remark}
%% \newtheorem*{commentary}{Commentary}

%% \theoremclass{Remark}
%% \theoremstyle{break}
%% \newtheorem{note}{Note}[section]

\theoremstyle{plain}
\theorembodyfont{\upshape}
\theoremsymbol{\ensuremath{\ast}}
\theoremseparator{}
%% \newtheorem{ednote}{Ed. note}[section]
\newframedtheorem{ednote}{Ed. note}[section]

\newtheorem*{todo}{TODO}
%% \newtheorem{eg}{Example}

\input{macros}

\usepackage{appendix}

%% \usepackage{csquotes}

\usepackage{setspace}

%% broken (doesn't work with tufte-handout):
\usepackage{zed-csp}
%% broken:
%% \usepackage{ltcadiz-fam}

\usepackage{fontspec}
%% \usepackage{xltxtra,xunicode}
\defaultfontfeatures{Scale=MatchLowercase}

%% \defaultfontfeatures{Scale=MatchLowercase}
%% \setmainfont[Mapping=tex-text]{Times New Roman}
%% \setsansfont[Mapping=tex-text]{Arial}
%% \setmonofont{Courier}

\setmainfont[Ligatures=TeX]{TeX Gyre Bonum}
\setromanfont[Ligatures=TeX]{TeX Gyre Bonum}
\setsansfont[Ligatures=TeX]{TeX Gyre Adventor}
\setmonofont[Ligatures=TeX]{TeX Gyre Cursor}


%% \setmainfont[Mapping=tex-text]{Minion Pro}
%% \setromanfont[Mapping=tex-text]{Minion Pro}
%% \setsansfont[Mapping=tex-text]{TeX Gyre Heros}

%% Bugfix: see https://code.google.com/p/tufte-latex/issues/detail?id=64
% Set up the spacing using fontspec features
\renewcommand\allcapsspacing[1]{{\addfontfeature{LetterSpace=15}#1}}
\renewcommand\smallcapsspacing[1]{{\addfontfeature{LetterSpace=0.0}#1}}

\usepackage{epigraph}
\setlength{\epigraphwidth}{.8\textwidth}

%% general symbols - degree, etc.
%% \usepackage{gensymb}

\usepackage [english]{babel}
\usepackage [autostyle, english = american]{csquotes}

%% nice double-stroke fonts
\usepackage{dsfont}

% Small sections of multiple columns
\usepackage{multicol}

% Provides paragraphs of dummy text
\usepackage{lipsum}

% The units package provides nice, non-stacked fractions and better spacing
% for units.
\usepackage{units}

%\usepackage{geometry}                % See geometry.pdf to learn the layout options. There are lots.
%\geometry{letterpaper}                   % ... or a4paper or a5paper or ...

\usepackage{xfrac}

\usepackage{hyperref}
\hypersetup{
  bookmarks=true,         % show bookmarks bar?
  bookmarksdepth=3,
  unicode=true,          % non-Latin characters in Acrobat’s bookmarks
  pdftoolbar=true,        % show Acrobat’s toolbar?
  pdfmenubar=true,        % show Acrobat’s menu?
  pdffitwindow=false,     % window fit to page when opened
  pdfstartview={FitH},    % fits the width of the page to the window
  pdftitle={Intuition and Exponentiation},    % title
  pdfauthor={G. A. Reynolds},     % author
  pdfsubject={Mathematics},   % subject of the document
  pdfcreator={G. A. Reynolds},   % creator of the document
  pdfproducer={Producer}, % producer of the document
  pdfkeywords={Exponentiation} {Mathematics}
  pdfnewwindow=true,      % links in new window
  colorlinks=true,       % false: boxed links; true: colored links
  linkcolor=blue,          % color of internal links
  citecolor=blue,        % color of links to bibliography
  filecolor=magenta,      % color of file links
  urlcolor=cyan           % color of external links
}

%% \usepackage[
%% bibstyle=numeric,
%% citestyle=authoryear,
%% hyperref,
%% bibencoding=utf8,
%% backref=true,
%% backend=biber]{biblatex}

%% http://tex.stackexchange.com/questions/66778/citation-alias-with-multibib-and-natbib
%% \makeatletter
%% \def\@mb@citenamelist{cite,citep,citet,citealp,citealt,citepalias,citetalias}
%% \makeatother

%% http://stackoverflow.com/questions/2496599/how-do-i-cite-the-title-of-an-article-in-latex
\defcitealias{z-iso-13568}{ISO 13568:2002 Information technology -- Z formal specification notation --
  Syntax, type system and semantics}

\usepackage{tikz}
\usepackage[markings,customcolors]{hf-tikz}
\usetikzlibrary{%
  arrows%
  ,calc%
  ,decorations.text%
  ,decorations.pathreplacing%
  ,fadings%
  ,positioning
  ,shapes.geometric%
}

\usepackage{tikz-3dplot}

\usepackage{pgfplots}
\pgfplotsset{height=7cm,compat=1.9}

\usepackage{tkz-euclide}
\usetkzobj{all}

%% prettier integral syms, but broken on miktex
%% \usepackage{esint}


%% \usepackage{MnSymbol}
%% \usepackage[misc]{ifsym}

%% \usepackage{morefloats}

%%%%%%%%%%%%%%%%%%%%%%%%%%%%%%%%%%%%%%%%%%%%%%%%%%%%%%%%%%%%%%%%
\title{HoTT Types}
%% \\
%% \Large Derived from the HoTT Book}
\author{}
%\date{}                                           % Activate to display a given date or no date

%%%%%%%%%%%%%%%%
%% tufte-latex customizations

\makeatletter
\let\runauthor\@author
\let\runtitle\@title
\makeatother

%% running headers
\newcommand{\changefont}{%
  \fontsize{7}{9.5}\selectfont
}
\fancypagestyle{plain}{
  \fancyhead[LO,LE]{\leftmark }
  \fancyhead[RO,RE]{\rightmark}
  \fancyfoot[CO,CE]{\thepage}
  \fancyfoot[LE]{\textsc{\runtitle}}
  \fancyfoot[RO]{\textsc{\runtitle}}
  \renewcommand{\headrulewidth}{0pt}
  \renewcommand{\footrulewidth}{0pt}
}
\pagestyle{plain}

\def\chpcolor{blue!45}
\def\chpcolortxt{blue!60}
\def\sectionfont{\LARGE}

\setcounter{secnumdepth}{5}
\setcounter{tocdepth}{5}        % sections and subsections for the toc

\makeatletter
%% Section:
\def\@sectionstrut{\vrule\@width\z@\@height12.5\p@}
\def\@makesectionhead#1{%
  {%\par\vspace{20pt}%
    \parindent -10pt\raggedleft\sectionfont
    %% \colorbox{\chpcolor}{%
    %%   \parbox[t]{90pt}{\color{white}\@sectionstrut\@depth4.5\p@\hfill
    %%     \ifnum\c@secnumdepth>\z@\thesection\fi}%
    %% }%
    \vspace{10pt}%
    \begin{minipage}[t]{\textwidth}%{\dimexpr\textwidth-90pt-2\fboxsep\relax}
      \@sectionstrut\hspace{-15pt}\textit{\textbf\Huge #1}
    \end{minipage}\par
    \vspace{5pt}%
  }
}
%% \def\@makesectionhead#1{%
%%   {\par\vspace{20pt}%
%%    \parindent 0pt\raggedleft\sectionfont
%%    \colorbox{\chpcolor}{%
%%      \parbox[t]{90pt}{\color{white}\@sectionstrut\@depth4.5\p@\hfill
%%        \ifnum\c@secnumdepth>\z@\thesection\fi}%
%%    }%
%%    \begin{minipage}[t]{\dimexpr\textwidth-90pt-2\fboxsep\relax}
%%    \color{\chpcolortxt}\@sectionstrut\hspace{5pt}\textbf{#1}
%%    \end{minipage}\par
%%    \vspace{10pt}%
%%   }
%% }
\def\section{\@afterindentfalse\secdef\@section\@ssection}
\def\@section[#1]#2{%
  \ifnum\c@secnumdepth>\m@ne
  \refstepcounter{section}%
  \addcontentsline{toc}{section}{\protect\numberline{\thesection}#1}%
  \else
  \phantomsection
  \addcontentsline{toc}{section}{#1}%
  \fi
  \sectionmark{#1}%
  \if@twocolumn
  \@topnewpage[\@makesectionhead{#2}]%
  \else
  \@makesectionhead{#2}\@afterheading
  \fi
}
\def\@ssection#1{%
  \if@twocolumn
  \@topnewpage[\@makesectionhead{#1}]%
  \else
  \@makesectionhead{#1}\@afterheading
  \fi
}
\makeatother

%%%%%%%%%%%%%%%%
%% macros

\newenvironment{important}[1][]{%
  \begin{mdframed}[%
      backgroundcolor={red!15}, hidealllines=true,
      skipabove=0.7\baselineskip, skipbelow=0.7\baselineskip,
      splitbottomskip=2pt, splittopskip=4pt, #1]%
    \makebox[0pt]{% ignore the withd of !
      \smash{% ignor the height of !
        \fontsize{32pt}{32pt}\selectfont% make the ! bigger
        \hspace*{-19pt}% move ! to the left
        \raisebox{-2pt}{% move ! up a little
          {\color{red!70!black}\sffamily\bfseries !}% type the bold red !
        }%
      }%
    }%
}{\end{mdframed}}

%% reversed integral sign
\makeatletter
\providecommand*{\curv}{%
  \mathrel{%
    \mathpalette\@curv\int
  }%
}
\newcommand*{\@curv}[2]{%
  \reflectbox{$\m@th#1#2$}%
}
\makeatother

%% \def\LaTeX{%
%%   L\kern-.36em
%%   {\setbox0=\hbox{T}%
%%     \vbox to \ht0{\hbox{\the\scriptfont0 A}\vss}}%
%%   \kern-.15em
%%   \TeX
%% }

%%%%%%%%%%%%%%%%

\newcommand\cspace{coordinate space}
\newcommand\Cspace{Coordinate space}
\newcommand\CSpace{Coordinate Space}

\newcommand\dspace{design space}
\newcommand\Dspace{Design space}
\newcommand\DSpace{Design Space}

\newcommand\Omg{\(\Omega\)}
\newcommand\sccs{standard cartesian coordinate space}
\newcommand\origin{\((0,0)\)}
\newcommand\ab{\((a,b)\)}

\newcommand\atypeA{\ensuremath{(a : A)}}

%% \newcommand\N{\(\mathds{N}\)}
%% \newcommand\R{\(\mathds{R}\)}
%% \newcommand\RR{\(\mathds{R}\times\mathds{R}\)}
%% \newcommand\Rtwo{\(\mathds{R}^2\)}
%% \newcommand\Z{\(\mathds{Z}\)}


\def\HoTT{%
  H\kern-.7pt
  {\tiny\raisebox{1pt}{o}}%
  %% {\setbox0=\hbox{T}%
  %%  \vbox to \ht0{\vss\hbox{\the\scriptfont0 o}\vss}}%
  \kern-1.5pt
  TT}

\def\HoTTB{%
  the H\kern-.7pt
  {\tiny\raisebox{1pt}{o}}%
  %% {\setbox0=\hbox{T}%
  %%  \vbox to \ht0{\vss\hbox{\the\scriptfont0 o}\vss}}%
  \kern-1.5pt
  TT Book
}

\newcommand\ML{Martin-L\"{o}f}

\newcommand\TTh{Type Theory}

\newcommand\tth{type theory}

\includeonly{%
introduction%
,preliminaries
,niceties
,pragmatism
,foundations
,types
,curry-howard
,proof
,semantics
,math
,brandom
,hotttypes
}

%%%%%%%%%%%%%%%%%%%%%%%%%%%%%%%%%%%%%%%%%%%%%%%%%%%%%%%%%%%%%%%%
\begin{document}
%% \ifx\traceon\undefined \tracingall \else \traceon \fi

\maketitle

\begin{abstract}
  Currently this doc contains a (mildly organized) set of notes
  followed by the intro and chapter 1 from the
  \href{http://homotopytypetheory.org/book/}{HoTT Book}.  Eventually
  (maybe) the intro and chapter 1 will contain annotations, comments,
  additional examples, etc., but I have not started that yet, so if
  you are already familiar with the text you need not read them -- I
  haven't (so far as I recall) changed anything.

  The idea is to winnow out some of the strictly mathematical stuff
  leaving the core ``philosophical'' stuff, and annotate the text with
  some comments and quotes from Martin-L\"{o}f, Brandom, etc.  Or
  maybe leave the math stuff in, but annotate it with more detailed
  explanation and examples in programming languages.  In any case the
  purpose is to more fully articulate the link between HoTT's ideas of
  type and judgment (etc.) to the philosophical debates about
  language, assertion, proposition from which they emerged.  Why?
  Because I find those bits of the HoTT a little murky, and
  philosophers like Brandom have a lot to say about the issues.  Also,
  to show more clearly how type theory differs from set theory and
  classic logic.  Another goal is to provide more practical guidance
  to programmers interested in exploring dependent types.
\end{abstract}

\tableofcontents
%% \setcounter{tocdepth}{2}        % chapters, sections, and subsections for the
%%                                 % metadata of the pdf
%% \cleartooddpage[\thispagestyle{empty}]

%% \mainmatter % Turn on roman page numbers and numbered chapters

%%%%%%%%%%%%%%%%%%%%%%%%%%%%%%%%
%%%%%%%%%%%%%%%%%%%%%%%%%%%%%%%%
\chapter{The Pragmatist Enlightenment}
\label{sect:enlightenment}

%%%%%%%%
\section{Liberation}
\label{subs:liberation}



%%%%%%%%
\section{Pluralism}
\label{subs:pluralism}

\begin{ednote}
  Not just propositions-as-types, but types-as-propositions.  Example:
  the type \N can be viewed as a proposition ``there exists a natural
  number''.  This means that there is no authoritative definition of
  what a type is, which means that pluralism is an essential aspect of
  type theory.  Is this a sharp contrast with traditional mathematics?
  For pre-modern mathematics, number was unequivocally quantity or
  magnitude - no pluralism there.  Modern mathematics discarded
  quantitative interpretations of number in favor of structural
  notions.  The issue of pluralism is not so clearly decided there.
  Once you have isomorphisms, you can't really say that one structure
  is \emph{the} structure for a given class.  Groups, for example.  So
  isn't modern math essentially pluralistic?  Well let's look at
  foundations - set theory doesn't seem to be very pluralistic; a set
  is a set is a set, and not something else.  You can come up with
  distinct set \emph{theories}, but they all depend on the primitive
  notion of set, or maybe set membership.  Type theory, by contrast,
  seems to be different.  It doesn't have this kind of unity.  In fact
  there are many distinct type theories, so we should probably always
  use the plural.  The primitive seems to be ``type''; but the concept
  of type is not primitive in all type theories---\HoTT{} being a case
  in point.  ``In fact, no type former is 'primitive' to the game of
  type theory in this sense: you can very well have a type theory with
  no type formers! But it won't be very interesting...'' (M. Shulman,
  \href{https://groups.google.com/d/msg/hott-amateurs/U1X0m4r6G-A/K5eeMSPXE5YJ})
\end{ednote}

``Type theory, formal or informal, is a collection of rules for
manipulating types and their elements.  But when writing mathematics
informally in natural language, we generally use familiar words,
particularly logical connectives such as “and” and “or”, and logical
quantifiers such as “for all” and “there exists”. In contrast to set
theory, type theory offers us more than one way to regard these
English phrases as operations on types. This potential ambiguity needs
to be resolved, by setting out local or global conventions, by
introducing new annotations to informal mathematics, or both.''\HoTTB, p. 101

%%%%%%%%
\section{Normative Pragmatics}
\label{subs:normprag}

  Chapter 1 of \cite{brandom_mie}

%%%%%%%%
\section{Inferential Semantics}
\label{subs:inferentialism}

  Chapter 2 of \cite{brandom_mie}


%%%%%%%%
\section{Expressivism}
\label{subs:expressivism}

See \cite{price_expressivism_2013}


%%%%%%%%%%%%%%%%%%%%%%%%%%%%%%%%
\section{Logics}
\label{sect:logics}

\begin{description}
\item [Traditional] terms are primitive; propositions are combinations of terms; judgments apply to proposotions
\item [Modern: classic] LEM, AC, etc.
\item [Modern: intuitionistic]
\item [Expressivism]  Brandom's version: propositions are primitive; relation to inferential semantics; Price's global expressivism
\end{description}

%%%%%%%%%%%%%%%%%%%%%%%%%%%%%%%%
%%%%%%%%%%%%%%%%%%%%%%%%%%%%%%%%
\section{Proof}
\label{sect:proof}

Traditional (classic) view: a proof is an epistemic device; it
displays, exhibits, makes \textit{visible} (if only to the mind's eye)
a form of \textit{certain knowledge}.\sidenote{The link between
  knowing and seeing runs very deep in Western culture.  Not
  surprisingly it is closely connected with representationalism and
  cartesianism generally.  It has pretty much dominated Western
  thinking since Descartes, but has come under strong attack from
  Pragmatists.  Dewey called it ``the spectator theory of knowledge.''f
  See \citep{rorty_philosophy_2009} etc.}

Alternatives to the spectator theory: pragmatism, know-how over know-that.

\begin{ednote}
  TODO: summary of concepts of proof.  Emphasize contrast between
  representationalism and inferentialism.  Representationalism is
  atomistic: you could have only one concept.  Inferentialism is
  holistic: you have to start out with at least two concepts, since
  every inference involves a premise and a conclusion.  Inferentialism
  is a natural fit for \HoTT.

  Question: can you have only one type?  In other words, is type
  theory essentially holistic or atomistic?
\end{ednote}


For \HoTT{}, as for most varieties of constructivism, it is better to
abandon traditional notions of proof as something you see in favor of a
more pragmatic notion of proof as something you do.

etc.

Critical point: in \HoTT we have two ``kinds'' of types: propositional
types and non-propositional types.\sidenote{This is not in general
  recognized in \HoTTB, but I think it should be emphasized, if only
  because it reflects intuition.}  If we are to also treat ``proof''
(or witness or whatever) as a fundamental principle of \HoTT, one that
complements the concept of type, then we need to treat both ``type''
and ``proof'' as genuses (genii?) of which propositional and non-propositional
are species.

\begin{ednote}
  General point (to be made elsewhere, maybe in
  \cref{sect:foundations}: the concepts of type and proof go together.
  You cannot have one without the other.  That's very different than
  set theory.  You can have sets and elements without proofs.
\end{ednote}



Long story short: we are in dire need of improved terminology.  My
suggestion is as follows:

\begin{description}
\item [Proof of a proposition] In contrast to the classic spectator
  view, we treat proof not as the exhibition (or: making available for
  inspection) of the form of a bit of certain knowledge, but as the
  \textit{demonstrative expression} of the proposition.
  Alternatively, the expressive demonstration of the proposition.  So
  whereas a classic proof is something that must be ``seen'' in order
  to be grasped, a type-theoretic proof is something that must be
  actively \textit{done}, not merely passively observed.  One must be
  able to follow the construction of the proof.

\item [Proof of a non-propositional type] Classically, one only proves
  propositions, not terms.  So the idea of e.g. ``proving'' the
  natural numbers doesn't even make sense; it reflects a category
  mistake.  But in \HoTT, the concept of ``proving'' a type is
  primitive; the problem is that ``proving'' is the wrong word.
\end{description}

So here's one way to look at it: we construct (make) proofs; but the
proofs we construct are expressions of the type (the thing we prove).

%%%%%%%%
\subsection{Of the Ambiguity of Of}
\label{subs:ofofof}

``Of'' supports two distinct readings.  Consider ``the conviction of
the defendant''.  If the court did the convicting, then ``of'' acts as
a kind of intermediary between a verbal noun (``conviction'' as act or
action of convicting) and its direct object (e.g. ``The court
convicted the defendant'').  The conviction affects the defendant from
the outside; it does not ``belong'' to the defendant but to the court.
On the other hand, if we take ``the conviction of the defendant'' to
refer to a belief to which the defendant is firmly committed, then the
conviction is ``internal''; it belongs to and comes from the
defendant.

This ambiguity of ``of'' afflicts phrases like ``proof of a
proposition'' as well.  If we can disambiguate it some of the mystery
of the relation between types and proofs will vanish.

%%%%%%%%
\subsection{Demonstrations and Demonstratives}
\label{subs:}

When we \textit{exhibit} a classic proof of a proposition, the proof
comes out as external to the proposition proved, just as a court's
conviction of a defendant is external to the defendant.  Such a proof
is something added or attached to the proposition.

But when we \textit{demonstrate} a proposition,\sidenote{Note: we
  demonstrate propositions, not proofs; a demonstration of a
  proposition \textit{is} a proof.} the demonstration (that is, proof)
is to be deemed an expression of the proposition in the internal
sense: an expression whose source, so to speak, is the proposition
itself, rather than the writer of the proof.  This may sound odd or
even ridiculously anthropomorphic, but if you think about it a bit it
makes perfect sense.  The mathematical proofs we write down are not
really expressions our our thought, but of mathematical structures,
entities, relations etc.  So they express
mathematics.\sidenote{Actually we should probably think of them as
  having a dual expressivism.  On the one hand they clearly express
  mathematics; but on the other hand, the particular form a proof
  takes is an expression of the writer's style or way of thinking.}

We can think of a demonstration in this sense as expressing a type's
structure, construed as the inferential articulation of the concept of
the type.\sidenote{See \cref{sect:brandom} for more on the inferential
articulation of conceptual content.}

The nice thing about this way of thinking is that it resolves the
tension between propositional and non-propositional types with respect
to proof.  In both cases, what \HoTT{} calls proof or witness is to be
taken as a demonstrative expression, or expressive demonstration, of
the type itself.  In the case of propositional types, favor the term
``demonstration'', with its connotations of progressive unfolding of a
logical structure, or better, rational argument.  In the case of
non-propositional types like \N, favor the term ``demonstrative'',
with its adjectival sense of ``something having a demonstrative
function'', rather than a nominal sense of ``act or action of
demonstrating''.  So an element\sidenote{We really must get rid of
  ``element''; it's too suggestive of set theory.  Maybe
  ``demonstrative'' fits the bill; instead of ``element of a type'' we
  would say ``demonstrative of a type''.  Or maybe ``demonstrator''.}
of a propositional type we would call a demonstration of the type, and
an element of a non-propositional type we would call a demonstrative
of the type.\marginnote{So $2$ is a demonstrative of the natural
  numbers; a proof that ``$2$ is even'' is a demonstration that
  expresses just that ``$2$ is even''.}

\begin{ednote}
  Demonstration qua demonstration of know-how?  Expression as
  expression of a type's structure - that is, its inferential
  articulation?
\end{ednote}

In both cases we have demonstration rather than proof of the type.

\begin{ednote}
  ``Demonstrator'' as the genus of ``demonstration'' and
  ``demonstrative''.  It has the virtue of paralleling
  ``constructor''.
\end{ednote}


%%%%%%%%%%%%%%%%%%%%%%%%%%%%%%%%
\section{Semantics}
\label{sect:semantics}

%%%%%%%%
\subsection{Meaning}
\label{subs:meaning}

%%%%%%%%
\subsection{Model-theoretic Semantics}
\label{subs:modeltheorysem}

%%%%%%%%
\subsection{Proof-theoretic Semantics}
\label{subs:proofsem}

``Proof-theoretic semantics is an alternative to truth-condition semantics. It is based on the fundamental assumption that the central notion in terms of which meanings are assigned to certain expressions of our language, in particular to logical constants, is that of proof rather than truth. In this sense proof-theoretic semantics is semantics in terms of proof . Proof-theoretic semantics also means the semantics of proofs, i.e., the semantics of entities which describe how we arrive at certain assertions given certain assumptions. Both aspects of proof-theoretic semantics can be intertwined, i.e. the semantics of proofs is itself often given in terms of proofs.''\cite{schroeder-heister_proof-theoretic_sep}


%%%%%%%%
\subsection{Inferential Semantics}
\label{subs:inferensem}



%%%%%%%%%%%%%%%%%%%%%%%%%%%%%%%%
\chapter{Mathematics}
\label{sect:math}

%%%%%%%%
\section{Traditional}
\label{subs:mathtrad}

%%%%%%%%
\section{Modern: classic}
\label{subs:mathmodclassic}

%%%%%%%%%%%%%%%%%%%%%%%%%%%%%%%%
\section{Modern: Intuitionism}
\label{sect:mathmodintuit}

\begin{ednote}
  Why Brouwer should be deemed a pragmatist.
\end{ednote}


%%%%%%%%%%%%%%%%%%%%%%%%%%%%%%%%
\section{Mathematical Pragmatism}
\label{sect:mathprag}

\begin{ednote}
  \HoTT is (largely) founded on \ML{}'s account of ``judgment''
  (assertion).  I don't know if that's entirely accurate, but it's my
  story and I'm sticking with it for now.  (\ML{} was quite specific
  that his project was motivated by ``purely philosophical''
  considerations.  See his 1972 paper.)  Brandom's account of
  assertion is part of a larger, very ambitious project that aims to
  explain the structure of rationality.  It's a thoroughly pragmatic
  account; everything comes down in the end to ``proprieties of
  practice'': conceptual activity (thinking and talking) is explained
  in terms not of what we know but of what we do (or what we know
  \textit{how} to do).

  Brandom's account of assertion is much more refined and
  sophisticated than \ML{}'s.  If we replace \ML{}'s account with
  Brandom's, then \HoTT comes out as a piece of ``mathematical
  pragmatism'' (or pragmatic mathematics): an account mathematics
  grounded in practice.
\end{ednote}

\begin{ednote}
  TODO: Brandom's philosophy, like most of contemporary pragmatism,
  subverts the dominant representationalist mode of thinking.  It
  turns things upside-down, or inside-out.  So it is with type theory.
  (In one of his papers \ML{} suggests something similar, pointing out
  that his take on judgment etc. reverts (in some sense) back to
  practices that preceded the ways of thinking that have dominated
  modern ``classic'' mathematics and logic.)  The to-do item here is
  to show how the relation of type theoretic to classic thinking in
  mathematics and logic parallels the relation between pragmatist
  (anti-representational, expressivist) thinking and representational
  (cartesian, platonistic) thinking in philosophy, about rationalism,
  conceptual content, etc.  Show how type-theoretic thinking turns
  traditional classic thinking inside-out.
\end{ednote}



%%%%%%%%%%%%%%%%%%%%%%%%%%%%%%%%
%%%%%%%%%%%%%%%%%%%%%%%%%%%%%%%%
\section{Foundations}
\label{sect:foundations}

\HoTT purports to offer a new foundational concept for mathematics.  If
we take assertion to be the foundational concept of type theory (I'm
not sure this works, but it seems plausible), then Brandom's account
of assertion can link type theory to a foundational account of
discursive practice (rationality).

Today set theory is the reigning foundational theory of mathematics.
It's fairly easy to present it as such: first you list the axioms,
then you show how to ``construct'' the natural numbers from sets,
using a successor function.  Or you might follow the lead of the Z
specification notation\sidenote{\cite{zed_spec}}, and proceed from
sets to relations and then to functions.  However you do it, it's all
pretty intuitive and relatively easy to explain, even to mathophobes.

What would such a foundational presentation look like for \HoTT?  If
\HoTT turns out to be a genuinely foundational theory, then it must be
grounded in intuition; specifically, we should expect that its basic
notions correspond in some way to some collection of pre-theoretic
mathematical intuitions, just as the axioms of set theory do, or as
the axioms of geometry match our ordinary intuitions about the
organization of space as we experience it pre-theoretically.

Presentations of set theory usually begin by discussing the axioms;
but even though axioms serve as ``unexplained explainers'', such a
presentation inevitably depends on a yet more primitive layer of
concepts.  Specifically, not only the (pre-theoretical) concepts of
set, subset, and membership, but also axiom and perhaps proof.  All of
these ``preliminary'' concepts---let's call them
``principles''---correspond more or less directly to intuitions
available to any concept-user.

In general, an explicit account of the fundamental \textit{principles}
of set theory is either omitted or informally glossed, before the
presentation moves on to the axioms.  But type theory, in the end, is
radically different from set theory at a very fundamental level, as
far as I can see.  ``Set'' and ``type'' are so easily grasped that it
is easy think of them as more-or-less the same sort of thing; but if
you look hard at them, they are very different, even fundamentally
different.  So I think a presentation of \HoTT would be well served by
beginning with an explicit account of principles, even before moving
on to consider primitives of the theory.

What are the pre-theoretical principles and primitives of \HoTT?  The
obvious place to start is ``type''.  The concept of ``type'' obviously
emerges from ordinary experience; indeed, it is arguably more
primitive than the concept of ``set''.  Just look at the vast
literature on the emergence of categorization in developmental and
cognitive psychology; the ability to categorize is undoubtedly one of
the most primitive human intellectual skills, if not the most
primitive.  It may even be a primitive animal capability--bees
categorize flowers, and every member of sexually reproducing species
categorizes possible mates.

What about ``axiom''?  At first glance it would seem that any
foundational account of mathematics (or anything else for that matter)
must rest on an axiomatic foundation.  Which is just another way of
saying that any explanation of anything must eventually bottom out on
a bedrock of unexplained explainers.  You can't explain everything
without entering an infinite regression.

On the other hand, we can view axiomatic explanation as just one
``style'' of explanation, one of many.  When you begin with axioms,
you present them as unequivocally (and unquestionnably) true.  But
this is really a bit of salesmanship; sometimes axioms turn out not to
be quite as axiomatic as they seem.  Reconceptualizations happen,
which may lead us to view axioms in a new light in which they do not
look quite as certain.  Then axiomatic explanations are still
intelligible, but are no longer unquestionnable.  The classic example
of this sort of evolution is to be found in the history of geometry.
Before the development of non-Euclidean geometries in the 19th
century, the axioms of Euclidean geometry were not only unquestioned
but unquestionable:\sidenote{I suspect I'm overstating the case here.
  Mathematicians: is this true?} the idea that parallel lines could
meet was not just wrong, but crazy.  Today, using axioms to define a
geometry is just a way of making clear the assumptions necessary to
make the theory work.  They no longer represent essential
connections to externally available bits of certain knowledge.

In other words, axioms are not a necessary condition of adequate
explanation.  So the question is whether or not the axiomatic style is
most appropriate for a presentation of \HoTT?  On the one hand, it
seems to me that it is not necessary; an adequate explanation of \HoTT
without axioms should be possible.  For example, we can treat the
concept of ``type'' as primitive, even if we cannot find a good way to
express it axiomatically.\sidenote{This is a little fuzzy; maybe it
  doesn't even make sense.  But as long as we're rethinking the
  foundations of mathematics, we might as well rethink everything.}

\begin{remark}
  Leaving full presentation of principles for later.  I think it
  includes at least proposition and judgment, maybe inference and
  proof.
\end{remark}

In any case, we'll have to begin somewhere, by stating some
fundamental principles; then we'll need an account of the primitives,
whether they take the form of axioms or not.  What are the principles
upon which \HoTT depends?  And once we have some principles (which are
external to the theory proper), what are the primitives (which are
``inside'' the theory)?\sidenote{Ok, ``primitive'' sounds a lot like
  ``axiom''.  But I think there's a difference, even if I can't quite
  articulate it.  Let's provisionally say that a primitive is an axiom
  without the concommitant commitment to unquestionned certainty.}

Here are some possibilities, based on my understanding of the material
in \HoTTB.  Please keep in mind this is coming from somebody who
thinks he has a fairly good grasp of what type theory is all about,
but is still grappling with \HoTT.

%%%%%%%%
\subsection{\HoTT Principles}
\label{subs:hottprinciples}

\begin{description}
\item [Type] Obviously a fundamental concept.  What to say about it, though, is
  not so obvious.
\item [Proposition]
\item [Judgment]
\item [Proof]\sidenote{from Latin \textit{probare} "to make good;
  esteem, represent as good; make credible, show, demonstrate; test,
  inspect; judge by trial" (source also of Spanish \textit{probar},
  Italian \textit{probare}), from \textit{probus} "worthy, good,
  upright, virtuous,"} Two kinds, corresponding to the two kinds of
  provables:\sidenote{Remember, we're talking about pre-theoretical
    principles (concepts) here, not about \HoTT per se.}
\begin{description}
\item [Demonstration] - \textit{rational argument} that compels assent
  to a proposition\sidenote{``Demonstration'' is intuitively
    satisfying, but conceptually misleading, insofar as it suggests a
    visual metaphor.  That would be classical; but for type theory we
    want metaphors of construction, not inspection.}
\item [Witness] - evidence that bears witness to the existence of a kind or category
\end{description}
\item [Inference]
\item etc.
\end{description}

\newthought{The concept of proof in type theory} deserves special
attention.  \Cref{sect:proof} examines it in detail; here, suffice it
to say that it extends beyond the traditional and intuitive notion of
proof as something one does to or with propositions.  In type theory,
propositional types represent propositions, so a type-theoretic proof
of a propositional type---call it a ``tt-proof''--- corresponds to an
ordinary proof of a proposition; it essentially involves inference,
for example.  But type theory also has lots of non-propositional
types, like \N.  These do \textit{not} represent propositions:
propositions have truth-values, natural numbers do not.  In set
theory, there is no connection between sets, elements, and proofs.  An
element either is, or is not, a member of a given set.  Period, full
stop.  The notion of proof never enters the set-membership
picture.\sidenote{That need not mean that proving membership is never
  an issue.  But you don't prove membership; rather, you prove that
  the element satisfies some predicate, which is a different concept.}
In particular, the existence of a set is not dependent on particular
members, and the fact that some element is a member of some set has no
significance with respect to the existence of the set.  By contrast,
in type theory construction of an element of a type counts as proof of
the type.  Etc.\sidenote{FIXME: fix this language.}  But this kind of
``proof'' is not like proof of a proposition; it does not involve a
proposition that may be true or false, and it does not involve
inference.  Instead it serves as a kind of evidence that shows the
type.

\begin{remark}
  Is there a significant distinction to be made between proof and
  witness?  I suspect there is, based on the difference between
  propositions and names.  Both count as evidence, but there is a
  difference between an inferential proof of a proposition and a
  ``testimonial'' witness to a kind.  Propositions-as-types unifies
  the two ideas, but does not erase the distinction.
\end{remark}

%%%%%%%%
\subsection{\HoTT Primitives}
\label{subs:hottprimitives}

\HoTT primitives are ....\sidenote[][-48pt]{A proper exposition would list 1) the name of the
  primitive, e.g. ``\(\Pi\)-type''; 2) the ``constructor'' symbol,
  e.g. \(\cross\) for product types; 3) the analogous concept from set
  theory, and then the ``rules'' for defining a type (formation,
  construction, elimination, computation, uniqueness).}

\begin{description}
\item [Function] ``Unlike in set theory, functions are not defined as
  functional relations; rather they are a primitive concept in type
  theory.''\sidenote{Or: set theory
    \textit{defines} a function as a set of ordered pairs whose domain
    has no duplicates; in other words, it treats a function and its
    ``graph'' as the same thing.  Question: what happens to the graph
    of a function in type theory?} \citep[p. 21]{hottbook}

\item [Product] Product types correspond to cartesian products in set
  theory.  The constructor symbol is the same as in set theory:
  \(\cross\).\sidenote{Why isn't this called the ``\(\Huge\cross\)-type''?}
  ``[U]nlike in set theory, where we define ordered pairs to be
  particular sets and then collect them all together into the
  cartesian product, in type theory, ordered pairs are a primitive
  concept, as are functions.''\citep[p. 26]{hottbook}

\item [Coproduct type] Coproduct types correspond to disjoint unions
  in set theory.  ``In type theory, as was the case with functions and
  products, the coproduct must be a fundamental construction, since
  there is no previously defined notion of ``union of
  types''.\citep[p. 33]{hottbook}

\item [Proposition type] Conceptually, at least, this seems primitive.
  Especially if the concept of ``proposition'' counts as a
  pre-theoretic principle.  Which implies that proof is also a
  pre-theoretic principle.  Propositions are fundamentally different
  than the other kinds of type, since they have truth-conditions. Etc.
  It follows that proofs are fundamentally different from other kinds
  of witness.

\end{description}

\begin{remark}
  For consistency, we might want to use symbols to designate all of
  the primitives, just as we do for \(\Pi\) and \(\Sigma\).  This
  would give us: \(\Huge\fun\)-types, \(\Huge\cross\)-types, and
  \(\Huge +\)-types.
\end{remark}


%%%%%%%%
\subsection{\HoTT Quasi-primitives}
\label{subs:quasiprim}

\noindent ``Fundamental''\sidenote{Obviously we need a better bit of
  terminology.  ``Quasi-primitives''?  ``Neo-primitives''?  These
  types are not primitive, strictly speaking, but on the other hand
  they are basic.  I think there is another fundamental principle at
  work here.  In set theory, for example, the concept of function is
  not only not primitive, it isn't necessary.  You could discard it
  and still have set theory.  But my intuition tells me that e.g. the
  concept \(\Pi\)-type is in a sense necessary or essential in type
  theory, even if it is not primitive.  Once you have the primitives,
  you necessarily have these non-primitive basic types.  Dunno if
  that's correct, but it would sure be nice if it were.} (but
non-primitive) types.  These types seem to be on a par with the
primitive types as far as importance goes, but they presuppose the
primitives, so cannot themselves be considered primitive.

\begin{description}

\item [Universe]  Is this a primitive?  Probably not, since it builds on the type concept.

\item [\(\Pi\)-type] Informally, ``dependent function''
  types.\sidenote{As a practical matter, I think it would be useful to
    have an informal term for these types that falls between
    ``dependent function type'' and \(\Pi\)-type.  Something like
    ``p-function type''.  \(\Pi\)-type is admirably concise, but I
    think it should mention ``function'', since it names a kind of
    function.}  The concept of \(\Pi\)-type is a generalization of the
  concept of function type, so it isn't primitive.

\item [\(\Sigma\)-type] Informally, ``dependent pair''
  type.\sidenote{Shouldn't this be called ``dependent
    \textit{product}'' type?  The type is product, not pair; pairs are
    ``elements'' of the type.  Informally, maybe ``sig-prod type?}
  The concept of \(\Sigma\)-types is a generalization of the concept
  of product type.

\end{description}

%%%%%%%%
\subsection{\HoTT Standard Type Library}
\label{subs:hottstdlib}

\begin{remark}
  By analogy to the usual ``standard library'' of programming
  languages.  The idea is to list commonly used types that are neither
  primitive nor quasi-primitive; ``application'' types, in a sense.
\end{remark}


\begin{description}
\item [Boolean] \citep[p. 34]{hottbook}
\item [$\nat$] \citep[p. 36]{hottbook}
\item [Propsition]
\item [Identity] 
\end{description}



%%%%%%%%%%%%%%%%%%%%%%%%%%%%%%%%
\chapter{Types}
\label{sect:type}

\HoTTB page 27 describes a ``general pattern for introduction of a new
kind of type''.  Martin-L\"{o}f does this too, somewhere.  In \HoTTB,
the list is

\begin{description}
\item [Formation Rules]
\item [Introduction Rules]  or constructors
\item [Elimination Rules] or eliminators
\item [Computation Rules]  ``which express how an eliminator acts on a constructor''
\item [Uniqueness Principle] which ``expresses uniqueness of maps into
  or out of that type.  Optional.
\end{description}


The question is where to place this stuff in the description of \HoTT.
Are these things primitives?  Do they form essential aspects of a
type?  Or in other words, can we have (think of) types without these rules?

\HoTTB introduces them almost as an afterthought, as a Remark in the
third major construction defined in Chapter 1.  But I suspect this is
a mistake or oversight; it looks to me like these rules are indeed
fundamental, essential to the concept of type.  In that case, they
should be presented along with the introduction of the type concept,
rather than in the middle of a description of a particular type.

%%%%%%%%%%%%%%%%%%%%%%%%%%%%%%%%
\section{Terms}
\label{sect:terms}

\begin{ednote}
  ``Terms'' is Awodey's terminology.  More common terminology include:
  witness; inhabitant.  Also proof.
\end{ednote}

``Under the Curry-Howard cor- respondence, one identifies types
with propositions, and terms with proofs...''\cite{awodey_tth}

%%%%%%%%
\section{Witness}
\label{subs:witness}

\begin{ednote}
  In what sense is a proof a witness to a type, or an ``inhabitant''
  of a type?  Intuitively this language does not work very well; we
  don't intuitively think of a proposition as a type ``inhabited'' by
  proofs.  The notion of proof as ``witness'' to a type is a
  substantive epistemological notion; it not only says that the proof
  is related to the type, but also it says something about the nature
  of that relationship.

  The trick is to see it from the perspective of the machine.  A
  proposition like \(1+1=2\) is just a form to the machine.  We can
  see that it is true just by looking, due to some mysterious
  epistemic capability.  But machines do not have epistemic abilities;
  a form is a form is a form to a machine.  Hammer, nail.  So in order
  for the machine to treat \(1+1=2\) as a \emph{true} proposition, we
  have to give it something more: a proof.  But ``proof'', again, is
  an substantive epistemic notion; the machine analog must be purely
  formal.  From the machine perspective, a proof is just another form,
  or rather, collection of forms (including inference rules as complex
  forms): to give the machine a proof of P we must provide it with a
  form or forms that ``lead to'' (produce, result in) P.  To prove a
  proposition to a machine, we give it forms and reduction rules such
  that the formal use of those forms and rules results in the form of
  the proposition to be proved.  (FIXME: a more accurately way of
  putting this would involve reduction of formulae to normal form,
  confluence, etc.)

  So we can think of a proof as a kind of device---just another
  machine (or machine description), but one whose sole output is the
  proposition to be proved.  Since for any given proposition there may
  be many ways of building such a proving device, we can treat these
  devices as forming a kind of equivalence class, which we can
  identify by taking (the form of) the proposition as a symbol
  referring to the class.  Now the connection to types and witness
  becomes clear: the equivalence class of such proving devices forms a
  type, the type of the devices (proofs), and each device (proof)
  ``inhabits'' (or as we would prefer, expresses) the type.
\end{ednote}


%%%%%%%%%%%%%%%%%%%%%%%%%%%%%%%%
\chapter{Curry-Howard}
\label{sect:curry-howard}

\section{Two kinds of proof}

Proof in logic and math: \textit{discursive} proof.

Proof in ordinary language: evidence, demonstration, etc.

Empirical v. logical proof.

Example: to prove to you that there is a coin in this purse I can open
the purse and display the coin.

The ``proof'' of the proposition is thus a performance of a certain
kind: a proof-performance.  But \textit{kinds} are abstract; the
specific performance should thus be construed as a \textit{token} of
the kind.  In this particular case (displaying a coin in a purse), the
performance can be repeated.  Of course, each performance will differ
from all the others in its fine detail, but insofar as each repeat
performance counts as proof \textit{of the same kind}, each counts as
a \textit{proof token} of the same type.

But it is not merely a proof-token; a proof-token of the type that
proves a particular proposition.  Every specific proof-token is a
proof of a particular proposition P.  So a given performance of this
sort - a repeatable proof-token - is classifiable as a proof-of-P
token (=performance).  The notion of \enquote{proof token} is a
generalization over proof-of-P tokens for all P.

In the case of ordinary provings like the example given above -
consider the sort of ``proving'' that goes on in schoolyards, where
proving means showing - proof is non-discursive: it does not involve
explicit reasoning.  Caveat: this may count as a kind of empirical
proof, but is not to be confused with inductive reasoning.  What makes
a given performance count as a proof is a deep question we won't go
into here, but we all know that e.g. displaying a coin in a purse
counts as proof of the proposition that the purse contains a coin.

NB: the original proposition ends up as the conclusion of a piece of
practical reasoning: I see a coin in the purse; therefore there is a
coin in the purse.

In the case of mathematical and logical reasoning, proof involves
discursive performance.

Written proofs as traces of discursive performances.

The proposition to be proved ends up as the conclusion of the proof.

The type of a proposition (statement, etc.) is \textit{not} the type
of its proofs.  The type of a proof of P is exactly
\textit{proof-of-P}, not P.  More exactly,
proof-whose-\textit{conclusion}-is-P.  That's the kind of thing such a
proof is: it's the sort of thing that counts as a proof of (proves) P.

So what is the type of a proposition?  The question is malformed; what
we really want to know is, what is the type of a proposition
\textit{token}.  The answer is obvious but hard to articulate clearly
in English, due to the inherent circularity of the type/token
distinction.  The type of a token is just its type; the tokens of a
type are, well, its tokens.  This page has many tokens of type
\enquote{the}.

\subsection{Tokens, Terms, Types}
\label{subsec:tokens-terms-types}

To communicate clearly about these issues, we need special notation.
Quote marks are insufficient; they turn an expression into a name of
the expression.  The famous example (Tarski's convention T) is:

\enquote{Snow is white} if and only if snow is white.

This sentence contains two tokens of type `snow is white'; the first
is mentioned, the second used.  Technically the quoted version
functions as a name (thus mention) of the sentence, while the unquoted
version is just the sentence (used).  The quoted version does
\textit{not} denote or indicate the type of the sentence.  For that we
can use a designated notation such as \(\ulcorner \urcorner\), so that
\(\ulcorner\)3\(\urcorner\) refers to the type of tokens of the form
`3'.  Call these token-type quote marks.  When we need to explicitly
refer to some symbol \textit{qua} token, we use \(\llcorner \lrcorner\)
and write \(\llcorner 3\lrcorner\).  So \(\llcorner 3\lrcorner\) is a
token of type \(\ulcorner 3\urcorner\).

\subsection{Structure of Proofs}
\label{subsec:structproofs}

So: we have a proposition P, and we have a discursive proof of P.
What \textit{kinds} of things are involved here?  What is the
structure of the kinds?  Kinds rather than types, because we want to
reserve the notion of type for syntactic duty: a type system is a kind
formal notation that combines the notions of syntactic calculus and
kindedness.

A proof of P is a proof token whose conclusion is a token of type P
(and type P is in turn a token of type Proposition.)

The \textit{written} form of the proof token as a trace of a proof
process or computation.  So conclusion of a (static) written proof ~
end result of a (dynamic) computation or construction.

Compare proof of an int and proof of a proposition.  An integer symbol
like \(23\) is a formula, just like a propositional formula.  It
denotes a device that computes a result.  This is true even of
``simple'' symbols like \(3\): in contrast to the denotational
perspective, under which \(3\) simply denotes the integer, under the
constructive perspective the symbol \(3\) denotes a device that
computes the integer.  The type/token distinction applies here just
like it applies to propositions and proofs: a computation (proof) of
\(3\) is a process/proof/computation/whatever whose conclusion is a
token of type 3.  That type in turn is a token in the type Z
(integer).  Similarly, the type of a proposition token is a token of
type Proposition (or we might call it type Provable, or Proven or even
True or the like).

Remark: token and term.  Same thing?  Not really.  Term contrasts with
type, just like token, but at a different level of abstraction.  By
example: 3 (on the page) is a token of type \(\ulcorner\)3\(\urcorner\),
which in turn is a term of type Z (here ``type Z'' means type of
values, rather than \(\ulcorner Z\urcorner\), the type of `Z' tokens).

Now how does this related to Curry-Howard?  In particular proof checking etc.?

We can interpret the usual formulation \enquote{a proposition is the
  type of its proofs} to be an abbreviated way of saying that the type
of a proposition serves to categorize proofs whose conclusions are
tokens whose type is the proposition.

Key concept: token-repeatability.  In the example of pulling a coin
from my pocket in order to prove the proposition that there is a coin
in my pocket, once I have performed the proof, I cannot repeat it,
since the coin is no longer in my pocket. (See: linear logic.)  But if
the proposition is that there is a coin in purse, I can prove it by
opening the purse and displaying the coin.  Since the coin stays in
the purse, I can repeat this proof as often as I like: produce as many
proof-tokens of this kind as I wish.

In the case of formal logic and computation, proofs are repeatable.

\section{Proofs and Propositions}

The usual formulation is something along the lines of \enquote{a
  proposition is the type of its proofs}.  But this obviously cannot
be quite right: propositions and proofs are distinct \textit{kinds} of
things, so how can a proof be a kind of proposition?.  We would never
say that a building is the type of its blueprint; why say that a
proposition is the type of its proofs?

The problem is that the standard terminology ``forgets'' about
computation.  They type of a compound expression is by definition the
type of its \enquote{result}.  In the case of mathematical
expressions, the result (of a computation) is a value of a certain
type; in the case of propositions, the result (of a proof) is either a
proposition or a truth value, depending on your preferred perspctive.
In both cases, it would be more accurate to talk of both the type of a
computation and the type of the result of a computation.



\section{Misc. notes}
\begin{ednote}
  Usually presented as ``propositions-as-types'', but this suggests an
  asymmetrical relationship; in fact the principle is that
  propositions \emph{are} types, and vice-versa.  This is a major move
  in type theory, introduced by \ML(?) based on work by Curry and
  Howard.  TODO: what exactly are the implications of this principle?
\end{ednote}

\begin{ednote}
  The critical point is that we go minimalist: start with the minimal
  logical language, which means combinatory logic.  It is the
  isomorphism between the logical constants and the combinators
  (Curry) that motivates Curry-Howard.  Once you see the connection at
  this minimalist level, it is easy to see it at any level, since the
  logical constants are the basic building blocks from which all
  propositions are constructed.
\end{ednote}


\begin{ednote}
  Start with Schoenfinkel and Curry, and the goal of finding the
  absolute minimum, which means eliminating variables.  Then the basic
  combinatorys, then the isomorphism to the logical constants.

  Equivalence of combinatory logics (no vars) and lambda calculus (vars)
\end{ednote}


Analogies.  Proof/proposition, term/type: ``There is also a one-to-one
correspondence between proofs of a certain proposition in constructive
predicate logic and terms of the corresponding type.'' (Dependent Types at Work)

%%%%%%%%%%%%%%%%%%%%%%%%%%%%%%%%
\section{bhk}
\label{sect:bhk}


\begin{ednote}
  Importance of metaphors.  See ML on BHK: proof as task to be
  accomplished, problem to be solved.  Add another metaphor:
  destination to be reached.
\end{ednote}


%%%%%%%%%%%%%%%%%%%%%%%%%%%%%%%%
\section{Assertion and Judgment}
\label{sect:assertionjudgment}

The account of judgment offered in the HoTT Book doesn't really work.
Ditto for Martin-L\"{o}f's account.  For example, it makes sense to
say ``P is a proposition'', but it doesn't make sense to say ``P is a
judgment''.  That's because judgment is a act, something one does.

On the other hand, ``judgment'', like ``proposition'', can be treated
as a verbal noun or as a ``plain'' noun.  Saying ``P is a
proposition'' is usually taken to mean that P refers to what has been
proposed.  There is no obvious reason not to treat ``P is a judgment''
in a similar manner: P refers to what has been judged.

However, there is a difference.  Judging a proposition (what was
proposed) amounts to \textit{evaluating} what was proposed, as good or
bad, true or false, or whatever.  By contrast, proposing a proposition
amounts to merely exhibiting it for consideration.  This arguably
involves an implicit evaluation - to propose a proposition is to
implicitly claim that it is good, or true, etc.  But proposing does
not involve offering an evaluation that is distinct from what is
proposed, whereas judgment does.  The two are distinct kinds of speech
act, and refering to the content of a speech act is not the same as
referring to the speech act itself.

Furthermore, it is not correct to treat the nominal sense of
``judgment'' as being the content, what has been judged.  The nominal
sense of ``judgment'' refers to the act of judgment itself, and not
the proposition judged.

Actually, by the same reasoning it is not correct to say that the
nominal sense of ``proposition'' is what-is-proposed; rather, it is
the act proposing, nominalized.  This makes perfect sense when you
consider that ``proposing'' can also be nominalized; ``the proposing''
is another way of saying ``the proposition''.

The same goes for all -tion words: suggestion, opposition, etc.  In
each case, the word can refer to the doing, or to what is done, and
what is done is always the act of doing itself -- not the subject or
object of the doing.

This suggests we should make a distinction between, for example, the
content of a proposition and ``proposition''.  But this term seems to
be a special case; it has the usual plain noun sense of
what-was-proposed, the usual verbal sense of ``proposing'', but also
the nominalized verbal sense of ``act of proposing''.

(But then the same considerations apply to ``judgment''.  The
difference must go back to semantics.)

\begin{remark}
  The Arabic grammatical tradition captures this distinction
  beautifully, mainly because the structure of the language makes it
  simple to do so.
\end{remark}

Or put it this way: when we judge a proposition like ``2+2=4'' to be
true, the what-was-judged is not ``2+2=4'' but the truth of ``2+2=4''.

\begin{remark}
  But how is this different from ordinary predication, like ``The
  triangle is red'' as a proposition?  Should we say that what is
  proposed is not that the triangle is red, but the redness of the
  triangle?  No, since we're treating it as a propostion, and the
  whole thing is proposed (exhibited).  If we judge it to be true,
  then again the judgment 
\end{remark}

So saying ``P is a judgment'' is incoherent if P is taken to refer to
nothing more than what is proposed.  If P refers to a claim of the
form ``X is true'' (or good, etc.), then ``P is a judgment'' seems to
make more sense; but it doesn't, really.  P still refers to an
unasserted content; to make sense, we would have to say something like
``P is a judgment when asserted''.  More explicitly, ``'X is true' is
a judgment'' (or better, ``'X is true' expresses a judgment'') only
\textit{exhibits} ``X is true'', which is a proposition, not a
judgment.  As a proposition it expresses a judgment; but when embedded
(equivalently, quoted) it does not express anything.

\begin{remark}
  Compare: ``Snow is white'' iff snow is white.  The quoted bit is a
  name of the sentence; it counts as a \textit{mention} of the
  sentence, which has no force.  The unquoted version of same is the
  sentence itself; it counts as a \textit{use} of the sentence, which
  has assertional force.  Obviously, the occurances of ``P'' in ``P is
  a proposition'' and ``P is a judgment are names of a proposition and
  thus mentions.  So they have no force.
\end{remark}

The key point is Frege's point: the content of a proposition is
distinct from the force of the utterance.  That means that P in ``P is
a proposition'' is unasserted, just as it is when embedded, as in ``If
P then Q''.  The truth of ``P is a proposition'' is independent of the
truth of P.

So even if we take the act of declaring ``P'' to be an act of
judgment, it does not follow that a reference to P is a reference to
the act of judging that P.  Hence there is no way to make ``P is a
judgment'' work.  If we take P to refer to what was judged, that again
is a proposition (or propositional content), so ``P is a judgment'' is
incoherent.

\begin{remark}
  We can assert that P, and we can assert P.  We can judge that P, but
  we cannot judge P.  I don't think this is a mere grammatical
  distintion; I think it reflects a genuine semantic difference.
\end{remark}

%%%%%%%%%%%%%%%%%%%%%%%%%%%%%%%%
\section{What's the Big Deal about Equality?}
\label{sect:equality}

\begin{ednote}
  Equality is arguably the most important concept of \HoTT{}, as far
  as I can tell, because of the ``Univalence Axiom''.
\end{ednote}

``In the intensional version of the theory, with which we are
concerned here, one thus has two different notions of equality:
propositional equality is the notion represented by the identity
types, in that two terms are propositionally equal just if their
identity type IdA(a,b) is inhabited by a term. By contrast,
definitional equality is a primitive relation on terms and is not
represented by a type; it behaves much like equality between terms in
the simply-typed lambda-calculus, or any conventional equational
theory.

If the terms a and b are definitially equal, then (since they can be
freely substituted for each other) they are also propositionally
equal; but the converse is generally not true in the intensional
version of the theory''\cite{awodey_tth}

``The constructive character, computational tractability, and proof-
theoretic clarity of the type theory are owed in part to this rather
subtle treatment of equality between terms, which itself is
expressible within the theory using the identity types IdA(a, b).''\cite{awodey_tth}

%%%%%%%%
\subsection{Substitution}
\label{subs:substitution}

As the quote from Awodey above suggests, the concept of
substitutability plays a basic role.

\begin{ednote}
  Compare substitution in lambda calculus, and in Brandom's model.
  Maybe something about combinatory logic and the elimination of
  variables?
\end{ednote}

%%%%%%%%%%%%%%%%%%%%%%%%%%%%%%%%
\section{Expressivity}
\label{sect:expressivity}

Instead of ``P is a proposition'' etc. we should say ``P expresses a proposition''.

%%%%%%%%%%%%%%%%%%%%%%%%%%%%%%%%
\section{Determinism}
\label{sect:determinism}

Hypothesis: classical math with LEM and AC is inherently
non-deterministic.  Constructive math(s) and logic(s) that discard LEM
and AC are deterministic.

%%%%%%%%%%%%%%%%%%%%%%%%%%%%%%%%
\section{Modality}
\label{sect:modality}

Classic proofs (that use LEM) are modal.  Consider the way a classic
LEM-dependent proof works.  You start by stating the hypothesis: P is
true.  You assume that P is not true; then you derive a contradiction.
The conclusion is not merely that P is true, however; it is that P
\textit{must} be true.

Constructive proofs, by contrast, are not modal.  They do not say what
must be the case, they say what \textit{is} the case.  Or rather, they
\textit{show} what is the case.  (I leave aside the question of
whether what is, is necessary.)

%%%%%%%%%%%%%%%%%%%%%%%%%%%%%%%%
\section{Habeus Corpus Logics}
\label{sect:habeus}

The principle of ``Habeus Corpus'', from the Latin ``(You shall) have
the body'', was once enshrined as a fundamental principle of
Anglo-American law.  It was used to force the State to present a
detainee in person before the court, back in the days when we
occasionally had the temerity to question the wisdom of the State when
it tried to disappear people.

Type theory, and constructive logics generally, operate under a writ
of habeus corpus that is permanently in effect.  Except that this writ
requires the production not of a detainee, but of a witness.  If you
claim to have a proof, you must produce a witness who is competent to
testify to that fact.  So its actually more like a law of evidence,
but I'm too lazy to come up with a clever legalism to express that
idea.




%%%%%%%%%%%%%%%%%%%%%%%%%%%%%%%%
\section{Frege}
\label{sect:frege}

{\todo The Frege Point; force v. content, etc.}


%%%%%%%%%%%%%%%%%%%%%%%%%%%%%%%%
\section{Martin-L\"{o}f}
\label{sect:ml}

{\todo Summarize ML's remarks on assertion, proposition, etc.}

\section{Brandom on Assertion}

\begin{remark}
  Relevance to type theory: type theory begins with an account of
  judgment, proposition, etc.  Robert Brandom offers a very
  sophisticated account of these concepts which IMO could be put to
  very good use in explaining the conceptual foundations of type
  theory.
\end{remark}

Brandom's ``deontic scorekeeping model of discursive practice'' is a
very sophisticated and ambitious philosophical project.  But the main
point of interest for us, his treatment of assertion, is relatively
easy to grasp.

First off, for Brandom logic is fundamentally \textit{expressive},
rather than epistemological.  ``Logic is for establishing the truth of
cerain kinds of claims, by \textit{proving} them.  But logic can also
be thought of in expressive terms, as a distinctive set of tools for
\textit{saying} something that cannot otherwise be made
explicit''. (AR p. 19) One of his favorite examples is the inference
from ``Pittsburgh is west of Princeton'' to ``Princeton is east of
Pittburgh''.  We can endorse that inference as a good material
inference - material because it follows from the meanings of the terms
the sentences contain - even if our language does not contain a
conditional connstruction like ``if...then''.  But once we extend our
language by adding such a device, we can make that endorsement
explicit by saying ``If Pittsburgh is west of Princeton, then
Princeton is east of Pittsburgh''.  So ``if...then'' is an expressive
device, rather than an epistemological one.

Brandom's model of assertion involves both a
social aspect and a structure of commitment and entitlement.  The
basic metaphor is that in the game of giving and asking for reasons,
interlocutors maintain a deontic scorecard for each other and for
themselves, tracking commitments and entitlements.

``According to the model, to treat a performance as an assertion is to
treat it as the undertaking or acknowledgment of a certain kind of
\textit{commitment}---what will be called a 'doxastic' or 'assertional'
commitment.  To be doxastically commited is to have a certain social
status.  Doxastic commitments are normative, more specifically
\textit{deontic} statuses.  Such statuses are creatures of the
practical attitudes of the members of a linguistic community-they are
instituted by practices governing the taking and treating of
individuals \textit{as} committed.  Doxasitc commitments are
essentially a kind of deontic status for which the question of
\textit{entitlement} can arise.  Their inferential articulation, in
virtue of which they deserve to be understood as propsitionally
contentful, consists in consequential relations among the particular
doxastic commitments and entitlements---the ways in which one claim can
commit or entitle one to others (for which it accordingaly can serve
as a reason).''  (MIE p. 142)

``Uttering a sentence with assertional force or significance is
putting it forward \textit{as} a potential reason.  Asserting is
givein reasons....The function of assertion is making sentences
available for use as premises in inferences.'' (MIE p. 168)

``The basic model of inferential practices that institute assertional
significance...is defined by a structure that must be understood in
terms of the interaction of three different dimensions.  First, there
are two different sorts of deontic status involved:
\textit{commitments}, and \textit{entitlements} to commitments...The
second dimension ... turns on the distinction between the
\textit{concomitant} and the \textit{communicative} inheritance of
deontic statuses.  This is the \textit{social} difference between
\textit{intra}personal and \textit{inter}personal uses of a claim as a
premise...The third dimension of broadly inferential articulation
that is crucial to understanding assertional practice is that in which
discursive \textit{authority} is linked to and dependent upon a
corresponding \textit{responsibility}.... In asserting a claim, one
not only authorizes further assertions (for oneself and for others),
but undertakes a responsibility, for one commits oneself to being able
to vindicate the original claim by showing that one is entitled to
make it.'' (MIE p. 168-171)

``At the core of assertional practice lie three fundamental ways in
which one can demonstrate one's entitlement to a clam and thereby
fulfill the responsibility associated with making that
claim... First... one can demonstrate one's entitlement to a claim by
\textit{justifying} it, that is, by giving reasons for it.  Giving
reasons for a claim always consists in making more claims: asserting
premises from which the original claim follows as a conclusion... The
second way of vindicating a commitment by demonstrating entitlement to
it is to appeal to the authority of another asserter.  The
\textit{communcative} function of assertions is to license others who
hear the claim to reassert it.  The significance of this license is
that it makes available to those who rely on it and rassert the
original claim a special way of ischargin thheir responsibiity to
demonstrate their entitlement to it.'' (MIE p. 174; the third way
involves invoking one's own authority as a reliable noninferential
reporter, which is discussed later in MIE.)

%%%%%%%%
\subsection{Propositional Content}
\label{subs:}

Note that Brandom's notion of what it is to understand a proposition
or proof looks very different from Martin-L\"{o}f's.  But they
converge on the essential point, which involves grasping the
inferential relations among concepts and reasons.  For ML,
understanding a proposition means grasping what counts as a proof (or
something like that); for Brandom, it involves grasping the
``inferential articulation'' of the concept - the network of
propositions and inferences relating them that consitutes the concept
itself.  This is more or less just like ML's idea: to understand a
proposition is to grasp what counts as a reason for the proposition,
or---what is the same thing---entitlement to commitment to the
proposition.

%%%%%%%%
\subsection{Applying Brandom's Model}
\label{subs:bapply}

Let's look at what mathematical assertions and judgments look like
from a Brandomian perspective.

To start: we have a propositional content, which we can write as
``unasserted P''.  We have commitment, entitlement, and justification
(proof).

Uttering---or, usually, writing down---a proposition P makes explicit
one's commitment to the content of P, and makes one liable to
demonstrate entitlement to that commitment.  Hearing---reading---a
proposition P entitles one (by ``deontic inheritance'') to undertake a
commitment to P if one is willing to ascribe entitlement to the
utterer/author.  Otherwise, it authorizes one to demand a reason.  One
can also record (on one's ``deontic scorecard'') the speaker's
commitment to P while declining to undertake the commitment oneself.

{\todo MLTT analyzes the structure of (mathematical) assertion interms
  of proposition, judgment, truth, etc.  Map this structure to
  Brandom's structure.  Brandom's account should turn out better since
  it is more finely articulated, and distinguishes explicitly between
  commitment and entitlement.}


%%%%%%%%%%%%%%%%%%%%%%%%%%%%%%%%
\section{From Truth to Testimony}
\label{sect:truth}

We have propositions as types, and we have non-propositional types
like $\nat$.  There is an obvious conflict of intuitions here.
Propositions, like \(1>0\), have truth conditions; names like \(\nat\)
do not.  How can they be the same kind of thing?

I think the way out of this embarrassment is recognition that the
classic concept of truth is not relevant to type theory; or, in a more
positive vein, that only a deflationary or minimalist notion of truth
should be used in type theory.  In type theory one does not say that a
proposition \textit{is} true or false; instead one says that a
(propositional) type is proven or disproven, or that either it or its
negation has a witness\sidenote{Or, a ``maker'' or ``constructor''.}.  Instead of a concept of truth we have a
concept of testimony.  Of course, ordinarily witnesses testify as to
the truth of some proposition; but the witnesses of type theory do
more than that---or rather they do something else, namely they produce
or ``make'' the proposition.

%%%%%%%%
\subsection{Proof, Witness, Constructor}
\label{subs:pwc}

Type theory seems to have settled on an idiom; one says, for example,
that types have or do not have proofs or witnesses.  But there are
problems with both of these terms.  The former covers too much ground
since it includes non-constructive proofs.  The latter invokes a
misleading metaphor, since a witness testifies to the truth, whereas a
type-theoretic witness to a type constructs (makes, produces, etc.)
something.  In the case of propositional types, constructors ``make''
the proposition (in the sense that they are inferences that terminate
in the proposition); in the case of non-propositional types,
constructors make ``elements'' of the type, which serve as proxies(?) for
the type.

\begin{remark}
  Problem: here again propositional and non-propositional types behave
  differently.  Every proof of a proposition has the proposition as
  its conclusion; they are all ``the same'' because they all have that
  element in the same structural position.  But the proof of
  e.g. \(\nat\) is different.  For example, \(2\) is a witness for
  \(\nat\).  Or rather, anything that constructs \(2\) is such a
  witness.  What all such constructions have in common is \(2\), not
  \(\nat\).  So they are all clearly proofs of \(2\), but we want them
  to be proofs of \(\nat\).  How do we get there?
\end{remark}

%%%%%%%%%%%%%%%%
\include{niceties}

\section{The Language of \HoTT}
\label{sect:hottlang}

One way to think about mathematics and logic is in terms of objects,
structures, relations, and the like.  etc.

But one can also think of it in terms of vocabularies (or idioms,
etc.).  Then mastering a discipline is not just a matter grasping some
content, but also of acquiring practical mastery over a vocabulary.

The vocabulary of set theory has dominated mathematical discourse for
most of the last 100 years or so.  Starting in the late 1940s, a
competing vocabulary based on category theory began to emerge.  Today
it is not uncommon to see both vocabularies deployed in the same
discourse (lecture, paper).

{\todo Type theory as a vocabulary - mostly confined to logic, then
  computer science.  Etc.  \HoTT{} as the latest distinctive vocab. -
  covering both math and compsci, also regions of logic.
  Significantly different that both set theory and classic logic.}

%%%%%%%%
\chapter{\HoTT{} Types}
\label{subs:hott}

\HoTT{} primitives are ....\sidenote{A proper exposition would
  list 1) the name of the primitive, e.g. ``Dependent Product-type'';
  2) the ``constructor'' symbol, e.g. \(\Pi\); 3) the analogous
  concept from set theory; 4) the ``rules'' for defining a type
  (formation, construction, elimination, computation, uniqueness).}

\begin{ednote}
  The concept of primitives probably isn't going to work for type
  theories proper.  Type theories seem to be inherently pluralistic,
  so there is no way to pick out some things as intrinsically
  primitive in all type theories.  Each theory might \emph{define} a
  set of primitives, but that would define conventions of the theory,
  not the notion of primitive that we're after here.  So if we want to
  talk about primitives it will have to be as above, involving
  principles antecedent to any type-theoretic talk. (?)
\end{ednote}

%%%%%%%%
\section{Simple Types}
\label{subs:simpletypes}

\begin{ednote}
  E.g. \N
\end{ednote}

``Proposition type'' Conceptually, at least, this seems primitive.
Especially if the concept of ``proposition'' counts as a pre-theoretic
principle.  Which implies that proof is also a pre-theoretic
principle.  Proposition types are fundamentally different than the
other kinds of type, since they have
truth-conditions.\sidenote{Actually this isn't quite right.
  Propositions have truth-conditions in classic logic, but not not in
  type theory.  In type theory they have proofs; a propositional type
  is not true or false, but proven or disproven.  But the larger point
  stands: the \textit{concept} of proposition is different than the
  concept of, say, natural number.} Etc.  It follows that proofs are
fundamentally different from other kinds of witness.

\begin{ednote}
  But Curry-Howard means there is no distinction between types and
  propositions, so it makes no sense to try to demarcate a
  ``proposition type''.  E.g. the type \N can be viewed as the type of
  ``there exists a natural number''.  This feature demarcates type
  theory; in classic logic and math, and esp. traditional logic, there
  is a fundamental difference between propositions and the terms from
  which they are constructed.  Not so in \tth{}.
\end{ednote}


%%%%%%%%
\section{Compound Types}
\label{subs:compountypes}

\begin{description}
\item [Function] ``Unlike in set theory, functions are not defined as
  functional relations; rather they are a primitive concept in type
  theory.''\sidenote{Or: set theory
    \textit{defines} a function as a set of ordered pairs whose domain
    has no duplicates; in other words, it treats a function and its
    ``graph'' as the same thing.  Question: what happens to the graph
    of a function in type theory?} \citep[p. 21]{hottbook}

\item [Product] Product types correspond to cartesian products in set
  theory.  The constructor symbol is the same as in set theory:
  \(\cross\).\sidenote{Why isn't this called the ``\(\Huge\cross\)-type''?}
  ``[U]nlike in set theory, where we define ordered pairs to be
  particular sets and then collect them all together into the
  cartesian product, in type theory, ordered pairs are a primitive
  concept, as are functions.''\citep[p. 26]{hottbook}

\item [Coproduct type] Coproduct types correspond to disjoint unions
  in set theory.  ``In type theory, as was the case with functions and
  products, the coproduct must be a fundamental construction, since
  there is no previously defined notion of ``union of
  types''.\citep[p. 33]{hottbook}

\end{description}

\begin{ednote}
  [Updated] [Update: M. Shulman pulled the scales from my eyes: ``From
    the point of view taken in the book, the difference is only one of
    perspective, and any type can represent a proposition by simply
    shifting our perspective on it. For instance, the type Nat
    represents the proposition "there exists a natural number".'']
  What I've called ``proposition type'' is not \textit{formally}
  distinct; the distinction I'm after is conceptual.  But this
  suggests that we need to add at least one more item to our list of
  primitives: ``ordinary'' or simple types.  Maybe it would be best to
  start with the natural numbers as an example of a simple type,
  rather than function types.  Then an example of a proposition type,
  before proceeding to function type.  The items listed (following the
  \HoTTB{}) are really constructions, or let's say complex types,
  built out of two other types.  So maybe the distinction we want is
  between simple and complex or compound types.  Then the simple types
  would come out as primitive, and the complex types as derived (just
  like dependent types.)  Compare the idea of constructions in
  category theory.  There categories are primitive and e.g. the
  product category is an example of a category constructed from other
  categories.
\end{ednote}

\begin{ednote}
  For consistency, we might want to use symbols to designate all of
  the primitives, just as we do for \(\Pi\) and \(\Sigma\).  This
  would give us: \(\Huge\fun\)-types, \(\Huge\cross\)-types, and
  \(\Huge +\)-types.
\end{ednote}


%%%%%%%%
\section{Dependent Types}
\label{subs:quasiprim}

\begin{ednote}
  Add sth re:  types indexed over n v. parameterized over \(\alpha\)
\end{ednote}

\begin{ednote}
  Major TODO: ``basic'' types (as in Z) plus constraints (predicates)
  v. complex dependent types.

  Example: the paradigmatic example for dependent types is a list.
  This combines a type parameter \(\alpha\) and an index \(n\):
  List~\(\alpha, n:\nat\) means a list of length \(n\) of values of
  type \(\alpha\).

  But that's only one possibility.  Let's add a real number:
  List~\(\alpha, n:\nat, x:\real\).  In this case \(x\) could mean
  anything: sum of the elements of the list, product, mean, standard
  deviation, etc.  In short \(x\) can be any \emph{statistic} computed
  over the list.

  This works intuitively; does it work in type theory?  Specifically,
  \ML{}-style theories like \HoTT?  Would these be genuine types or
  types plus additional constraints?  In other words, is the
  interpretation of \(x\) built-in to the type, or is it an ``extra''
  constraint applied after the fact, as it were.
\end{ednote}

\noindent ``Fundamental''\sidenote[][-28pt]{\begin{ednote}
  Obviously we need a better bit of
  terminology.  ``Quasi-primitives''?  ``Neo-primitives''?  These
  types are not primitive, strictly speaking, but on the other hand
  they are basic.  I think there is another fundamental principle at
  work here.  In set theory, for example, the concept of function is
  not only not primitive, it isn't necessary.  You could discard it
  and still have set theory.  But my intuition tells me that e.g. the
  concept \(\Pi\)-type is in a sense necessary or essential in type
  theory, even if it is not primitive.  Once you have the primitives,
  you necessarily have these non-primitive basic types.  Dunno if
  that's correct, but it would sure be nice if it were.
  \end{ednote}}%
(but non-primitive) concepts and types.  These types seem to be on a par
with the primitive types as far as importance goes, but they
presuppose the primitives, so cannot themselves be considered
primitive.

\begin{ednote}
  We can make a distinction between the concept of dependent type, and
  the two specific dependent types introduced here.  Neither is
  primitive; you can have a type theory without the concept of
  dependent types.  Most programming languages fit this description,
  whether they have an explicit type discipline or not.
\end{ednote}

\begin{ednote}
  Regarding the notion ``quasi-primitive'': not a very satisfactory
  term, but I can't come up with a better one at the moment.  What I'd
  like to show is that the concept of dependent type (maybe also type
  universe) follows ``naturally'' or necessarily or essentially from
  the more primitive concepts.  Maybe the right concept here is
  ``induction'': the primitive concepts (types) ``induce'' the concept
  of dependent type.  That would be nice esp. if induction is a
  primary principle.
\end{ednote}

\begin{description}

\item [Universe]  Is this a primitive?  Probably not, since it builds on the type concept.

\item [\(\Pi\)-type] Informally, ``dependent function''
  types.\sidenote{As a practical matter, I think it would be useful to
    have an informal term for these types that falls between
    ``dependent function type'' and \(\Pi\)-type.  Something like
    ``p-function type''.  \(\Pi\)-type is admirably concise, but I
    think it should mention ``function'', since it names a kind of
    function.}  The concept of \(\Pi\)-type is a generalization of the
  concept of function type, so it isn't primitive.

\item [\(\Sigma\)-type] Informally, ``dependent pair''
  type.\sidenote{Shouldn't this be called ``dependent
    \textit{product}'' type?  The type is product, not pair; pairs are
    ``elements'' of the type.  Informally, maybe ``sig-prod type?}
  The concept of \(\Sigma\)-types is a generalization of the concept
  of product type.

\end{description}

%%%%%%%%
\section{Standard Type Library}
\label{subs:hottstdlib}

\begin{ednote}
  By analogy to the usual ``standard library'' of programming
  languages.  The idea is to list commonly used types that are neither
  primitive nor quasi-primitive; ``application'' types, in a sense.
\end{ednote}


\begin{description}
\item [Boolean] \citep[p. 34]{hottbook}
\item [$\nat$] \citep[p. 36]{hottbook} But there's a problem here;
  actually several.  First of all, the section on the naturals in
  chapter one does not actually show how to construct them; its really
  a section about induction, not the natural numbers.  Second, what it
  does say about the naturals is that they start from zero.  That
  obviously won't do; zero is not a natural number, and there is no
  intuitive notion of zero as a number.  You can't even think of it as
  a number until you've severed the link between the concept of number
  and the concepts of quantity and/or magnitude.  So the natural
  numbers really must start with 1, not 0.
\item [Proposition]  Moved to Primitives section.
\item [Identity]
\end{description}



\clearpage
\appendix
\begin{appendices}

\include{introduction}

\include{preliminaries}

%%%% Bibliography
%% \bibliographystyle{halpha}
%% \phantomsection % black magic to get TOC to point to correct page
%% \addcontentsline{toc}{part}{\bibname}
%% \markboth{}{\textsc{Bibliography}}
%% {\renewcommand{\markboth}[2]{} % Prevent bibliography from resetting the header to something silly
%% \OPTbibliographyfont
\bibliography{references}
\bibliographystyle{plainnat}

\end{appendices}

\end{document}
