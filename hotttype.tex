\documentclass[12pt,toc]{tufte-handout}
%% \documentclass[reqno,12pt]{tufte-book}
%% \usepackage{trace}
%% \documentclass[reqno,12pt]{article}

\usepackage{draftwatermark}

% BLACK & WHITE
\input{opt-black-white}

% FORMATTING DEPENDENT ON PAPER SIZE
\input{opt-letter}

\usepackage{etex}

%%%%%%%%%%%%%%%%%%%%%%%%%%%%%%%%%%%%%%%%%%%%%%%%%%%%%%%%%%%%%%%%
%% packages included by original hott main.tex

%%% For table {tab:theorems}
\usepackage{pifont}

%%% Multi-Columns for long lists of names
\usepackage{multicol}

\usepackage{graphicx}
\usepackage{comment}

\usepackage{fancyhdr} % To set headers and footers

%% \usepackage{nextpage} % So we can jump to odd-numbered pages

\usepackage{amssymb,amsmath,amsthm,stmaryrd,mathrsfs,wasysym}
\usepackage{enumitem,mathtools,xspace}
%% \numberwithin{equation}{subsection}

\usepackage{xstring} % For generating singluars and plurals in \backref

%% \usepackage{xcolor,mdframed}
\usepackage{xcolor} % For colored cells in tables we need \cellcolor
\usepackage{wallpaper} % For the background image on the cover page

\usepackage{booktabs} % For nice tables
\usepackage{array} % For nice tables

\definecolor{linkcolor}{rgb}{\OPTlinkcolor}
\usepackage{aliascnt}
\usepackage[capitalize]{cleveref}
\usepackage[all,2cell,cmtip]{xy}
\UseAllTwocells
%\usepackage{natbib}
\usepackage{braket} % used for \setof{ ... } macro

\usepackage{tikz}
\usetikzlibrary{decorations.pathmorphing,arrows}

\usepackage{etoolbox}           % hacking commands for TOC

%% \usepackage{mathpartir}         % for formal.tex appendix, section 3

\usepackage[numbered]{bookmark} % add chapter/section numbers to the toc in the pdf metadata
%%%%%%%%%%%%%%%%%%%%%%%%%%%%%%%%%%%%%%%%%%%%%%%%%%%%%%%%%%%%%%%%
\input{macros}

\usepackage{appendix}

%% \usepackage{csquotes}

\usepackage{setspace}

%% broken (doesn't work with tufte-handout):
\usepackage{zed-csp}
%% broken:
%% \usepackage{ltcadiz-fam}

\usepackage{fontspec}
%% \usepackage{xltxtra,xunicode}
\defaultfontfeatures{Scale=MatchLowercase}

%% \defaultfontfeatures{Scale=MatchLowercase}
%% \setmainfont[Mapping=tex-text]{Times New Roman}
%% \setsansfont[Mapping=tex-text]{Arial}
%% \setmonofont{Courier}

\setmainfont[Ligatures=TeX]{TeX Gyre Bonum}
\setromanfont[Ligatures=TeX]{TeX Gyre Bonum}
\setsansfont[Ligatures=TeX]{TeX Gyre Adventor}
\setmonofont[Ligatures=TeX]{TeX Gyre Cursor}


%% \setmainfont[Mapping=tex-text]{Minion Pro}
%% \setromanfont[Mapping=tex-text]{Minion Pro}
%% \setsansfont[Mapping=tex-text]{TeX Gyre Heros}

%% Bugfix: see https://code.google.com/p/tufte-latex/issues/detail?id=64
% Set up the spacing using fontspec features
\renewcommand\allcapsspacing[1]{{\addfontfeature{LetterSpace=15}#1}}
\renewcommand\smallcapsspacing[1]{{\addfontfeature{LetterSpace=0.0}#1}}

\usepackage{epigraph}
\setlength{\epigraphwidth}{.8\textwidth}

%% general symbols - degree, etc.
%% \usepackage{gensymb}

\usepackage [english]{babel}
\usepackage [autostyle, english = american]{csquotes}

%% nice double-stroke fonts
\usepackage{dsfont}

% Small sections of multiple columns
\usepackage{multicol}

% Provides paragraphs of dummy text
\usepackage{lipsum}

% The units package provides nice, non-stacked fractions and better spacing
% for units.
\usepackage{units}

%\usepackage{geometry}                % See geometry.pdf to learn the layout options. There are lots.
%\geometry{letterpaper}                   % ... or a4paper or a5paper or ...

\usepackage{xfrac}

\usepackage{hyperref}
\hypersetup{
  bookmarks=true,         % show bookmarks bar?
  bookmarksdepth=3,
  unicode=true,          % non-Latin characters in Acrobat’s bookmarks
  pdftoolbar=true,        % show Acrobat’s toolbar?
  pdfmenubar=true,        % show Acrobat’s menu?
  pdffitwindow=false,     % window fit to page when opened
  pdfstartview={FitH},    % fits the width of the page to the window
  pdftitle={Intuition and Exponentiation},    % title
  pdfauthor={G. A. Reynolds},     % author
  pdfsubject={Mathematics},   % subject of the document
  pdfcreator={G. A. Reynolds},   % creator of the document
  pdfproducer={Producer}, % producer of the document
  pdfkeywords={Exponentiation} {Mathematics}
  pdfnewwindow=true,      % links in new window
  colorlinks=true,       % false: boxed links; true: colored links
  linkcolor=blue,          % color of internal links
  citecolor=blue,        % color of links to bibliography
  filecolor=magenta,      % color of file links
  urlcolor=cyan           % color of external links
}

%% \usepackage[
%% bibstyle=numeric,
%% citestyle=authoryear,
%% hyperref,
%% bibencoding=utf8,
%% backref=true,
%% backend=biber]{biblatex}

%% http://tex.stackexchange.com/questions/66778/citation-alias-with-multibib-and-natbib
%% \makeatletter
%% \def\@mb@citenamelist{cite,citep,citet,citealp,citealt,citepalias,citetalias}
%% \makeatother

%% http://stackoverflow.com/questions/2496599/how-do-i-cite-the-title-of-an-article-in-latex
\defcitealias{z-iso-13568}{ISO 13568:2002 Information technology -- Z formal specification notation --
  Syntax, type system and semantics}

\usepackage{tikz}
\usepackage[markings,customcolors]{hf-tikz}
\usetikzlibrary{%
  arrows%
  ,calc%
  ,decorations.text%
  ,decorations.pathreplacing%
  ,fadings%
  ,positioning
  ,shapes.geometric%
}

\usepackage{tikz-3dplot}

\usepackage{pgfplots}
\pgfplotsset{height=7cm,compat=1.9}

\usepackage{tkz-euclide}
\usetkzobj{all}

%% prettier integral syms, but broken on miktex
%% \usepackage{esint}


%% \usepackage{MnSymbol}
%% \usepackage[misc]{ifsym}

%% \usepackage{morefloats}

%%%%%%%%%%%%%%%%%%%%%%%%%%%%%%%%%%%%%%%%%%%%%%%%%%%%%%%%%%%%%%%%
\title{HoTT Types \\
\Large Derived from the HoTT Book}
\author{}
%\date{}                                           % Activate to display a given date or no date

%%%%%%%%%%%%%%%%
%% tufte-latex customizations

\makeatletter
\let\runauthor\@author
\let\runtitle\@title
\makeatother

%% running headers
\newcommand{\changefont}{%
  \fontsize{7}{9.5}\selectfont
}
\fancypagestyle{plain}{
  \fancyhead[LO,LE]{\leftmark }
  \fancyhead[RO,RE]{\rightmark}
  \fancyfoot[CO,CE]{\thepage}
  \fancyfoot[LE]{\textsc{\runtitle}}
  \fancyfoot[RO]{\textsc{\runtitle}}
  \renewcommand{\headrulewidth}{0pt}
  \renewcommand{\footrulewidth}{0pt}
}
\pagestyle{plain}

\def\chpcolor{blue!45}
\def\chpcolortxt{blue!60}
\def\sectionfont{\LARGE}

\setcounter{secnumdepth}{5}
\setcounter{tocdepth}{5}        % sections and subsections for the toc

\makeatletter
%% Section:
\def\@sectionstrut{\vrule\@width\z@\@height12.5\p@}
\def\@makesectionhead#1{%
  {%\par\vspace{20pt}%
    \parindent -10pt\raggedleft\sectionfont
    %% \colorbox{\chpcolor}{%
    %%   \parbox[t]{90pt}{\color{white}\@sectionstrut\@depth4.5\p@\hfill
    %%     \ifnum\c@secnumdepth>\z@\thesection\fi}%
    %% }%
    \vspace{10pt}%
    \begin{minipage}[t]{\textwidth}%{\dimexpr\textwidth-90pt-2\fboxsep\relax}
      \@sectionstrut\hspace{-15pt}\textit{\textbf\Huge #1}
    \end{minipage}\par
    \vspace{5pt}%
  }
}
%% \def\@makesectionhead#1{%
%%   {\par\vspace{20pt}%
%%    \parindent 0pt\raggedleft\sectionfont
%%    \colorbox{\chpcolor}{%
%%      \parbox[t]{90pt}{\color{white}\@sectionstrut\@depth4.5\p@\hfill
%%        \ifnum\c@secnumdepth>\z@\thesection\fi}%
%%    }%
%%    \begin{minipage}[t]{\dimexpr\textwidth-90pt-2\fboxsep\relax}
%%    \color{\chpcolortxt}\@sectionstrut\hspace{5pt}\textbf{#1}
%%    \end{minipage}\par
%%    \vspace{10pt}%
%%   }
%% }
\def\section{\@afterindentfalse\secdef\@section\@ssection}
\def\@section[#1]#2{%
  \ifnum\c@secnumdepth>\m@ne
  \refstepcounter{section}%
  \addcontentsline{toc}{section}{\protect\numberline{\thesection}#1}%
  \else
  \phantomsection
  \addcontentsline{toc}{section}{#1}%
  \fi
  \sectionmark{#1}%
  \if@twocolumn
  \@topnewpage[\@makesectionhead{#2}]%
  \else
  \@makesectionhead{#2}\@afterheading
  \fi
}
\def\@ssection#1{%
  \if@twocolumn
  \@topnewpage[\@makesectionhead{#1}]%
  \else
  \@makesectionhead{#1}\@afterheading
  \fi
}
\makeatother

%%%%%%%%%%%%%%%%
%% macros

\newenvironment{important}[1][]{%
  \begin{mdframed}[%
      backgroundcolor={red!15}, hidealllines=true,
      skipabove=0.7\baselineskip, skipbelow=0.7\baselineskip,
      splitbottomskip=2pt, splittopskip=4pt, #1]%
    \makebox[0pt]{% ignore the withd of !
      \smash{% ignor the height of !
        \fontsize{32pt}{32pt}\selectfont% make the ! bigger
        \hspace*{-19pt}% move ! to the left
        \raisebox{-2pt}{% move ! up a little
          {\color{red!70!black}\sffamily\bfseries !}% type the bold red !
        }%
      }%
    }%
}{\end{mdframed}}

%% reversed integral sign
\makeatletter
\providecommand*{\curv}{%
  \mathrel{%
    \mathpalette\@curv\int
  }%
}
\newcommand*{\@curv}[2]{%
  \reflectbox{$\m@th#1#2$}%
}
\makeatother



\newtheorem{theorem}{Theorem}
%\newtheorem{cor}{Corollary}
%\newtheorem{lem}{Lemma}
%% \newtheorem*{defn}{Definition}
%% \theoremstyle{remark}
\newtheorem{remark}{Remark}
\newtheorem*{commentary}{Commentary}
\newtheorem*{todo}{TODO}
%% \newtheorem{eg}{Example}

\newcommand\cspace{coordinate space}
\newcommand\Cspace{Coordinate space}
\newcommand\CSpace{Coordinate Space}

\newcommand\dspace{design space}
\newcommand\Dspace{Design space}
\newcommand\DSpace{Design Space}

\newcommand\Omg{\(\Omega\)}
\newcommand\sccs{standard cartesian coordinate space}
\newcommand\origin{\((0,0)\)}
\newcommand\ab{\((a,b)\)}

\newcommand\atypeA{\ensuremath{(a : A)}}

%% \newcommand\N{\(\mathds{N}\)}
%% \newcommand\R{\(\mathds{R}\)}
%% \newcommand\RR{\(\mathds{R}\times\mathds{R}\)}
%% \newcommand\Rtwo{\(\mathds{R}^2\)}
%% \newcommand\Z{\(\mathds{Z}\)}

\includeonly{%
introduction%
,preliminaries
,niceties
,brandom
}
%%%%%%%%%%%%%%%%%%%%%%%%%%%%%%%%%%%%%%%%%%%%%%%%%%%%%%%%%%%%%%%%
\begin{document}
%% \ifx\traceon\undefined \tracingall \else \traceon \fi

\maketitle

\begin{abstract}
  Currently this doc contains the intro and chapter 1 from the
  \href{http://homotopytypetheory.org/book/}{HoTT Book}, plus some
  notes on some of the basic ideas like type and judgment.  The idea
  is to winnow out some of the strictly mathematical stuff leaving the
  core ``philosophical'' stuff, and annotate the text with some
  comments and quotes from Martin-L\"{o}f, Brandom, etc.  The purpose
  is to more fully articulate the link between HoTT's ideas of type
  and judgment (etc.) to the philosophical debates about language,
  assertion, proposition from which they emerged.  Why?  Because I
  find those bits of the HoTT a little murky, and philosophers like
  Brandom have a lot to say about the issues.
\end{abstract}

\tableofcontents
%% \setcounter{tocdepth}{2}        % chapters, sections, and subsections for the
%%                                 % metadata of the pdf
%% \cleartooddpage[\thispagestyle{empty}]

%% \mainmatter % Turn on roman page numbers and numbered chapters

\include{introduction}

\include{preliminaries}

\include{niceties}

\section{Brandom on Assertion}

\begin{remark}
  Relevance to type theory: type theory begins with an account of
  judgment, proposition, etc.  Robert Brandom offers a very
  sophisticated account of these concepts which IMO could be put to
  very good use in explaining the conceptual foundations of type
  theory.
\end{remark}

Brandom's ``deontic scorekeeping model of discursive practice'' is a
very sophisticated and ambitious philosophical project.  But the main
point of interest for us, his treatment of assertion, is relatively
easy to grasp.

First off, for Brandom logic is fundamentally \textit{expressive},
rather than epistemological.  ``Logic is for establishing the truth of
cerain kinds of claims, by \textit{proving} them.  But logic can also
be thought of in expressive terms, as a distinctive set of tools for
\textit{saying} something that cannot otherwise be made
explicit''. (AR p. 19) One of his favorite examples is the inference
from ``Pittsburgh is west of Princeton'' to ``Princeton is east of
Pittburgh''.  We can endorse that inference as a good material
inference - material because it follows from the meanings of the terms
the sentences contain - even if our language does not contain a
conditional connstruction like ``if...then''.  But once we extend our
language by adding such a device, we can make that endorsement
explicit by saying ``If Pittsburgh is west of Princeton, then
Princeton is east of Pittsburgh''.  So ``if...then'' is an expressive
device, rather than an epistemological one.

Brandom's model of assertion involves both a
social aspect and a structure of commitment and entitlement.  The
basic metaphor is that in the game of giving and asking for reasons,
interlocutors maintain a deontic scorecard for each other and for
themselves, tracking commitments and entitlements.

``According to the model, to treat a performance as an assertion is to
treat it as the undertaking or acknowledgment of a certain kind of
\textit{commitment}---what will be called a 'doxastic' or 'assertional'
commitment.  To be doxastically commited is to have a certain social
status.  Doxastic commitments are normative, more specifically
\textit{deontic} statuses.  Such statuses are creatures of the
practical attitudes of the members of a linguistic community-they are
instituted by practices governing the taking and treating of
individuals \textit{as} committed.  Doxasitc commitments are
essentially a kind of deontic status for which the question of
\textit{entitlement} can arise.  Their inferential articulation, in
virtue of which they deserve to be understood as propsitionally
contentful, consists in consequential relations among the particular
doxastic commitments and entitlements---the ways in which one claim can
commit or entitle one to others (for which it accordingaly can serve
as a reason).''  (MIE p. 142)

``Uttering a sentence with assertional force or significance is
putting it forward \textit{as} a potential reason.  Asserting is
givein reasons....The function of assertion is making sentences
available for use as premises in inferences.'' (MIE p. 168)

``The basic model of inferential practices that institute assertional
significance...is defined by a structure that must be understood in
terms of the interaction of three different dimensions.  First, there
are two different sorts of deontic status involved:
\textit{commitments}, and \textit{entitlements} to commitments...The
second dimension ... turns on the distinction between the
\textit{concomitant} and the \textit{communicative} inheritance of
deontic statuses.  This is the \textit{social} difference between
\textit{intra}personal and \textit{inter}personal uses of a claim as a
premise...The third dimension of broadly inferential articulation
that is crucial to understanding assertional practice is that in which
discursive \textit{authority} is linked to and dependent upon a
corresponding \textit{responsibility}.... In asserting a claim, one
not only authorizes further assertions (for oneself and for others),
but undertakes a responsibility, for one commits oneself to being able
to vindicate the original claim by showing that one is entitled to
make it.'' (MIE p. 168-171)

``At the core of assertional practice lie three fundamental ways in
which one can demonstrate one's entitlement to a clam and thereby
fulfill the responsibility associated with making that
claim... First... one can demonstrate one's entitlement to a claim by
\textit{justifying} it, that is, by giving reasons for it.  Giving
reasons for a claim always consists in making more claims: asserting
premises from which the original claim follows as a conclusion... The
second way of vindicating a commitment by demonstrating entitlement to
it is to appeal to the authority of another asserter.  The
\textit{communcative} function of assertions is to license others who
hear the claim to reassert it.  The significance of this license is
that it makes available to those who rely on it and rassert the
original claim a special way of ischargin thheir responsibiity to
demonstrate their entitlement to it.'' (MIE p. 174; the third way
involves invoking one's own authority as a reliable noninferential
reporter, which is discussed later in MIE.)

%%%%%%%%
\subsection{Propositional Content}
\label{subs:}

Note that Brandom's notion of what it is to understand a proposition
or proof looks very different from Martin-L\"{o}f's.  But they
converge on the essential point, which involves grasping the
inferential relations among concepts and reasons.  For ML,
understanding a proposition means grasping what counts as a proof (or
something like that); for Brandom, it involves grasping the
``inferential articulation'' of the concept - the network of
propositions and inferences relating them that consitutes the concept
itself.  This is more or less just like ML's idea: to understand a
proposition is to grasp what counts as a reason for the proposition,
or---what is the same thing---entitlement to commitment to the
proposition.

%%%%%%%%
\subsection{Applying Brandom's Model}
\label{subs:bapply}

Let's look at what mathematical assertions and judgments look like
from a Brandomian perspective.

To start: we have a propositional content, which we can write as
``unasserted P''.  We have commitment, entitlement, and justification
(proof).

Uttering---or, usually, writing down---a proposition P makes explicit
one's commitment to the content of P, and makes one liable to
demonstrate entitlement to that commitment.  Hearing---reading---a
proposition P entitles one (by ``deontic inheritance'') to undertake a
commitment to P if one is willing to ascribe entitlement to the
utterer/author.  Otherwise, it authorizes one to demand a reason.  One
can also record (on one's ``deontic scorecard'') the speaker's
commitment to P while declining to undertake the commitment oneself.

{\todo MLTT analyzes the structure of (mathematical) assertion interms
  of proposition, judgment, truth, etc.  Map this structure to
  Brandom's structure.  Brandom's account should turn out better since
  it is more finely articulated, and distinguishes explicitly between
  commitment and entitlement.}


%%%% Bibliography
%% \bibliographystyle{halpha}
%% \phantomsection % black magic to get TOC to point to correct page
%% \addcontentsline{toc}{part}{\bibname}
%% \markboth{}{\textsc{Bibliography}}
%% {\renewcommand{\markboth}[2]{} % Prevent bibliography from resetting the header to something silly
%% \OPTbibliographyfont
\bibliography{references}
\bibliographystyle{plainnat}


\end{document}
