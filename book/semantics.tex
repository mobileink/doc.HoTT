%%%%%%%%%%%%%%%%%%%%%%%%%%%%%%%%
\section{Semantics}
\label{sect:semantics}

%%%%%%%%
\subsection{Meaning}
\label{subs:meaning}

%%%%%%%%
\subsection{Model-theoretic Semantics}
\label{subs:modeltheorysem}

%%%%%%%%
\subsection{Proof-theoretic Semantics}
\label{subs:proofsem}

``Proof-theoretic semantics is an alternative to truth-condition semantics. It is based on the fundamental assumption that the central notion in terms of which meanings are assigned to certain expressions of our language, in particular to logical constants, is that of proof rather than truth. In this sense proof-theoretic semantics is semantics in terms of proof . Proof-theoretic semantics also means the semantics of proofs, i.e., the semantics of entities which describe how we arrive at certain assertions given certain assumptions. Both aspects of proof-theoretic semantics can be intertwined, i.e. the semantics of proofs is itself often given in terms of proofs.''\cite{schroeder-heister_proof-theoretic_sep}


%%%%%%%%
\subsection{Inferential Semantics}
\label{subs:inferensem}

