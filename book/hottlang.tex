\chapter{The Language of \HoTT}
\label{sect:hottlang}

\begin{ednote}
  TODO: disentangle commonly used terminology: parameterized types,
  generic types, algebraic types, generalized algebraic types, types
  indexed/parameterized, polymorphism, parametric polymorphism,
  polytypism etc. etc.
\end{ednote}

\begin{ednote}
  Type formers as entries in a \HoTT{} lexicon, serving not as
  definitions but as (normative) rules of usage.  They don't say what
  the terms mean, they set out how to use them.  That means,
  specifically, the rules governing the notation, not rules governing
  denoted entities.

  NB: rules of use \emph{of vocabulary} \(\neq\) rules of construction
  \emph{of objects}.  But the idea is for one set of rules to work
  both ways.  That's pretty much how model-theoretic semantics
  connects vocab to semantic domain (completeness and consistency).
  The difference is that the TT semantic domain here (i.e. objects and
  their rules of construction) is not a passive, platonistic realm of
  ``real'' objects, but a pragmatic ``field'' of action.  So rules of
  vocab use and rules of construction converge while remaining
  conceptually distinct.  IOW the difference is not metaphysical.

  On this view we treat \HoTT{} as truly a vocabulary rather than a
  theory about something.  Or more precisely as a regimented idiom or
  dialect.  The user is free to treat e.g. \N as ``defining'' a true
  model of the natural numbers, but \HoTT{} makes no such claim.
\end{ednote}

One way to think about mathematics and logic is in terms of objects,
structures, relations, and the like.  etc.

But one can also think of it in terms of vocabularies (or idioms,
etc.).  Then mastering a discipline is not just a matter grasping some
content, but also of acquiring practical mastery over a vocabulary.

The vocabulary of set theory has dominated mathematical discourse for
most of the last 100 years or so.  Starting in the late 1940s, a
competing vocabulary based on category theory began to emerge.  Today
it is not uncommon to see both vocabularies deployed in the same
discourse (lecture, paper).

{\todo Type theory as a vocabulary - mostly confined to logic, then
  computer science.  Etc.  \HoTT{} as the latest distinctive vocab. -
  covering both math and compsci, also regions of logic.
  Significantly different that both set theory and classic logic.}

``it is possible to directly formalize the world of homotopy types
using the class of languages called dependent type systems and in
particular Martin-Lof type systems.'' V. Voevodsky
\url{http://www.math.ias.edu/~vladimir/Site3/Univalent\_Foundations\_files/univalent\_foundations\_project.pdf}

Note: ``class of languages called dependent type systems'' -
languages, not theories

``Type systems are syntactic objects which are specified in several
steps. First one chooses a formal language L which allows the use of
variables and substitution. Then one chooses a collection of relations
on the sets of L-expressions with a given set of free variables which
is stable under the substitutions. These relations are called the
reduction rules and the equivalence relation generated by the
reduction rules is called the conversion relation....  A type system
based on L is defined as a pair of subsets BB and BBg in the sets of
pre-contexts and pre-sequents respectively which satisfy a number of
conditions with respect to reduction and substitution. Elements of BB
are called the (valid) contexts of a type system and elements of BBg
the (valid) sequents of the type system.'' same, p. 3


