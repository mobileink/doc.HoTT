\documentclass{article}

\usepackage{hottmacros}

\usepackage{fontspec,xltxtra,xunicode}
\usepackage{fontspec}
\defaultfontfeatures{Scale=MatchLowercase}

%% \setmainfont[Mapping=tex-text]{Times New Roman}
%% \setromanfont[Mapping=tex-text]{Times New Roman}
%% \setsansfont[Mapping=tex-text]{Arial}

\setmainfont[Mapping=tex-text]{TeX Gyre Pagella}
%% \setromanfont[Mapping=tex-text]{TeX Gyre Pagella}
%% \setsansfont[Mapping=tex-text]{TeX Gyre Heros}

\usepackage{tikz}
%% \usetikzlibrary{arrows,shapes,patterns,backgrounds,spy}
%% \usepackage{pgffor}

%% \usepackage{animate}

%%%%%%%%%%%%%%%%%%%%%%%%%%%%%%%%%%%%%%%%%%%%%%%%%%%%%%%%%%%%%%%%
\begin{document}
\title{Unit Type Families}
\maketitle
\tableofcontents
\vfill
\large


HoTT without Equality

or: Equality/Identity type as a generalization of Unit type.

We can discard both the concept of equality and the principle of
induction.

First note that our ordinary sense of equality does not involve proof
as a first-class object.  To show that a=b the best we can do is
display a and b in reduced/normal form, look at them, and see that
they are equal.  The HoTT book puts it this way: "[t]he basic way to
construct an element of a = b is to know that a and b are the same."
(p. 47) But this is a cop-out.  The problem is that knowledge is not
something that we can display for inspection and manipulation.  We
inspect a and b, and "see" (that is, know or come to know) that they
are they same; how can we then display that knowledge itself?  We
cannot.

Type theory thus requires an extension of our ordinary concept of
equality.  It's not enough just to know that two things are the same;
we must have a way of displaying that knowledge in some form.  HoTT
solves this problem by fiat: it simply declares that refl(a) is to
count as proof that a=a.  But note that this does not really say much
about equality; after all, labels like "refl" and "Identity" for
tokens and types are arbitrary.  They are chosen to suggest
(informally) well-know concepts, but they have no intrinsic meaning -
refl suggests the mathematical principle of reflexivity, but that is
literally not what it means.  It is just a constructor; all we know
from refl(a):(a=a) is that we have a token of a type.  We could change
the names to aqjd(a):(a qedrtw a) and it would make no difference to
the meaning.  There is nothing in the type-theoretic meaning of refl
etc. that bears any relation to the concept of equality.

HoTT's use of the induction principle for equality.  First, a concept
of induction that does not include an inductive hypothesis can hardly
be considered an induction principle.  So HoTT's "induction principle
for equality" is an abuse of language; whatever it is, it does not
involve genuine induction.  In fact it seems to be no more than a
disguised version of substitution salve veritate: equals can be
substituted for equals.  The critical point is that it relies on an
external concept of equality, drawn from homotopy theory.

Informally expressed, what HoTT says is that to prove P for all
equality tokens, it is sufficient to prove it for one distinguished
equality token, namely the reflexivity token.  What makes this work is
the interpretation of a = b (given a:A, b:A) as a path between points
a and b in space A.  More precisely, the proof of a=b is taken to be a
continuous function f on the unit interval [0,1] into space A with
f(0)=a and f(1)=b.  So to prove a=b, it is sufficient to prove a=a -
that is, to construct a continuous, constant function g on [0,1] into
A with f(x)=a for all x in [0,1].  What makes this sufficient is the
homotopy between f and g; that is, f can be continuously deformed
until it is point-wise equalt to g, and vice-versa.  That makes f and
g homotopy-equivalent, and this homotopy equivalence is interpreted as
evidence a proof a (that is, path between) a and a is "equal" to any
proof of (path between) a and b, so if you can prove something for the
former you can prove it for the latter.

How can we construct a proof of a=a?  That is, how can we construct
the appropriate constant function?  Unfortunately, we cannot, not in
type theory; we just have to assume it, and the way we do that is to
punt, and just declare that the constructor refl(a) just is that
function.  So it takes the form of the sole constructor for type
(a=a); by definition, refl(a):(a=a).  So if you prove something for
refl(a), you've proved it for every x:(a=b); or rather, a refl(a) is
equivalent to every proof of (a=b).

This looks suspiciously like an existential conditional: if
refl(a):(a=a), then /there exists/ some x such that x:(a=b).  But that
is not quite right; the principle does not address the question of
whether such an x exists.  Rather, it says that if it does exist, it
will satisfy any P that refl(a) satisfies.  In other words, the HoTT
approach says nothing about how to go about finding some x:(a=b); it
only addresses how we can use such things /if/ they exist.

In summary: the homotopy interpretation says that any proof of (a=b)
is equivalent to a proof of (a=a), and vice-versa.  Path induction,
which depends on the homotopy interpretation, says that we can define
functions on (a=b) in terms of functions on (a=a) - another way of
saying that to prove something for all (a=b) it is sufficient to prove
it for (a=a).

It's all very elegant, but it depends essentially on something
external to type theory, and it involves a very questionable notion of
induction.

Summary:

    * equality/identity types and the refl constructor have nothing to do with equality

    * "path induction" does not involve induction

    * 

* An alternative approach: Dependent Unit types

We have:

*    : Unit0

*a   : Unit1 a

**a  : Unit2 a a

a,b:A, *a, *b : Unit1 a |- a=b : Unit2 a b

***a : Unit3 a a a

etc.



We add some eliminators; for now we'll restrict our attention to
Unit and Unit2.  Expressed as inference rules:

x : Unit3 a b c |-  **a : Unit2 a a

x : Unit3 a b c |-  **b : Unit2 a a

x : Unit3 a b c |-  **c : Unit2 a a

x : Unit2 a b |-  *a : Unit1 a

x : Unit2 a b |-  *b : Unit1 a

x : Unit1 a   |-  *  : Unit0

x : Unit0     |-  ???

\section{Explanation}

Like HoTT's 1 type, each Unit type has only a single constructor, so
every element is "equal" to that constructor.  This is the source of
the interpretation of Unit2 as the equality type.  Given x:Unit(a,b),
we know a*:Unit(a) and b*:Unit(a); and since there is only one element
of Unit(a), we know that a* and b* are "the same".

Erm, except that to prove that every element of Unit(a) is "equal" to
*a, we need an equality type.

So the whole thing depends on the idea that every element of a type is
"equal" to one built by a constructor.  Or: every element is
expressible as a canonical form, where the latter means a form whose
principal term is a constructor.

Aside: the notion that we need to prove such a thing is an artifact of
model-theoretic semantics.  See Bauer's blog at
http://math.andrej.com/2013/08/28/the-elements-of-an-inductive-type/

HoTT, p. 171
%% (https://books.google.com/books?id=LkDUKMv3yp0C&lpg=PP1&pg=PA171#v=onepage&q&f=false):

"The view of identity types as inductive types has historically caused
some confusion, because of the intuition mentioned in §5.1 that all
the elements of an inductive type should be obtained by repeatedly
applying its constructors."

For identity types "[t]here is only one constructor, refl, but not
every path is equal to the constant path.  More precisely, we cannot
prove, using only the induction principle for identity types (either
one), that every inhabitant of a = a is equal to refl(a)".

Ok, the problem is a misconstrual of the role of canonical form.  It's
not that every element is produced using constructors, but that every
element can be expressed as a canonical form.  And this is a principle
of type theory, not an artifact of a particular model.  (It means
among other things that you can always use the elimination rules.)

Example of and element that is not a constructor?

\section{Problems with MT semantics}

1.  Just use the reals as the semantic object assigned to the type
name Nat.  Then universal quantification catches reals; the model
forces us to interpret forall n:Nat as forall n in R.  This is because
we've mixed types (Nat) and sets (R), and this is probably unaviodable
for MT semantics.

2.  Same problem with e.g. Pi and Sigma types.  Anywhere we have a
typed variable, say, x:Nat, we have the same problem: how do we know
which element of R is assigned?

The problem with the problem Bauer sees in our interpretation of
induction is that he confuses induction and quantification.

3.  Another way to see it: we're using quantification in a language
that presupposes a discreet domain, but we're actually quantifying
over a continuous domain.

4.  There's another problem but I can't recall it.  Drat.

5.  If you want HoTT to be a foundation framework, you cannot use some
other foundation - i.e. set theory - to define HoTT.  But that is what
HoTT does.  Well, some might dispute that, so let's refine the
argument.  HoTT, as a model-theoretic enterprise, depends on or
presupposes the same classic representational semantics/logic that
classic set theory depends on.  So even if you can argue successfully
that HoTT does not directly depend on set theory, it does presuppose
all the "machinery" that set theory depends on.



But do we need to prove that Unit proofs are unique?  Can't we just
define them so?  E.g.

\begin{align}
  x{:}\,Unit(a) \vdash\ *a{:}\,Unit(a)
\end{align}

So we can infer *a from any x:Unit(a).  This does not mean that every
such x is the same as *a.

Do we need an elimination rule for Unit?  I gues we have:

    *a : Unit(a) |-  a:???

or: *a : Unit(a) |-  * : Unit0

\section{Interpretation}

The concept of equality is based on the notion of a fair trade, which
we make concrete in the form of a contract.  Suppose we want to trade
a quantity of rice for a quantity of beans.  We sign a contract.  In
addition to obliging each of us to deliver the goods, the contract
serves as a concrete proof of the equal /value/ of those goods.  This
is true even for barters that do not involve money; the mere fact of a
contract agreed to by both parties means the trade is fair - an
exchange of equal values.

So we can informally interpret Unit2 as the type of such contracts:
\(p{:}\Unit(a,b)\) is proof (in the form of a "contract") that \(a\)
and \(b\) have the same exchange value (i.e. \(*a, *b{:}\Unit(x)\) for
that \(x\) - for p?  i.e. \(p{:}\Unit(a,b)\vdash\ *a, *b{:}\Unit(p)\)).

The Unit type can be viewed as a kind of appraisal type: given
\(a{:}A\), \(*a\) represents the ``exchange value'' of \(a\) ``under''
\(\Unit(a)\).


\section{Paths, Graphs, etc.}

In what follows, assume a domain space \(X\); e.g. \(a\) means \(a\in
X\), \(\forall x\) means \(forall x\in X\).

NB: the examples assume a two dimensional space, but the concepts
apply to any n-dimensional space.  For example, a path between two
points in the plane may be 1-dimensional: a "line".  A path between two
points is 3-D space may be a line, but could also be 2-dimensional ...?


\subsection{What is a path?}

A \emph{path function} between two distinct points \(a\) and \(b\) in
space \(X\) is a continuous function \(f\) over the unit interval
\([0..1]\) into \(X\) with \(f(0)=a, f(1)=b\).

We distinguish between path functions and their graphs.  We call the
latter \emph{path graphs}.

There are infinitely many path graphs between any two points \(a\) and
\(b\).  A path may contain loops, dips, oscillations, etc.; it also
may double back on itself.  It may contain its endpoints more than
once; for example, imagine a straight line path that starts at \(a\)
and ``yo-yos'' through its endpoints multiple times before coming to
rest at \(b\).

In other words, the \emph{graph} of a path function need not itself
look like a function graph.

For every path \emph{graph} except the canonical reflexive path (see
below) there are infinitely many path \emph{functions}.  That is
because, although each such function has the same graph, each may
``draw'' that graph in a different way; that is, at different rates as
we move from \(0\) to \(1\) in the domain.  To make this concrete,
imagine a straight-line path across a campus from the bookstore to the
saloon.  To go from books to booze you can walk at a steady pace, or
you can start out running and then slow to a walk, or a crawl, or
vice-versa - you can pace yourself in an infinite number of ways.

A programmer can think of different path functions having the same
path graph as disinct ``implementations'' of the graph.  Another way
to think of it is in terms of ``easing'' functions for animation; see
the examples at http://easings.net (Actually those examples are not
quite right, because each draws a different path.  Another tool:
http://matthewlein.com/ceaser/)

Special case: in 2D at least, the linear path from a to b with
constant rate of change.  We can call this the canonical path from a
to b.

We will see later that HoTT uses this concept of a path to express the
notion of equality.

\section{Reflexivity}

When both endpoints of the path are the same point \(a\), we call the
path \emph{the reflexive path (function) at a}, or just the reflexive
path.  There are infinitely many reflexive path functions for every
point; each one can be thought of as a loop.

As a special case, we have the \emph{constant path function at a},
defined as above, except that both endpoints are \(a\) and \(f(x)=a\)
for all x.  This path is doubly constant: both its rate of change and
its graph are constant.

We will see later that HoTT uses this kind of path to express the
mathematical notion of reflexivity (which is why we call it a
reflexive path).

\section{Homotopy}

\begin{quote}

A homotopy between a pair of continuous maps \(f : X1 → X2\) and \(g :
X1 → X2\) is a continuous map \(H : X1 × [0, 1] → X2\) satisfying
\(H(x, 0) = f (x)\) and \(H(x, 1) = g(x)\).

(HoTT Book p. 59)
\end{quote}

This is the general definition of homotopy, which applies to any
functions, not just path functions.  A homotopy takes each point in
the domain from its image under f to its image under g, as the
"control pointer" moves across the unit interval from 0 to 1.  This
has the effect of "sliding" the image of f onto the image of g,
without interruption.  Keep in mind, this applies to spaces of any
dimension.

TODO: animated examples.

\begin{quote}
In the specific case of paths p and q from x to y, a homotopy is a
continuous map H : [0,1] × [0,1] → X such that H(s,0) = p(s) and
H(s,1) = q(s) for all s ∈ [0,1]. In this case we require also that
H(0,t) = x and H(1,t) = y for all t ∈ [0,1], so that for each t the
function H(–,t) is again a path from x to y; a homotopy of this sort
is said to be endpoint-preserving or rel endpoints.  (HoTT Book p. 59)
\end{quote}

TODO: animations

Note the parallel with product types!  Homotopy as product of paths;
path as product of points.  Then note parallel to binary dependent
type Unit - -.

Homotopy as 2-D path between 1-d paths; path as 1-D path between 0-D
points.

Homotopy as continuous deformation: useful, but misleading.  The
end-paths of a homotopy do not deform; nothing every changes in math.
Instead we need to think about moving from dimension to dimension,
ie. shifting frames of reference, and thus interpretation.  A path may
be thought of as a 1-D object embedded in 2-D space.  But how can a
2-D space contain anything other than 2-D objects?  Better to think of
a path as a 1-D unifier of its 0-D endpoints, and a homotopy as a 2-D
unifier of its 1-D endpaths.  In each case we can think of unifying as
a sort of interpolation of points or paths in order to form a
continuous unity. But notice that the original endpoints/endpaths are
not truly unified; rather the unit formed is a new kind of thing, of a
different order or dimension.  In a sense the end objects become
lifted out of their isolation and into a new synthesis or unity.

After all, in the beginning the endpoints of a path are not endpoints,
they're just distinct points.  The endpoints of the path are something
else again; a kind of translation of the original isolated points into
parts of a whole.

And in such unities, every part is "equal" to every other part, and to
the whole.  Every part of a path is a path, etc.  So they are a
special kind of unity: fractal.  Equality as self-similarity.  Which
works for both the standard HoTT interpretation and my Unit interp.

Leading naturally to the use of dependent Unit types to model equality.

The fact that a path graph can be drawn in 2-D space does not make it
a 2-D object - the graph, not the path is 2-D.  The path is just a
(1-D) collection of points.

Which suggests we should distinguish between a path graph (a 2-D
object) and a path image (the set of points that is its range).  A
path graph is an interpretation of a path image.

We can think of a homotopy between paths as a function on the unit
square.

A path between a and b can also be expressed as a homotopy between
reflexive paths at a and b; loosely, a homotopy between points.  It's
not really between points, but between reflexive paths at points.





\section{Equality}

Reflexivity, equality, and identity are distinct concepts.

Reflexivity: a=a.  In homotopy theory, this means a path function from
a to itself; we will call such a function a \emph{path function for
reflexivity at a} or, for brevity, a reflexive path (function) at a.
There are infinitely many such functions.


The \emph{canonical path function for reflexivity at a} is the
constant path function at \(a\): \(\forall x,\ f(x) = a\).  All other
path functions for reflexivity at \(a\) are non-canonical; they are
distinct from the canonical path function, but can be \emph{expressed}
as the canonical function.  The mechanism for this is homotopy -
continuous deformation of one non-canonical path function into another
path function.  Later, we will see that HoTT interprets this as
``equality'', but it is critical to see that two paths that we call
equal in this sense are not identical; they remain distinct paths.  It
is the interchangeability enabled by the homotopy that justifies
calling this relation equality; if \(p\) and \(q\) are distinct paths,
\(p=q\) is just another way of saying that they are homotopic.

What gets deformed?  The path function or the path graph?  The path
function - the graph is just a one-sided representation of the
function.

\end{document}
