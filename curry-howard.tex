%%%%%%%%%%%%%%%%%%%%%%%%%%%%%%%%
\chapter{Curry-Howard}
\label{sect:curry-howard}

\begin{ednote}
  Usually presented as ``propositions-as-types'', but this suggests an
  asymmetrical relationship; in fact the principle is that
  propositions \emph{are} types, and vice-versa.  This is a major move
  in type theory, introduced by \ML(?) based on work by Curry and
  Howard.  TODO: what exactly are the implications of this principle?
\end{ednote}

\begin{ednote}
  The critical point is that we go minimalist: start with the minimal
  logical language, which means combinatory logic.  It is the
  isomorphism between the logical constants and the combinators
  (Curry) that motivates Curry-Howard.  Once you see the connection at
  this minimalist level, it is easy to see it at any level, since the
  logical constants are the basic building blocks from which all
  propositions are constructed.
\end{ednote}


\begin{ednote}
  Start with Schoenfinkel and Curry, and the goal of finding the
  absolute minimum, which means eliminating variables.  Then the basic
  combinatorys, then the isomorphism to the logical constants.

  Equivalence of combinatory logics (no vars) and lambda calculus (vars)
\end{ednote}


Analogies.  Proof/proposition, term/type: ``There is also a one-to-one
correspondence between proofs of a certain proposition in constructive
predicate logic and terms of the corresponding type.'' (Dependent Types at Work)
