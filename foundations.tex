%%%%%%%%%%%%%%%%%%%%%%%%%%%%%%%%
\section{Foundations}
\label{sect:foundations}

\HoTT purports to offer a new foundational concept for mathematics.  If
we take assertion to be the foundational concept of type theory (I'm
not sure this works, but it seems plausible), then Brandom's account
of assertion can link type theory to a foundational account of
discursive practice (rationality).

Today set theory is the reigning foundational theory of mathematics.
It's fairly easy to present it as such: first you list the axioms,
then you show how to ``construct'' the natural numbers from sets,
using a successor function.  Or you might follow the lead of the Z
specification notation\sidenote{\cite{zed_spec}}, and proceed from
sets to relations and then to functions.  However you do it, it's all
pretty intuitive and relatively easy to explain, even to mathophobes.

What would such a foundational presentation look like for \HoTT?  If
\HoTT turns out to be a genuinely foundational theory, then it must be
grounded in intuition; specifically, we should expect that its basic
notions correspond in some way to some collection of pre-theoretic
mathematical intuitions, just as the axioms of set theory do, or as
the axioms of geometry match our ordinary intuitions about the
organization of space as we experience it pre-theoretically.

Presentations of set theory usually begin by discussing the axioms;
but even though axioms serve as ``unexplained explainers'', such a
presentation inevitably depends on a yet more primitive layer of
concepts.  Specifically, not only the (pre-theoretical) concepts of
set, subset, and membership, but also axiom and perhaps proof.  All of
these ``preliminary'' concepts---let's call them
``principles''---correspond more or less directly to intuitions
available to any concept-user.

In general, an explicit account of the fundamental \textit{principles}
of set theory is either omitted or informally glossed, before the
presentation moves on to the axioms.  But type theory, in the end, is
radically different from set theory at a very fundamental level, as
far as I can see.  ``Set'' and ``type'' are so easily grasped that it
is easy think of them as more-or-less the same sort of thing; but if
you look hard at them, they are very different, even fundamentally
different.  So I think a presentation of \HoTT would be well served by
beginning with an explicit account of principles, even before moving
on to consider primitives of the theory.

What are the pre-theoretical principles and primitives of \HoTT?  The
obvious place to start is ``type''.  The concept of ``type'' obviously
emerges from ordinary experience; indeed, it is arguably more
primitive than the concept of ``set''.  Just look at the vast
literature on the emergence of categorization in developmental and
cognitive psychology; the ability to categorize is undoubtedly one of
the most primitive human intellectual skills, if not the most
primitive.  It may even be a primitive animal capability--bees
categorize flowers, and every member of sexually reproducing species
categorizes possible mates.

What about ``axiom''?  At first glance it would seem that any
foundational account of mathematics (or anything else for that matter)
must rest on an axiomatic foundation.  Which is just another way of
saying that any explanation of anything must eventually bottom out on
a bedrock of unexplained explainers.  You can't explain everything
without entering an infinite regression.

On the other hand, we can view axiomatic explanation as just one
``style'' of explanation, one of many.  When you begin with axioms,
you present them as unequivocally (and unquestionnably) true.  But
this is really a bit of salesmanship; sometimes axioms turn out not to
be quite as axiomatic as they seem.  Reconceptualizations happen,
which may lead us to view axioms in a new light in which they do not
look quite as certain.  Then axiomatic explanations are still
intelligible, but are no longer unquestionnable.  The classic example
of this sort of evolution is to be found in the history of geometry.
Before the development of non-Euclidean geometries in the 19th
century, the axioms of Euclidean geometry were not only unquestioned
but unquestionable:\sidenote{I suspect I'm overstating the case here.
  Mathematicians: is this true?} the idea that parallel lines could
meet was not just wrong, but crazy.  Today, using axioms to define a
geometry is just a way of making clear the assumptions necessary to
make the theory work.  They no longer represent essential
connections to externally available bits of certain knowledge.

In other words, axioms are not a necessary condition of adequate
explanation.  So the question is whether or not the axiomatic style is
most appropriate for a presentation of \HoTT?  On the one hand, it
seems to me that it is not necessary; an adequate explanation of \HoTT
without axioms should be possible.  For example, we can treat the
concept of ``type'' as primitive, even if we cannot find a good way to
express it axiomatically.\sidenote{This is a little fuzzy; maybe it
  doesn't even make sense.  But as long as we're rethinking the
  foundations of mathematics, we might as well rethink everything.}

\begin{remark}
  Leaving full presentation of principles for later.  I think it
  includes at least proposition and judgment, maybe inference and
  proof.
\end{remark}

In any case, we'll have to begin somewhere, by stating some
fundamental principles; then we'll need an account of the primitives,
whether they take the form of axioms or not.  What are the principles
upon which \HoTT depends?  And once we have some principles (which are
external to the theory proper), what are the primitives (which are
``inside'' the theory)?\sidenote{Ok, ``primitive'' sounds a lot like
  ``axiom''.  But I think there's a difference, even if I can't quite
  articulate it.  Let's provisionally say that a primitive is an axiom
  without the concommitant commitment to unquestionned certainty.}

Here are some possibilities, based on my understanding of the material
in \HoTTB.  Please keep in mind this is coming from somebody who
thinks he has a fairly good grasp of what type theory is all about,
but is still grappling with \HoTT.

%%%%%%%%
\subsection{\HoTT Principles}
\label{subs:hottprinciples}

\begin{description}
\item [Type] Obviously a fundamental concept.  What to say about it, though, is
  not so obvious.
\item [Proposition]
\item [Judgment]
\item [Proof]\sidenote{from Latin \textit{probare} "to make good;
  esteem, represent as good; make credible, show, demonstrate; test,
  inspect; judge by trial" (source also of Spanish \textit{probar},
  Italian \textit{probare}), from \textit{probus} "worthy, good,
  upright, virtuous,"} Two kinds, corresponding to the two kinds of
  provables:\sidenote{Remember, we're talking about pre-theoretical
    principles (concepts) here, not about \HoTT per se.}
\begin{description}
\item [Demonstration] - \textit{rational argument} that compels assent
  to a proposition\sidenote{``Demonstration'' is intuitively
    satisfying, but conceptually misleading, insofar as it suggests a
    visual metaphor.  That would be classical; but for type theory we
    want metaphors of construction, not inspection.}
\item [Witness] - evidence that bears witness to the existence of a kind or category
\end{description}
\item [Inference]
\item etc.
\end{description}

\newthought{The concept of proof in type theory} deserves special
attention.  \Cref{sect:proof} examines it in detail; here, suffice it
to say that it extends beyond the traditional and intuitive notion of
proof as something one does to or with propositions.  In type theory,
propositional types represent propositions, so a type-theoretic proof
of a propositional type---call it a ``tt-proof''--- corresponds to an
ordinary proof of a proposition; it essentially involves inference,
for example.  But type theory also has lots of non-propositional
types, like \N.  These do \textit{not} represent propositions:
propositions have truth-values, natural numbers do not.  In set
theory, there is no connection between sets, elements, and proofs.  An
element either is, or is not, a member of a given set.  Period, full
stop.  The notion of proof never enters the set-membership
picture.\sidenote{That need not mean that proving membership is never
  an issue.  But you don't prove membership; rather, you prove that
  the element satisfies some predicate, which is a different concept.}
In particular, the existence of a set is not dependent on particular
members, and the fact that some element is a member of some set has no
significance with respect to the existence of the set.  By contrast,
in type theory construction of an element of a type counts as proof of
the type.  Etc.\sidenote{FIXME: fix this language.}  But this kind of
``proof'' is not like proof of a proposition; it does not involve a
proposition that may be true or false, and it does not involve
inference.  Instead it serves as a kind of evidence that shows the
type.

\begin{remark}
  Is there a significant distinction to be made between proof and
  witness?  I suspect there is, based on the difference between
  propositions and names.  Both count as evidence, but there is a
  difference between an inferential proof of a proposition and a
  ``testimonial'' witness to a kind.  Propositions-as-types unifies
  the two ideas, but does not erase the distinction.
\end{remark}

%%%%%%%%
\subsection{\HoTT Primitives}
\label{subs:hottprimitives}

\HoTT primitives are ....\sidenote[][-48pt]{A proper exposition would list 1) the name of the
  primitive, e.g. ``\(\Pi\)-type''; 2) the ``constructor'' symbol,
  e.g. \(\cross\) for product types; 3) the analogous concept from set
  theory, and then the ``rules'' for defining a type (formation,
  construction, elimination, computation, uniqueness).}

\begin{description}
\item [Function] ``Unlike in set theory, functions are not defined as
  functional relations; rather they are a primitive concept in type
  theory.''\sidenote{Or: set theory
    \textit{defines} a function as a set of ordered pairs whose domain
    has no duplicates; in other words, it treats a function and its
    ``graph'' as the same thing.  Question: what happens to the graph
    of a function in type theory?} \citep[p. 21]{hottbook}

\item [Product] Product types correspond to cartesian products in set
  theory.  The constructor symbol is the same as in set theory:
  \(\cross\).\sidenote{Why isn't this called the ``\(\Huge\cross\)-type''?}
  ``[U]nlike in set theory, where we define ordered pairs to be
  particular sets and then collect them all together into the
  cartesian product, in type theory, ordered pairs are a primitive
  concept, as are functions.''\citep[p. 26]{hottbook}

\item [Coproduct type] Coproduct types correspond to disjoint unions
  in set theory.  ``In type theory, as was the case with functions and
  products, the coproduct must be a fundamental construction, since
  there is no previously defined notion of ``union of
  types''.\citep[p. 33]{hottbook}

\item [Proposition type] Conceptually, at least, this seems primitive.
  Especially if the concept of ``proposition'' counts as a
  pre-theoretic principle.  Which implies that proof is also a
  pre-theoretic principle.  Propositions are fundamentally different
  than the other kinds of type, since they have truth-conditions. Etc.
  It follows that proofs are fundamentally different from other kinds
  of witness.

\end{description}

\begin{remark}
  For consistency, we might want to use symbols to designate all of
  the primitives, just as we do for \(\Pi\) and \(\Sigma\).  This
  would give us: \(\Huge\fun\)-types, \(\Huge\cross\)-types, and
  \(\Huge +\)-types.
\end{remark}


%%%%%%%%
\subsection{\HoTT Quasi-primitives}
\label{subs:quasiprim}

\noindent ``Fundamental''\sidenote{Obviously we need a better bit of
  terminology.  ``Quasi-primitives''?  ``Neo-primitives''?  These
  types are not primitive, strictly speaking, but on the other hand
  they are basic.  I think there is another fundamental principle at
  work here.  In set theory, for example, the concept of function is
  not only not primitive, it isn't necessary.  You could discard it
  and still have set theory.  But my intuition tells me that e.g. the
  concept \(\Pi\)-type is in a sense necessary or essential in type
  theory, even if it is not primitive.  Once you have the primitives,
  you necessarily have these non-primitive basic types.  Dunno if
  that's correct, but it would sure be nice if it were.} (but
non-primitive) types.  These types seem to be on a par with the
primitive types as far as importance goes, but they presuppose the
primitives, so cannot themselves be considered primitive.

\begin{description}

\item [Universe]  Is this a primitive?  Probably not, since it builds on the type concept.

\item [\(\Pi\)-type] Informally, ``dependent function''
  types.\sidenote{As a practical matter, I think it would be useful to
    have an informal term for these types that falls between
    ``dependent function type'' and \(\Pi\)-type.  Something like
    ``p-function type''.  \(\Pi\)-type is admirably concise, but I
    think it should mention ``function'', since it names a kind of
    function.}  The concept of \(\Pi\)-type is a generalization of the
  concept of function type, so it isn't primitive.

\item [\(\Sigma\)-type] Informally, ``dependent pair''
  type.\sidenote{Shouldn't this be called ``dependent
    \textit{product}'' type?  The type is product, not pair; pairs are
    ``elements'' of the type.  Informally, maybe ``sig-prod type?}
  The concept of \(\Sigma\)-types is a generalization of the concept
  of product type.

\end{description}

%%%%%%%%
\subsection{\HoTT Standard Type Library}
\label{subs:hottstdlib}

\begin{remark}
  By analogy to the usual ``standard library'' of programming
  languages.  The idea is to list commonly used types that are neither
  primitive nor quasi-primitive; ``application'' types, in a sense.
\end{remark}


\begin{description}
\item [Boolean] \citep[p. 34]{hottbook}
\item [$\nat$] \citep[p. 36]{hottbook}
\item [Propsition]
\item [Identity] 
\end{description}

