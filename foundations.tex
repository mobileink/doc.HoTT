%%%%%%%%%%%%%%%%%%%%%%%%%%%%%%%%
\chapter{Type Theory: Foundations}
\label{sect:foundations}

\HoTT{} purports to offer a new foundational concept for mathematics.  If
we take assertion to be the foundational concept of type theory (I'm
not sure this works, but it seems plausible), then Brandom's account
of assertion can link type theory to a foundational account of
discursive practice (rationality).

Today set theory is the reigning foundational theory of mathematics.
It's fairly easy to present it as such: first you list the axioms,
then you show how to ``construct'' the natural numbers from sets,
using a successor function.  Or you might follow the lead of the Z
specification notation\sidenote{\cite{zed_spec}}, and proceed from
sets to relations and then to functions.  However you do it, it's all
pretty intuitive and relatively easy to explain, even to mathophobes.

What would such a foundational presentation look like for \HoTT{}?  If
\HoTT{} turns out to be a genuinely foundational theory, then it must be
grounded in intuition; specifically, we should expect that its basic
notions correspond in some way to some collection of pre-theoretic
mathematical intuitions, just as the axioms of set theory do, or as
the axioms of geometry match our ordinary intuitions about the
organization of space as we experience it pre-theoretically.

Presentations of set theory usually begin by discussing the axioms;
but even though axioms serve as ``unexplained explainers'', such a
presentation inevitably depends on a yet more primitive layer of
concepts.  Specifically, not only the (pre-theoretical) concepts of
set, subset, and membership, but also axiom and perhaps proof.  All of
these ``preliminary'' concepts---let's call them
``principles''---correspond more or less directly to intuitions
available to any concept-user.

In general, an explicit account of the fundamental \textit{principles}
of set theory is either omitted or informally glossed, before the
presentation moves on to the axioms.  But type theory, in the end, is
radically different from set theory at a very fundamental level, as
far as I can see.  ``Set'' and ``type'' are so easily grasped that it
is easy think of them as more-or-less the same sort of thing; but if
you look hard at them, they are very different, even fundamentally
different.  So I think a presentation of \HoTT{} would be well served by
beginning with an explicit account of principles, even before moving
on to consider primitives of the theory.

\begin{ednote}
  Todo: add somewhere in here an explanation of what we mean by
  ``primitive''.  The basic idea is conceptual independence.  A
  primitive is like an axiom in the traditional sense: an unexplained
  explainer.  If it depends (conceptually) on some other concept, then
  it is not primitive.

  For example: on Brandom's account, representation is not primitive,
  since it depends on other things.  Nor is conceptual content, for
  that matter, since it is instituted by proprieties of practice,
  which are primitive.

  Another way to look at it: if you can get along without it---discard
  it and still count as playing the same game---then it is not
  primitive.  So for example dependent types are not primitive: we can
  play the type-theory game without them.

  On the other hand, I'm trying to distinguish between conceptual
  primitives---what you have to be able to think in order to think
  type-theoretically---and the specific bits of a particular theory
  which are to count as primitive for that theory.  In other words, we
  have the primitives of rationality, and we have the primitives of
  specific theories.  \HoTT{} counts as a specific theory; type theory in
  general, on the other hand, is closely tied (or at least tie-able)
  to at least one foundational account of rationality (Brandom's).

  This might not be worth the trouble, but then again type-theory
  strikes me as sufficiently radical to merit this kind of close
  philosophical scrutiny.  I think it's hard to overstate the extent
  to which a type-theoretic mode of thinking differs from classic (set
  theoretic, classic logic) modes of thinking, and the differences can
  be quite subtle.  After all, its a pretty radical step to say (with
  Huw Price\citep{price_naturalism_2010}) that we should
  discard the concept of representation \textit{in toto} since it has
  no useful theoretical work to do.  And I think that type theory and
  \HoTT{} amount to ``natural'' idioms for such pragmatist modes of
  thinking.

  Finally: the pragmatic angle, where the ultimate primitive is just
  practice.  From that perspective, we ought to be looking not so much
  at primitive concepts as primitive practices.  Two obvious
  candidates are categorization (=conceptualization) and counting.
  Pre-linguistic creatures can do both (language is not a necessary
  condition for counting), and linguistic communities invent
  terminology that allows there members to say what they otherwise
  would only be able to do---say \textit{that} there are three lions
  down by the watering hole rather than pantomiming or pointing them
  out directly.  This suggests that one way to present type theory is
  treat the type of natural numbers as, if not primitive, only one
  step up from primitive (type).
\end{ednote}

What are the pre-theoretical principles and primitives of \HoTT{}?  The
obvious place to start is ``type''.  The concept of ``type'' obviously
emerges from ordinary experience; indeed, it is arguably more
primitive than the concept of ``set''.  Just look at the vast
literature on the emergence of categorization in developmental and
cognitive psychology; the ability to categorize is undoubtedly one of
the most primitive human intellectual skills, if not the most
primitive.  It may even be a primitive animal capability--bees
categorize flowers, and every member of sexually reproducing species
categorizes possible mates.

What about ``axiom''?  At first glance it would seem that any
foundational account of mathematics (or anything else for that matter)
must rest on an axiomatic foundation.  Which is just another way of
saying that any explanation of anything must eventually bottom out on
a bedrock of unexplained explainers.  You can't explain everything
without entering an infinite regression.

On the other hand, we can view axiomatic explanation as just one
``style'' of explanation, one of many.  When you begin with axioms,
you present them as unequivocally (and unquestionnably) true.  But
this is really a bit of salesmanship; sometimes axioms turn out not to
be quite as axiomatic as they seem.  Reconceptualizations happen,
which may lead us to view axioms in a new light in which they do not
look quite as certain.  Then axiomatic explanations are still
intelligible, but are no longer unquestionnable.  The classic example
of this sort of evolution is to be found in the history of geometry.
Before the development of non-Euclidean geometries in the 19th
century, the axioms of Euclidean geometry were not only unquestioned
but unquestionable:\sidenote{I suspect I'm overstating the case here.
  Mathematicians: is this true?} the idea that parallel lines could
meet was not just wrong, but crazy.  Today, using axioms to define a
geometry is just a way of making clear the assumptions necessary to
make the theory work.  They no longer represent essential
connections to externally available bits of certain knowledge.

In other words, axioms are not a necessary condition of adequate
explanation.  So the question is whether or not the axiomatic style is
most appropriate for a presentation of \HoTT{}?  On the one hand, it
seems to me that it is not necessary; an adequate explanation of \HoTT{}
without axioms should be possible.  For example, we can treat the
concept of ``type'' as primitive, even if we cannot find a good way to
express it axiomatically.\sidenote{This is a little fuzzy; maybe it
  doesn't even make sense.  But as long as we're rethinking the
  foundations of mathematics, we might as well rethink everything.}

\begin{ednote}
  \HoTTB addresses this explicitly, p. 20-21.  Roughly, it replaces
  the axioms of set theory with rules.  But that's not entirely
  accurate: the rules of \HoTT, after all, are themselves axiomatic.
  You really cannot get by without some sort of axiom; every journey
  must start from somewhere.  So what is the difference between the
  axioms in set theory and those in \tth?  I think it boils down to
  the difference between conditional and unconditional statements, but
  I'm not yet sure what the significance of this is.
\end{ednote}

\begin{ednote}
  Leaving full presentation of principles for later.  I think it
  includes at least proposition and judgment, maybe inference and
  proof.
\end{ednote}

In any case, we'll have to begin somewhere, by stating some
fundamental principles; then we'll need an account of the primitives,
whether they take the form of axioms or not.  What are the principles
upon which \HoTT{} depends?  And once we have some principles (which are
external to the theory proper), what are the primitives (which are
``inside'' the theory)?\sidenote{Ok, ``primitive'' sounds a lot like
  ``axiom''.  But I think there's a difference, even if I can't quite
  articulate it.  Let's provisionally say that a primitive is an axiom
  without the concommitant commitment to unquestionned certainty.}


\begin{ednote}
  ``Type theory, formal or informal, is a collection of rules for
  manipulating types and their elements.''\HoTTB p. 101
\end{ednote}

Here are some possibilities, based on my understanding of the material
in \HoTT{}B.  Please keep in mind this is coming from somebody who
thinks he has a fairly good grasp of what type theory is all about,
but is still grappling with \HoTT{}.

%%%%%%%%
\section{Principles of Type Theory}
\label{subs:hottprinciples}

\begin{ednote}
  The idea here is to discuss the ``primitives'' of type theory; that
  is, the concepts that are essential to any type theory.
\end{ednote}

\begin{description}
\item [Type] Obviously a fundamental concept.  What to say about it, though, is
  not so obvious.
\item [Proposition]
\item [Judgment]
\item [Proof]\sidenote{from Latin \textit{probare} "to make good;
  esteem, represent as good; make credible, show, demonstrate; test,
  inspect; judge by trial" (source also of Spanish \textit{probar},
  Italian \textit{probare}), from \textit{probus} "worthy, good,
  upright, virtuous,"} Two kinds, corresponding to the two kinds of
  provables:\sidenote{Remember, we're talking about pre-theoretical
    principles (concepts) here, not about \HoTT{} per se.}
\begin{description}
\item [Demonstration] - \textit{rational argument} that compels assent
  to a proposition\sidenote{``Demonstration'' is intuitively
    satisfying, but conceptually misleading, insofar as it suggests a
    visual metaphor.  That would be classical; but for type theory we
    want metaphors of construction, not inspection.}
\item [Witness] - evidence that bears witness to the existence of a kind or category
\end{description}
\item [Inference]
\item [Induction] Seems pretty basic to me, and as far as I know,
  nobody has ever been able to explain it in terms of more primitive
  notions.  Without a Principle of Induction (of some kind), we would
  not be able to, for example, form the Natural Numbers in type theory.
\item etc.
\end{description}

\newthought{The concept of proof in type theory} deserves special
attention.  \Cref{sect:proof} examines it in detail; here, suffice it
to say that it extends beyond the traditional and intuitive notion of
proof as something one does to or with propositions.  In type theory,
proposition types represent propositions, so a type-theoretic proof
of a proposition type---call it a ``tt-proof''--- corresponds to an
ordinary proof of a proposition; it essentially involves inference,
for example.  But type theory also has lots of non-propositional
types, like \N.  These do \textit{not} represent propositions:
propositions have truth-values, natural numbers do not.  In set
theory, there is no connection between sets, elements, and proofs.  An
element either is, or is not, a member of a given set.  Period, full
stop.  The notion of proof never enters the set-membership
picture.\sidenote{That need not mean that proving membership is never
  an issue.  But you don't prove membership; rather, you prove that
  the element satisfies some predicate, which is a different concept.}
In particular, the existence of a set is not dependent on particular
members, and the fact that some element is a member of some set has no
significance with respect to the existence of the set.  By contrast,
in type theory construction of an element of a type counts as proof of
the type.  Etc.\sidenote{FIXME: fix this language.}  But this kind of
``proof'' is not like proof of a proposition; it does not involve a
proposition that may be true or false, and it does not involve
inference.  Instead it serves as a kind of evidence that shows the
type.

\begin{ednote}
  Is there a significant distinction to be made between proof and
  witness?  I suspect there is, based on the difference between
  propositions and names.  Both count as evidence, but there is a
  difference between an inferential proof of a proposition and a
  ``testimonial'' witness to a kind.  Propositions-as-types unifies
  the two ideas, but does not erase the distinction.
\end{ednote}

%%%%%%%%
\section{Type Formers}
\label{subs:hottrules}

\begin{ednote}
  There's a lot to be said here.  ``Rules'' is not quite right;
  construction rules, for example, are not just laws governing
  correctness, but ``how-to'' rules, productive procedures.

  If such ``rules'' are primitives, then we have a very sharp contrast
  with classic mathematics and logic.  The \HoTT{}B{} doesn't dwell on
  the rules; it seems to treat them instrumentally, as devices we need
  in order to come up with new types and elements.  But they are
  substantive; they define what it is to be a type or element.

  Conceptually: can we have types without these rules?  Intuitively,
  we obviously can: we don't need (explicit) rules in order to think
  of, say, $2$ as a natural number.  Categorization, apparently, does
  not rely on any notion of construction.  This suggests that these
  rules are not primitive---we could discard them and still have
  types.  But I think maybe that's the wrong way to think about
  things.  It's too representationalist.  On the inferentialist model,
  which is holistic, you cannot think ``$2$'' without also grasping
  its inferential structure.  Can we treat this structure as
  constructive?  I think so, but it will take some work to articulate
  just why and how.  This will require considerable refinement of what
  is meant by ``construction''.
\end{ednote}

\begin{description}
\item [Type formation rules]
\item [Introduction rules] i.e. construction rules
\item [Elimination rules]
\item [Computation rules]
\item [Uniqueness principle]
\end{description}

